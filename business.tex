\documentclass[12pt]{memoir}

\usepackage{mathpazo}

\usepackage{fontspec}
%	\setmainfont{PalatinoLinotype-Roman}[afrc,cpsp,frac,kern,liga,onum,pnum,smcp,subs,sups]
\setmainfont{PalatinoLinotype-Roman}[Fractions = Alternate,Kerning = On,Ligatures = Common,Numbers = Lowercase,Numbers = Proportional,ItalicFont = PalatinoLinotype-Italic, BoldFont = PalatinoLinotype-Bold]
	\setsansfont{LinBiolinumO}[Scale=MatchLowercase]
	\setmonofont{LinLibertineMO}[Scale=MatchLowercase]

%\usepackage{mathpazo}
%\setromanfont{Junicode}
%
%\newICUfeature{StyleSet}{1}{+ss01}
%\newICUfeature{StyleSet}{2}{+ss02,-liga}
%\newICUfeature{StyleSet}{3}{+ss03}
%\newICUfeature{LigType}{disc}{+dlig}
%\newICUfeature{LigType}{hist}{+hlig}
%\newICUfeature{IPAMode}{on}{+mgrk,-liga}
%\newICUfeature{Compose}{off}{-ccmp}
%\newICUfeature{Contextual}{on}{+calt}
%\newICUfeature{Swash}{on}{+swsh}
%\newICUfeature{Fractions}{on}{+frac}
%\newICUfeature{Superscripts}{on}{+sups}
%\newICUfeature{Subscripts}{on}{+subs}


%\usepackage[osf,sups,scosf]{cochineal}
%\renewcommand{\sfdefault}{iwona}
%\renewcommand{\ttdefault}{txtt}
%\usepackage{Typocaps}

%\usepackage{unicode-math}
%  \setmathfont[Scale=MatchUppercase]{libertinusmath-regular.otf}
%\usepackage{lettrine}


%\usefont{U}{Typocaps}{xl}{n}

%\newcommand{\InitalTypocap}[2]{\lettrine[realheight=true]{\usefont{U}{Typocaps}{xl}{n} #1}{#2}}



%%% Personal page style
\copypagestyle{myPageStyle}{ruled}
	\makeevenfoot{myPageStyle}{\thepage}{}{}
	\makeoddfoot{myPageStyle}{}{}{\thepage}





\renewcommand\eminnershape{\upshape\scshape}
\usepackage{ulem}\normalem

\pagestyle{ruled}

%\setcounter{secnumdepth}{3}
%\setcounter{tocdepth}{2}

\usepackage{textcomp}
\usepackage{microtype}
\usepackage[unicode=true,
 bookmarks=true,bookmarksnumbered=false,bookmarksopen=false,
 breaklinks=true,pdfborder={0 0 0},pdfborderstyle={},backref=page,colorlinks=false]
 {hyperref}
\hypersetup{pdftitle={Business Perspectives From The Word},
 pdfauthor={Thomas Allen Bathgate},
 pdfkeywords={business, Bible, wisdom},
 pdfborderstyle={}}


\makeatletter

%%%%%%%%%%%%%%%%%%%%%%%%%%%%%% LyX specific LaTeX commands.

\@ifundefined{date}{}{\date{}}
%%%%%%%%%%%%%%%%%%%%%%%%%%%%%% User specified LaTeX commands.
\usepackage[xspace]{ellipsis}
\usepackage{parskip}
%%% ToC down to subsections
\settocdepth{subsection}
%%% Numbering down to subsections as well
\setsecnumdepth{subsection}

\PassOptionsToPackage{hyphens}{url} % url is loaded by hyperref

\urlstyle{same}  % don't use monospace font for urls


%%%These calculations are for 9x6 pages.
%\setstocksize{11in}{8.5in}
%\settrimmedsize{9in}{6in}{*}
%	%\setlxvchars[\sffamily]
%	%\setxlvchars[\sffamily]
%	\setlength{\trimtop}{0pt}
%%	\setlength{\trimedge}{\stockwidth}
%%	\addtolength{\trimedge}{-\paperwidth}
%%	\settypeblocksize{6in}{4in}{*}
%	\settypeblocksize{6in}{4in}{*}
%	\setulmargins{*}{*}{1.618}
%	\setlrmargins{*}{*}{1.618}
%	\setheaderspaces{*}{*}{1.618}
%	\setbinding{0.75in}
%	\checkandfixthelayout

%%%These calculations are for 8.5x11 pages
%%%and using Golden Mean dimensions

%\setstocksize{11in}{8.5in}
%\settrimmedsize{11in}{8.5in}{*}
%\setlxvchars[\sffamily]
%\setxlvchars[\sffamily]
%\settypeblocksize{7.333in}{5.6667in}{*}
%	\settypeblocksize{6in}{4in}{*}
%\setulmargins{*}{*}{1.618}
%\setlrmargins{0.676in}{*}{}
%\setheaderspaces{*}{*}{1.618}
%\setbinding{0.7in}
%\checkandfixthelayout


%% --------------------------------------------------- %%

\usepackage{amsfonts}
\newcommand*{\plogo}{\fbox{$\mathfrak{MER}$}} % Generic publisher logo

\makeatother

\usepackage[texindy,nonewpage,splitindex]{imakeidx}
	\indexsetup{level=\section*,toclevel=chapter}

\makeindex
\makeindex[columns=3,name=gen,title=Genesis]
\makeindex[columns=3,name=exo,title=Exodus]
\makeindex[columns=3,name=lev,title=Leviticus]
\makeindex[columns=3,name=num,title=Numbers]
\makeindex[columns=3,name=deu,title=Deuteronomy]
\makeindex[columns=3,name=jos,title=Joshua]
\makeindex[columns=3,name=jdg,title=Judges]
\makeindex[columns=3,name=rut,title=Ruth]
\makeindex[columns=3,name=1sa,title=1 Samuel]
\makeindex[columns=3,name=2sa,title=2 Samuel]
\makeindex[columns=3,name=1ki,title=1 Kings]
\makeindex[columns=3,name=2ki,title=2 Kings]
\makeindex[columns=3,name=1ch,title=1 Chronicles]
\makeindex[columns=3,name=2ch,title=2 Chronicles]
\makeindex[columns=3,name=ezr,title=Ezra]
\makeindex[columns=3,name=neh,title=Nehemiah]
\makeindex[columns=3,name=est,title=Esther]
\makeindex[columns=3,name=job,title=Job]
\makeindex[columns=3,name=psa,title=Psalms]
\makeindex[columns=3,name=pro,title=Proverbs]
\makeindex[columns=3,name=ecc,title=Ecclesiastes]
\makeindex[columns=3,name=son,title=Song of Solomon]
\makeindex[columns=3,name=isa,title=Isaiah]
\makeindex[columns=3,name=jer,title=Jeremiah]
\makeindex[columns=3,name=lam,title=Lamentations]
\makeindex[columns=3,name=eze,title=Ezekiel]
\makeindex[columns=3,name=dan,title=Daniel]
\makeindex[columns=3,name=hos,title=Hosea]
\makeindex[columns=3,name=joe,title=Joel]
\makeindex[columns=3,name=amo,title=Amos]
\makeindex[columns=3,name=oba,title=Obadiah]
\makeindex[columns=3,name=jon,title=Jonah]
\makeindex[columns=3,name=mic,title=Micah]
\makeindex[columns=3,name=nah,title=Nahum]
\makeindex[columns=3,name=hab,title=Habakkuk]
\makeindex[columns=3,name=zep,title=Zephaniah]
\makeindex[columns=3,name=hag,title=Haggai]
\makeindex[columns=3,name=zec,title=Zechariah]
\makeindex[columns=3,name=mal,title=Malachi]
\makeindex[columns=3,name=mat,title=Matthew]
\makeindex[columns=3,name=mar,title=Mark]
\makeindex[columns=3,name=luk,title=Luke]
\makeindex[columns=3,name=joh,title=John]
\makeindex[columns=3,name=act,title=Acts]
\makeindex[columns=3,name=rom,title=Romans]
\makeindex[columns=3,name=1co,title=1 Corinthians]
\makeindex[columns=3,name=2co,title=2 Corinthians]
\makeindex[columns=3,name=gal,title=Galatians]
\makeindex[columns=3,name=eph,title=Ephesians]
\makeindex[columns=3,name=phi,title=Philippians]
\makeindex[columns=3,name=col,title=Colossians]
\makeindex[columns=3,name=1th,title=1 Thessalonians]
\makeindex[columns=3,name=2th,title=2 Thessalonians]
\makeindex[columns=3,name=1ti,title=1 Timothy]
\makeindex[columns=3,name=2ti,title=2 Timothy]
\makeindex[columns=3,name=tit,title=Titus]
\makeindex[columns=3,name=phm,title=Philemon]
\makeindex[columns=3,name=heb,title=Hebrews]
\makeindex[columns=3,name=jam,title=James]
\makeindex[columns=3,name=1pe,title=1 Peter]
\makeindex[columns=3,name=2pe,title=2 Peter]
\makeindex[columns=3,name=1jo,title=1 John]
\makeindex[columns=3,name=2jo,title=2 John]
\makeindex[columns=3,name=3jo,title=3 John]
\makeindex[columns=3,name=jud,title=Jude]
\makeindex[columns=3,name=rev,title=Revelation]


\tightlists
\begin{document}
%\indexmarkstyle{\normalfont\tiny\ttfamily}
%\showindexmarks

\frontmatter


\begin{titlingpage}
\thispagestyle{empty}

\begin{center}
{\Large Business Perspectives From The Word}
\end{center}

An idea for a cover of the book: what about instruments (slide rule? construction equipment?) and other symbols of your work (mathematical formulae, for instance) in a somewhat transparent style, overlayed with a cross?


Multi-point Moment:


\begin{displaymath}
\mu_n = \int r^n \, \rho(r) \, dr.
\end{displaymath}

Entropy:

\begin{displaymath}
\textrm{d}S = \frac{\delta Q}T.
\end{displaymath}

\cleardoublepage

\newlength{\drop}
\newcommand*{\titleDB}{\begingroup%
%\FSfont{5pl} % FontSite URW Palladio (Palatino)
\drop = 0.14\textheight
\centering
\vspace*{\drop}
%{\Large }\\[\baselineskip]
{\Huge\scshape Business Perspectives}\\[\baselineskip]
{\Huge\scshape From}\\[\baselineskip]
{\Huge\scshape The Word}\\[1.5\baselineskip]
{\large By}\\[\baselineskip]
{\LARGE Thomas Allen Bathgate}\par
\vfill
%{FOREWORD BY AN OTHER}\\[8\baselineskip]
\vfill
{\small\sffamily Reindeer Games Publishing House}\\\plogo\par
\endgroup}

\thispagestyle{empty}
\titleDB
\clearpage
%\end{titlingpage}

%\vfil 
%
%\hbox{ % Horizontal box
% \hspace*{0.1in} % Whitespace to the left of the title page
% \rule{1pt}{1\textheight} % Vertical line
% \hspace*{0.05in} % Whitespace between the vertical line and title page text
% %
%\begin{minipage}[b]{3.5in}%
%{\HUGE Business Perspectives \\[1ex] From The Word}
%
%\vspace{0.5\textheight}
%
%
%\textsc{\Large{}{}Thomas Allen Bathgate}{\Large\par}
%
%\vspace{0.3\textheight}
%
%
%{Reindeer Games Publishing House}\\
% \plogo %
%\end{minipage}}


\thispagestyle{empty}
\begin{center}
\textcopyright 2017 by Thomas Allen Bathgate. All rights reserved, including those pertaining to the contents and design of this book. 
\par\end{center}

\begin{center}
\vspace{0.25in}
 \end{center}

\begin{center}
No copies may be made in any form without the express written permission of the author. 
\end{center}

\begin{center}
\vspace{0.25in}
 \end{center}

\begin{center}
Failure to comply with both the letter and the spirit of the law, together with the will, wishes, and desires of the author will result in your immediate and total annihilation. 
\par\end{center}

\begin{center}
\vspace{0.25in}
 \end{center}

\begin{center}
You have been warned. 
\par\end{center}

\begin{center}
\vfill{}
 
\par\end{center}

\begin{center}
\textsc{Just see if we're kidding!} 
\end{center}
\end{titlingpage}



%\cleardoublepage{}

\renewcommand*{\contentsname}{Short contents}
\setcounter{tocdepth}{0}%  chapters and above
\settocdepth{0}
\tableofcontents

\begin{comment}
\clearpage

\renewcommand*{\contentsname}{Contents}
\setcounter{tocdepth}{2}%  subsections and above
%\settocdepth{2}
\tableofcontents


%\tableofcontents*
\end{comment}


\index{business!ethics|see {ethics}}
\index{employee supervision|see {supervision}}

\index{motivation of employees|see {motivation}}
\index{personal conduct|see {conduct}}
\index{simple ethics|see {ethics}}
\index{work!ethic|see {ethics}}


\cleardoublepage

\chapter{About the Author}

Thomas A. Bathgate---Tom---resides in Abington, PA where he and his wife Natalie have raised their two children, Meredith and Bradley. Tom is an architectural engineer and CEO of PWI Engineering, a company focusing on the design of energy systems for buildings and campuses, with work in 29 countries around the world.

Tom grew up in the mountains of Appalachia and was educated at Penn State University. Tom married his childhood sweetheart, Natalie, and moved to the big city after serving in the U.S. Army to build a career. On this path, he became ``born again'' with the prodding of his children and wife, and he joined them in their church: St. John's Anglican Church in Southampton, PA.

Tom has been blessed. God has provided him with professional success, family health and love, and a wonderfully broad experience-based life. God has consistently led brothers and sisters in Christ across Tom's path to assist and guide him in his spiritual growth.

Being ``born again'' at the age of 38, Tom felt that he needed to catch up on his understanding of the Bible, Christ, and God. After reading and re-reading the Bible (5--10 times, with many footnotes written), Tom started to create a database of verses that seemed most helpful to him.

One of Tom's Christian brothers mentioned that he thought that the Bible was God's version of an owner's manual for us humans. Taking that concept to heart, Tom started recording in his database his perspectives on business, parenthood, legal, and other topics from God's Word.

This business focused Bible study is also the product of joint studies and discussions amongst Christians involved in business. This Bible study is not a personal development of Tom's, but it can be considered as an assembly of the work of many---Frank Allen, Steve Kennerly, Duane Miller, Phil Lyman, Ray Friesheim, Bob Voth, Dave DeFlavis, John Pacana, and many, many others.  

Being an engineer at heart, Tom has had a difficult time committing his thoughts into proper English and he's very grateful to Susan Sayer for editing his work of love.

While Tom is still in the twilight of his engineering career, he remains an active hunter, fisherman, and grandfather in his beloved mountains and is quick to share the Gospel with anyone crossing his path. He continues to re-read the Bible and he continues to find more meaning and wisdom as his spiritual life matures. Tom's pulpit is his life: businessman, father\slash{}grandfather, sportsman, and teacher, and will continue to be that way until he's called home.   


\chapter{Introduction}

\begin{quote}
All Scripture is God-breathed and is useful for teaching, rebuking, correcting and training in righteousness, so that the man of God may be thoroughly equipped for every good work.''\footnote{2 Timothy 3:16, NIV.}
\end{quote}

\emph{Business Perspectives from The Word} is an interactive
guide for retrieving business perspectives from the Bible. By topic,
by business category, or simply by the sequence of chapters of business
categories, the readers of this guide will find rich references to
business perspectives. The reader \emph{must} combine this guide with a Bible: a version of their choice will suffice.

Christian business persons are not only Christians on their day of
worship, but every day and hour of the week and year.

Christians are often maligned as being too soft and easy, as those who cannot succeed in business---certainly not in a competitive environment. However, my experiences are quite to the contrary. Christian business
persons appear to reap more long term success, can handle stress and
trials with greater ease, and seem to understand the values of principled
leadership, fair play, and the marketing value of trust. Christian
business persons appear to stand out as admirable examples of honor
and (self)-control. They have purpose and a plan. They promote the
synergy of teamwork. They seem to know where they are going.

You may have heard the saying, that ``the Bible is God's version
of an Owner's manual for life'' and it certainly is. From the perspective
of the world of business, of operating and maintaining a business
in the United States, I find it invaluable.

The Bible has always fascinated me. It is from the depth of God's
wisdom. I can read the same verse several different times under different
life conditions and it will provide me with a different perspective
of wisdom and guidance each time. The perspectives continue to evolve
with my personal maturity and aging in the passage of life.

Having spent nearly all my career as a leader and executive in the
design engineering business---a very competitive environment---I
was blessed in meeting several strong and successful Christian business
persons. Their advice, fellowship, and courage have had a profound
effect on my concentration to promote the Christian business perspective.

What is the Christian business perspective? With the help of many
Christian brothers and sisters, I have been collecting Bible references
as guidance for me and others in the business world. I utilized the
references as a structure for study of the Bible and as support and
guidance in my business. While I don't consider these Business Perspectives
to be an exhaustive study of the Word, I share them simply with the
hope that they may help other brothers and sisters in their missions
within their business fields. I will vouch for the truth and value
of the references and I can proclaim that adopting these guides of
business wisdom will promote success and \index{fulfillment}fulfillment.

For years, I have been carrying this collection of verses with me,
around the world. I have used it to start my day and to express my
gratitude before my sleep on the business trail. As a top executive
in my companies, I have found that there are others who enjoy a more
specialized approach to the references---business or law or marketing.
Therefore, I decided to categorize the references into the areas of
business focus, which are presented in these chapters. Please recognize
that while you may be most interested in one or more of the categories,
I suggest that you expand your horizons into all of the categories.
The values are adaptable to all callings in business, and the principles
and virtues are universal---not solely germane to any one specialty.

\index{Bible!reading!importance of}
\index{prayer!importance of}
I encourage you to further study the Bible beyond the scope of these
pages. I am sure that there are many other verses that guide business
folks and I would appreciate your sharing them with me and with your
friends. The business-mission field is rich with work. It needs many
more skilled workers. Our skills must be constantly improved by study
and prayer, so that we are prepared.

With the following citation from 1 Peter, allow me to lead you into
your further study and spiritual growth. 
\begin{quote}
Who is going to harm you if you are eager to do good? But even if
you should suffer for what is right, you are blessed. `Do not fear
what they fear; do not be frightened.' But in your hearts set apart
Christ as Lord. Always be prepared to give an answer to everyone who
asks you to give the reason for the hope that you have. But do this
with \index{gentleness}gentleness and \index{respect}respect, keeping a clear conscience, so that those
who speak maliciously against your good behavior in Christ may be
ashamed of their slander.\footnote{1 Peter 3:13--16, NIV.} 
\end{quote}

\mainmatter

%\part{Business Perspectives\protect \\ from ``The Word''}

\thispagestyle{empty}

\pagestyle{myPageStyle} \pagenumbering{arabic}

\chapter{Organizational Activities and Teams}

Teams and team building are areas of focus not only for business leaders; they are also important for \emph{all} business participants. Whether you are leading a work group,
or simply participating in a work group, it is important to know group
dynamics, to know how organizations function, and to be an informed
participant, if not a multi-discipline player.

Group dynamics and the synergy of teams performing work or solving
problems is the lifeblood of business success worldwide. The Bible
offers much advice and guidance in this area of business.

\section[Attitude, Positive, Team Building]{Attitude, Positive, Team Building, 2 Timothy 1:6--7}
\index{attitude!positive}
\index{team!building!positive attitude}
\index[2ti]{1:06--07@1:6--7}

Take a lesson from the disciples and leaders of the early church of
Jesus Christ! With no timidity, fill yourself with confidence and
the Spirit while organizing a functional team for business. With God's
blessing, your business ventures, efforts, and endeavors will be filled
with a ``spirit of power, of love, and of self-discipline''. Like
most things in life, if you do not believe that you have a chance
for success, you are doomed to failure. With God's guidance and Spirit,
fear not.

I can remember taking a golf lesson on putting some years back. The
teacher advised that if I did not believe that the putt would go in
the hole, he assured me that it would not. For those of you who play
golf, you will understand the need for confidence and concentration
in sinking putts. Similarly, success in the team environment requires
belief in the team's success. We might call it positive thinking.
Others would call it believing in God's success and God's answering
of our prayers. If we are sure of why we intend to organize and proceed,
and the reason is God-based, then surely we and you will succeed.

\section[Behavior, Goals]{Behavior, Goals, Proverbs 16:7}
\index{behavior}
\index{goals}
\index[pro]{16:07@16:7}

Often the fear of Christians in business is their vulnerability to
their adversaries and enemies. Christians are expected to turn the
other cheek, to forgive, and to accept life as God's doing. We and
others interpret these directives as being weak and easy prey for
the aggressive and evil.

Overcoming this fear requires confidence building and faith in our
God. The Word advises that if we concentrate on pleasing our God,
that even our enemies will be peaceful to us. I can vouch for
this from my personal business experiences. My personal recommendations
include not only prayer, but study and focus on what God wants from
us during times of crisis. Effectively, as we focus on serving our
God, we find ourselves performing at a much higher plane of graciousness
and goodness, almost godliness, that is observed by our adversaries
and teammates.

The pettiness of human weakness and sin are often overlooked, almost
ignored, to bring the competition up to your level. At this level,
perspectives differ. The mundane issues of the world are properly
weighed and prioritized from a universal perspective. It is very difficult
for your enemies to not admire and respect\index{respect} you, particularly when
it is obvious that they are competing against ``more than you''.
In all of your endeavors, please recognize the source of your strength---God.

\section[Business Philosophy, Ben \& Jerry Perspective]{Business Philosophy, Ben \& Jerry Perspective, Philippians 2:1--4}
\index{business!philosophy}
\index{ben \& jerry perspective}
\index[phi]{2:01--04@2:1--4}

Ben and Jerry's Ice Cream Company and their employee programs and
philanthropy are well documented and published. It is for this reason
that I subtitled these verses as such. These verses from Philippians
explode with team-oriented business principles of alignment, \index{respect!mutual}mutual respect, team camaraderie, and \emph{all for one and one for all}.
In business, sports, and combat, there must be a single purpose of
existence that is simply the goal of all of the team participants.

This single-mindedness of existence is critical to performance when
the chips are down. Being one in spirit will promote individual
performance beyond the average expected. And in addition being
one in spirit and purpose with Christ will provide you with a model
of easy following, and learning.

\section[Commonwealth of Israel, The Beginning]{Commonwealth of Israel, The Beginning, 1 Samuel 30:21--25}
\index{commonwealth of Israel, beginning}
\index[1sa]{30:21--25}

\index{Bible!characters!David}
This is the wisdom of God through David in building the team attitude for Israel, its army, and its people. Time and again in business there is prejudice against some contributors to the cause, production,
or profit making. In the engineering business, non-technical employees
such as accountants, legal staff members with a law education, and
marketing and sales people are openly looked upon as second class
contributors to the business effort.

This prejudice is built upon ignorance, of course, and the narrow,
self-centered human mind which focuses just beyond one's nose. While
the labors of our time and effort in life are the center of our attention,
we are commissioned to look beyond ourselves into the larger world
around us. In most instances, we will see the importance of all of
the contributors to the cause and business, promoting the attitude
and sense of commonwealth, which is critical for complete team success---in
the long term.

\section[Communicating and Team Building]{Communicating and Team Building, Acts 4:23--33}
\index{communication}
\index{team!building!communicating}
\index[act]{04:23--33@4:23--33}

The success of the early disciples and their church are examples of
team building that create a model for us. Recognize that in worldly
terms, the chances for success of the ``movement'' called Christianity---the
church---were not good. By the world's measure, the outlook was bleak
and a cause for pessimism.

However, as a team and with all of the synergy of a team, the early
disciples and servants of Christ moved forward and their success is
history, if not our legacy. Please note their process: they 
\begin{enumerate}
\item prayed to God for his help and guidance; 
\item became filled with the Holy Spirit, by God's grace; 
\item became a team, of one mind and spirit; and, 
\item shared with one another. 
\end{enumerate}
How many times have we heard our business leaders or prophets talk
of team building in our time using terminology like ``pockets of
power'', ``bundles of capabilities'', ``alignment'', and on and
on. Are these human explorations the efforts of folks in search of
something, or are they simply offering identity to God's provisions?

Don't overlook the simplicity and pureness of the early Christians
and their ability to form and mobilize a successful team. The ability
to communicate to and amongst the teammates, develops through the
process of team building itself, creating a unity of purpose, mind
and spirit. Effectively, the process of team building is one of communication
in itself.

\section[Courage, Resistance to Evil]{Courage, Resistance to Evil, Ephesians 6:10--20}
\index{courage}
\index{resistance to evil|see {evil}}
\index{evil!resistance to}
\index[eph]{06:10--20@6:10--20}

Having spent time as an athlete and a soldier, these verses are a
call to arms as a Christian soldier. They also have a similar effect
on my business sense. However, the key to my excitement lies in verse 12\index[eph]{06:12@6:12} and starts with a re-evaluation of the basis for my business. What
is the purpose of my business and my business activity? Are my efforts
and my intentions righteous; do they glorify God? With the confidence
and assurance spiritually that I am doing God's work and fulfilling
his intentions for me, I can then begin to learn, to become inspired
by, and follow God's Word in these verses. ``For our struggle is
not against flesh and blood, but \dots against the spiritual forces
of evil in the heavenly realms.''

Yes, I do believe that we can and should treat our business endeavors
with the same passion and commitment as that of the early Christians,
missionaries, and clergy. If our business is God-intended,\footnote{`God-oriented', perhaps?} God-led,
and righteous in His eyes, then we should treat our endeavors as part
of God's efforts. Our business days should be lived with God as our
partner, with His defenses, His scouting, His intelligence, and His
courage and strength.


The business world, at least that to which I've been exposed, has
few easy paths. Success involves struggles. We take on the struggles,
with God's help, with strength, courage, and His wisdom. From that
perspective, we are teammates, soldiers, and sheep of God's flock.
We are expected to give our all, work our tails off (so to speak),
and with God's help, we will fulfill our mission to serve him. We
will meet his goals, for which he has prepared us. We must always
remember to be prepared to give him the credit in our endeavors.
\index{Bible!characters!Paul}God-focused
endeavors and consistent recognition of God's help will prepare us to be his worthy servants, such that we might fulfill Paul's request\slash{}prayer in verses 19 and 20.\index[eph]{06:19--20@6:19--20} We must all fearlessly declare the Gospel.

\section[Discrimination, Basis of]{Discrimination, Basis of, Romans 12:3--8}
\index{discrimination, basis of}
\index[rom]{12:03--08@12:3--8}

Our attitude toward our role in our business, on a team, or simply
in a society, is prepared in these verses, for team building. Our
discrimination of others is difficult, if not impossible, if we take
to heart, in faith, the guidance of these words of love and cooperation.
Just as we humble ourselves before God, we must humble ourselves with
sober judgment as we perceive our individual roles on teams, in society,
and in business.

As individuals, we are simply contributors with unique abilities,
each of us. Our unique abilities are not stand-alone, but complementary,
if not gears in the overall, big machine. As our abilities are to
be respected, so are others. Each of us needs the others, or we are
simply alone, certainly not a team, or a business, or a society. In
fact, we all do belong to one another.

As we know, the team composed of specialists applying their skills
as an orchestra might produce well beyond the abilities of the sum
of the individuals. Without the appreciation for the differences of
the individuals, as they compose the team, we are discriminating individuals\footnote{Is this what you mean to say?}.

\index{respect}With the respect and appreciation for teammates, with individual humbleness,
and with our emphasis on the performance and the success of the team
before our individual gain and recognition, the team will reach a
level of success that can be sensed almost spiritually, just as Christ's
church.

\section{Discipline}
\index{discipline}
\subsection[Following Orders]{Following Orders, Proverbs 13:13}
\index{discipline!following orders}
\index{following orders}
\index[pro]{13:13}

While the words of this proverb are reasonably simple to understand,
there is a deeper appreciation for them in the complex workings of
organizations and teams. Discipline in following orders is key to
the success of a team or organization, not just an army. In order
to be able to accept orders, individuals must be prepared to respect
the singular importance of the need to follow instructions and orders.

This is not to propose that things might not be accomplished without
orders, as many of the defeated armies of the world have found. But
when orders are given, they must be followed; likewise instructions
must be followed.

Often the instructions or orders are given in a very narrow perspective.
This approach leads to a lack of understanding of the logic for the
action. There may be good reason why the logic and reasoning for the
instruction is withheld. It might be security of information. It might
require too much time to explain. It might be beyond the comprehension
of the team (often true of God's instructions). Sometimes it is impossible
to take the time to explain the entire workings of the whole.

With all team members following their instructions, or their orders,
certainly, there is a better chance for success. The alternative is
chaos. Orders must be followed.

\subsection[Soldierly]{Soldierly, 2 Timothy 2:3--7}
\index{discipline!soldierly}
\index[2ti]{2:03--07@2:3--7}

At one point of my business career, I had visions of expanding operations
into a new business that was not even related to the engineering design
business, in which we had been blessed with success. Prior to investing
in this new business, I consulted with some older, smarter advisors
whom I respected. One of my wise friends sat back and reminded me
of my formal education in engineering, my experienced development
in engineering design, and wondered if I wasn't drifting from my area
of focus. His final words, looking out over his pipe were, ``Mind your knitting!''

\index{Bible!characters!Paul}\index{Bible!characters!Timothy}
I can't read these verses from 2 Timothy without thinking of my friend
and his wise advice. Is Paul encouraging Timothy to keep his concentration
on the mission, task, and goal, as a soldier must do when he's
disciplined? In addition, those performing within the boundaries of
the rules to the best level will win.

We celebrate the performance of our best athletes on four year cycles
in the world's Olympics. Our games have rigid rules, timing devices,
and drug testing to assure that the athletes are competing within
the rules. They must be courageously disciplined, to properly train,
compete, and win.

When they do follow the rules, and are disciplined to properly train,
they are rewarded. So is the case in business as we mind our knitting
with discipline. As we work hard, like a farmer would and must, we
will reap our rewards. If our business is to grow food, as the farmer
does, we will reap food. What is the purpose of our business?

\section[Employee Supervision]{Employee Supervision, Deuteronomy 25:4}

\index{supervision}
\index[deu]{25:04@25:4}

Good teams are led by good captains, coaches, supervisors, and bosses.
Good bosses are sensitive to what makes their employees or team members perform.

During the training and development stages of individuals, the application of restrictive discipline is important.
It is instruction in the limits of what is good and what is acceptable. 
Once trained, if the employee or the team member performs, let him run like a good thoroughbred race horse.
Continued discipline at this stage will simply be restrictive and inhibit performance.
With the proper discipline instilled, the drive to perform will come from within the team and its members.

\section[Equality of Man]{Equality of Man, Galatians 3:26--29}
\index{equality of man}
\index[gal]{3:26--29}

In order to build a team with an attitude in which the individuals
realize that the team is more important than their personal success,
some coaches of sports teams forbid the use of names, the use of individualized
uniform garnishing, and special publicity to individuals. All are
equal, and the team is above the individuals.

In his published writings, Theodore Roosevelt\index{people!Roosevelt, Theodore} campaigned that folks in the US should forget their immigration heritage and declare themselves
Americans, not German Americans, or Irish Americans, etc. Our melting pot, which President Roosevelt loved with a passion, was certainly more important than the roots from whence we came, and in America, people are Americans which makes them equal under the law of the land,
and under God, as published, guaranteed, and died for.
This attitude promotes spiritual and personal commitment from all, toward the whole.
It is an ingredient for success.
America belongs to its people.

A business, with a staff of dedicated employees, with a sense of ownership,
will outperform that of a team of ``disinterested folks''. Equality
requires a sense of membership, dignity, and mutual respect.\index{respect} It is
the founding basis for the greatest church\footnote{How about ``organization''?} in the world---Christ's church.

\section[Ethics, Business]{Ethics, Business, Colossians 3:12--15}
\index{ethics}
\index[col]{3:12--15}

While my notes simply say, ``Ethics---for a Christian business person'',
these verses offer guidance in many kinds of situations. When dealing
with the evil of the world, they are more than ethics, especially
when our instincts call for us to lash out, attack, and get even.
They remind us of our obligations to our Savior and God and the kind of person that we are expected to be.

When following these directions in business with other Christians,
they give us a glimpse of what our eternity, after this life, might
be like. This level of comfort, trust, and genuine, loving respect
is exceptional in our world today. Why must it be exceptional behavior?
Why can it not be the standard, if not common behavior in the business
world?

Change and revolution start with individuals. Let us each attempt
to clothe ourselves ``with compassion, kindness, humility,
gentleness and patience'' in our business lives.

Our individual lives should become models for others. Our contribution
to teamwork will be valued even more so, as the team fabric is strengthened.
The values of ``binding'', ``perfect unity'' and being at one
with Christ, offer not only personal peace, strength, and comfort,
but also reinforcement to the structure of a team.

\section[Expanding through Righteousness]{Expanding through Righteousness, Romans 5:5--17}
\index{expanding through righteousness}
\index[rom]{05:05--17@5:5--17}

The power of one man starts many major things. Raging infernos begin
with small sparks. We can also start major movements, if not revolutions
to initiate change with our own behavior. Our own righteousness naturally
spreads the Word, while at the same time promulgating righteous
behavior. Expansion and growth through righteousness will succeed.

From the business scale, expanding one's business to other geographical
locations, or even expanding ownership into other business paths,
is often seen as risky. If righteousness pervades your business, its
purpose, and its organization, your business should be blessed in God's
eyes. With God's blessings, your business will have a contributor
better than any mankind can offer.

Business philosophers for years have warned business folks of drifting
too far from one's expertise, or getting into new businesses, or wandering
from the trade that made us whole. ``Stick to your knitting!'' is
one of those warnings.

Let me suggest changing that statement to ``First, know your knitting, and why it is that you're knitting.'' Successful business persons, whether in a management or supporting role, envision their business as more of a campaign or mission.
Christian business persons usually view their campaign or mission in business as having a righteous cause, if not the blessing and grace of Jesus Christ.

When expanding the mission of the business, whether changing the structure,
expanding the market of products or services, opening new ventures
in a new geography, or investing in new technology, start with a righteous
cause for the expansion. Be sure of what you hope for, and that hope
should be righteous.

Just as our salvation was initiated by one man and one God, steeped in righteousness, so can our activities and missions be steeped in righteousness by God's grace and love.


\section[Goal Setting]{Goal Setting, Proverbs 16:3}
\index{goals!setting}
\index[pro]{16:03@16:3}

My engineering background forces me to consistently break all issues, questions, and problems into basic principles.
Problems can be solved, when the problems are defined.
Successful business ventures do not just succeed, they are well conceived and well planned.

Military and business leaders have been warned for millennia to not proceed without good planning, good intelligence, and good training.
Based upon Proverbs 16:3\index[pro]{16:03@16:3}, let me suggest that a critical, initial ingredient might be commitment.

\index{commitment}What is commitment? Commitment can be defined as ``dedication or direction of ownership''.
In the case of the activities in our life, commitment is recognizing
that our activities, work, and rewards are ``to be dedicated to God''.
Hopefully, if our intentions or ``goals'' are dedicated to our God,
our actions will glorify God. Let us not fool ourselves. Let us be
honest with ourselves that we are setting our goals and plans for
the glory and work of God---not ourselves. If properly targeted,
Proverbs 16:3 assures that our plans will succeed.

\section{Helping}
\index{helping}
\subsection[Justification For]{Justification For, 2 Corinthians 1:3--7}
\index{helping!justification for}
\index[2co]{01:03--07@1:3--7}

Throughout much of scripture and even the philosophies of man, blessing
others as we have been blessed, is promoted. When it is applied throughout
our society, and especially in business, ``the helping of others''
is often viewed as the exception, or unusual. Often such an act is
accompanied by a derogatory comment that ``we have another good Samaritan''.
That in itself is a sad testimony to our society and our business
world.

Recognizing the unique value of helping, marketing people influencers might promote helping others as a good and positive procedure in building business relationships, and promoting sales.
First build the relationship, then promote the product.

\index{Bible!characters!Paul}Paul's word to the Corinthians would have Christians, simply adopt
helping of others as a command, because Jesus Christ helped
us. Christians do not need a business excuse to help, nor do they
need any justification to help, other than we owe it to Christ as
His servants. Yes, Christians do stand out as good people. They do
help even though they expect nothing in return. Yes, Christians do
make wonderful marketing and sales people, not because they want something
in return, not because they are setting up a potential sale, but because
they are inspired by their Savior.

\subsection[Overcoming Evil]{Overcoming Evil, Romans 12:9--21}
\index{helping!overcoming evil}
\index{overcoming evil|see {evil}}
\index{evil!overcoming}
\index[rom]{12:09--21@12:9--21}

These words have become one of my favorite reminders, if not my personal battle cry, in business and the rearing of my children. I have difficulty
keeping a dry eye when I read them, because they explode
in meaning and virtue. I have always understood them as being directed
in the imperative---as commands from God, and I do wish that someday
I can live by them completely.

Verses 9--16\index[rom]{12:09--16@12:9--16} are guides to individuals which are invaluable in team-building,
in personal character, and spiritual development. Each verse is
powerful in meaningful criteria, for what actions, and feelings
we should target to be considered good servants of our Lord.
As a business person, athlete, or any other competitor, these words
are performance goals to establish personal righteousness, so that
you might further enhance a team or society. They are critical.

While competing in our personal worlds, as individuals and teams,
verses 17-21\index[rom]{12:17--21} provide us with an attitude, if not a reason for confidence.
My high school football coach promoted hard hitting, which he used
to call ``crunching the opponent''. He also insisted that we then
offer a helping hand to bring our opponents to their feet. After a
successful crunch, as I extended my hand, some opponents would
accept it and give me a pat on the backside, with a compliment to
further heighten the fun of the competition, while other competitors
would despise the offer, anguishing in their grief, often to a level
of inhibiting their competitive ability.

Christians are competitors, and they compete at a level of performance that at times appears unfair---they're not alone. However, Christians, as competitors, compete fairly, with righteousness, and for God's glory. As we are promised that ultimately goodness will overcome evil, we are committed as Christians to work at the same goal individually.

\section[Motivation Of Employees]{Motivation Of Employees, Nehemiah 2:17--18}
\index{motivation}
\index[neh]{02:17--18@2:17--18}


\index{Bible!characters!Nehemiah}Nehemiah is one of my management heroes. He didn't simply order
his workers to re-build Jerusalem's walls and gates, but he employed
the following techniques of employee (or team) motivation: 
\begin{enumerate}
\item He shared the situation, and big picture importance with the team; 
\item He promoted team work with his selection of the words ``we'' and ``Come, let \emph{us}''; 
\item He stated his authority, establishing the righteousness of the matter; 
\item He personalized the situation, by indirectly outlining that they would be disgraced if they didn't participate. 
\end{enumerate}
In business, as in many life endeavors, do not overlook the power
of the participants, especially if they are passionate in their endeavors.
Most people do not simply work for the sake of work. They have a reason, and the reason is of importance. An encouraged person will always outperform a whipped, submissive person. Give folks the
reasons, personalize the mission, share the authority, outline the perils, and remind them of the universal importance of their actions.

Nehemiah's approach has worked for me repeatedly over the years. Try it!

\section{Organization}
\index{organization}

\subsection[Chemistry]{Chemistry, Romans 12:3--8}
\index{organization!chemistry}
\index[rom]{12:03--08@12:3--8}

Again, these verses provide a different perspective, when the reader's focus
changes. ``Organizational chemistry'' or ``people chemistry''
or ``team-chemistry'' are terms used to define the relationship
and characterization of a group of people. Good chemistry defines
a complementary group, who can perform admirably together as a team and have mutual respect for one another. Bad chemistry usually implies
that the reason the team is not performing well is because the relationships
among the people are an impediment to the performance of the team, often
called ``bad blood''. Funny terms we humans use.

\index{Bible!characters!Paul}
These verses clearly define guidelines for our relationships as team
members, and as human beings in our society (while originally intended
by Paul for the early church members). Can you imagine, when putting
these guidelines into place that they would create bad chemistry?
Of course not. They are not only good guidelines for relationships,
but also words of encouragement for us to mobilize our abilities,
whatever they might be, to serve Christ's goals and mission. Please
do not ignore them.

\subsection[Pyramid Structure]{Pyramid Structure, Deuteronomy 1:9--18}
\index{organization!pyramid structure}
\index{pyramid structure}
\index[deu]{01:09--18@1:9--18}
\index{management}\index{leadership}

Organization is control and order, whether at an individual level
or at a group level. In the case of Moses\index{Bible!characters!Moses}, the Israelites were growing
to the point where control, fairness, and judgment concerned him. It
was becoming difficult for Moses to handle the requirements of leadership
for all of them, without some method of organization, in which his
responsibilities could be delegated and shared. For those of you in
``burnt out'' leadership roles, does this sound familiar?

There are several points of interest jump from these verses that are noteworthy: 
\begin{enumerate} 
\item Note that Moses asked each of the tribes to select their own leaders,
representative of each of the tribes. It would appear that Moses'
understanding of the principles of democracy may have pre-dated the
Greeks. Regardless, it makes good management sense and it promotes
peace and team chemistry.

\item Note that Moses asked the tribes to identify their representatives,
and that they should be ``wise, understanding, and respected men''.
I would like to propose these same criteria for our representatives
in our democracies of today. 

\item In \index[deu]{01:14@1:14} verse 14,  reminds the tribes that they had accepted his proposal,
and he reminds them that they agreed with this approach to organization.
This management tactic to revise organization employing the review
and input of the working staff, expands the ``ownership'' of the
idea to those that it affects. In other words, it never hurts to take
the time to review proposals, take comments, and give the ``team''
a chance to agree with it, if not modify it, so that ``they'' make
it work. 

\item Recognize that Moses created a pyramid of leadership and communication.

\index{management!rule of seven}Today we talk about the ``rule of seven'' (i.e., no one should manage more than seven people at one time). While Moses didn't work at our present ratios, the principle appears to be similar. Effectively he was establishing a chain of authority and communication.
The communication
was a two way operation---both up and down. Such an organization,
offers hope to all, which is critical to peace and order. Never destroy human hope. 

\item In a democratic and peaceful society, judges and their judgment can
show no partiality. I believe that the same holds true in business,
particularly with regards to promotions, layoffs, hiring, and personnel
practices in general. There must be a higher purpose that we all serve,
and that higher purpose is the basis for all judgment, and it cannot
be partial. This is not to exclude judgment being compassionate. 

\item Notice that when cases were too hard, the lower judges could request a higher counsel---Moses.
This sounds very similar to today's court systems.
Similarly, most successful organizations have an outlet and method for appeals of employees and supervisors, for counsel and judgment on difficult issues. 

\item Recognize that Moses was preparing the Israelites to relocate them
to their promised land, and to also mobilize their society toward
a new way of living---to develop a society. Organizational planning
is the theme. Effectively, Moses, with God's direction, was envisioning
the possibilities and probabilities of the future, and preparing the
Israelites' organization for that situation. As a manager, if not
creator of organizations and businesses, I can feel the workings
of a flexible organization to handle change and evolution. 
\end{enumerate}

\subsection[Staff and Line Responsibilities]{Staff and Line Responsibilities, Nehemiah 4:12--23}
\index{responsibility!personnel, staff and line}
\index[neh]{04:12--23@4:12--23}
\index{organization!staff and line responsibilities}

\begin{quote}
``Specialization, while creating expertise, often leads to boredom.''

``Variety is the spice of life.'' 
\end{quote}

Human beings can identify with these truths. Good teams and effective business organizations often have dual responsibilities for
their team members, just as \index{Bible!characters!Nehemiah}Nehemiah. Nehemiah, who is one of my heroes
in the field of successful management, outlined a staff responsibility (which I often refer to as work to support the existence and potential to exist) and a line responsibility (which is often the first line priority
of the individual, or the dynamic function of the group). In the case of Nehemiah's worker\slash{}warriors, construction was their line responsibility
or first line mission. Defense was their staff responsibility (the second line). Defense permitted their primary mission to succeed.

In today's military, the roles are reversed. An airplane pilot would consider his flying to be his line responsibility, while that pilot might also serve as the fire safety officer---their staff responsibility.

In successful organizations, dual roles are often created to meet the mission role, when resources are limited, or personnel are multi-discipline. Often they are simply created to minimize
boredom.

In Japanese industry, the secondary roles have been classically directed toward quality control circles, where the team's perspective plays a role in the direction or re-direction of the functions and organization. It builds a sense of ownership and involvement. It promotes care and hope.

You may want to re-read these verses; as you do so please note the following:

\begin{enumerate}
\item Nehemiah issued his orders for defense and construction by families, which would promote camaraderie, mutual respect, and a natural, fierce sense of defense that only family members might be willing to exert. ``Never walk between a mother bear and her cubs!''

\item The trumpeter, the center of communications, remained with the leader for the purpose of command and control. Leadership demands communications, quickly and effectively. False communications can do more harm at times than none.

\item The workers put their faith in God, for their defense and success, and then worked at it.

\item The level of \index{commitment}commitment was maintained at a high pitch. The workers never
left their weapons. They were always armed. There was a sense of commitment, mission, and urgency. People perform best under these conditions. 
\end{enumerate}

This last point is interesting. One of the ingredients to successful project management in the design and construction industry is to create timely crises to develop environments of urgency, commitment, and mission, even if they are not natural. Research on success has
identified this as a key ingredient.
Conditions and environments characterized by exclamations such as ``we had no choice but to perform'', ``we had such little time, we had to bypass the typical rules of procedure'', ``we were under the gun from the beginning to the end'' often accompany successful efforts in business.

Finally, please re-read verses \index[neh]{04:19--20@4:19--20} 19 and 20. There are two critical principles that come to mind: 

\begin{enumerate}
\item Every organization must have a method or procedure by which all the resources and team members of the organization can be brought to bear on a problem, a development, or a defense.
Without this focusing method, the organization is simply a group of scattered individuals. 

\item When team members have a goal that is even higher than simply the
completion of a project or effort, or in the case of the wall, their
performance will be above average. If they are performing with the
understanding of a universal truth such as ``we are righteous in
so doing'', or ``we are fighting for justice'', or even because ``our God will fight for us!'', their confidence and resolve will make the effort put forward a stronger effort.
\end{enumerate}

\subsection[Structure]{Structure, Exodus 18:17--23}
\index{organization!structure}
\index[exo]{18:17--23}

Moses\index{Bible!characters!Moses}\index{Bible!characters!Jethro}
Businesses, teams, societies, institutions, and other organizations
must be structured not only to optimize the amount of work produced
by each person, but must be structured for overall performance. Each
person should have the capability of contributing long term, with
a reasonable level of life quality. Jethro, Moses father in law, was
quick to notice the strain continuously on Moses, and to bring it
to his attention.

There are many tombstones erected for those who didn't have the benefit
of Jethro, or any outside source, to note their predicament.

In business, most organizations are structured to promote efficiency
of production. The late 1980s and early 1990s brought us revolutionary
concepts in organization for businesses, concepts that respected the quality of peoples' lives, the empowerment of people. In some instances, these modern programs are taking a note from Jethro's advice to Moses.

The pyramidal structure discussed in these verses, might appear to
be common sense in today's thinking, if not trite. However, please
pay special note to the following:

\begin{enumerate}
\item Jethro's advice was offered ``in the presence of God''. He not only observed the overworked condition of Moses, but he also suggested that Moses not handle the judgments by himself, but be the peoples' representative and present them to God for resolution---something we all need to keep in mind.

\item Jethro promoted education of all people. Let the people know and understand government and the laws---the basis for a successful \index{democracy}democracy and
the basis for a successful people empowered business.

\item Verse \index[exo]{18:21} 21 is a wonderful statement of criteria for government officials, elected leaders, appointed business leaders, and leading individuals on teams. They must be capable ``men who fear God, trustworthy men who hate dishonest gain''. 
\end{enumerate}

\section{Peace}
\index{peace}

\subsection[A Blessing]{Blessing, Numbers 6:22--26}
\index{peace!blessing}
\index[num]{06:22--26@6:22--26}

May we all be blessed by God and receive his peace.

\subsection[Prayer For]{Prayer For, 1 Timothy 2:1--4}
\index{prayer!for peace}
\index[1ti]{2:01--04@2:1--4}

\begin{quote}
``That we may live peaceful and quiet lives in all godliness and holiness''! 
\end{quote}

Too often we take for granted this state of blessing that God has
given us.

Several years ago, as my engineering business was booming, a friend
asked me what markets I had targeted to dominate. I really thought
that he was joking as lawyers often do, but then I realized that he
was serious. Isn't it funny how a simple statement like that sticks
with you for years. At the time, I was happy to be financially alive,
to be employed, have employees, and to have a business future of
which I could see almost 40 days ahead---and my friend is talking
about domination. Frankly, I was happy to be at relative peace
and have a quiet life.

We humans have a problem with our goals sometimes. Can you imagine
how the twentieth century, with its major wars, might have been different,
had the people of the world not thought of domination or revenge,
but instead had targeted peace and quiet in their lives?

Last year, I was conducting an engineering analysis to save energy
at a chemical plant in Resende, Brazil. At first, I was having a difficult
time with personal relations with the plant engineering staff---not
with language, not with cultural differences, but with what was to
be perceived as different personal and national goals. After a few
days of work, we had the opportunity to relax together at a Brazilian
picnic, which they call a barbecue. As the conversation\slash{}questions
and answers on the differences in life styles, focused on American
life, I was asked what my goals were, for my family, and my business. I offered the following in my broken Portuguese\slash{}English:

\begin{enumerate}
\item For my business, I hope to keep my staff employed until they retire with work that is challenging to them. 

\item For my country, I hope that we can maintain peace in the world, so that all peoples can raise their families, have their babies, grow old, and die in peace. 

\item For my family, I pray that my children find happy lives, in which they feel they properly serve their God. 
\end{enumerate}

My friends from Brazil were taken aback. Their previous vision of
Americans included images of domination, greed, hypocrisy (we destroyed
our forests, but complain of their destruction), a people of little
principle, and few values. As they shared this image with me, I assured
them that there were probably Americans who fit their visions, but
for the most part, Americans want peace and a quiet life to raise
their families. I told them that God and family are the centers and focus of American lives---it is really a simple, but wonderfully fulfilling lifestyle!
I do hope that I was not misleading them.

\section[Performance Self Evaluation]{Performance Self Evaluation, Romans 12:3--8}
\index{performance!self evaluation}
\index{self evaluation}
\index[rom]{12:03--08@12:3--8}

\begin{quote}
``Think of yourself with sober judgment \dots'' 
\end{quote}

As a youngster, I had a friend who had contracted polio, which crippled
him. I felt so sorry for my friend, as we all did. He couldn't play
with us, which I perceived as a large and important part of our young
lives.

Concurrent with this experience, I recall being informed in school
that our country had been founded on the principle that all men
were created equal, not only from the government's perspective,
but also God's. When thinking of my friend, I could not understand
how we could all be equal, because his body wasn't equal to my body,
and he couldn't play to our level. When I asked my father how we could
be equal, of course he told me of the special gifts that my friend
had, that I didn't. My friend's strength of mind, his sense of humor,
his patience with time, and his learning ability, were all far superior
to mine. Those gifts and abilities, when perceived from an overall
perspective, made us equal.

We are not the greatest in all areas of performance as human beings.
Self evaluation starts with recognizing this fact, and identifying
what our gifts are, and where we can contribute to the mission. In
some cases, our abilities may be just in support of others. Supporting
roles, are the foundations for success in society, church, and even
in buildings. Success is not possible without them, nor can a building
stand. ``\dots  with sober judgment \dots  '' evaluate your role,
and what it should be, and how it can be improved. Self improvement
will assist the team, and its unique players, to improve the team's
success.

Improvement always seems to start with ourselves.

\section{Personnel}

\subsection[Criteria]{Criteria, 1 Timothy 3:1--7}
\index[1ti]{3:01--07@3:1--7}
\index{personnel!criteria}

In business, each employee, and every person identified with your company creates an image of service and presents an example of character, that becomes the perception of your company.
In the service business, this is even more so, since reputation is critical to success.

In the engineering design business, we have concluded that all of
our people must be qualified as ``overseers''. They all have contacts
with our clientele, they carry our company names to many a forum,
and they deliver our services directly or indirectly. They will be,
what we are. Therefore, each is treated as a leader, with the qualities
of the overseer.

As you read these verses, please pay special note to the following:

\begin{enumerate}
\item ``gentle, not quarrelsome'' will promote improved team relationships and performance, along with client confidence.

\item ``not a lover of money'' will promote trustworthiness, a simple and absolute ingredient that cannot be compromised within and beyond your organization. 
\end{enumerate}

I smile with surprise at some of the trends in business buzz that offer revolutionary statements on paying close attention to the character of your personnel.

\subsection[Recruiting]{Recruiting, 1 Samuel 16:7}
\index[1sa]{16:07@16:7}
\index{personnel!recruiting}
\index{recruiting}

Let me suggest that from a business perspective, recruits should be
scrutinized for what their ``heart'' offers.

In the engineering and design business, a good college engineering
education is a sound starting point for a career. It provides a foundation
for learning. The engineering business, like most high technology
businesses, requires continuing education programs to further develop
the culture and environment of the organization. The continuing education
promotes curiosity, improvement, and a sense of adventure in creating
new design concepts, and business procedures. The sense of adventure
and curiosity are driven by the heart.

The quality of personnel, still the most important ingredient in most business, is most often based upon character---character that has been molded by each person's heart and spiritual beliefs.

When interviewing prospective candidates, contemplate questions that
will expose their sense of ethics, morality, resolve for improvement
of the world, and sense of commitment to their mission. If they have
these characteristics in good quantity, then technical and procedural
abilities can be added in short order.\emph{ }

\subsection[Résumé]{Résumé, 2 Corinthians 3:1--3}
\index[2co]{03:01--03@3:1--3}
\index{personnel!resume@résumé}
\index{resume@résumé}

\index{Bible!characters!Paul}
St.~Paul is stating some universal truths regarding qualifications and performance.
While we can publish information regarding our accomplishments, abilities, and experiences, the proof is often obvious in our behavior, attitudes, and sensitivities, ``not on tablets of stone but on tablets of human hearts''.

I find that in a world wrought with distrust, deviousness, and almost pride in deceit, a righteous Christian in today's business world stands out like a prize rose in a flower garden of weeds. Prior to my own ``re-birth'', I often thought that Christians could not perform in the business world.
Christians might be too naïve, taken advantage of, and certainly not to be trusted because of their honesty.
I'm sure that you've heard that ``Good guys finish last!'' I was wrong.

I find that real Christians are strong, reliable, sensitive to others needs, wise beyond their years, and trustworthy.
They do stand out, and they are admired. In many instances, the heathens of the world attempt to imitate Christians, and are constantly wondering why they are so good.
``What is it about that person to whom I am attracted?'' is a common question. We do know the answer.

Like product and service advertising, the product and service must live up to the ``booking'' given, or the success will be very short lived. Similarly, one's résumé can make many claims and can cite much in the way of abilities. While the résumé may get you an interview, your character and true self will be exposed, if not at the interview, sometime later. Let your heart be your letter of recommendation and let your righteousness shine.

\section[Rejection, Brothers in Christ]{Rejection, Brothers in Christ, Romans 15:5--7}
\index[rom]{15:05--07@15:5--7}
\index{rejection}

Throughout scripture, the apostles, prophets, and Christ himself alert us to how we will become different from the man of the world as we accept God's role and mission. We may even feel a sense of rejection by our worldly brothers and sisters. Do not despair.

These verses are comforting for several reasons.

\begin{enumerate}
\item Please be reminded that God will give us ``endurance and encouragement''. 

\item With our Christian brothers and sisters, we are not alone, but in fact, we can, through God's spirit, sense the unity of purpose and commitment;.

\item Our difference in the world is not a bad difference, but in fact, a good difference; we are aligned in the Spirit of God for his service.

\item As we are reborn, our mission is to ``glorify the God and Father of our Lord Jesus Christ''.
\end{enumerate}

We can accomplish this by ourselves, with others, but always through the grace and guidance of our God.

\section[Relationships, Building of]{Relationships, Building of, 1 Corinthians 9:19--23}
\index[1co]{09:19--23@9:19--23}

``When in Rome, do as the Romans!'' This overworked saying was originally
intended to offer a means of building relationships, not simply to
conform.

In an effort to build a relationship, for whatever reason, one of
the simple ingredients is to develop a perception of similarity, either
physical, emotional, or experience-based.
\index{Bible!characters!Paul}As St.~Paul, in these verses,
outlines his methods of building relationships for the purpose of
spreading the word, the same holds true for business relationships.

The most successful business person that I had ever met used to always remind me to build a relationship first, then marketing and sales can take place.
I have adopted this fundamental principle for not only business, but also evangelizing.
Build relationships!!!

There are active and there are passive relationship building methods.
Passive methods are letting it happen, if not serendipitous.
Yes, God will assist in building such.
However, in marketing and sales, along with sharing of the Word, active methods are the point of reference in these verses.
In other words, it is our business to build relationships, so therefore we are going out of our way to make it happen.

In order to build relationships, one of the easiest methods is to
find common ground.
\index{Bible!characters!Paul}If the common ground can be expressed and shared,
you are well on the way to building a relationship, just as St.~Paul
explains in these verses.

\index{Bible!characters!Paul}
Please recognize that I do not believe that St.~Paul is promoting
deception, or misinformation. I believe that St.~Paul is simply sharing
a marketing\slash{}sales principle toward relationship building. Keep in
mind that if a relationship can be built, even with the ``hardest
of hearts'', and the communication path is opened, not only can the
product be sold in business, but the soul might be saved in the spiritual
world, which is the highest of stakes.

\section[Religions, Differences of]{Religions, Differences of, Romans 14:5--12}
\index{religions, differences of}
\index[rom]{14:05--12@14:5--12}

There are different religions, even within Christianity. In the world
of business, especially those of us who build our businesses around
their mission to God, the ``differences'' of religion in our teammates
should not hinder the goals and performance of the team.

It is not critical that each of us on a team, has the same religion,
Church or spiritual practice. What is important is that each of us
has an absolute commitment to our God and our individual practices
to worship and glorify God.

We can still be a high performance team, composed of members with individual spiritual beliefs and practices, as long as each and every one of us responds and reports to God.


\section[Rights, Reinforcing of]{Rights, Reinforcing of, Acts 16:16--40}
\index[act]{16:16--40}
\index{rights, reinforcing of}
\index{injustice}
\index{injustice|seealso {justice}}

\index{Bible!characters!Paul}\index{Bible!characters!Silas}
Paul and Silas passed up their chance to quickly retreat from the
injustice of imprisonment and punishment, to stay and take a stand to reinforce the law and their rights, for the sake of others, and for the sake of the law itself.

Throughout our lives, we experience injustice and unfair treatment.
In the USA, civil rights violations are often a daily occurrence.
It is not fair, and it is not right.
\index{Bible!characters!Paul}\index{Bible!characters!Silas}We are reminded by the courage
of Paul and Silas, that not only can we not tolerate such injustice
and unfair treatment, but we must take a ``stand against it'', even
if it means enduring unjust punishment, if not inhumane, treatment.

It is often easier for us to turn our back on injustice, and to not
make waves.
\index{Bible!characters!Paul}\index{Bible!characters!Silas}It probably would have been easier for Paul and Silas
to simply leave the prison, when the earthquake loosened their chains.
However, look at what would have been lost. In addition, Paul and
Silas were not alone---God was with them, guiding their actions and
words. God will be with us also.

In business, there are many situations of injustice, unfair practices
and absolute violations of God's commandments. I have become nauseated
by the words, ``Well, that's just business!'' \emph{It isn't and
we must not permit such activities to pass without justice and responsible
awareness!}

The business ethics and practices of not only the USA, but the world
rely on individuals to practice fairness, promote fairness and conduct
business in a righteous fashion. Like the growth of plants, it starts
with a seed of small dimension. It will grow, spread and reproduce.

Righteousness, through God, must start within us, within our own teams,
and within our companies. We must take stands against injustice, unfairness,
and deceitful practices in life and business. I pray for all of us
that God give us the wisdom to recognize these situations, and the
strength to correct them.

\section[Safety Principle]{Safety Principle, Deuteronomy 22:8}
\index{safety principle}
\index[deu]{22:08@22:8}

Doesn't this verse sound like a message from OSHA (the Occupational Safety and Health Administration)? It really could be their motto.

Growing up in the country, I worked farms, helped my father in his welding shop, and participated on many different construction sites. Safety wasn't simply the distribution of signs, but it was a common thread in everything that we did. Nearly every action was mentally preceded by the challenge of safety. We weren't directed to wear hard hats, safety glasses, or breathing masks. If the chore was made safer, we applied the gear voluntarily.

Being a lover of hunting and target shooting, safety comes first in all that you do when handling weapons. If there is any doubt whatsoever as to the safety of the condition, a shot is not taken. I share with you my father's wisdom, ``if a weapon is pointed at something, it is meant to kill it. Therefore, never point any weapon at anyone, unless you mean to kill them.'' If applied, this principle would eliminate most accidental shootings. It also points to the responsibility of the individual in maintaining safety.

Safety is everyone's responsibility, not just a governmental agency.
It starts with deliberate, well thought plans, that consider consequences.

\section[Teaching, Reception of]{Teaching, Reception of, Proverbs 12:15}
\index[pro]{12:15}
\index{teaching, reception of}

In what appears to be a simple principle, this proverb explodes with
meaning to me in my life and business experience.

In the engineering design business, the successful firms or individuals are those that apply, develop or maintain new technology. As we say,
it is important to stay atop the cutting edge, not the bleeding edge. This requires a vigilance toward new knowledge, new methods, and new approaches to even old problems. It is not good business to
be ``set in your ways'' or stagnant in career development. However,
I fear that many businesses do not embrace this vigilance.

In recruiting engineering personnel, one of our character questions
is ``how good an engineer are you?'' If the response is ``very
good'', ``or great'', we often find these candidates to be either
``dangerously arrogant'' or ``complacent''.

If the candidate considered himself or herself to be above average
or below average, but ``striving to be better'', then we have the
material that we can mold into a growing, ever-curious, and advancing
design engineer.

Too often in business today, I find ``concrete'' attitudes, unwilling
to hear another opinion. In some cases, our society rewards these
types of individuals, as heroes. Open minded, ever-developing individuals,
should be rewarded for their development, and not mistaken as being
unsure of themselves.

\section{Teams}
\index{team}

\subsection[Team Building]{Team Building, 1 Thessalonians 5:12--19}
\index{team!building}
\index[1th]{5:12--19}

\index{Bible!characters!Paul}
These words were written by St.~Paul to provide guidance to a young, but strong church in Thessalonica.
They are powerful words of direction and motivation.

I find them inspirational at the start of a ``stressful''
or critical business day, especially since my business revolves around
team success, not personal performance. They are words to a team,
which focus on the critical performance criteria of the team. The
words and thoughts appear to be in best context when thinking about
a long term goal. As an example, if these words were to be presented
to a football team, I would think that they would be best presented
at the beginning of a season, since they are geared toward building
the team, and motivating the individuals.

The specific thoughts to keep in mind include: 
\begin{enumerate}
\item always respect hard work and hard workers; 
\item always respect superiors and those who offer constructive criticism
as advice; 
\item do not fight with your teammates, ``live in peace\dots  ''; 
\item do not condone mediocrity, ``warn those who are idle''; 
\item be patient with others and help them along; 
\item follow Christ's commandments; 
\item always be joyful, give thanks and pray; and last but not least, 
\item always maintain God's spirit in all that you and your team do. 
\end{enumerate}
Start with these team building activities and guides, and the character
of the team will shine, and the strengths of the members will be synergized
to an optimum level.

\subsection[Team Success]{Team Success, Ecclesiastes 4:8--12}
\index{team!success}
\index{success}
\index[ecc]{04:08--12@4:8--12}

When I studied engineering in college, many of my faculty had been
self-employed in the engineering design business. If they were not
sole proprietors, they had been the leaders of their respective design
firms. All of the students were encouraged to direct their career
development toward the same. The seed fell deeply into my soul, as
I focused all of my career decisions toward developing my own business
where I could be front and center stage. With such egotistical
goals, I was driven toward success, which would make me
rich and famous. I deliberately selected employment which would
educationally expose me to the facets of the design business that
I would need to fulfill my self-centered goals.

On one of my assignments, I worked for one of the largest engineering
design firms on the East Coast. The firm was manned with many professionals, most of whom were dedicated to the team concept. I had originally
planned to stay with this firm for only a period of five years before
starting my own company. I stayed in their employ for nearly ten years.
During these ten years, I was constantly exposed to new and different
experiences and design opportunities. In addition, the magnitude of
the projects, the complexity of the technology, and the level of management
coordination demanded that the design projects be accomplished by
teams, with good team leadership. Being absolutely in love with my
work, I did not even recognize my need for relief from my work from
time to time. However, the team concept eventually permitted me some
relief and diversity.

This experience convinced me of the strengths of the team, and when given the opportunity in my own business, I installed and promoted the team concept. Even in my moderate size design business of just over 100 persons, we refer to ourselves as the team. Each component of the organization is also called the team. Our management is a team effort.

Some would say that businesses cannot be managed by committees and
I agree. However, they can be managed by teams; teams are not committees.

\subsection[Team Work, Interrelations]{Team Work, Interrelations, 1 Corinthians 12:14--26}
\index{team!work!interrelations}

Team building and teamwork succeed in proportional to the strength
of the participants' relationships.

Those who have performed on great business and sports teams talk of
the spirit of the team, and some sort of extended dimensional
force that propels the team to greater levels of performance. I have
experienced this phenomenon myself, both in sports and business. The
elevated performance has always been accompanied by a sense of understanding
and belonging beyond that of the material world in which we live.
It also produces a profound respect for the other teammates.

One of the more famous National Hockey League coaches, just prior
to the final game of the Stanley Cup Championships, simply wrote his
message to his team, a small group of Canadian young men, just prior
to the moment that these men had dreamed about all of their young
lives. The simple message was ``Win today, and we walk together forever!''
This message transcends the mechanics of any sport, the physical ability
of any athlete, the ability and education of any business performer.
It focuses on the importance of the relationship and identifies the
human value of memorializing performance.

Successful teams, everywhere, recognize the value of all of the participants.
Verse 26 is so important to recognize. Yes, there will be failures and there will be stars. However, the failing members and the
stars on successful teams will always recognize the value of the team as the difference between winning and losing. If you can't
personally dedicate yourself to team situations, you're losing out
on one of the better experiences of life in this world.

Team relationships start with the re-direction of individual focus,
self-centered, to that of the mission of the team. It is very difficult
for some, and easy for others. The level of egotism and egocentricity
are usually the difference. While the ego drives some to new levels
of individual performance, it is usually an impediment to team success.

In business, stars are often considered dangerous, if not self-centered.
As a manager, I do not condone stars, but demand
team play. A high level of team play will require performance equivalent
to that that any star can offer. Without the spirit of team, the stars cannot be synchronized and will frustrate
the team. The focus must be team success, not individual success.
The team goals must be met, even at the expense of individual sacrifices.

Even from a financial perspective, team success will always exceed
the sum of individual financial performance, resulting in higher rewards
for the individuals of the team.
\index{Bible!characters!Paul}Paul's successful attempts at building
teams for the Church are strong examples of this team-oriented wisdom.


\subsection[Team Work, Organization]{Team Work, Organization, John 15:1--8}
\index{team!work!organization}
\index{organization}
\index[joh]{15:01--08@15:1--8}

Christ has defined for us in these verses, the spiritual organization
of His Church and His religion. He is the ``vine'' that nurtures
our activities and existence. This definition of Christ's relationship
with the Church also is an exemplary organizational relationship
between \index{leadership}\index{management}executive management and personnel in many successful businesses.

Many of the management consultants of the 1980s and 1990s have worked hard at reinventing business structures. Revolutionary concepts in management\slash{}personnel relationships have been developed and in some
cases, used successfully. Traditional organizations are built around
the pyramid structure, with the executive management at the top, and
the labor force at the bottom, with middle management in between the
two units. This may be the oldest, and best understood of the management
structures. The ``upside-down'' organization, in contrast to the
pyramid structure, places executive management and staff functions
at the bottom as a service agency. The client or customer is at the
top, with the point of service or contact immediately under the client
or customer. The point of service or contact is acknowledged as the
most critical of functions, while other organizational elements are
primarily intended as a point of service to the critical functions.

In effect, the ``upside-down'' organization has executive management
in a posture that is the roots and vine of the organization. Its
responsibility is to service the critical elements of a highly honed
team.

\subsection[Team Work, Roles and Attitudes]{Team Work, Roles and Attitudes, Romans 12:3--8}
\index{team!work!roles and attitudes}
\index{roles and attitudes}
\index[rom]{12:03--08@12:3--8}

Working with some of the best engineering design managers in my field, I was taught to promote a consciousness of what we, as individuals enjoy and do best. So often, folks pursue careers in fields that others have promoted, or others have admired. Most successful individuals, in a competitive environment are doing what they enjoy the most and do the best. It becomes critical in team success to establish a level of consciousness within the team members, of their own strengths and contributions to the team.

If they are not aware of what those strengths are, steps must be taken
to identify them, so that long term confidence can be built into the
team members. The strengths of the individuals will lead to the role
assignments. The roles and the role members will contribute to the
team effort.

Team building starts with molding of individual contributors. Once
the roles are defined, the focus for team success has to change to
the team level. ``So in Christ we who are many form one body, and
each member belongs to all the others.''

\section[Work Attitude]{Work Attitude, Ecclesiastes 5:18--20}
\index{work!attitude}
\index{attitude}
\index[ecc]{05:18--20@5:18--20}

I'm sure that you've heard the saying, ``time flies when you're having fun!'' These verses offer some definition and understanding into the saying.

``To accept his lot and be happy in his work'' is a blessing or gift of God.
Some years ago, as my understanding and faith in Jesus Christ was
developing, I felt compelled to ``do more'' for my Savior. Concurrent
with this drive, a close friend was undergoing ``his call'' to seminary.
As he shared his pains, his confusion, his desires, and his uncertainty,
I became suddenly excited about the possible ``calling'' to myself.
I began to become sensitive to my life's activities, how I might better
serve God, and intensified my prayers for guidance. Through a powerful
series of events (a story in itself), God answered my prayers. It
is my mission to be his servant through my business path.

This experience also made me aware of the ``shortness'' of life,
the importance of recognizing God's ``assignments'' for periods
of time, and the fact that we should never lose hope.

While we may love our jobs, our calling in life, we must also fulfill
our universal expectations by serving our God to the utmost of our
ability, which is His gift to us.


\chapter{Personal Conduct and Leadership in the Business World}
\index{conduct}
\index{leadership}

As we go through life, we often have constructive criticism of others, suggestions for the improvement of others, and usually sound advice for the advancement of others.
This chapter does not focus on others.
It focuses on \emph{us, ourselves, and me (I)} in the first person.
This is where we can take a stand.
This is where we can begin to improve the world.
This is also where we can be most effective.

As Jesus Christ has come into our lives, He has made a change that
is noticeable to others. Our joy and glow cannot be hidden.
Others do begin to not only notice us, but they begin to respect us
and in some cases, follow our lead.
Whether we are given the stripes of leadership\index{leadership} in our life, or whether we simply come to know Jesus
Christ, we are in a leadership role, whether we like it or not.
Recognize it and live with it. However, live with it as Christ would
have us live with it.

Improving ourselves, and adding to or improving on our leadership
skills, requires expanded knowledge of our God's universe. It also
requires a greater sensitivity to others and to situations surrounding
us. May God's Word envelop you in such wisdom.

\section[Action On Dreams]{Action On Dreams, Proverbs 13:12}
\index{dreams}
\index[pro]{13:12}

This Proverb has been repeated in many places throughout the world.
It has consistently been an encouragement and motivation for action.
It promotes meeting one's goals, and not procrastinating. At first
glance, these interpretations appear to be correct.

In re-reading this proverb several times, new meanings have jumped
to the front of my mind. Please consider the following: 
\begin{enumerate}
\item ``Hope'' may be one of the most important ingredients in the existence
of man. Hope includes the expectations of our goals and our dreams.
It is the ``positive'' projection of what we expect to occur, and
what we work toward in our endeavors. 
\item To defer our ``hope'' is to ``hold it off'', deflect it, and in
any case make it unfulfilled. Without hope, we humans are not much
more than our biological composition. 
\item The existence of ``hope'' in people is critical. 
\item Hope may be the difference between life and death, war and peace,
and anxiety and comfort. 
\end{enumerate}

As God is our Creator, we must recognize God's wisdom and majesty, in providing us with his son, Jesus Christ. Certainly, Jesus is the
\index{fulfillment}fulfillment of our hope and God's fulfillment of our need.

\section{Alcoholism}
\index{alcoholism}

\subsection[Fate of]{Fate of, Proverbs 20:1}
\index{alcoholism!fate of}
\index[pro]{20:01@20:1}

This warning against the possible evils of drinking spirits is directed toward the effects of the drinks, and is listed as a warning.
Being ``\dots led astray by them \dots'' is an insightful way of presenting the effects.
It would seem that our real selves are asked to leave the presence, while the new person created by the spirits then takes over our body. How nearly true this can be.

Additionally, some things never change in culture.
During the enjoyment of wine amongst friends and associates, I have experienced an increase of mockery, cynicism, and gossip.
Similarly, as a young soldier, beer drinking always led to brawling.
Is it possible that these less than desirable activities are lurking potentially within us, waiting for the booze to free them?
Control and self-discipline are needed in the use of spirits.
Additionally, the same control and self-discipline are required in sober conditions.

\subsection[A Warning]{A Warning, Proverbs 23:29--35}
\index{alcoholism!warning}
\index[pro]{23:29--35}

Oh my, the plight of the alcoholic. Some science studies have reported that most of us have the tendency toward alcoholism, but that it is just not triggered. Let's not push it.

These verses provide us with the pain and crying of an alcoholic, more real than any Oscar winning performance. While in pain, depths of desperation, and acknowledgment of the addiction, there is the drive and craving for the next drink. Isn't this true of any substance abuse?

I've personally not attempted the use of illegal drugs, but I have
been a tobacco smoker. Nearly thirty years of addiction, with many
attempts to kick the habit, provide me with insight into these verses.
While I may consider myself healed of this addiction, after a year
of absence of use, these verses reinforce my conviction, and make
me realize that I'm one cigarette away from the addiction. I must
continue to resist. I will never be cured, but simply, I am just one
more day away from it.

I do pray that these verses help someone, someplace, control their
addiction.

\section{Attitude}
\index{attitude}

\subsection[Christian]{Christian, 1 John 4:11--12}
\index{attitude!Christian}
\index[1jo]{4:11--12}

As Christians in the business world, in the workday and on-the-job,
what should our attitude be? These verses from the Word simplify the
answer.

Throughout our day and week at work, I often find myself wrapped
around the axle of some issue or problem.
My focus is directed
away from God, and my Savior, Jesus Christ. The intensity of the issues,
and the thought and concentration needed to resolve the business or
engineering problems, almost makes me forget what is most important
to me in my life. Recognizing the ease with which I can be diverted,
and also the ease in which I can love for my fellow Christians
and for my fellow workers, I have had to develop an almost hourly
requirement for regular meditation to refocus and reprioritize
my focus and concern. Regular focus and concentration help me
recall the purpose of my existence, and the attitude with which
I am to live my life.

Please be reminded, ``if we love one another, God lives in us and his love is made complete in us.''

\subsection[Hope]{Hope, Romans 15:13}
\index[rom]{15:13}
\index{attitude!hope|see {hope}}
\index{hope}

Our hope is critical to our existence. God is our hope, and
He is the God of Hope. We pray to our God, to fulfill our lives, to
guide our development, and to complete our maturity. Prayer is critical
to involvement with God.

When these goals are defined and structured, the value of prayer,
and the prayer of these verses becomes meaningful and powerful.
``To overflow with hope by the power of the Holy Spirit'' is something that I can understand. When filled with the Holy Spirit,
my levels of joy and peace are so complete, that personally I have
a difficult time keeping this Spirit within. It no longer is a secret
in my life. The intensity of the joy and peace are too much to keep
to myself.

The in depth meaning of this prayer of hope makes it a prayer that
can be used quite frequently, still promoting contemplation and commitment.
I pray for you.

\section[Born again, Definition]{Born again, Definition, Romans 6:1--23}
\index{born again, definition}
\index[rom]{06:01--23@6:1--23}

I have heard these verses referred to as the long definition of being
``born again''. Often being asked what these terms mean, I am drawn
to these verses for the explanation. The logic and steps are simply
defined, but sometimes difficult to implement. 
\begin{enumerate}
\item The evil existence that we've been living, along with our treasures
and longings, generated by the pettiness of the world, must \emph{die}. 
\item Our new existence, founded upon our Savior, is our new life into which
we are reborn. 
\end{enumerate}
In many aspects of our worldly lives, changes in our behavior and
attitude are required. As we change are we not being reborn?

As an example, in my business---the engineering design business---we attempt to recruit college graduates who are driven, display self-initiative, exhibit the ability to work based upon their grades in college, and display a righteous character.
Most often these folks are expectedly self-centered, and know little or nothing about teamwork, team attitudes, and often have little in the way of people skills.
Unfortunately, our business approach requires teamwork. We have no place for egocentricity.

Our in-house educational programs and on the job training are directed
toward breaking egocentricity and promoting team play.
Effectively, we are killing self-centeredness and promoting a new spirit of teamwork.
Our staff is being ``reborn'' in business.
While this evolution is critical in life's business, the more important evolution is being reborn in Christ Jesus.

Recalling my personal rebirth brings tears to my eyes.
They are tears of joy.
It is my real birthday. While I was thirty-seven years old, chronologically, I finally began to live.

\section[Brotherly Love]{Brotherly Love, 1 Thessalonians 4:9--12}
\index{love!brotherly}
\index[1th]{4:09--12@4:9--12}

These are powerful words from my perspective. They outline our commands of how to live with one another in love, but at the same time they
also outline how we are to be perceived by the people of the world.

\index[1th]{4:11}Verse 11 is probably the most revealing. What does it mean to be
quiet and to work with your hands? Throughout scripture we
are reminded to control our tongues, to be quiet and gentle,
illustrating respect and integrity.
Isn't this the way Jesus behaved
in front of the rulers and courts before His death? Christians are
righteous, and they are not ``babbling loudmouths''.

In the engineering profession, there is a fear of being classified
as a ``textbook engineer'', which is to have only the experience
of that provided by reading. ``Real engineers'' have ``hands on''
experience. Their confidence is greater; they can jump right in to
handle problems in operations. Their capabilities of working with
their hands permit them to mobilize action quicker. They can handle
the earthly problems. ``Working with hands'' is an experience that
is not rooting us to the world and its ways. Frankly, I find myself
closer to God when I am working with my hands. Working with your hands
roots you in God's world, not man's world.

\section[Change Implementation]{Change Implementation, Nehemiah 2:13--16}
\index{change implementation}
\index[neh]{02:13--16@2:13--16}

\index{Bible!characters!Nehemiah}Nehemiah kept his intentions, plans and dreams for the rebuilding
of the wall, confidential as he developed his plans for construction.
When formulated, the plans were presented formally, with logic and
meaning, so that there was little dissension, complete understanding,
and little diversion from the work at hand.

So often in business, changes to operating procedures, organization
and mission are contemplated. May I suggest that you take advice from
the ages and the Word. Do not publicize the information, until it
is well engineered and ready for presentation and challenge. Many
of the best-laid plans are discarded long before they become public,
for often the right reasons. By maintaining confidence over possible
plans of change and then privately challenging and modifying them,
they either develop into real plans or go by the wayside.

I have adopted this policy for not only my business activities, but also my family activities.
This policy promotes control and discipline over what is said, and proper timing for such.
It is very similar to teachings of the book of James---controlling one's
tongue.
It is not wise to think out loud.
Success requires more than good thoughts and plans.
It requires proper presentation, i.e., communication, which in turn requires proper timing.

I believe that the contemplative activities of developing plans of change involve giving God plenty of chances to assist us in formulating our lives and activities. Let God be involved!

\section[Character, Perception]{Character, Perception, Proverbs 14:17}
\index{character!perception}
\index{character}
\index[pro]{14:17}

From the time of my youth, my mother often stated that I should not ``choose my friends, but permit them to happen''. While this may
appear to be a very serendipitous approach to life, it also requires attentiveness to the character of the acquaintances that we make,
if they are to be our friends for long.

As this Proverb promotes, listen to their words, and their tongue will expose their character.
If you have any question, test folks by leading them into discussions of  alternative topics, not just what they would prefer to discuss. It often doesn't take long to assess their worthiness.

Please also be alert to the deviousness of some, in that their words may appear to be quite righteous, when in fact, their character
and soul are not.
Our history points to the repeated abuse of mankind by these serpents. Wars are started, people are abused and cheated,
and good is lost to these folks. Beware!

\section[Commitment To God]{Commitment To God, Romans 12:1--2}
\index{God!commitment to}
\index{commitment}
\index[rom]{12:01--02@12:1--2}

These few verses are to me some of the most powerful, spiritually
renewing in the Bible. I read these verses, when my life, work, and
family life feel insignificant, meaningless, and in a rut. The
spiritual act of sacrificing my existence to God orders my
life. The mountainous problems finally appear at the proper scale,
the demands of this world's pressures are placed in proper priority,
and I find peace in my relatively busy life. I am transformed, and
I am able to better understand what God's intentions are.

This spiritual act of worship is an act that I must perform
on more than a weekly, or even daily basis. I must remind myself
frequently, sometimes on an hourly basis, as the pace of my worldly
existence heats up. It does renew my mind and focuses
my thinking. It sharpens my spiritual ears and eyes to see God's path
set before me. It also sensitizes my compassion for others in their
struggle to find God through the obstacles of this world.

If you have not been through this experience, I must advise and warn
you that it is a very powerful, mind-straightening experience. It
will make you shake your head in amazement, sigh with relief, and
possibly cry with happiness and joy. It is not something for the worldly
to comprehend. With time and repeated dedication of one's existence,
you will find yourself becoming one with Him and He with you.

\section{Communication}
\index{communication}

\subsection[Confidentiality]{Confidentiality, Proverbs 20:19}
\index{communication!confidentiality}
\index[pro]{20:19}

``Loose lips, sink ships!'' Sailors of World War II were reminded
not to share information regarding their ship, cargo, destination
or time of departure. They were directed not to discuss their activities.

In business, particularly in a highly competitive field, maintaining
control of information that aids your competitive posture is important.
In my business, the engineering design business, competition is keen.
It is keen from the perspective of technologies applied, fees, and
approaches to design. Often projects are obtained by bidding.

Over the years, I have hired and consulted with marketing and sales
consultants (experts). The experts were invited for specific missions,
but most related to improving our marketing and sales. Time
and again, as they would speak to all of our leaders, I would hear
them remind all that ``all information is important'' and ``do not assume that any information is not important''. In many instances
brief pieces of information may complete the picture and contribute
to the success of the competition.

In my experience, most folks are gossips and most people cannot contain themselves when pumped for information. Alas, it may warrant you to be secure in your information and limit its dissemination, certainly of critical information.

\subsection[Delivery\textsubscript{1}]{Delivery\textsubscript{1}, Titus 2:1--8}
\index{communication!delivery}
\index[tit]{2:01--08@2:1--8}

The same message and the same words from two different people can
be interpreted differently. The delivery and the source of the message
can determine how we perceive it. A message from a respected and righteous
person, would be received differently if it came from a drunkard or
braggart.

When communicating a message, consider the image created by your presence and demeanor. Your dress, carriage, projected mood are all part of the delivery.

\index{Bible!characters!Paul}
As St.~Paul reminds us in these verses, our image and the perception
others have of us, will have an impact on how folks will believe
us, or how folks might even receive the Word as shared by us.

In the engineering design business, many project competitions are
won as a result of formal presentations and interviews. Drawings
and models can not stand alone. Our presentations of concepts and
thoughts are an integral part of the program. The way we dress,
the attitudes that we project, the surroundings of the presentation
room, the methods of graphic projection, are all part of the fabric
of the image portrayed. Human nature is to first analyze and judge
acquaintances. In the case of an interview or a competition presentation,
it is paramount.

In order to communicate, the listener must first believe that the
speaker is sharing the truth, the information is of value, and the
source of the information is worthy. The listener is often passing
judgment on the presenter, even if it is only skin deep.
\index{Bible!characters!Paul}Paul
has recognized this fact for all ages of Christians.

It is a valuable lesson for us not only as Christians, but also as
businesspersons.

\subsection[Delivery\textsubscript{2}]{Delivery\textsubscript{2}, 1 Timothy 4:9-16}
\index{communication!delivery}
\index[1ti]{4:09--16@4:9--16}

\index{Bible!characters!Paul}Whether sharing the Word, or presenting a business proposition, St.~Paul's words of advice for communications are sound, and I can personally attest to their success, if followed.

Humans are critical in their judgment of persons put in a position
of authority or on the speakers' box, especially during the
initial moments of the encounter. One very prominent bias is age.
There is a tendency to denigrate a youthful speaker, at the start.
\index{Bible!characters!Paul}As St.~Paul advises, ``set an example for the believers in speech, in life, in love, in faith and in purity.'' My paraphrasing is to
work through it with righteousness---a righteousness that is complete from all observed perspectives.

Because of my own business needs for presentation skills, I have studied
image makers and presentation techniques, both informally and
formally. I have also insisted that our young leading engineers be trained by any and many of the formal public speaking and communications programs available. I must share with you that these programs while beneficial, can be greatly enhanced by following St.~Paul's advice.
\index{Bible!characters!Paul}If your own personal development in these skills need
improvement, whether for Christ's mission, or for business development,
focus on St.~Paul's advice to the early church leaders.

\subsection[Importance of]{Importance of, Proverbs 18:21}
\index{communication!importance of}
\index[pro]{18:21}

This verse goes along with \index[pro]{19:01@19:1}Proverbs 19:1. Again, it promotes control of the mouth (tongue), while at the same time not overlooking its importance.

While many of us are not great orators, we all have opportunities
to serve God in sharing His Word. In order to best serve Him, we need
to practice and prepare. While memory skills are useful in such activities,
letting God's Spirit flow from your heart in a controlled pace is
often more effective.

I personally feel so ineffective at times. A few years ago, while
on business in Texas, I had an especially poor business result. I
had time to reflect on my losses in Scripture as I waited for hours
for my flight home to Philadelphia. I thought to myself that I had
wasted my money in traveling to Texas, the business opportunity was
lost, I didn't think that I had planted any seed for future
business or personal relationships, and I had a long haul home.

As I entered the jet, my seat was in the back row, next to the galley.
I was hoping that I would be alone in the row to read my Scriptures
quietly. As the plane filled, I noticed a huge fella in a sweat suit,
ambling his way down the aisle. I knew he was headed for my seat row.
Yes, he had the wall, because there were no windows. He took up nearly
both seats, and since there were three, we had enough room. Before
they closed the hatch on our earthly ship, a young English girl joined
us. We raised the arms of the chairs to somehow make it work. Bill,
the big fella from Oklahoma, asked what I was reading, and I showed
him my Bible. He then said quite loudly, ``Ah, we have a believer.''
He then announced that he was a believer, a Baptist missionary headed
back to Africa.

He then looked at the young gal between us, and said, ``You've got
to be a believer, no?'' She went on to say that she went to church,
but wasn't quite sure. From that point, Bill, then patiently asked
her where she had been, where she was going, and what she was doing
with her life in general. Like most young folks, she was smitten with
love, and was truly interested in hooking up with a fellow from Oregon
that she had met at a celebration at Trafalgar Square. As a widower,
single parent, he was taking his time to look for a good mother for
his children. Bill, picked up on this and the almost divine arrangement
of meeting in Trafalgar Square, and asked her how she couldn't believe
in God. He then went right at her heart with the three step approach: 

\begin{enumerate}
\item Do you do unto your neighbors, as \dots ? 
\item Do you love God with all of your heart \dots ? 
\item Do you believe that Jesus Christ is the Son of God, and the Savior \dots ? 
\end{enumerate}

At this point, a young man from the seat in front of us was looking
back over the seat and listening hard. Bill would look over at me,
and say, `` you believe this don't you Tom?'' I would quietly respond
``Yes!'' The young girl would look at me then the boy in front.
I would tell them that I was pretty ignorant of these universal truths
for years, but I finally saw the Truth.

Bill gave the young man and young lady Bibles for their further reading,
in which they appeared to be quite interested. I stayed behind with
Bill, as the plane emptied. I wished him well, suggested that he be
careful, and was about to say Goodbye. Throughout my words, Bill stared
seriously at me, and as I finished he said, ``If we don't do it,
who's gonna? Huh, Tom?'' I dropped my head with Bill and prayed for
strength, courage, and God's wisdom to fill us both, and thanked God
for putting Bill in my life. We parted, and I further thanked God
for putting a ``billboard'' of direction in my life.

\subsection[Mission Mark]{Mission, Mark 16:15--20}
\index{communication!mission}
\index{mission}
\index[mar]{16:15--20}


He said to them, ``Go into all the world and preach the good news to all creation''. This is not a plea, not a suggestion, not an alternative. It is a direct command by any terms including military. It is a direct command by the Commander in Chief of our army in the spiritual wars. Insubordination, or disobeying this command, as in our own military, will result in unpleasant consequences. If you believe, you are directed to share the Word, or be condemned.

Each of us has individual gifts and missions from God. As we serve our God in our life, let us remember our commands, when the opportunities avail themselves. I personally know that there are daily opportunities in my business life to share the Word. I pray that I do not ignore the opportunities, in effect, denying Jesus Christ.

\subsection[Perspective]{Perspective, Proverbs 10:19--32}
\index{communication!perspective}
\index[pro]{10:19--32}

These verses offer the perspective of working to better understand those that are communicating to us. At the same time, they tell us something
about ourselves in our quest to be godly.

Again and again in scripture, we are reminded to control our tongue. Part of my consistent message to young engineers in business is to
not think out loud. Think and formulate, then determine for yourself the wisdom of the act of speaking your thoughts. Techniques include asking yourself the following:

\begin{enumerate}
\item ``So what?'' In other words, what was the point of my message?

\item Will I offend someone with this thought?
 
\item Will I serve God with this thought? 
\end{enumerate}

I'm sure there are also other more specific challenges to the ``starting of the tongue''. It should never be an automatic start. It must be controlled.

It seems to me that when I'm quickly responding to everything in my path verbally---or as my father used to tell me, ``you're running at the mouth!''---it was because I was excited about something, fearful of losing the speaker's platform, or emotionally distraught.

Being fearful of losing the platform is generally a sign that the people around you are not good listeners, and probably not good learners. Wisdom is difficult to gain, if you do not observe and listen.

The next time you find yourself at a reception or party, attempt to observe the folks around you that might be listeners, not simply talkers. You'll often find the listeners to be quite wise, filled with confidence and peace, and probably living a more fulfilled life.

\subsection[Salvific\slash{}Saving]{Salvific\slash{}Saving, Acts 4:8--12}
\index{communication!salvific\slash{}saving}
\index[act]{04:08--12@4:8--12}

``Plain truth-talk!''

I wrote these same words about these verses, twice, some two years apart. These words become the foundation of whether you consider yourself
a Christian---or not.

My observations of St. Peter's words focus on two points: 1) he was filled with the Holy Spirit, and, 2) he publicly acknowledged that salvation is found in no one else.

When we attempt to serve God by sharing the Word in an attempt to
save others, always recognize that if we are to have any chance of
helping, we must be filled with the Holy Spirit. I can't count the
numerous times that I thought I would cleverly debate the merits of
my beliefs, and found myself totally powerless. Similarly, I cannot
tell you the number of times when I saw or felt the opportunity to
share the Word that I was assisted by the Spirit, and was able
to communicate sensible thoughts that carried my friends further in
their spiritual growth.

One of my dear, close friends whom I consider to be a brother in
Christ, uses the terminology ``we can get them to first base,
second base, or third base, but that only God can bring them home''.
As each of us has different gifts in the sharing of the Word, my trivial
successes have often included the following quick actions, upon
hearing the ``alarm'' alerting me to a chance to share.

\begin{enumerate}
\item I immediately thank God for the chance to serve Him.

\item I ask God to fill my existence with His Spirit, that my words may be His.

\item I maintain a confident, controlled pace that is tempered with respect and gentleness, noting that a foundation of spiritual wealth must be built if there is hope.

\item I recognize that I cannot bring my friends and acquaintances the whole way home, but if I can prod and push them from one base to another, there is hope. 
\end{enumerate}

After concluding my sharing of the Word, I again thank God in prayer, and ask Him for further guidance and follow-up. I remind myself that I must be patient. I remind myself of the years that I must have tortured God with my spiritual wandering. I remind myself that God is looking for souls that have decided for themselves, believe for themselves, and love God for themselves.

\subsection[Truth]{Truth, John 14:15--21}
\index{communication!truth}
\index{truth}
\index[joh]{14:15--21}

If you don't know the truth, feel the truth and live the truth, how
can you communicate the truth? These are words that I heard for years
from counselors as a young man. These principles were further expanded
by engineering professors who espoused that science was the quest for truth in the universe. If so, engineering is the application
of the results of that quest for the truth.

I believe that these principles are true. It disappoints
me when I hear the non-believers of the world discuss science
as something that belongs to man, and is man's personal development.
As a lover and student of science, the more I study, the more I am
convinced that science is simply man's way of giving entity\footnote{Is this the best word? I'm not sure what you mean. Is it something like, ``science is simply man's way of giving voice to God's universe''?} to God's
universe. From that perspective, science never contradicts God, or
His Word. Science simply gives us more insight, and I must add, that
I believe only at the pace that God permits.

The Counselor that Christ is discussing is a different Counselor
than I mentioned above. It is a spiritual Counselor. It is the spiritual
dimension of guidance upon which we are focusing, not man's science,
not man's world's rules, but a dimension beyond this world.

If you are interested in communicating, sharing the Word, and promulgating
the Joy of God, please focus and concentrate on filling yourself with
the Spirit of God, the Spirit of truth, which will provide you with
the Counseling you need.

\subsection[Value]{Value, Proverbs 12:14}
\index{communication!value}
\index[pro]{12:14}

Good communication skills, both verbal and written, will be rewarded
in not only the world, but in God's eyes. Communication skills require
diligence and work. They come easy to no one that I've known in my
life.

My second boss in my working life was as well educated as you get
in the USA. I was consistently complimented by him, not for my engineering
skills, but for my communication skills---mostly speaking, unfortunately.
This fell on a very fond heart, since I always admired my boss's writing
and speaking ability. My ego was well out of control at times.

After approximately two years of sharing information on communication
improvements through the inter-office mail, I received a six step
correspondence course on ``Improving Your Writing Skills'' in the
mail from him. Of course, he had underlined all of the written material
and had completed the workbook assignments. I was flabbergasted that
my boss, at 60 years of age, with advanced degrees from Harvard, was still working on his writing skills.

For all of those who might be listening, the key to success in your
profession and success in sharing the Word is through good
communication skills. Do not neglect them, but spend every extra hour
working on them. Learn rhythm from poetry, organization from technical
reading, plotting \sout{and ploying}\footnote{``Ploy'' is a noun. Do you mean ``planning''?} from the great novelists, writing and
speaking from educators, and truth from the Word.

\subsection[Verbal and Written]{Verbal and Written, 2 Corinthians 10:7--11}
\index{communication!verbal and written}
\index[2co]{10:07--11@10:7--11}

If you were to meet me, it is very probable that you would find my
words and demeanor falling short of your expectations of me. While
I do not really have a gift for this language, I do have the commitment
and time to spend at the keyboard, dumping my heart into this outline.
I should also tell you that I spend time before each of my writing
sessions asking God to fill my writings and my existence with His
Spirit, that my words may be His, or at least worthy in His sight.

I can't number the times in my life when I've found myself speechless,
but within weeks I was able to communicate my thoughts \emph{via} the written
word in a better fashion than I would have expected from my mouth.
I personally have a problem of saying too much. I say what I think,
rather than thinking, preparing and presenting.
\index{Bible!characters!Paul} However, as St.~Paul
points out in these verses, actions are not just words, but the spirit of the heart in dynamic action.

Growing up as a farm hand, one's \index{ethics!work}work ethic is a point of not only pride,
but existence. Your ability to outwork others was important.
Be kind, quiet and outwork all. That was the motto of the day. Well,
to that extent, I am a product of my upbringing.
While my words often are limited in their ability to share my heart, my actions, if given
the chance, will outwork most for the sake of Christ Jesus, my Lord.

\section[Conduct, Personal]{Conduct, Personal, Romans 12:9--21}
\index{conduct!personal}
\index[rom]{12:09--21@12:9--21}

I have been asked repeatedly which Book of the Bible or verses of
the Bible are my favorite. As I normally respond, it is like picking
which one of your children that you love the most. The children, like
the Bible verses, are so unique and different in their special way.
They are all to be loved.

However, it seems that these verses are so special, so often.
\index{Bible!characters!Paul}In the
competitive business world in which I live, these words and expressions
of the heart of St.~Paul are especially important, timely and meaningful to
me. If I can target the criteria of living that St.~Paul commands,
I am closer to my spiritual goals than ever. I also can say that I
am trying, when I face God in my prayers.

If you are in a tension filled career, under a great deal of stress
most of your day, and repeatedly competing to survive, may I suggest
that you mark your Bible with these verses, and start with them most
mornings, first thing.

\section{Courage}
\index{courage}

\subsection[Armor of God]{Armor of God, Ephesians 6:10--18}
\index{courage!armor of God}
\index{armor of God}
\index[eph]{06:10--18@6:10--18}

There is a great untruth floating about, which promotes that Christian
business persons might be weak and lacking in courage. This is far
from the truth.

There are many folks who might be running from their fears, the evils
of the world, or their failure to face God. However, in the development
of one's Christian soul, we reach a point of spiritual development,
where we are required to either take a stand, or retreat in our faith.
Taking a stand, or standing your ground are not easy by
yourself.

\index{Bible!characters!Paul}Following St.~Paul's recommendation of donning the armor of God builds
a partnership between you and God. Standing your ground becomes not only easy, but it builds a closer relationship between you and
God. You become a team with God. The teamwork with God is complete
with communication---prayer.

Standing your ground is no longer a conflict in nature, but it might be considered spiritually blossoming. I think that you'll find that standing your ground will take you to new heights
of spiritual development.

\subsection[Classical Guidance]{Classical Guidance, Psalm 23}
\index{conduct!classical guidance}
\index[psa]{023:00@23}

``Leading sheep to the slaughter'', ``as timid as a sheep'', and other sayings suggest the vulnerability of sheep. Sheep have
been domesticated for millennia, to the point where they can't fend
for themselves, and in fact, they even have a difficult time giving
birth without assistance. To sheep, the shepherd is not only their protector, but their source of existence and direction.

In many ways, human beings are really quite vulnerable, particularly
when you put our existence in relative perspective with the universe
and the power of God in the universe. Closer to our lives,\footnote{I suspect you mean here something like ``in our own experiences \dots''.} we
are plagued with disease, crime, and generally the \index{evil!human}evil of fellow
human beings. It becomes apparent how much we need God to sustain
our existence.

When I read or recite this Psalm, I put myself in the place of the
vulnerable sheep and permit myself symbolically to feel the rod and
staff of the shepherd, guiding me through my life. Believing and knowing
that God is with me, leading me, and assisting me, develops a courage
that is obvious, if not contagious. It is a courage that is not arrogant,
but one of quiet confidence. It can even be a courage that others envy and find interesting.

Last week, on the business road, my cab driver from the airport was
discussing the local crime scene and the difficulty of city living.
He asked how I was able to deal with it, and my answer was one of
calm confidence. He quickly asked if I was afraid to die and my quick
confident response ``not at all'' required his repeated question
to me. While I suggested that I was not crazy about a painful death,
I really was not afraid to die and am not afraid. He then asked if
I was a ``Christian, or some other crazy martyr''. Frankly, I found
this to be quite a compliment and calmly advised that in fact I was
a Christian and a believer. He then asked if this was what made me
so strong and courageous, and I said ``Absolutely, yes!'' I then
had the opportunity to share with him my belief.

Again, the sharing of the belief and the sharing of the Word of
God requires courage, and it requires God's support to be meaningful
at all. He must be with us as a good Shepherd would be, constantly
protecting and directing us.

\subsection[Fear, Elimination of]{Fear, Elimination of, 1 John 4:16--21}
\index{courage!fear, elimination of}
\index{fear}
\index[1jo]{4:16--21}

\index{love!commitment}Love is a long term commitment, that goes well beyond infatuation.
Love is the commitment and total involvement for marriages that make
strong and complete families. It is dedicating one's existence to
something or someone beyond oneself. Love is what encourages us to
help and serve others, just as God helps and loves us.

These verses share with us the courage, and the resulting elimination
of fear, that love brings us. For those of us who have had an all
encompassing romance in our lives, we know what it is like to be immersed
in that love to the exclusion of everything around us. There was no
such thing as fear in our lives at that time.

With these verses, our love of God and our brothers in this world,
must be intense and reach the level of eliminating fear in a similar
fashion. ``Perfect love drives out fear.'' Love is not simply an
emotional reaction or autonomous response. It requires deliberation and effort. As skills are honed, so is the ability to love. Perfect
love requires concentration and effort. With it, will come the joy that knows no fear.

\subsection[Goals]{Goals, 1 Peter 5:6--11}
\index{courage!goals}
\index{goals}
\index[1pe]{5:06--11@5:6--11}

Why be courageous and steadfast in your life? These verses advise
on the goals awaiting such, and also put into perspective the challenges
of life.\footnote{I think that this needs some clarity.}

As many passages from the Word teach us, life is a maturing process
for our souls. Our souls will not mature without trials to mold them.
Recognize this fact, place your trust in God, and your courage will
grow, even as you might suffer. Ultimately, the Word promises us that
we will be restored, and we will become ``strong, firm and steadfast''.

Set your focus on these goals, and proceed from this day forward.

\subsection[Hope]{Hope, 1 Peter 1:3--9}
\index{courage!hope|see {hope}}
\index{hope}
\index[1pe]{1:03--09@1:3--9}

I am convinced that without hope, no human can live. Socially
and economically deprived folks around the world must be given hope.
They must be able to dream and see the spark or opening for their deliverance or escape from their deprived condition. If
the dream or spark does not exist, then our mission in helping mankind
is clear---make a spark and give them a dream which is real.

One of those powerful, promised and fulfilling dreams and hopes is
Jesus Christ himself. With Jesus Christ as our Leader and Savior, we will
always have the promise, and the Hope.

\subsection[Logic]{Logic, Romans 5:1--5}
\index{courage!logic}
\index{logic}
\index{reasoning}
\index{knowledge}
\index[rom]{05:01--05@5:1--5}

Once again, understanding the maturing process that we're all exposed
to in this world and life, eliminates fear through logic and reasoning.\footnote{Is this subsection more about logic, or knowledge?}
The sufferings of this life have a logical maturing effect on
us as is shared in these verses. The evolutionary process is defined
for all to see, and with this understanding courage replaces fear
and desperation.

I recall our apprehension when my wife became pregnant with our first
child, Meredith. Of course, we were fearful, almost as if this was
to be the first child ever born. Our friends encouraged us to consider
having the child with the minimum level of sedation, which required
that we learn about the entire birth process. We subscribed to a natural
childbirth group, that spent a great deal of time informing us as
to what was, and what was not, going to happen to my wife's body during
this pregnancy and delivery. We were astonished as to how our fears
subsided, as we learned more about the process. Our courage did increase.
Our courage was even stronger with our second child.

Fear is bred from ignorance. Courage replaces fear, when knowledge
is gained.

\subsection[Perseverance]{Perseverance, Hebrews 10:19--30}
\index{courage!perseverance}
\index{perseverance}
\index[heb]{10:19--30}

As we learned in Romans 5, our ``sufferings produce perseverance''
on the ladder of development leading to hope and salvation. There
are some helpful methods of assisting in the persevering process,
which include teamwork, as outlined in the referenced verses of Hebrews.

Christian brothers and sisters rely upon one another for strength
and perseverance. Their shared insights provide knowledge and truth
to the team. The corporate prayers are powerful and especially well
received by God. The ``collective'' nature of the team, in building
character and soul with one another is inspiring.

I'm often reminded by my minister that my attendance at church on
a regular basis may not only be important to me alone, but my attendance,
and words of encouragement may be important to my brothers and sisters
in Christ. We do rely upon one another.

When encouraged by a brother or sister in Christ, I hear, see and
feel my own love of God. It heightens my passion and intensity to
serve. It strengthens my courage to not only face life, but to almost
attack it with a spirit of purpose and oneness. I pray that you also
find this team, and recognize its strength.

\section[Creation, Human]{Creation, Human, Psalm 139:13-16}
\index{creation, human}
\index[psa]{139:13--17}

Do you believe in coincidence, in chance, in luck? Do you believe
that you and your existence are an accident, a product of statistics?
I don't.

These verses remind me of the manifestation of God's management skills,
his planning and his deliberateness. I believe that my existence is
within God's control and plan. When believing this, I recognize my
purpose in serving God. It has meaning, I develop confidence in my
ability to perform for God, and then the depressing aspects of this
life have less importance.

God created us! The methods that He used are not known within our
science and technology.

God created us! Face it, accept it, and let it be the basis for your
existence. He wouldn't have created us without purpose. We have reason
for existing. Do not despair! Search for the reasons! Fulfill your
purpose for your existence!

God created us! Believe it and find confidence and purpose in your
life!

\section[Crisis Management]{Crisis Management, Proverbs 20:25}
\index{management!crisis}
\index[pro]{20:25}

This verse is a warning, that in a time of crisis or passion or life
directing decision, do not make your decision quickly, without deliberation.

One of my business partners has consistently handled crises by what
he has called ``planned procrastination''. When faced with a heavy
decision, he intentionally has focused on schedule planning. His
mission was to stretch the decision making process to the limit of
the schedule. In many instances, he would report that he found much
more time than was originally reported or conceived. In some instances
his lack of decisiveness would uncover the fact that the crisis was
not real and that it was simply a perceived problem. Yes, sometimes
his procrastination produced problems, but very seldom.

I do not advocate procrastination in any situation in
life and I'm not promoting procrastination here. But I am encouraging the following in time of crisis:
 
\begin{enumerate}
\item calm the mind, so that it can clearly decipher the facts and alternatives---weigh
the information to make a good decision; 

\item focus on the schedule and identify what are the possible solutions
within the time allotted;

\item act only after thinking things through;

\item then act with commitment, since the decision has been made. Act with resolve. Do not look back. Bring the action to completion.
\end{enumerate}

The difference between winning and losing, success and failure, and life and death
is often just ``getting it done''. The resolve to complete the action must be there.

\index{commitment}So often I've heard that life is too complex to decipher. Certainly
God's abilities and His world are more complex than we can normally
understand. However, do not use this as an excuse. Attempt to make
the world and its decisions black and white, ``yes or no''. In mathematical terms, we refer to commitment as ``absolute or relative''. Be absolute, without reserve in your vows and commitments. With God's help and guidance, you'll find living to be more productive and quite simply \dots easier.

\section[Criticizing, Using Analogies]{Criticizing, Using Analogies, 2 Samuel 12:1--10}
\index{criticizing, using analogies}
\index[2sa]{12:01--10@12:1--10}

\index{Bible!characters!David}
\index{Bible!characters!Nathan}
You may want to read Chapter 11\index[2sa]{11:00@11}, if you're not familiar with the entire
story of David's misdeed. What I observed in Nathan's approach to
David was his use of an analogy to share God's message, to permit
David to see himself from God's perspective.

Analogies are wonderful tools in communicating.

Communicating constructive criticism is very often difficult, if not
impossible. People in general do not receive constructive criticism
well. I'm not sure why they don't. Due to my lack of knowledge in
this field, I will simply defer to the experts and suggest that it
is probably a product of our society, our environment, and yes, Satan.


Perhaps we're all too self-centered to accept constructive criticism
well. It is usually easier to find fault with others, than to face
our own faults. Most people enjoy this. Call it gossip, mudslinging,
if not playing God. Since folks are good at passing judgment
on others, especially remotely (a friend calls it a bombing run from
30,000 feet), rather than recognizing this propensity and using analogies to communicate.

\index{Bible!characters!David}
\index{Bible!characters!Nathan}
It worked for Nathan, as David quickly realized the severity of his
misdeed, most importantly from God's perspective. If you'll finish
chapter 12, you'll find the sincere heartache of repentance in David's
life. Of course, those of us who believe recognize that Nathan was going to communicate with David one way or another, because ``the
Lord sent Nathan to David''.

\section{Death}
\index{death}

\subsection[Hope]{Hope, 1 Corinthians 15:50--55}
\index{hope}
\index{death!hope|see {hope}}
\index[1co]{15:50--55}

I can recall when one of my closest friends at work, while often mocking
my spiritual activities, asked me to join him for lunch. He told me
at lunch that he was having nightmares that he was dying, and that
he was evolving into a dimension of ``nothingness, and state of void''.
He asked for my beliefs, because while he had made fun of my Christianity,
he was interested in knowing more about it, because of having to face
death, at least through his dreams. He asked if this nothingness was
Hades, and I advised that according to the Word, Hell will be ``eternal
torment''.

I of course took the opportunity to share what Heaven should be, and
what it would take to live there eternally. Our discussions led to
what part of us will reside in the hereafter, and how this might be
explained by science, since we were both heavily educated in the sciences.

The verses referenced above talk of a change in dimension, from the
perishable to the imperishable, mortal to the immortal. We believe
that this transformation can take place on earth, through our faith
and belief. As the Spirit fills us, we are well aware of the immortality
of our existence, while knowing that a ``change of state'' awaits
us.

Our luncheon discussion led to the topic of what makes up a soul,
and where does it reside within the perishable body, attempting to
use medical terms. At most ``turns'' of the discussion, we ran out
of scientific terms to describe what will happen to us. I could sense
my friend's cynicism arise, as we discussed these phenomena. I then
was led to take him back in time (history), when man did not understand
the unseen forces of the world, such as radiation, atomic structure,
and even light. As we speak (or you read), man is continuing to uncover
(or should I say that God is disclosing to man), explanations of the
unseen, such as things of smaller and larger scale (micro and universal),
and things of a different dimension that we have difficulty sensing.
I suggested that we must believe that our soul will be in a different
dimension than that with which we are most familiar.

I was then asked what the soul might be. Since I didn't have my Bible
with me, God led me to suggest that when we think of others, how would
we describe their ``real being'', not their physical appearance.
Characteristics such as loving, kind, humorous, sensitive, and the
many other traits that combine to create ``us''. Immediately the
cynicism stopped, and my friend cried. He believed, and to this day
has grown into a strong Christian.

I pray that we don't let our lack of wisdom and understanding of God's
universe get in the way of our beliefs. Sometimes our education and
slight exposure to God's world makes us think that we know more than
we do. In fact, we are simply seeing the ``tip of the iceberg''.

\subsection[``The Light'' (Near Death Experiences)]{``The Light'' (Near Death Experiences), Psalm 49:19}
\index{death!light@``the light''}
\index{NDE}
\index{near death experiences|see {NDE}}
\index[psa]{049:19@49:19}
\label{NDE_2_15}


Several years ago, one of my engineers came to me with a book that was listed on the popular best selling lists regarding a near death experience. He handed the book to me and said, ``I know
that you're religious and I want to know if this is true!''

I laughed
to myself, because of the apparent faith my engineer had in my opinion.
Oh, if life were only that simple!

Well, in reading this book, I became fascinated by the phenomena of near death
experiences (NDE).\footnote{For more on NDE, see also \ref{NDE_2_46} (p.\ \pageref{NDE_2_46}) and \ref{NDE_2_49} (p.\ \pageref{NDE_2_49}).} I asked my engineer where he got the book and
he advised that he was given the book by his doctor, when his mother
was dying of cancer. The doctor advised that it might help explain
things. I took the time to research my Bible, to offer explanations
as they were available, and pointed out were the Bible contradicted the book he had given me. I did a ``mini-thesis paper'' and presented it to him along with a new Bible, so that he might also enjoy the mysteries of the universe.

I didn't stop there, but I continued to read about NDE for years.
There are thousands of cases documented, and by far the majority have
many similarities. Nearly all of them discuss approaching a light.
Those NDE's that are Christians describe finitely\footnote{Is this the word you mean?} that the light is Jesus Christ.

I should point out that these documentaries have
been tested for contrivance and found to be independent experiences.
Using scientifically approved procedures, NDE'ers have been studied and found to be quite different, even in their inherent bio-electrical make-up, possibly as a result of being exposed to the ``light''.
I should also share with you that most NDE'ers are not really pleased
to have returned to life. And yes, some NDE'rs also experienced some
horrific torment---believed to be an introduction to heaven's alternative.

When we believers talk of the light, let there be no mistaking that it is God the Father, God the Son, and the Holy Ghost.

\section[Decision Making, Gamaliel Tactic]{Decision Making, Gamaliel Tactic, Acts 5:29--39}
\index{decision making}
\index[act]{05:29--39@5:29--39}

Is it of God, or is it of men? This is the fundamental and quite simple question that Gamaliel suggested be asked. In the case of the apostles, 2,000 years of growth of the Church, would indicate that it was of God.

Decisions confront us throughout our lives. The results of our decisions
are often right or wrong, good or bad, and sometimes life or death.
Generally, we all like to make informed decisions. The level
of information is critical. If given the chance, we all would appreciate
enough information, so that our decisions are ``no brainers''---so
informed that the decision is easy. This seldom happens and often
we do not have the time or resources to collect all of the information.

Gamaliel's tactic was to make the decision upon whether the outcome
will be God-driven, or not. Is it of God, or is it of men? As Gamaliel states, if it is of God, there is nothing that men will be able to
do to stop the results. If it is of men, then the results will be
meaningless, certainly not threatening. Frankly, this is comforting
to me when I make decisions. I also find that sharing the burden of
the decision making with God, inviting Him to participate, makes the decisions easier and the outcome more righteous, if not to His
liking.

Some would suggest that the Gamaliel tactic might have been employed
by \index{people!Lee, Robt.~E.}Gen.\ Robt.~E.~Lee at Gettysburg, with his fateful charge of the
Union middle by his General Pickett\index{people!Pickett, George}. It has been reported that, in effect, Lee was putting the outcome of not only the battle, but also the war in God's hands. Win the battle and move forward with a Confederate
victory, possibly over-running Washington, DC, or at least negotiating
a truce. Lose the battle with such a bold charge and recognize God's wishes for the defeat of the Confederacy.

Unfortunately, Gen.~Lee did not have sufficient time to research as
to whether his cause was God inspired, or not. I do recommend such\footnote{``Such'' what? I think it might help to state this propositionally, rather than by reference.}
as a key criteria in your decision-making. Take the time to research
whether your goals and motives are of God, or of men. Please
trust me when I suggest that you consider making your decisions based
upon the motives of God, not of men. Your success rate as a result
of your decision making will be dramatically increased.

\section{Depression}
\index{depression}

\subsection[A Prayer For]{A Prayer For, Psalm 86:1--17}
\index{prayer!for depression}
\index[psa]{086:01--17@86:1--17}

My personal bouts of depression are often caused by the stress of
business, problems with business partners and associates, unfair proceedings
by my adversaries, or problems with my family. They are often accompanied
by periods of sleeplessness at night, a tightening of the stomach
during the day, and physical fatigue. I must also share with you that
I believe these bouts of depression are stumbling blocks in my ability
to serve God. They must be eliminated, so that I can better serve
my Master.

Psalm 86 is a prayer for the elimination of depression. Steps include: 

\begin{enumerate}
\item recognition of God's place in the universe; 

\item our need for an undivided heart;

\item our need to fear God;

\item our need to be reminded of God's success in the past to relieve us of these pressures;

\item our need to build our faith and confidence in our position in the world, and;

\item the importance of our recognition of our servitude to God. 
\end{enumerate}

God will not let us down. We are often the obstacles to
healing, to resolving depression. This prayer to God reminds us of
what is involved in healing. Please keep it nearby, and resort to
it when in need.

\subsection[Identification]{Identification, Luke 21:34--36}
\index{depression!identification}
\index[luk]{21:34--36}

Ahh, the pleasures of life, or so they appear to be. Living in an
affluent and peaceful society still brings us depression. These verses
tell us that ``all those who live on the face of the whole earth''
will be exposed to dissipation, drunkenness, and anxiety.

In order to treat depression, you have to recognize it. Jesus Christ
is telling us that pleasures and frustrations of this world will affect
all of us. Beware, it will happen to you. Be sensitive to the fact
that you may in fact be suffering from such and, as a result, you may not be conducting your business and life in the fashion that you should be. Your judgment and objectivity in decision making, direction
taking, and job performance may be clouded by depression.

Depression
is natural and to be expected. When depression is suspected, be alert
to it, recognize it and overcome it. You won't overcome it with substance
abuse or emotional outbursts, but with a reminder of our obligations
to God. My best reminders are regular Bible reading, regular communication
with Christian brothers and sisters, and regular prayer. The need
of these reminders is directly proportional to my work and stress
load. When I'm very active and stressed out, I need to call on my
reminders more frequently.

God tells us that depression, brought on by dissipation, drunkenness
and anxiety is normal for the world and none of us will escape it.
Don't be afraid of it, for with our focus on God, we will overcome
it.

\section[Details, Managing]{Details, Managing, Luke 12:6--7}
\index{details, managing}
\index[luk]{12:06--07@12:6--7}
\index{management}

As I rose through the ranks of management in my career, I was always
concerned with what level of detail I should be involved. In the world
of technology, my career path led me from the bottom of the big pyramid,
where I specialized and was involved in all of the details of my smaller
sector of design responsibility. As I rose in the pyramid, I often
wondered (as do many folks) how much of my detail management should
be delegated to others. As most recognize in the team performance
arena, team members need to have some level of freedom to permit them
to perform and meddling by superiors in the details can sometimes
be just that---meddling.

My favorite football coach prescribes that for successful football,
you must pay attention to the details and then the big picture activities
will take care of themselves. He used to focus on details of performance in practice, allowing the aptitude and qualities of the athletes to
rule during the contests. By the way, he was one of the most successful college coaches of all time.

Similarly, in business, there is need for the attention to the details.
Many of us in the management of engineering design use our methods
of detail management, similar to an active zoom lens. From our
position up the pyramid, on a regular, planned, and sometimes random
basis, we zoom into the details of our operation at a lower
level on the pyramid. This approach keeps us alert, informed and it
also reinforces the importance of detail management to the subordinates.
The timing of the zoom is almost art form and certainly the
sign of excellence in management when properly done.

God rules our world, with control and attention to detail. Might I
suggest that we consider modeling our lives after what we know of
God, and for our own successes, manage our affairs similarly---managing
the details.

\section{Discipline}
\index{discipline}

\subsection[Children]{Children, Proverbs 13:24}
\index{discipline!children}
\index[pro]{13:24}

As a parent and manager of people in business, my observations confirm
the teachings of these verses. That is, that discipline is critical
to child raising and discipline is critical to design engineering
training. I can attest to the accuracy.

As we read verse 24, please note that the words include ``careful
to discipline him''. It is preceded by ``he who loves him''. Yes,
a lack of love can result in a lack of discipline. Conversely, a lack
of discipline usually indicates a lack of love and care. Care is required for discipline.

Several of my friends have spent their twilight years (not retirement,
but the second half of their life) in child raising and counseling.
Problem children, criminal children, and children of the court have
been their subjects. They tell me that nearly all of the children
have suffered from a lack of discipline and love. They go hand in
hand. Discipline without love is abuse, if not exploitation. Love
without discipline illustrates a lack of caring. Some of my closest encounters with my own children were when we embraced and softly discussed their wrong-doing after a discipline session.

Discipline is mandatory for: 

\begin{itemize}
\item societal living,
\item teamwork,
\item military experience,
\item health,
\item education, and,
\item many other aspects of life. 
\end{itemize}

Why is discipline so neglected? Discipline is not simply punishment; it is the training that teaches self-control, something desperately needed in our society.

\subsection[Reasons For]{Reasons For, Proverbs 3:11--12}
\index{discipline!reasons for}
\index[pro]{03:11--12@3:11--12}

Most often, the process of discipline is not a pleasant experience.
In order to make it a valuable experience, we need to focus on the objective.
Beware that sometimes, it is not easily recognizable.

As a father, I recognize the value of discipline. I can even recall
my own upbringing, and have found that perhaps my own father could
have spent some more time disciplining me. The long term value of
the discipline is invaluable. Discipline is necessary for our society
and civilization to continue. Taking this further, can you imagine
eternal life in Heaven, with spirits and souls free to exist, without
discipline? I can't. In fact, for this reason alone, working on self-discipline
has purpose.

God's love is the driving force for the discipline that we receive.
The experiences, trials and loves of this life, are a part of His
love and discipline. It is a process of maturing our souls for the
future in His Eternal Kingdom. Know the purpose of His discipline!
Recognize the value, and persevere! Focus on the Prize!

\section[Endurance Against Temptation]{Endurance Against Temptation, 1 Corinthians 10:12--13}
\index{endurance!against temptation}
\index[1co]{10:12--13}

Time and again, I've heard these verses uttered by friends in desperation.
I can recall my cousin crying them on my shoulder in the midst of
her husband's long battle with cancer. I can recall my mother crying
them following a year of emptiness after the death of my Father. My
observations of life would indicate that our lives are very similar
to a steeplechase horse race. As soon as we clear one hurdle, there will be another hurdle, possibly of different configuration coming
at us. And yes, we are judged on our form as we attempt to clear
the hurdle.

Of course the hurdles are the trials and temptations of life. God,
the referee and organizer of this steeplechase we call life,
has organized our temptations and hurdles at such a level of difficulty,
that we can in fact clear them. As these verses promise, the hurdles
are common to all of man and we can overcome them. We are not that
unique in our temptations and sufferings. There will be times and
trials which seem to be insurmountable. Do not despair!

God will provide a way out. You must recognize that God is with you,
and you must rely on him for direction, and aid. I can
attest that if you do not focus and rely on God's help, your personal
problems and trials will seem overwhelming. In fact, they will appear
to be monumental. Why do we always make things more difficult than
they really are?

\section[Environmental Responsibility]{Environmental Responsibility, Ezekial 34:17--19}
\index{responsibility!environmental}
\index{environment@the environment}
\index[eze]{34:17--19}

God reminds us through these verses that we are not alone on this
earth and that we have a responsibility for caring for our natural environment.
In my life, I have watched what appears to be a natural tendency for
most of mankind to plunder the landscape, destroy the foliage and
growth, and litter the beauty of God's nature with waste. As stated
in verse 18, ``Is it not enough \dots ?''

One of America's great architects used to promote the notion that
if a new building is not a site improvement, it should not be built.
If we followed his notion, there would not be many buildings, nor
civilized development. This may be extreme in concept, but our attitude
toward the earth and environment should be protective and responsible.
Do not squander this precious corner of God's universe. To date,
our space exploration indicates that God has not placed any refuge
for mankind within reachable distances. We have no alternatives for
existence but this world. We cannot trample it to a level of
environmental chaos.

As God provides our existence, do not take more than necessary and
do not destroy the environs for others. Let them also partake of the
beauty and richness. Be reminded that we are simply caretakers for
a brief period of time.

\section[Ethics, Simple]{Ethics, Simple, Proverbs 12:2--3}
\index{ethics}
\index[pro]{12:02--03@12:2--3}

God establishes the differences between right and wrong, ``good and
crafty'', and ``wickedness and righteousness''. From one end of
the scale to the other, God is actively judging. These verses tell
us that he is favoring ``good'' and condemning the ``crafty''.
Not only is ``He taking sides'', but God is advising us that there
is no stability, no foundation and no permanence in ``wickedness''.
Only the ``righteous'' have permanence and cannot be ``uprooted''.

These verses simply tell us what God expects from us, and advise of
the consequences of our actions. Are we listening?

In most activities in life, I've found that when thinking about whether
I'm living up to God's expectations, I'm trying to determine which
side of His scale I'm on---good side or bad side. Frankly, it's usually
pretty easy to determine, particularly if you do a ``heart and purpose''
check. These checks evaluate the basis for our actions. Are the purposes
and reasons righteous in God's eyes? Very few things in life are at
the fulcrum of good and bad. It's not that complicated.

\section[Evil, Correcting It]{Evil, Correcting it, Luke 6:41--42}
\index{evil!correcting it}
\index[luk]{06:41--42@6:41--42}

Yesterday, I had a Christian brother advise me that he was disgusted
with the evil in the world and actually found himself thinking that
it would be a good thing for God to simply come and take him away.
While I believe that he was just kidding, the evil and the actions
of the people of this world can often make us frustrated and make
us wonder whether there is \index{hope}hope for this world. Sometimes the anxieties
make us wonder whether we can really make a difference. Can we make
a difference? Can we do something about the condition of this world,
about the evil of the world, about the attitudes of folks in the world?

Of course, the answer is ``yes''. These verses from the ``Owner's Manual'' prescribe how and where to start to make a difference. We're usually on the inside looking out---our point of perception. Being human
beings ourselves, we are part of the problem. 

Let's become part of
the solution. Step outside and look in. God has given us choice and
free will, for our own actions and decisions. Let's make sure that
we're walking and talking in a righteous fashion, first. Our walk and talk will then begin to work on others. Yes, the righteousness
will multiply and grow. Correcting evil in the world starts with the
man in the mirror---and his Savior.

\section[Excesses, Moderation]{Excesses, Moderation, Proverbs 25:16}
\index{excess}
\index{moderation}
\index[pro]{25:16}

``Most things in moderation are acceptable'' is a phrase that I've
heard over and over in my life. Often it is used to describe the
drinking of wine and alcoholic beverages, eating of rich foods, and
sex. This verse from Proverbs reminds me of this phrase and the basic
truth that it represents. In fact, this verse prompts even further truths in the business world.

\index{management}\index{leadership}Business management does not limit itself to top management,
but to all who participate in business. Management should heed this verse in
the treatment of staff. When I use the term ``staff'', I am including
those employees that report to you, peers that respect you, and others
that might look up to you, or respect you and model your ways. In your treatment of staff, you can overdo the rewards to the point where
they are not appreciated and perhaps even detrimental to good performance.

In the engineering design business, salary and pay compensation are
relative to market conditions, and what the market will bear. Regions
may vary, but often there is parity within competitors for a given
region. As college graduates start work in the business and grow
through the first five to ten years, deliberate effort has to be exerted
to raise compensation regularly to reward the growth, the service
and contribution which is ever growing.

My first boss told me never to offer raises of greater than 15\%, since increases over this
amount had a tendency to be squandered by folks and could also build
a level of greed that would be difficult to satisfy in the future.
When he told me this, I discounted his advice, thinking that he was
``old fashioned'', of the era when employers took advantage of their
staff.

Let me share that I violated his advice approximately two years
later, giving an individual a 50\% increase. That engineer resigned
six months after his big raise to take work with a competitor at
even more income.

My competitor told me that he had hired him because
of his salary, noting that I had a reputation of fairness and this
fellow must have been really good. He was fired after nine months
on the job, took a job with another competitor at a much lower rate
than even I was paying him, his wife left him, and he died a young
man with a drinking problem.

Unfairness and inequitable business practices should not be in the repertoire of the Christian businessperson. In fact, such practices and treatment of those who serve or report to you should be distasteful in your eyes. Good, long term business practices promote the establishment of reachable goals for all participants, and the fair reward for reaching those goals. Communication of the goals and rewards is critical to maintaining the relationship.

Excess sweets and rewards are often not appreciated. There is something
to be said for a fair wage for a good day's work.

\section{Faith}
\index{faith}

\subsection[Definition]{Definition, Hebrews 11:1}
\index{faith!definition}
\index[heb]{11:01@11:1}

This verse is one of my favorite definitions of faith. Reading
and saying it, remind me often of my mission in life and the importance
of simply believing.

Personally, my faith has few blind spots.\footnote{Here I think you mean ``blind leaps''. ``Blind spots'' is usually a phrase that means areas of ignorance.}
Most of my faith is built upon some level of knowledge. While the
universe is puzzling to science and technology, I attempt to understand
the unknown with some level of explanation that is God-centered. With
this little bit of knowledge, I then attempt to build a belief that
is sure.

I recall the strength of this verse showing up in the bankruptcy proceedings
of a friend and his family business. I asked him where had he gone
wrong. Rather than advising me of such things as inventory control,
outstanding accounts receivables and payables, bad debt, extended
credit, and other such accounting analyses, he advised me that
he ``had not been sure of what he had hoped for''.

He led me right to \index[heb]{11:01@11:1}Hebrews 11:1 and put my nose in it. He then admitted that the driving reasons behind the founding and running of his business were to feed his ego with money and fame. While he was a believer, he prayed that God would save his business, his house and car, all for the wrong reasons.

My friend lost his business, his home, his beach home, his boat, his
car, and his place of fame. His wife, being a devout believer,
stood beside him through all of the trials and liquidations.
He asked God for forgiveness and prayed that, if given another chance,
he would not forget to establish reasons acceptable in God's eyes
for the business and then he would be sure of what he hoped for.
My friend, with nothing to start from other than his profound faith
in God, has rebuilt his business and has made his business a platform
from which to serve God.

Please be assured of what you hope for, and be sure of the righteousness
of your hopes in the eyes of God. When business decisions appear,
test the alternatives in God's eyes, with prayer and study in the
Bible. They will lead you through all of your decisions.

\subsection[Value Of]{Value Of, Romans 5:1--2}
\index{faith!value of}
\index[rom]{05:01--02@5:1--2}

Now for a trick question. What is the value of faith? These verses
tell us that we will have peace with God, access to God's grace, and
joy in the hope of the glory of God. As an individual, this is more
than I could ever hope for, simply by having faith.

If I go back to \index[heb]{11:01@11:1}Hebrews 11:1, faith also is defined as believing in
something that cannot be seen. Examples of faith are not exclusive
to religious faith. Much of science includes faith.

The existence
of electrons is an example. Electrons have never been seen, felt as particles, and never weighed.\footnote{In the classical system, this may have been true; but is it still true in the present quantum system? A quick internet search says that electrons are $\approx 1/1836$\textsuperscript{th} the mass of a proton. A better example now might be something like ``dark energy'' (on which the idea of quantum mechanics depends, but for which there is no method of detection)?} They're too small, too active and too light in mass. The theory of electrons and their existence and characteristics seems to make sense. The theory of their existence has been the basis for electrical energy theory for some decades now. Our science has accepted their existence and electron existence and theory have been taught, if not promoted, for nearly a century. In effect, we and science have faith in the theory of electrons and their part in sub-atomic structure.
Isn't it odd that our science, government, and society cannot similarly have faith in God and Jesus Christ, even though our history has repeatedly recorded their existence and the acts attributable to their existence?

With this in mind, I am amazed at the lack of faith. There is more
proof for the existence of God than science can offer for the existence
of electrons, and to top that off, there are tremendous, eternally
lasting benefits of the belief and faith in God. I thank God for my
own belief, and I am so sorry, if not embarrassed for the years of
disbelief.

\section{Family}
\index{family}

\subsection[Generations]{Generations, Proverbs 17:6}
\index{family!generations}
\index[pro]{17:06@17:6}


My daughter was eight and my son was four years old when I first stepped
through the door of accepting Jesus Christ as my Savior and believed
fully in the existence of God. Prior to that time, my life and business
existence were very self-centered, and I was driven by an egocentric
belief that my existence would better mankind and certainly the engineering
design industry. My understanding of my existence was not a wholesome
situation, nor was it very deep and fulfilling in its understanding.

However, as my children and God became the focus of my life, I quickly
came to value the significance of my own existence, by having God
remind me of my children and perhaps even the generations to follow
them. This verse in Proverbs has consistently motivated me to put
my own life into the perspective of the universe and the march
of mankind through the centuries.

Having done this often, I am convinced that my greatest contribution
to mankind may be through the development of my children, and even
those that they influence and develop. We must remember that
what is at stake are many generations and friends of friends to come.
Avoid the self-centered perspective that only what you see and hear
in your life is of importance. Often the impact of our words and actions
go well beyond our hearing and reach, especially through our children.
Also, be reminded that the words and actions we exhibit in our life,
have a profound influence on our children, and even their friends.
Hopefully, and I pray constantly, that all of our words and actions
be a point of pride to not only our children, but our God.

\subsection[Universal]{Universal, Luke 8:19--21}
\index{family!universal}
\index[luk]{08:19--21@8:19--21}

Christian brothers and sisters are throughout the world. They are
our eternal family. Often, I sit back in our church during Sunday
worship and look around at my brothers and sisters in Christ, and I smile at the comfort that I feel in knowing of my eternal existence
with these folks-souls. Like me, the most important thing in their
life is God. We are one in purpose, and unified in the goals
of life. Even the most remote and foreign of characters suddenly become
kin. Why? Because it is a matter of priority. Our God and our servitude
to Him is of utmost importance. Our differences fall away. 

We may be separated by language, by skin color, by country, by family,
by beliefs and values, but not by our God. It is the one and only
God. 

I pray that you step up to the task, become part of the momentum and start introducing yourself to your brothers and sisters in Christ
to let them know of your common interest and dedication. I believe
you'll find, as I have, that they are in fact truly kin. As the ``man in the mirror'', are you part of the family?

\section[Fear, Coping With]{Fear, Coping With, 1 John 4:16--18}
\index{fear!coping with}
\index[1jo]{4:16--18}

By many standards, I am a new Christian.\footnote{Would you like to revisit this paragraph for dates and thoughts?}
At the time of this writing,
it has been twelve years since I accepted Jesus Christ as my Savior.
Over those twelve years, I've worked at learning more of God, at knowing
His teachings, and as a result I've grown spiritually. With God's
help, I hope to continue to grow further. I noticed a
remarkable inverse relationship with the spiritual growth.
As I've grown spiritually, my fears have subsided, by an inversely proportional ratio. With my spiritual growth, my love for God has
grown, almost proportionally. My life has been what these verses of
1 John state.

In the early days of my Christian life, I had difficulty stating with
confidence to God that I loved Him with all of my heart, mind and
soul. In fact, I revised my daily prayers to God, to ask Him to help
me love God with all of my heart, mind, and soul. I can share with
you that there are growth levels of love, and the more complete the
love, dedication, and commitment to God, the less fear of the things of the world and beyond. 

Like in a good and complete marriage on earth, a relationship of love
means sharing. It offers collective powers and defense. Strength in
numbers and a union create strength. So too in your relationship
with God. Growing in your love of God will create a union of strength,
mitigating and minimizing the fears of this world. With God, you have
nothing to fear. I pray that you learn to not only love God, but that
your love for God continues to mature and become complete, eternally.

\section[Freedom of Conscience]{Freedom of Conscience, Romans 6:1--23}
\index{freedom of conscience|see {conscience}}
\index{conscience!freedom of}
\index[rom]{06:01--23@6:1--23}

We are free to choose between sin and righteousness. As the passages
say, one leads to death and the other to eternal life, but we are
still free to choose.
\index{Bible!characters!Paul}Upon choosing, we become dedicated, committed
and absorbed in our choice. St.~Paul appropriately chose the word
``slave''. We become enslaved to either sin or righteousness. 

I can recall that one of my pre-Christian fears was that by becoming
a Christian, I would be limiting my life, cutting off the things that
were fun, and even possibly becoming a social outcast. I thought that
the self-discipline of Christianity would be limiting my abilities
and I would lose my freedom. Yes, I was really, really wrong and misled.
Becoming a ``slave to righteousness'' or committed to God, in fact
opens new horizons that are only available to believers. The ``freedom''
of being a slave to righteousness even on earth, is immense and challenging.
In fact, I believe that these freedoms that I have, are training for the ultimate freedom---eternal life. The freedoms that God gives us
on earth, especially after we commit ourselves and our existence to Him, can be very good maturing tests, to help us grow spiritually strong in His eyes. They prepare us for the limitless freedoms that we will experience in His eternal kingdom. 

\index{Bible!characters!Paul}
These writings of Paul, as God inspired them, can be quite confusing and on the surface appear contradictory. As my love for God grows, as my commitment to God becomes more complete, the meaning explodes in my heart with acceptance and understanding. There truly is freedom when you become a slave to God and His righteousness. You do throw away the shackles of sin. As you bathe in the love of God, you are transformed into an existence of almost ``free flight''. Words become
limiting in describing the experience. Freedom is the word of choice. 

\section[Friendship, Joy]{Friendship, Joy, Proverbs 27:9}
\index{friendship}
\index{joy}
\index[pro]{27:09@27:9}

As I read this verse, I am often reminded of my daughter and her friend
next door when they were both 4--5 years old. As they would emerge
from the house to play each day, they would run toward each other
screaming with joy and start the day with a huge, genuine hug of
love and joy.

Our friendships are precious. I believe that it is important and fulfilling to recognize our friendships, and not to take them for granted.

A friend suggested to me that I should step aside from myself and put my life in perspective, as if I were to write an autobiography.
Doing something like that forces you to recognize and acknowledge your friendships and how important they really are.
Ponder friendship and spend the appropriate
``quiet time'', finding out about yourself, your friends and your life. Then, as you encounter your friends each time, be reminded of
their value and experience the joy of their friendship, and do not
be afraid to offer a genuine, loving hug. Who knows? This kind of
joy and love might become contagious.


\section[God, Perspective of]{God, Perspective of, Deuteronomy 10:12--22}
\index{God!perspective of}
\index[deu]{10:12--22}

Our science and technology have identified many unseen forces and
things in our world. X-rays, radiation, sound waves, sub-atomic particles,
and the wave-particle duality of light are just examples of some of
them. Much of the universe and our world is held together by unseen
forces and particles. We, as creatures, are very limited in our ability
to see and understand.

In this context, we are very weak in ``seeing''
and defining God. 
This definition of God in Deuteronomy is as fine a definition as I've
been able to find, beyond the love of my heart. It concentrates on
defining God's love, His character, and His actions. I believe that
these defining characteristics are those that are most important.
They are the characteristics that are eternal.

I was recently asked by a searching friend if I believed in
God, and after-life existence. After confidently responding ``yes'', I
was asked to explain the possible transition from this earth to the
afterlife, including the definition of a soul or spirit.

My friend was hung up on the transition from the physical body and scientifically
attempting to define the transition. I advised him that man's science
cannot yet define the ingredients that go into making a human being,
that man can't understand the structure of our universe, and by far,
science has not been able to grasp the transition from the body to
the soul in eternal life.

However, using the approach of characterizing God in Deuteronomy, one can begin to define the characteristics and make-up of the soul after life on earth. The defining ingredients of love, character, and purposeful actions are the terms of definition.
Aren't they also the ingredients of what we really are? Our bodies with their physical attributes are simply temporary quarters for the real us. 

May I suggest that you consider looking at your friends, and brothers
and sisters in Christ, beyond their bodies and appearances. Start looking at them for their love, character and purposeful actions. Meet the real people, not their façades.


\section[God's Ways, Following]{God's Ways, Following, Jeremiah 17:10}
\index{God!His ways!following}
\index[jer]{17:10}

Much of the Bible can lead us to the way God conducts Himself, and
in most cases, we are invited to copy His ways. This verse from Jeremiah
outlines God's method of evaluating men. It is also quite appropriately adaptable to business. As God searches the heart and examines
the mind in His evaluation of a person's conduct, He is identifying
the motives for the conduct. The purpose for the actions is more important
than the actions themselves. It is the heart, soul and mind that
make the person. 

In business, many critical positions of performance require more than
simply doing a day's work. The stress levels, the creativity and ingenuity required, and the above-average initiative needed to perform can make recruiting for some positions a challenge. It requires exploration into what motivates a given candidate to perform---his or her heart,
mind, dreams, and goals. These characteristics are the enduring factors
for long term performance. The more critical the role in an organization,
the more important it is to consider these real characteristics.

Again, as a point of perspective in exploring the Bible, there are
many parallels in God's conduct that we should consider in living
our lives.
He and His Son, Jesus Christ, lead us with many examples,
which are adaptable to all aspects of life. 


\section[Government, Praise to God]{Government, Praise to God, Psalm 72}
\index{government!praise for}
\index[psa]{072:00@72}

I am pleased and blessed to be an American. I was raised in the patriotic
outback of America, served in the US Army, and nearly cry every
time I hear the National Anthem. My patriotism forces me to rise to
attention, with hand over heart at the start of ballgames. I get chills
on the 4\textsuperscript{th} of July. I love my country, its government,
its people of many backgrounds, and the idea that every person has
the opportunity to grow beyond his or her parents' accomplishments.
I love my country.

I am thankful that my country permits different religions and beliefs.
I love the fact that our country is a haven for eccentric religious
groups. My country permits them to practice their religion---as long
as no one is offended\footnote{You might want to consider the distinction between ``offended'' and ``hurt'', especially in this present day, when ``being offended'' means that I can prevent another person's free speach or free exercise rights under the Constitution.} or hurt by it. My country and its government
were founded to promote freedom. Through the centuries of its
life, my country has had many trials and wars of its own. It has survived
them, and emerged from the trials stronger and wiser than ever before.
Our \index{democracy}democracy and our culture have been blessed by God. 

\index{Bible!characters!David}While Psalm 72 was a prayer of David, to proclaim the gratitude of
Israel to God, they are wonderful praises to God for the United States
of America. Next Independence Day, when you take time to consciously
give thanks for the USA, remember Psalm 72 and praise God for this
wonderful country of ours.


\section[Grace \emph{and} Law]{Grace \emph{and} Law, Romans 4:1--16}
\label{chap2:law_and_grace}
\index{grace and law}
\index{law!and grace}
\index[rom]{04:01--16@4:1--16}

\begin{center}
\emph{For related thoughts on law and grace, see \ref{chap7:law_and_grace} on page \pageref{chap7:law_and_grace}.}
\end{center}

Grace and law, as two different foundations for belief, are not easily
understood. The differences are even more difficult to explain.
\index{Bible!characters!Paul}St.~Paul does a great job in these verses. In order to appreciate the
beauty of his thinking and words, try to focus on ``the foundation
of faith''---or what is the basis for belief.

I can recall shortly after I had become a Christian, I asked my spiritual
leader and friend, Phil, to help me further grow in Christ and to
guide me in making more of my life as a servant to God. Phil began
to describe a journey into the spiritual growth of a human being.
It was a journey of knowledge, passion and faith. It was founded upon
prayer and study of the Bible. I can recall him talking about appreciating
the difference between grace and law, as being one of the thresholds of spiritual development. While I don't profess to be spiritually
mature or complete, I have enjoyed learning and studying grace and
law. They do require contemplation and thought. When unfolded,
the concepts of grace and law provide us with guidance in our spiritual
development, and also promote a heartfelt passion for our faith, and
enormous appreciation for God.

\index{Bible!characters!Paul}
Grace and law, as presented by Jesus Christ and St.~Paul, outline
the threshold of spiritual development, which can be expressed as
truly believing and understanding. The closest analogy that
I can offer as an engineer and business person is the learning of
mathematics. I was educated in the math tables, trigonometry, and geometry
by law. I initially memorized the tables and the theorems, accepting
them as law without understanding.

``Two plus two equals four'' was a law. I was not initially taught that if I added two apples to
two more apples, I'd have four apples. It was after years of applying
the law, that suddenly, I understood why two plus two equaled four.
Eventually, the mathematics began talking to me and I could
see the workings of it beyond the memory and the acceptance of the
law. Advanced engineering requires not only memory and acceptance
of science, but an understanding and passion for the laws of nature, laws that can be applied for the betterment of mankind. Whether it's building
huge bridges, buildings, automobiles or airplanes, good engineers
do not simply apply tables, but they feel and understand the numbers
in their hearts and minds, knowing the outcome with confidence and
belief. So it is with faith.

I was brought up in a Christian church. I learned the principles
of my religion by memorizing the ten commandments, Christ's summary
of two commandments, and our Christian creeds. Unfortunately, it took
me nearly twenty more years to understand the beliefs and to passionately
accept them. 

If we accept the beliefs and the laws of our faith, and if we conduct
our lives in compliance with the beliefs and laws, then we are righteous according to the law.\footnote{Do you want to say something here about how we will still not measure up to God's standards, no matter how hard we work at it? Otherwise, it may sound as if you are advocating the idea that we \emph{can} be accounted righteous before God through our obedience, without Christ's atonement. Also, this could be a great spot to again drive home the gospel.} We are conforming to the standard set before us. We
are following the program, almost as if we were a robot or machine.

However, if we truly believe in God, love Him with all of our heart
mind and soul, and truly love others as ourselves through the love
and passion of our Lord Jesus, then we have faith by grace,
which was offered to us through Jesus Christ. Upon this faith, as
compared to that of by law only, we can apply our faith in life,
and our works will be righteous, and it can be expected that our works
will comply with God's laws---not because we are required, but because
we do it of free will. We do it, because we choose to.


\section[Grace \emph{versus} Law]{Grace \emph{versus} Law, Romans 6:1--23}
\label{chap2:law_v_grace}

\index{grace and law}
\index{law!and grace}
\index[rom]{06:01--23@6:1--23}

\begin{center}
\emph{For related thoughts on law and grace, see \ref{chap7:law_and_grace} on page \pageref{chap7:law_and_grace}.}
\end{center}

We can practice our faith by law or by grace, or by both. By law,
we are simply following the practices and principles, perhaps without
heart and passion. By law only, we are but mere robots. We are not
using our hearts and minds, but simply following orders. By following
orders, we are slaves to the track that we're following. We're simply
taking the path that we're directed to follow. While our actions may
appear righteous, they are mindless and mechanical. 

However, by grace, our faith is practiced through our free will. We
hopefully comply with the laws, but we are conducting our works because
our heart and mind are initiating our actions through love and obedience.
We are free!
We are free, because of the grace that God brought to us through our Savior Jesus Christ.

There are obligations with this freedom. It requires education and
study in the Word of our Lord. Understanding His Word will help us
in making proper decisions. It requires frequent, regular communication
with God. Prayer will provide guidance and pureness of heart for our
mission. Living by grace requires a boldness, that is refreshing and
free. The rewards of a righteous life, by grace, are the ultimate---eternal
life.

\section[Gratitude, Remembrance Of]{Gratitude, Remembrance Of, Deuteronomy 6:10--12}
\index{gratitude}
\index[deu]{06:10--12@6:10--12}

July 4\textsuperscript{th}, Independence Day. As I write these thoughts,
the sky is blue, from my study I can see our flag waving in the breeze,
and I can hear the fire crackers popping. There is a lull in my business
and family life allowing me to step back and acknowledge the
value of our freedom. I can't do it without remembering the source
of our gifts and our existence. These verses, while originally intended
for the society of Israel, are so meaningful to the American Society.
We don't have to travel far around the world to find countries and
cultures void of the hope and opportunities we have. In fact, most
of the world does not have our opportunities. 

Some will say that they've earned their homes, daily bread,
and lifestyle themselves. They believe that their hard work, good
mind, and cleverness are the reasons entitling them to their success.
They forget Who gave them their life, their minds, their education
and the opportunities that our democracy offers us. Please be reminded
of the source of our things and our existence---God. 

As we are reminded of our wealth and rich lifestyle, we must be ever mindful of the need to give God thanks. At all times and in all places we should give thanks.\footnote{Nice citation from the BCP!} Do not forget the source---God. 

\section{Growing And Growth}
\index{growth}

\subsection[Through Righteousness]{Through Righteousness, Romans 5:5--17}
\index{growth!through righteousness}
\index[rom]{05:05--17@5:5--17}

I look for every opportunity to discuss God and Jesus Christ with friends and associates.
Most, who are not born again, will freely offer that they believe in God but have not adopted all that other stuff. I also spent years feeling this way. I was raised in a Christian Church, but fell short of understanding the significance of Jesus Christ.

Upon becoming a believer, I was advised by many Christian brothers
and sisters that spiritual growth requires work. It demands substantial
reading and prayer. In all cases, however, all must be done with a
focus on Jesus Christ, our Savior. Too often, we church-goers overlook
the powerful significance of Jesus Christ and the Holy Spirit.
\index{Bible!characters!Paul}Paul's
words in these verses remind us of how we can grow, how to do it, and where to focus. 

Pay special note to the following: 

\begin{enumerate}
\item ``God has poured out His love into our hearts by the Holy Spirit, whom He has given us.'' 
\item ``God's abundant provision of grace and of the gift of righteousness reign in life through the one man, Jesus Christ.'' 
\end{enumerate}

\subsection[Professional and Spiritual]{Professional and Spiritual, 2 Peter 1:5--8}
\index{growth!professional}
\index{growth!spiritual}
\index[2pe]{1:05--08@1:5--8}

I believe that part of the success of our company in the engineering
design business has been our promotion and nourishment of professional
growth and advancement. We have attempted to instill a never-ending
quest for growth in all of our staff, with a special orientation program
for the youngsters out of college. Consistently growing and maturing
will produce not only personal rewards, but team and company results
that are appreciated. Yes, professional growth is very similar to
spiritual growth. Both endeavors are never ending, and have ``predictable
passages'' in the development phasing. 

These verses from St. Peter's testimony outline the spiritual growth
process, which frankly, also parallels professional growth. They are
progressive steps and identify intermediate goals of advancement and
development.
\begin{enumerate}
\item Start with ``faith''. As a young Christian, I was taught to ``walk
through the door'' and believe. You must have faith, as a start.
In engineering and business, you must believe in what you've been
taught, and you must believe in yourself. Then \dots
\item Add goodness. Fill your existence with righteous thoughts, and throw
away evil. Think positively about your mission and your accomplishments.
Be sure of what you hope for. Then \dots 
\item Develop knowledge, then; (Identify a continuing education program.
Consistently read and gather information. Identify experts, and resources
of advanced information. Know the specialists, and use them for your
personal development. Begin to build and nurture a library of information.) 
\item Develop self-control, then; (Recognize that your ego has driven you
to apparent success in some instances, but that it can be a hindrance
in the future, if not controlled. The ego often carries emotions which
can explode like a volcano, allowing untimely comments and words lacking
wisdom. Control the emotions and the tongue. Be deliberate. If you're
a Christian, ask God to fill your existence with His Spirit, such
that your words might be His words, and will work to serve Him.) 
\item Proceed with perseverance, then; (Be aware that professionally aggressive
people, along with developing Christians, will be obvious (standing
out in the crowd), and not necessarily appreciated by others, who
might want to maintain the status quo. Expect to be criticized, ostracized,
and even persecuted for your desire to make the most of what you've
been given. Strengthen yourself with a consistent image of your goal
and destination. Be prepared to explain the hope that you have, and
what the future might hold and why. Persevere, you are not alone!) 
\item Blossom with godliness, then; (As you have matured and developed to
this level, you will begin to see the world and the universe in a
different light. Businesses will not just be a place for making a
living. They will become a forum for people development and betterment.
They can be an environment in which families blossom. As a Christian,
you will understand more of what God has in store for your life and
mission. The value of your life will go well beyond what your old
vision might have been. You will feel the power of God, and the mysteries
will unfold before you.) 
\item Resulting in brotherly kindness, then; (At this level, I suspect that
your business people skills will have advanced to the optimum point
of communications, relationships, and your ``sphere of influence''
in business will be great. You will recognize in business that virtue
and righteousness are of value. Negative advertising and opinions,
along with evil manipulations, are not long term success factors.
Even in business, you will exert extra effort to train, educate, and
promote yourself as a ``good fellow''. Christians are commanded
to do such.) 
\item You'll have love, then; (This may be the one point of variance in
the comparison. Professional and business development will probably
not add love to your life. Spiritual development with God will. It
is God's guarantee. ) 
\item Continue to improve and add to all of the above. (Please be reminded
that it is a never-ending growth, and we are thankful for it.) 
\end{enumerate}

\section[Guilt, Degree of]{Guilt, Degree of, Numbers 15:27--31}
\index{guilt}
\index[num]{15:27--31}


God identifies the types of guilt, brought about by sinning. He lists this guilt as unintentional versus that which is defiant. The unintentional can be further defined as accidental, or unintended. Most of my acquaintances are unintentional in their sin.

However, there are those folks who are intentional and defiant in their actions of sin and guilt. God's solution is to cut him off, remove him from His people, from His society. 

In today's business environment of stressful competition, the criticality of teamwork for success, and the requirement for high levels of education, character is of importance in the make-up of your business teams. It is to be
expected that people will make mistakes and unintentional errors. Those mistakes are forgivable. 

And yes, while we are to continuously forgive our fellow man in their
transgressions, we must be aware of all of the alternatives that forgiveness
brings. If one of your team members defiantly wrongs the group, he has hurt all involved. He is a threat and frankly, a distraction
to performance for all involved.

I recommend that you contemplate cutting him off from the group, either through dismissal or
reassignment. The dismissal or reassignment should be accompanied with a communication of the faults that you have found and the need
for the individual's repentance and change. Your dismissal\slash{}firing of that individual, or their reassignment may be the single most important breakthrough
in that individual's professional life to date! Most folks receive it as a wake-up call. Most also move on to more productive and serving lives.

Dismissals and firings are some of the most difficult tasks that I've
faced in business. They are stressful and guilt ridden. Dismissals
and firings are often accompanied with a feeling of failure on my
part. It is as if I could have done something that I didn't, or that I've done something that I shouldn't have.

However, these situations
are trials that will mature us. They will not only mature us, but also those receiving the dismissal. We are part of a multi-person
trial. In retrospect, they have been building experiences, even more so for those that I've dismissed. While there haven't been that many dismissals under my command, those that were dismissed have become better people for it. 

Please remember that our businesses do not simply exist for bread-winning, but they are arenas for our personal maturity and platforms for our spreading the Word!

\section[Help, Seeking it]{Help, Seeking it, Proverbs 13:10}
\index{help!seeking}
\index[pro]{13:10}

Often we are too busy or too proud to seek help in our concerns and
problems. I find myself forgetting to ask for God's help in my day
to day activities. In order to avoid getting too wrapped up in my work, and lest I forget the purpose of my existence, I have organized
my day into quarters and benchmarks. 

When I am in the office I treat my day like a football game. At the
end of the first and third quarter, I break for prayer as a reminder
of my mission and as a reminder that I can't go it alone. I must be with God and He with me. I need His Spirit within me. 

At noon, I lock myself away in my office for quiet time and Bible reading. That is the specific case as I write these thoughts.
I find refreshment in the Word, and recharge by the Spirit. 

On the road, I employ similar intermissions, as I call them.
They are breaks in my day and travel that permit me quiet time for meditation. I find myself recalibrating my goals
and perspectives with respect to God as my benchmark.

Examples include every time I leave my automobile. It is a time for thanks, a time for request for assistance, and a time to worship God. Just
as I tell my wife and children, that I meaningfully love them,
I need to tell God the same, ``with all of my heart \dots'' Other
benchmarking occurs between takeoff and cruising altitude in plane
flights, before I've met my fellow passengers and before meals, and snacks are fed. And of course, before each meal---a time of thanks, requests for blessing, and a request for advice and direction. 

In addition to asking for God's help in our endeavors, we should not
be too proud nor afraid to ask other for help or advice. Most folks
take a Request For Advice (RFA) as a compliment. It also illustrates
to others your sensitive side, your quest for knowledge and wisdom,
and your intent to respect them. I advise most of my young engineers
to maintain a list of experts that they can rely upon throughout their career. Don't be afraid or too proud to ask for advice, and
then give heed to it.

\section[Homosexuality, Commentary on]{Homosexuality, Commentary on, Romans 1:24--27}
\index{homosexuality}
\index[rom]{01:24--27@1:24--27}

I have had several homosexual and lesbian friends. I have enjoyed
their company, and have found them to be well-meaning, sensitive people.
I have found them (in most cases) to be honorable in business practices.
Most folks would consider them good people. Frankly, it has been very
difficult for me to deal with them as sinners. It has been difficult
for me to understand why their ways and practices are not acceptable
in God's eyes. 

In my pre-parenthood life, I simply accepted the Word, such as that found in Romans~1. I left the differences and judgment in God's hands, simply as a matter of faith. 

With my two children nearly grown at this time,\footnote{Does this deserve an edit?} I now have a better
appreciation for God's direction and Word to us. I believe that our
sexuality and tendency for sex is a wonderful and natural trait,
which permits us to give birth to the next generation. The next generation is, of course, our hope and pride. The basis for our family structure, as we've known it in our history, is heterosexual, with sex intensified
for reproduction. Frankly, my most memorable sexual moments were those
in which our sex was for the purpose of creation of our children.
It was very powerful sex, as Hollywood would describe it. 

This powerful drive can be led astray. Throughout history, man has
found diversions for the sex drive. Many and most diversions, including
(in my opinion) homosexuality detract from our social organization, our ability to perform within our society, and our mission. We must be aware of the consequences.

Our God is forgiving, as we should be. I pray that His Word is received.

\section[Hope, A Prayer for]{Hope, A Prayer for, Ephesians 3:14--19}
\index{hope}
\index{prayer!for hope}
\index[eph]{03:14--19@3:14--19}

Many Christian church service include popular prayers of hope, such as \index[rom]{15:13}Romans 15:13. However, I wanted to bring Ephesians 3 to your attention, since it is just a little different from my perspective. 

The Ephesians 3 prayer of hope appears to be intended for Christians (believers), who are in the process of further maturing. The focus is on His Spirit
and His Love. It understands that God's grace is what makes us more hopeful and directs us in our lives.

It is a prayer that gives me chills and tears in my quiet times. Hopefully, as a Christian who is attempting to continuously
mature in my spiritual being and my love of God, this is as good as it gets, and I thank God for it.

\section[Idolatry, False Gods]{Idolatry, False Gods, Deuteronomy 4:15--19}
\index{idolatry}
\index{false gods}
\index[deu]{04:15--19@4:15--19}

We humans have so many temptations and so many opportunities to worship
things other than God. Upon achieving successes, we can be diverted
from God with praise and admiration from our peers, and we can quickly
forget the source of our success. We begin to worship ourselves and
our abilities. Some begin to believe that there are other answers as to who or what is responsible for success or failure.

I am quite impressed with the natural universe that God has given
all of us. The dimensions of the universe, the magic of motion and radiation, the power of storms, are all fascinating and often
mesmerizing. I can begin to understand how the ancients mistook them
for gods, rather than simply tools of God. 

In today's business world, there are other mistaken entities and false
gods: money, the market, profit margins, market share, market domination,
Forbes 500, and other material measurements of success. These are
man-made concepts and tools---products of the mind of man, for the
purpose of guiding business. They should not become personal treasures
or false gods. As in Hebrews \index[heb]{11:01@11:1}11:1, be ``sure of what we hope for''.

\section[Self Image]{Image, Self, Luke 14:1--11}
\index{self!image}
\index[luk]{14:01--11@14:1--11}

I once heard a mediocre design engineer say that if he didn't ring his own bell, who would. While there are many times during the work week that I somewhat feel like this, there is plenty enough reason not to submit to this tendency.

I was asked by a friend who works in retail sales, a business that pulls out the stops to attract consumers into the stores, what it was like to be in an ethically driven profession like engineering.
He was assuming that in engineering, like medicine and law in previous
years, advertising was not permitted.

I advised him that in the professions, our advertising is like the ethics promoted in Luke 14. Our advertising is to promote others to do it for us, just as people most often act in situations in society. Do not take the place of honor by yourself, but have others encourage and assist you. 

In the professions, even the hierarchy of business structure is often
judged on performance, not who you know. Being of substance
is more important than being on the front page, or the loudest mouth. In the professions, I can share with you many, many experiences
of validation that ``everyone who exalts himself will be humbled, and he who humbles himself will be exalted.''

\section[Israel As An Example Of Hospitality]{Israel As An Example Of Hospitality, 1 Kings 8:41--43}
\index{hospitality!national}
\index[1ki]{08:41--43@8:41--43}

Being a country raised boy who moved to the city for his career,
I often find myself keeping to myself and not extending a welcome
or gesture of hospitality to not only friends, but also Christian
brothers and sisters. The coldness of culture in the city is foreign to me, but I've learned too well to adapt to it. I'm disappointed
in myself.

I can easily see that God wants us to be hospitable. He has directed Israel to welcome those who come to see, to worship, and to experience God in the land.

While we pride ourselves in our religious freedoms and protect the rights of all religions, I believe that
America is God's land. Our government and its constitution are built upon His Word and His Law. Our democracy is inspired by Him. The simply spectacular beauty of our natural landscape is His Making. The blending and sharing of different ethnic groups, the opportunities for personal development, and the cooperative efforts of many organizations
for good are created through God's eyes and thoughts.

As Israel has a responsibility to welcome all to their land, so do we in the United States of America. We are blessed by His Hand. 


\section[Jesus' Commitment To Us]{Jesus' Commitment To Us, John 6:35--40}
\index{Jesus!commitment to us}
\index{commitment!of Jesus}
\index[joh]{06:35--40@6:35--40}

Recently, in a heated and closely competed golf match, my friend and
opponent finished with a spectacular 3 wood shot, from 240 yards to
within inches of an eagle to defeat me with one fell swoop. As he
uncoiled from the swing, he arrogantly looked toward me and said,
``Any questions?''. The golf shot was so dramatic in the match and so emphatic in defeating me, there was nothing that could be argued, debated, or appealed. That was it. It was a finite and emphatic statement and act.

Very few things in our lives are so direct, absolute and complete.

However, Jesus' commitment to us in John 6 \emph{is} one of these statements. Our Savior was a supreme communicator. There is no debate or appeal. If
you believe, you will not hunger or thirst. ``\dots  everyone who
looks to the Son and believes in him shall have eternal life and I will raise him up at the last day.'' \emph{Any questions}?


\section[Judaism, Synopsis of]{Judaism, Synopsis of, Nehemiah 9:6--37}
\index{Judaism!synopsis of}
\index[neh]{09:06--37@9:6--37}

I've been amazed at the bloodshed and misunderstanding between Jews and Christians (along with other religious sects) over the centuries since Jesus walked the earth.

My first employer in engineering was a devout Jew, who was free in his discussion of his religion. He suggested that I read these verses from Nehemiah when I asked what Judaism was all about, since I was raised in a Christian church. My boss, who eventually became one of my dearest friends, was surprised at my lack
of understanding of the differences. 


After reading these verses, I then asked him again for the differences,
noting that I thought it came down to the belief in Jesus' role in
the universe. Being from a ``reformed congregation'', he looked
at me with a perplexed look, and suggested that I needed to live amongst
his neighbors; then I would understand it.

I'm still searching for those differences, and perhaps I'll not ever find them. I don't feel that Christians are different except for their belief and reliance upon Jesus Christ as the Son of God. Our church history shares all, up to about 2000 years ago. Our God is the same God. Our prophecies are the same. Our understanding of God's fulfillment of the prophecies are different.\footnote{Good insight, well put!}

We are still all God's children, attempting to mature well in His
eyes. I do pray that our similarities are noted and promoted, and
that our differences are simply respected. As God gives us choice
of belief and commitment, let us also. Let us all pray for peace.

\section{Judgment}
\index{judgment}

\subsection[Of Others]{Of Others, Romans 2:1--4}
\index{judgment!of others}
\index[rom]{02:01--04@2:1--4}

Time and again in our society and in my business, credit giving, awards
and applause, and ``bell-ringing'' accompany good works. \index{praise}Praise
and \index{reputation}reputation create images that precede and follow people. Often,
we overlook those who simply do their work effectively, but without
much recognition. I have many such folks in my firm. Some of the more
quiet, but effective folks fall below the radar of management.
The self-praising and consistently self bell-ringing folks are usually not scrutinized, even when they are not performing. Why is
it that we pick on the quiet ones?

I've seen the same judgment passing in churches, whereupon the
quiet ones are discriminated against.
It is expected that they do
not have the passion for God and His Son, Jesus Christ. Please do
not fall into this trap. There are many hard working and worthy
servants of Christ in our churches that simply don't take the time
to impress other folks, but they are focusing on God. After all, we
must answer to God, not to men. Let Him be the judge.


\subsection[Spiritual]{Spiritual, Romans 14:1--5}
\index{judgment!spiritual}
\index[rom]{14:01--04@14:1--5}

Several years ago, as I was sharing a lunch with a Christian business brother, I complained to my friend that I was pulling an ``uneven yoke'', in that my wife did not apparently have the same passion for her faith that I did. My friend Ray simply smiled at me and asked me to read Romans 14. He assured me that these verses would help me in my concerns. 

Of course, after reading the verses, I realized how arrogant I had
been. Throughout scriptures, God is showing us the way up and out,
by having us focus on our own sins, not those of others. After all,
very seldom are we successful in changing the ways of others, but
we should be able to do something about our own activities and behavior. 

May I suggest that you keep verse 4\index[rom]{14:04@14:4} close to you, as I now do. The
beginning of the verse is my own personal humbler. My interpretation is that when I become arrogant and start judging my
Christian brothers and sisters, I must be serving a different master.
Self improvement and self focus are critical in spiritual growth.
How can we be a positive influence on others, when we can't control
ourselves?

\subsection[Ultimate]{Ultimate, Romans 2:6--11}
\index{judgment!ultimate}
\index[rom]{02:06--11@2:6--11}

There will be judgment. You can count on it. We will be accountable
for not only our actions, but for the motives behind those actions. 

\index{NDE}\label{NDE_2_46}I've spent several years, on a part time basis, studying ``near death experiences'', called NDE.\footnote{For more on NDE, see also \ref{NDE_2_15} (p.\ \pageref{NDE_2_15}) and \ref{NDE_2_49} (p.\ \pageref{NDE_2_49}).} I've read many different accounts of NDE, all of which were first person and amazingly similar. 

Most start with an out of body experience, looking at themselves and loved ones. They are then rushed through a tunnel and approached a light. For Christian NDE'ers, the Light has been Jesus Christ.

After meeting their Savior, there is a detailed review or ``audit'' of their life. In most instances, they see both good and bad. The review has been described to be more than just a viewing or discussion. The review
seems to be in a form of multi-media that approaches virtual reality.

The review is like re-living the experience(s). It includes feelings, emotions, and maximum sensory information, all in some new and unfamiliar form of communication which they've never before experienced. Some recall proceeding into an area of joy, singing, and wonderful colors. They recall a wonderful sense of peace. Most NDE'ers have then been told that it is not time, that there is more to be done, and that they must go back. They then return---most reluctantly\footnote{I'm assuming that you mean here ``most (of them) returned reluctantly'', rather than ``they returned \emph{most} (i.e., very) reluctantly''.}---to their bodies on this earth. And most return as transformed people.

Some have become so inspired by their experiences, that they've written and spoken about it. They have provided comfort and peace to believers (who knew it all along) and to others who are still grieving over the loss of a loved one. In some cases, their books have become best sellers, and their accounts have generated much excitement. 

If you pursue the subject further, you'll find that many NDE'ers do \emph{not} have a wonderful experience. This group has experienced being thrust into a state of lost control where they are tormented by monstrous creatures. Their experiences, even when merely reading about and listening to them, are frightening.

\begin{quote}
It was almost as if I were in deep water
with man-eating creatures swimming adroitly around me, and feeding
on me. It was the most terrifying experience of my life.\footnote{Do you have the source for this citation? Or is this simply a paraphrase of what you have read in several places? If it's a paraphrase, I need to take it out of this blockquote style.}
\end{quote}

This group of NDE'ers has not written a best seller, and they are not on most public speaking tours. I believe that they should be. 

As I try to understand the non-believer, most seem to ignore the existence of the spiritual world---good and bad. I do wish, for their sake, that they listen to someone, who has had a bad NDE experience. The horrors are real. It is worth taking the time to consider.

Read the words of these verses and believe!

\section[Justice On Earth]{Justice On Earth, Psalm 64}
\index{justice!on earth}
\index{justice|seealso {injustice}}
\index[psa]{064:00@64}

Time and again in my life, the first half of this Psalm describes my feelings toward my adversaries, rebellious cohorts, and sometimes even members of my church. It also seems to me that our leaders in Washington, D.C. feel this way toward the press at times. The schemes and plots of injustice are not just paranoia on my part. 

Recently a Christian brother, who is a business associate, was complaining of the same. In fact, he was at his wits end with the evil doers, so much so that he felt like asking God to bring him ``home''.

``Surely the mind and heart of man are cunning.'' It never seems to cease. It is a plague within our society. It is self-serving, and it is a cancer that could eventually be the destruction of our society and way of life, if not for God. This Psalm could have been written today, or tomorrow. We are to recognize the injustice(s) as simply a part of our trials. We are not to ignore the injustices, but to have hope to overcome them with God's help. 

Psalm 64 does not pull any punches. God is with us. He will deliver
justice. Count on it.

I recognize that there are times when it does not seem this way, but God does not necessarily work within the time parameters that we would like. Throughout our recorded history, you will read of repeated injustice(s) of man upon man. It has happened many times regarding sins in the past and will continue to occur into the future. However, it will not sustain itself forever.

Our investment in justice on earth is not a short term gain; but it is also \emph{not} a futile attempt. It is not simply a flash in the pan. Have heart and hope! Our investment as Christians is long term---eternity. 


\section{Leadership}
\index{leadership|seealso {management}}

\subsection[A Prayer For]{A Prayer For, 2 Chronicles 1:6--12}
\index{prayer!for leadership}
\index{leadership!a prayer for}
\index[2ch]{01:06--12@1:6--12}

When finding ourselves in a leadership role, what should our attitude and goals be, recognizing that we are at the same time servants of God?

I believe that Solomon's prayer, as he approaches his kingship, is a classic example of what a servant's prayer should be. Solomon asks for the proper tools to do his job, while thanking God for the opportunity.
Solomon does not ask for success, wealth, bonuses, glory, news coverage, or other worldly valuables. His desires are to best serve God, not
himself. He lacks greed. His focus is on what he needs to do his job.

Each year I have the opportunity to counsel high school youth who are possibly interested in engineering, or attending the university at which I
studied. From time to time, the youngsters will ask me which career
path will bring the most financial gain.

My response is pretty pat.
I tell them that the career path which excites them for forty years,
every day, will be a good path. If it is a career path with \index{fulfillment}fulfillment from the perspective of making the world a better place, then that
career path might be the most financially rewarding to them. It will
be an inspired career. Most inspired careers accompany performance
and reward.

I also warn them that the world is filled with folks who make a lot of money and don't enjoy their work or their place in life. I know many. At first glance, they have a façade which indicates that they possess contentment and happiness. They don't.

In fact, they are in a state of depression. They spend most of their time protecting their assets and attempting to gather more material wealth. Their existence is centered in further gain. They'll never be satisfied. They have a huge hole in their existence---they have no purpose, no inspiration. Their
lack of excitement and inspired purpose in their lives even affects the happiness of their families. 

Let me suggest that Solomon's prayer is not only for leaders, but
for all of us. It is a prayer that I repeat every morning
of my life. I direct the prayer not only with regard to my business, but to my
family, to my relationship with my wife, to my interaction with my
friends on a daily basis, and to just about all situations, as they
are all in service to God.

I don't ask for winning, but start by giving
thanks to God for the opportunity to compete and perform. I then ask
God for wisdom and knowledge to serve Him properly, this day and forever
more. As blessed as I've been, I can vouch for the effectiveness of
this approach.

\subsection[Character]{Character, 1 Corinthians 13:1--12}
\index{leadership!character}
\index[1co]{13:01--12@13:1--12}

Over the years in business, I've noticed that love is one of the primary
character traits of many long-term, successful, business leaders.
Love, as defined in these verses, promotes dedication and commitment among employees, team members, and staff to the leader and his goals.

Perhaps I should recruit our next managers, directors and leaders with a notice that calls for candidates to have the qualities of love described in these verses. The recruitment document might read ``should be patient, kind, not envying, not self-boasting, not proud, not rude, not self-seeking, not easily angered, not keeping a record of wrongs, not delighting in evil, rejoicing in the truth, always protecting, always trusting, always hoping, always persevering''!

These are the qualities of the people with whom I enjoy partnering,
working, and teaming. In fact, I find most difficulty in attempting
to work with folks in my company and elsewhere that are void of these
qualities.
\index{Bible!characters!Paul}I believe that we have to search for individuals who have
the love that St.~Paul describes. Not only must we search for others,
but we must find it and continue to build it in the man in the
mirror.

Where do you find the love, you say?
Jesus Christ is the
answer! His example and His words in the Gospels will provide excellent
direction for you.

\subsection[Commitment]{Commitment, John 10:1--16}
\index{leadership!commitment}
\index[joh]{10:01--16@10:1--16}

Eventually we can reach the level of spiritual maturity whereupon we recognize that our accomplishments are \emph{not} really ours, but they are the blessings of God.

In business, whether you're a corporate officer, or a sole proprietor, your perspective on the success of the business changes.
The company is no longer just an income maker, or a profit making machine.
It is not simply a money maker.
It takes on a character of inspired mission, well beyond the day to day perspective
of business.

When we reach this level of maturity, we look at our
employees or staff as if they are our ``flock''. We become caretakers
of the business(es) that God has given us; and the employees and team
members are our charge. It is our mission to provide them with challenging
work and with an environment that promotes professional and character growth. As responsible leaders, we are to provide our employees with fair compensation that they might adequately and comfortably raise their families. And we are to set an example of how a Christian business person should live their lives. 

While we'll never live up to the Good Shepherd's performance standard,
He has established a benchmark against which all compare. While I don't
think any of us will compare well with our Savior's standard, or
live up to His example, I do think that just taking on the attitude
of Good Shepherding goes a long way toward being a good leader.

Do not run away when your charges are threatened.
Protect and care for them.
As a leader, you are in a role that will permit you to bear witness for God.
\index{leadership!commitment}The Good Shepherd presents us with a complete role model for leadership. Keep your eyes on the Leader. When you're making leadership decisions, don't hesitate to ask for help, and always be reminded
of what \emph{He} would do.

\subsection[Image]{Image, 1 Peter 3:13--16}
\index{leadership!image}
\index[1pe]{3:13--16}

When you are a leader, you are often watched, studied, and mimicked.
People will talk about you, sometimes write about you, and often tell
their families about you. You are up front and on center stage, whether
you like it or not. What you do, what you wear, what you say, and
other personal features will be scrutinized. How you handle a crisis,
how you carry yourself, how you react in emergencies, how you behave
when you win or when you lose, will be noticed and mimicked. What
you eat, what you read, what you drive, and what you do when you're
not working will be the talk of the office.

The lack of privacy in
these situations often makes me wonder why I want to be a leader.
I often long for privacy and quiet time. Perhaps this constant exposure
is what drives many corporate executives to lock themselves away in
their corporate castle-like headquarters, well away from their people.
I can understand their emotions.

Can it be worse than this, especially for self-consciously Christian people? Yes!

The attention and judgment are magnified and intensified for Christian business people, and especially Christian business leaders. Their appearance, demeanor, and conduct should be becoming of a Christian. It is critical
for Christian business leaders to maintain a focus on what they hope for and on their leader Jesus Christ. 



If you are a Christian business leader at any level and you focus
on Jesus Christ, you'll quickly find a reduction in fear, a clear
path ahead, and purpose in your life. In fact, you will find the people
side of your mission come alive. God will use you to help ``shepherd His flock''. He will bring staff and employees to you to find the reason for the strength and hope that you should continuously be displaying.

I can share with you that there is no greater satisfaction than to
be a valued servant to God, no matter how insignificant your contributions
may seem. Being prepared to serve God is part of our mission. 

\subsection[Personnel]{Personnel, Deuteronomy 17:14--20}
\index{leadership!personnel}
\index[deu]{17:14--20}
\index{personnel}

In these verses from Deuteronomy, we learn of God's prescription for a king of the Israelites. Some of the key characteristics include:
\begin{enumerate}
\item Coming from within the Israelites (not an outsider); 
\item Not being materialistic and being of modest means; 
\item Maintaining a focus on God and His law.
\end{enumerate}

These criteria and characteristics are also good for business applications.
In fact, they are critical for businesses relying upon people, teams, and their initiative. 

For some years now in my own business, we've been setting the stage for our in-house leaders to step forward, take control, and eventually become owners of the business. I am convinced that if we were to bring in the next generation of leaders and owners from the outside, it would significantly demoralize the staff and reduce dramatically the performance of our business. As a compromise, we have added some ``complimentary talent'' to the staff, with the specific target of promoting folks to leadership positions in due time.

Business leaders should lead and motivate people and the performing leaders should be compensated fairly. They should not take advantage of their leadership role to gain material advantage. A materialistic and greedy leader will quickly lose the respect of his staff. 

It's true that shrewdness may lead to short term wins and flashes in the pan. Long term success in leadership requires righteousness, not shrewdness. If you are in a leadership role, or are about to take one, please keep in mind that it is quite possible that you will be leading some
very righteous people, who might be deeply devout brothers or sisters.
They may be well beyond your spiritual maturity level and capable
of helping you.\footnote{What a startlingly humble and wise insight!}  Do not lose their confidence. Maintain your dignity and work at being righteous.

How do you become righteous? Keep your focus on God, read His Word regularly, and pray regularly each day---true and time-tested advice.

\subsection[Responsibility]{Responsibility, Ezekiel 34:7--10}
\index[eze]{34:07--10@34:7--10}
\index{leadership!responsibility}

To paraphrase other scripture, to those whom much has been given,
much more is expected.\footnote{\index[luk]{12:48}Luke 12:48.} As I have experienced so-called success
and advancement in business, I am constantly reminded of our commitments
God. We are not placed in positions of leadership to simply
serve ourselves. God has a plan and a reason for granting and awarding
us with success. It is His success, not our success. We bear responsibility to make good use of His success.

We are responsible for our employees, our servants, and our charge. We are responsible for their understanding and learning. God holds us accountable for His flock. If we are not responsible, He'll take it from us.

What this means to me in the engineering design business are these things.
\begin{enumerate}
\item I am responsible to be the character and spiritual leader of my staff.
\item I am responsible to attempt to provide long term, career advancing \index{employment}employment for all of my staff. 
\item I am responsible for compensating my staff not only at a fair level, but at a level that is responsible.\index{compensation}
\item I am to promote and develop a \index{team!culture}culture within the teamwork frame of my company that permits love and spiritual richness to develop.
\item I am simply the caretaker in this window of time for my business. The business is not really mine, it is God's and I'm simply His
assigned caretaker.
\item I am responsible ``at all times and all places'' to give thanks
to God for the opportunity that He has given me to serve Him. 
\end{enumerate}

\subsection[Work Ethic]{Work Ethic, Proverbs 12:24}
\index{leadership!work ethic}
\index{ethics!work}
\index[pro]{12:24}

I thank God repeatedly for our democracy and economic system in America.
My father was a laborer with a vision and dream for his children.
He drove all of his children to seek higher education.
All of his children are college educated (chemical engineer, architectural engineer, and special education teacher), and his grandchildren are not only repeating the feat, but out-performing the previous generation.
Praise the Lord!

Our family's educational and professional accomplishments can be tracked back to the push and encouragement of my father.
He was relentless in this endeavor.
With God's help, each generation has adopted his values.
He used to say, ``There is always good work for good, hard working people.'' 

\index{leadership}The referenced Proverb still rings true.
The difference between leadership and slave-like labor is the diligence of the effort.
It is not, and should not be a matter of birth right, who you know, race,
religion, sex, and other characteristics, but simply performance.

I have been blessed with business success. My company has been deliberately
organized with a focus on personnel, people. It blends social
programs for the common good---benefits, etc.---with an environment that
promotes good competition and professional growth. Good performance
is fairly rewarded. Performance at the expense of others is not tolerated.
Team performance is rewarded above individual performance. It all
starts, however, with individual hard work and diligence.

I've often heard the term ``God Bless America'', and He has. Sometimes
I wonder if America in its organization and principles doesn't in
fact, embody God and what He would have us do in most instances?
I do pray that we continue to direct and lead our country in His way. 

\section{Life}
\index{life}

\subsection[An Attitude For]{An Attitude For, Psalm 90:5--12}
\index[psa]{090:05--12@90:5--12}
\index{life!attitude for}

Understanding our existence, and maintaining a universal perspective on our lives helps us in stress control, working through the trials of this life, and establishing priorities in our lifetime.
While each of us has a unique life, we share a common existence. It
is short in duration; but a brief experience to mature our souls. By keeping this in mind, our perseverance is strengthened, our resolve is hardened, and our objectives are easier
to achieve. 

\index{NDE}\label{NDE_2_49}I've read research on folks who have experienced a Near Death Experience
(NDE).\footnote{For more on NDE, see also \ref{NDE_2_15} (p.\ \pageref{NDE_2_15}) and \ref{NDE_2_46} (p.\ \pageref{NDE_2_46}).} 
A near death experience is very dramatic. I think that we could easily compare it to a wake-up call. After the near death experience, folks have been described as living a fuller life, valuing their present existence (living for
the present), becoming more spiritual, etc. It is a constant reminder of the frailty and shortness of life.

Do we need to actually have a near death experience to understand
this? Hopefully, we will be wise enough to understand just how fragile,
short and precious our life is.

In addition, as we think of our existence, our work, and our accomplishments,
verses \index[psa]{090:13--17@90:13--17}13--17 provide us with a fresh perspective. The work of our
hands, our accomplishments are in God's control---as they should
be. Our duty is to perform them to the best of our ability. We are
God's servants. We are not to expect rewards, awards, and compensation.
As God's servants, our compensation is His acceptance of us and eternal
life with Him. 

\subsection[Purpose of]{Purpose Of, Romans 8:28--30}
\index{life!purpose of}
\index[rom]{08:28--30@8:28--30}

To be called by God is to be justified and glorified. To me it is
a miracle for any of us to be called to God---to be honored in His presence---to be a descendant of Jesus Christ.

From my early youth, I've heard the question of ``what is the meaning and purpose of life?'' The question comes up in heady, mental explorations.
It also comes up after tragedy, when we stand-back from the rush of our daily lives, and catch a glimpse of our overall existence. 

For those of us who are believers, we have been called to be servants
of God. We are to serve God's purpose, not our own. We are ``predestined
to be conformed to the likeness of His Son'', Jesus Christ. It is
our calling, our mission, and our destiny. We are to be servants in
God's works, for our own good, because we love Him. We will serve
His purpose, whatever it may be---good or bad in our eyes, but always good in His eyes. 

If direction and guidance in this life are needed, we need only look
to the ``first-born'', Jesus Christ. We are following in His footsteps,
and His path. We are one of the brothers. We are one of the descendants
of Jesus Christ, who is our model of goodness and servitude.

\section[Light, Universal]{Light, Universal, Psalm 139:11--12}
\index{light, universal}
\index[psa]{139:11--12}

There are many common human fears: fear of the dark, fear of death, fear of pain, and fear of nightmares are a few. As a youngster, a nightmare
would drive me to leave my bed and join my older brother, asking him
for comfort and protection. I recall the comfort of knowing that
my big brother would not let anything happen to me, since in my mind,
he was older, stronger and capable of defending me from most of my
mental villains.

As a middle aged man, I still have nightmares from time to time, so
vivid that I've been known to flail about my bed. They are so vivid
that when I wake, I fear the darkness of my room and home. In a just
a few short moments, however, my fears subside as I focus on God's
power and His ability to overcome darkness. I am comforted by His
strength and ability to defend me, regardless of the odds. My love
for God, which is great, is far exceeded by God's love for me. God is my Protector, my Shield, and my Source of Courage. 

\section{Listening}
\index{listening}

\subsection[Learning]{Learning, Proverbs 18:2}
\index[pro]{18:02@18:2}
\index{listening!learning}

Listening is at the heart of learning and understanding. In my business,
we conduct many meetings, workshops and deliberations in an effort
to discuss missions, problems, and our procedures in general. The purpose
of the congregation is to share opinions and perspectives.

As a leader of these deliberations, I have to constantly remind myself that I am attending, not to hear my own thoughts, but those of others. I usually start the meetings by injecting provocative thoughts and
even ``misinformation'', just to activate the minds and thoughts
of others. I have to remind myself that once the fire is started, I can back off and listen and learn. 

Some of my heroes of history---Theodore Roosevelt is one such---promoted
the same approach. Remember that the quiet ones are those who are listening
and learning. Verbosity and the ability to talk do not make for intelligence.
The mind has to receive information to process it and create concepts.
If it doesn't have a chance to receive information, because it is
constantly transmitting, then it becomes starved and self-centered. 

\subsection[Value Of]{Value Of, Proverbs 18:13}
\index[pro]{18:13}
\index{listening!value of}

Throughout Proverbs, there are many passages touting the value of
accepting constructive criticism, controlling the tongue, and the
value of wisdom and techniques for learning. This Proverb brings to light
a common misunderstanding that a leader needs to always be the smartest,
the bearer of all wisdom. 

I can recall as a young engineer, when placed in a role of leadership\index{leadership!listening}
in business, I always felt the need to impress my superiors, peers,
and subordinates. Frankly, I lacked confidence in my own abilities.
I often would provide answers to questions before the question was
fully presented, simply to impress folks. However, I often didn't
have the question properly predicted.

I learned to recognize that, in effect, I was indirectly telling the interrogator that their
question was such an easy one that:
\begin{enumerate}
\item I really didn't think much of the value of the question; 
\item I really didn't think much of the person nor the time it was taking
to formulate and answer the question. 
\end{enumerate}
When leaders take the time to calmly listen and calmly discuss answers
with folks, the experience is more than simply the value of the thoughts.
It is a period of counsel. It shows that leadership cares, that leadership
is interested in the concerns and problems of all, and these leaders
might even be basing their behavior on their model---God. Can you
even imagine God or His Son, Jesus Christ, not taking the time to
listen to our questions or requests? As servants, let us follow our model.

\section[Love, Commitments Of]{Love, Commitments Of, Romans 8:28--39}
\index{love!commitments of}
\index[rom]{08:28--39@8:28--39}

\index{Bible!characters!Paul}
In verse 38 of this passage, St.\ Paul states that he is ``convinced''.
I believe that being convinced might just be the key ingredient
to faith. Being convinced of and believing in God's love is to recognize
God's commitment to us and His love for us. When Jesus assures us of peace for following Him, He is assuring us that nothing, anywhere, and forever will intercede between
God and us. 

I believe also that we similarly must develop and nurture our own
love of God. I believe that we are to build a love for God that knows
no boundary, that involves all of our heart and mind and soul. Our
love must be a mutual commitment to God that bonds us for eternity. 

I also believe and have experienced that when we build a greater love
for and focus on God, that love spreads throughout our daily lives.
Our love for our spouses, our children and family, and our love for
our friends grows greater. God's commitment of love to us should give
us a better idea of what a real commitment to love can be. God's commitment and love for us goes to depths beyond our comprehension.
However, it provides us with a model for what our commitments of love
to others on this earth should be. 

\section{Management}
\index{management|seealso {leadership}}

\subsection[Attitude For Success]{Attitude For Success, Luke 22:24--27}
\index{management!attitude for success}
\index[luk]{22:24--27}


Having been the president and owner of an engineering firm that grew
from twelve people to over one hundred people within five years under
my leadership, I often forget that I really was not responsible for
this success. It's pretty easy for me to become ego-centric, straining
my arm to pat my back. Fortunately, God brings me back to earth from
time to time, and I then recognize that I am simply a caretaker of
this forum for which God gave me temporary responsibility. It is
like a window of opportunity through which your character is matured
and reviewed. 

In other words, I really didn't lead the company, nor own the company
with this success and growth. From a spiritual or universal perspective,
I was simply on watch with God's permission. When we take a universal
or cosmic perspective in this fashion, it is difficult to envision
management being anything more than servants. We are servants to our
ideas, to our charge of employees and their ability to perform and
grow, and to God Himself.

Like King Solomon, our management throne is not for us to be served,
but for us to serve. 

\subsection[Charter For]{Charter For, 1 Peter 5:1--4}
\index{management!charter for}
\index[1pe]{5:01--04@5:1--4}

These few words summarize tremendous wisdom for leaders. The provide
direction and an attitude for leaders. As a leader, I continuously
need to be reminded of my mission. These words are that reminder. 

Time and again, I become weary in my job. I often develop a spirit
of exasperation, entrapment, and wonderment at why I continue to do
what I do. There are times when I feel that I've been unjustly called
to lead. I sometimes simply want to be left alone. I believe that
much of my frustration and weariness lies in the stress level
of the business that I'm in, and also in the apparent lack of appreciation
by my staff. 

These words remind me that God wants me to lead, but He is not demanding
it. I will lead as long as God calls. I need not perform for my staff,
but I am accountable to God. I don't have to receive thanks from my
employees, but simply a nod from God. I need not lord over the staff,
but set an example for them to follow. I must drive myself toward
being a good shepherd.

Who's my example and model? As verse 5 tells us, the Chief Shepherd
is my model. As Jesus Christ is the Chief Shepherd, I suggest that
we all keep our eye on our example and model. Following His cue, following
His advice, and sharing each crisis with Him, will serve us all well.
As His servants, may I suggest that we all strive to gain His approval
and His acceptance. Even His nod of approval will be well appreciated
by this simple servant.

\subsection[Criteria for]{Criteria for, Luke 16:10--12}
\index[luk]{16:10--12}
\index{management!criteria for}

Briefly envision yourself having a position of authority and judgment.
It is your responsibility to choose people to be managers in life and in your business. Or more critically, you are to choose managers for God's business. What criteria do you use in such decisions?

As we've learned from other readings, management should serve, should
lead, and should set positive examples. Other criteria might include
integrity and honor. In these verses, Jesus Christ delivers to us
the key ingredients for successful management. Management must be
more than capable. Management must be righteous in character. Honor
and integrity are fundamental. Management is being the shepherd of
the flock, and there is no room for impropriety. 

Let me further suggest that such criteria are not only applicable
to management, but to all servants of God. My reasons are that our
ultimate ``riches ''are eternal life in paradise---God's heaven.
In this paradise, if we are so blessed to attain it, there will be
complete freedom. We will not be limited by many of the forces of
this world---gravity, biological needs, the speed of our automation.
Our freedoms will be true riches that we must earn in this world.
We must earn God's confidence and approval by displaying our trustworthiness
and righteousness.\footnote{These sentences needs to be rethought. It sounds as if you're saying we can earn heaven. You may be thinking of 1 Corinthians 3:10--15?} These character traits are absolute terms, with
no compromise. You either are, or you are not. There are no partial
commitments. God will not accept acting. He monitors the hearts of
us all.

These principles and criteria might appear to be difficult to adopt.
As most of us are tolerant of a ``good try'', we would all like
to think that God will also be tolerant of us. Please keep in mind
that these verses are orders from Jesus Christ. As orders, they
are not to be questioned, simply followed. It is our duty as obedient
servants, to display our trustworthiness, our righteousness in complete
and absolute terms.

\subsection[Procedure]{Procedure, Psalm 78:1--8}
\index{management!procedure}
\index[psa]{078:01--08@78:1--8}
\index{leadership}
In the engineering design business, much of management is teaching and training. In directing even the most typical of projects, leadership spends a great deal of time reminding, prompting, and motivating the
staff. It is a process of boiling concepts down to basic principles
and pushing the principles. Most of these concepts are age old.
Even in our high tech world with the application of new science, there is not much that is new. This is especially true in leadership
and management. 

\index{leadership!learning}At a relatively young age (28), I was put in charge of a 100+ person
design team. My obligation was to lead it through several projects.
While I was not publicly shy of the leadership responsibilities, I
was privately scrambling to learn what I could about leadership. I spent
every waking hour reading contemporary books of business management
and modern leadership. As a new vocabulary evolved from my reading,
my concepts began to all appear as things from of old. They were truths
from the ages, with new wrappings. I also now realize that most of
them were modeled after ``praiseworthy deeds of our Lord''. The
dynamics of management include the constant reinforcement and alignment
of these truths, as we apply them in each and every project. That
is how we train our children, our employees, and our design teams.
The parables of old still apply. 

I am still amazed to read the biographies of leaders, in which the
qualities of leadership and management are overlooked. Such biographies
often focus on the leaders' ability to perform---not manage. Often,
detailed investigation of these leaders discloses that the key to
their success was teaching, reminding, advising, motivating, and empowering
their team. Understanding this, leadership and management can be simply
defined. 

\section{Marriage}

\subsection[Attitudes]{Attitudes, Ephesians 5:21--33}
\index{marriage!attitudes}
\index[eph]{5:21--33}


When I read these recommendations, if not encouragements and commands for couples in marriage, I'm experiencing a feeling of love for my wife, that is well beyond infatuation. It is a mature love that welcomes
responsibility, and long term \index{marriage!commitment}commitment. These words put petty disagreements,
conflicts and the everyday problems that we encounter in our marriage,
in proper perspective. They are certainly something that we can easily
overcome. When I share these words with my wife, there is a sense
of mutual respect, admiration, and love.

Unfortunately, the words and feelings too quickly slip away with time
and exposure to life. Within a short period of time, work, family
activities, problems of friends, and the typical stresses of life
become a big part of our consciousness. This fog of the world hides the true feelings, the love of our relationship. I need to be constantly reminded of these verses. By being reminded
of my love, I am kept from drifting from my commitments. 

Once again I find another reason for which I need to regularly read
my Bible. My reading is not just to advance my spiritual development.
It deepens my love for my wife, it reduces my fears, it and lightens
my load in my life. It helps me focus on my Leader.

\subsection[(Un)Shared Faith]{(Un)Shared Faith, 1 Corinthians 7:12--20}
\index{marriage!sharing faith}
\index[1co]{07:12--20@7:12--20}

I believe that I've heard most of the stories of how couples met,
where they met, and why they eventually married. The stories vary
as much as do the animals and birds of the world. However, in most instances,
the continuing marriages are made of people of different backgrounds,
interests, and character. Sometimes, the couples are also of different
spiritual backgrounds and intensities. 

The individuals making up successful couples often complement one another in their differences. 
\index{Bible!characters!Paul}
St.~Paul recognizes these differences and advises acceptance of the
differences, spiced with the hope of possible conversion. In fact,
St.~Paul is advising one to maintain their position of servitude to
God. It is not necessary to change your profession, change your lifestyle,
or change your job. God calls all of us to serve Him in our original
position in life. This is not to preclude any of us from changing.
\emph{We will change}, surely. We cannot help but change, but we are
to control our exuberance. In addition, we are to respect our spouse's
position. We are not to despair of a spouse's lack of belief, compassion,
or spiritual worship of our God. Be patient and confident. 

As usual, we are to be prepared to explain and communicate the hope
that we have, and to testify for our Savior. Yes, it is not easy to
bear an uneven yoke. Yes, it would be much easier to share life with
a passionate Christian spouse. Yes, it is not always to be expected.
Do not despair, but be patient and ever willing to share your faith.
Hearts break over this, hearts bleed over this, and tears fall over
this. Do not despair!

\section{Maturity}
\index{maturity!suffering}

\subsection[By Sufferings]{By Sufferings, Romans 5:1--5}\label{maturity_by_suffering}
\index[rom]{05:01--05@5:1--5}
\index{Bible!characters!Paul}
In Paul's letter to the Romans, he describes the steps of spiritual
maturity and spiritual development. Spirit, character, and the soul develop and grow through tests, trials and experiences. Their growth
is a maturing process which is described in these verses. The process
appears to simply be suffering when the trials are experienced, but
in fact, the process is one of bettering and developing the
spirit, character, and soul. 

Sufferings produce perseverance. Without perseverance, there can
be no character development in an individual. Without character, there is no hope
in a person. Hope is one of the basic ingredients of the soul, and comes from the love of the Holy Spirit. 

As I write this, I am surrounded by several young graduate engineers,
who desire to make their careers in my profession. My mission is to
train them as quickly and effectively as possible. While I try to
warn and alert them to the sufferings that they will face---almost in an attempt to help them avoid the pain and suffering---I know that
they must experience these sufferings as I did in order to grow.
They must work late into the night, start early in the morning, and
still find that not all of their work is accurate. They will learn
that if they persevere, their work will eventually be accepted and
used in the construction world. If they do not persevere, they will
wash out and leave the design field. There appears to be no other
way.

Is this true in all business fields? I have no difficulty noting that with perseverance, there must be
character and with character there must be hope, even in the business
field. Those with hope, those with dreams and visions, and those in
roles of teaching and leadership have character. They have a soul.
It is almost visible, certainly unhidden to me, a believer. I can
also tell very quickly whether the soul is filled with the Spirit.
If it is filled with the Spirit, it is an everlasting hope, a loving
hope, and a compassionate hope, that appears to be capable of lasting
for an eternity.

I pray that all believers develop
to this level, such that our light shines.

\subsection[Forgiving, Strength of]{Forgiving, Strength of, Proverbs 19:11}
\index{maturity!forgiving, strength of} \index{forgiveness} \index[pro]{19:11}

It is a strength to forgive. Forgiveness reaps glory in the eyes of God. My oh my---how we can forget this truth!

In our world of business, people, organizations and procedures are
less than perfect. From time to time, people will make mistakes, organizations
will overlook contributors, and procedures will offend someone. How
strong are we as individuals to forgive and forget? How much wisdom
do we have as individuals to recognize the imperfection of the world? 

As a simple rule, look to our Master for guidance. How much has He
forgiven us, and how much has He forgiven mankind?\footnote{MER note: is this a good place for ``and forgive us our debts, \emph{as we also have forgiven our debtors} \dots (Matthew 6:12)?} As we fill our
existence with His Spirit, I pray that we will find the wisdom to
forgive.

\subsection[Trials]{Trials, Proverbs 17:3}
\index{maturity!trials} \index{trials} \index[pro]{17:03@17:3}

\begin{center}
\emph{Should this section be coupled together with the ``By Sufferings'' sub-section above?}
\end{center}

Just as each metal has an optimum procedure for hardening and curing, so do each of our souls. Each of us have experienced and will experience trials to aid in forming and developing our character and soul, so that it might be worthy of entering the Kingdom of God.\footnote{MER note: one cannot be made worthy of God's Kingdom by our efforts, as you know. I think you might mean to say something like what you say in subsection \ref{maturity_by_suffering} above, especially by referencing Romans 5:4?}

In athletics, practice is used to develop the individual and the team.
Repetitive exercises aid in this. Scrimmages and test meets are used
to hone the skills. Comes the game day, those that have practiced hard
and well generally are best prepared for the contest. As most athletes
will tell you, game day is the reward in athletics. 

In business, we have parallel paths to development. It often includes
strenuous education (engineering school, in my case). The academic
education is followed by years of engineer-in-training work,
or as other careers call it, internship. The internships of value are very structured, exposing the newcomer to the trials of
business on a gradual basis. Ultimately, business performers arrive at their destination, some with certificates, others simply with satisfaction.
With God's help, they reach their game day, their reward. 

In life, trials are God's preparation of our character and soul---overall
the heart. How do we strengthen it, hone it, and prepare for
our game day? Do not despair. Do not cringe or turn from God's preparation
of our hearts. Look the trials right in the eye with a confident smile. Go right at the trials for what they are. Make the most of
the trials, as they develop us. Do not fade away, but note that God
is with us every step of the way. Give joy for the trials, for they
will make us ready.

\section[Hope as a Medicine]{Hope as a Medicine, Proverbs 17:22}
\index{hope} \index[pro]{17:22}

I've thought about this principle from both within and outside of one's own body.

It is fact. Cheerful hearts do promote good feelings, both in others
and in the body of the cheerful heart. Yes, there is a strong link
between mental health and hope relative to physical well being. Love and kindness do heal many illnesses. The spiritual enhancement that they
provide, develop and nourish hope. Hope, of course provides perseverance
and steadfastness. These are the things that medical schools discuss
when teaching bedside manner.

In contrast, a crushed spirit despairs and despair leads to depression. Mental illness can lead to physical illness; lack of hope to loss of healing desire. Without hope, without
cheerful hearts, our bodies cannot perform to their fullest. Without
our bodies working to the best of their varied abilities, we are not
performing our mission to the fullest extent. We are then less than
complete servants.

Let me also suggest that not only will our bodies suffer without a
positive, cheerful and upbeat attitude, our businesses, family, and
other concerns will also suffer. May I suggest that we all take a regular look in the mirror and pick up the beat, so that
we might fulfill our purpose.

\section[Nation, God's Perspective]{Nation, God's Perspective, Exodus 19:3--6}
\index{nation, God's perspective} \index[exo]{19:03--06@19:3--6}

\index{Bible!characters!Moses}
You've probably heard these verses and words many times before. However,
I am going to ask you to step back for a second and ponder the verses
once more. I'd like you to think about God's direction to Moses as
if it were a guiding order to you, or even to us as a nation
of people. To me the term \emph{nation} refers to an entity that God has given to his collection or choice of people. 

Some time ago, when I had a life-threatening malady, my wife asked
me to identify any values or principles that she might share with
our infants, should I check out of this world (as I nearly did).
This question evolved into years of contemplation, study and the recording
of values that we had identified. We contemplated how we might best
share these values with our children, noting that simply teaching
and preaching doesn't always work.

In addition to our regular regimen of sharing values, we decided to spend focused vacation and travel
time with our kids to share the values. From our own life experiences,
Natalie and I decided that these travel times were experiences in
which our words and influenced experiences might better stick with our children. Through this program of value sharing we have traveled
the world, but more intensely, the USA. There are not too many parts
of the USA that we've not visited with our children.

I must share with you, that God truly has shed His light on us. It is a spectacular country, with wonderful loving folks and a varied terrain that is not much matched in the world. It is land that is blessed.

But is it a nation in the eyes of God?

Because of my exposure to people in business in the northeastern US, I can say that I pray that someday we will become a nation in His eyes. At the time of this writing, our nation is plagued with ruthless,
evil folks, who seem to almost be intent on destroying our nation.
It is a time of trial for all of us. However, that is why I believe
that God has brought us to this place and point. We are His servants.
We are His soldiers. We are His shepherds. We can not simply turn
our heads to this evil. We must take a stand and rise up to the challenge
in the name of our God. 

I pray that with the rising of day to day Christians to face the threat, even in unity through revival, we may become ``a
kingdom of priests and a holy nation''. 

\section[Nature, Learning From]{Nature, Learning From, Proverbs 30:24--28}
\index{nature!learning from} \index[pro]{30:24--28}

I was blessed by God to grow up in the mountains as a man of nature.
I'm quite familiar with the basics of life, and what living might mean.

I find myself a bit clumsy in the midst of society. Having grown up in the wilds, my values are basic. They evolve around what in life is necessary to live and exist.

Let me list them as Proverbs
would teach them.
\begin{enumerate}
\item Prepare for a rainy day and save for the necessities of life (food, shelter, etc.).

\item Secure for yourself and your family a home, a shelter, a gathering place for family and friends.

\item Recognize that many folks working as a team, are much more effective than the sum of the individuals---synergy!

\item On this earth, we are experiencing trials to mature. It is not heaven. There is the perspective of our reality, our frailty, and our mortality
with which we all must live.
\end{enumerate}

These are the basics of our existence on earth.

Society attempts to cover up the facts. The cover up includes worldly matters
and issues. The worldly matters and issues seem to cloud the real issues of life.

However, let me suggest that all Christian brothers and sisters
attempt to keep these real issues on the screen, front and center,
as the real priority at all times. The real issues which are the basics
of life are a major part of the fundamentals of our earthly visit. 

\section[Neighborly Graciousness]{Neighborly Graciousness, Deuteronomy 23:24--25}
\index{neighborliness} \index{graciousness} \index[deu]{23:24--25}

The wisdom of the ages within the Bible never ceases to amaze me. Of course, why should it? It \emph{is} God's word.

I've been blessed in my life to live in several different cultures,
all within the USA. I grew up in a rural, mountain area. My children often say that I was a hillbilly. Folks treated each other with a
significant amount of trust, respect, and often even a sense of excitement.
Perhaps the excitement came from the solitude that the mountainous
country brought. I further spent time in dairy farming country, in
Texas with the cowboys, in a northeastern metropolitan area, and in suburbia.

In each of these areas, the warmth with which you extended a welcome varied. While it might have varied, it always included an extension of your wealth, or a sense of sharing. It might be interpreted as your graciousness. 

However big or small, it is important to extend one's self in neighborliness, whether by comfort,
food, or by some other measurable gift. It is just as important for
the stranger, visitor, or guest to accept this grace. In most of the USA, it is expected that you will accept it. 

These verses share with us that in Israel's early years it was also proper to offer and to accept graciousness. Similar to today, we are
warned not to be piggish in our acceptance of graciousness. 

Let the warning be heeded, because I believe that this warning extends beyond social protocol,
into good business practice. Perhaps you've heard Wall Street analysts say, ``bears prosper and pigs die!'' Similarly, you've probably
heard sages suggest that ``you shouldn't put all your eggs in one basket''. Often this approach is driven by greed, or the desire for
excess profits. 

Just as it is socially acceptable and appropriate to sample, graciously the offerings of our friends, I believe that in business it is healthy to be open to diversity---both in marketplace and geography. We should promote a diverse business reliance on alternative markets that complement one another, just as alternative geographical
areas balance one another.

In building your business upon these principles
of diversity, your foundation will be broader and stronger. The business
will be able to withstand downturns in some of the marketplaces, in
some areas of the country, without dramatically affecting the whole
of the company. As a business philosophy, it doesn't promote get rich quick concepts, but it promotes stability and long term growth.

I've read many business journals and books that promote specialization or niche marketing. While being an expert or filling a niche gets our businesses
started, they should be merely stepping stones. In my opinion, diversity
of business services should be a top priority after start-up businesses reach a point of business stability and momentum.

\section[Ownership and Business]{Ownership and Business, Leviticus 25:23--24}
\index{ownership}
\index[lev]{25:23--24}

While these were and are God's direction to the Israelites in their
handling of land, it bears great meaning to my perception of our worldly
things. We are but tenants and aliens in this world.
All belongs to God, not to us. 

I can recall the near terror of borrowing money against the value
of my home equity in order to buy my engineering business. When I had paid
the bank for the loan, there was a sense of relief. I'm embarrassed
to say that there was also a sense of pride and possession. I felt
that I had earned the money; I had created the business; and I actually
believed that I owned the business for eternity.

It took several years
to understand and finally realize that I was and am merely a landlord
at most. One of my partners refers to our ownership as our watch. We are the third generation of owners and if we do it right, we'll
just be one of many watches that will be kept over it. Hopefully the business will continue
to provide professional growth, financial compensation for the raising
of families, and a forum for spiritual growth. 

When and if we can recognize the real perspective of land, business, and the ownership of things, then we can begin to place values on
our actions. We must act as responsible tenants, landlords, or watchers. It is important to strive to be good tenants, landlords, or watchers.
We are to do a good job, not squandering the resources, while gaining our just compensation from working of the land and business.

Upon redemption of the land, business or things, we should be able
to clearly illustrate the diligence of our care of the land or business,
our improvements to them, and most importantly their readiness to be
further developed. 

Responsible care of land, businesses and things is part of our mission. Will we be able to submit our land, businesses and things to the Owner without reservation when it is time? Will the Owner say
to us, ``Well done, good and faithful servant''?


\section[Paranoia of Powerful and Despair]{The Paranoia of the Powerful and Our Despair, Psalm 56}
\index[psa]{056:000@56} \index{paranoia of the powerful} \index{despair} 

These words have been very meaningful to me. At times, I've thought that I could have been the one to write them to describe my predicaments and situations. I've not been alone among man. Many of my Christian brothers and sisters in business have felt this way.

Is this paranoia? Perhaps not so much!

In business, people skills are critical to success. When combined with Christian values, the mixture results in extra sensitivity to the feelings and schemes of people. We are placed under a microscope, our actions are scrutinized and slandered at every turn, and conspiracies
against our goals and dreams, even informal, grow in strength and action.

These are the times when our focus on God is critical. Our
focus on God brings perseverance, patience and strength. We will overcome
it all with God's help. We will have the strength to work through
it. 

These are trials to strengthen our faith, to promote our spiritual
growth, and to mature us for our ultimate mission---to live with
God forever. Do not despair!

\section{Patience}
\index{patience}

\subsection[Self-Control]{Self-Control, Proverbs 12:16}
\index[pro]{12:16} \index{patience!self-control}

I really enjoy hearing and telling a good joke.
Like most folks, I don't always do a good job of story telling.
I often use poor word selection, have a lack of emotion or diction, or pause too long to properly deliver the punch line.
I believe that the same is true in all of my verbal communications and the written word.
Many times, words come out that are not intended and other times emotions come out that provoke words that were not intended.
The results of these slips are cruelty, insults, rumor passing, and emotional injury.
They are the sparks that start emotional fires.

Perhaps you have had similar experiences. Certainly, as we recall
these experiences in our lives, it almost seems that the emotional
eruptions that evolved from such words almost seemed to be looking
for a way out our minds and bodies.
It is almost as if this emotionally volatile situation was an explosion waiting to happen.
In retrospect, we can realize that patience and calmness would have eliminated the reactionary explosion of emotion.
And the embarrassment that follows such an explosion of emotions would not have occurred had we been strong and calm.

Is it lack of self control? Is it excessive focus on ourselves---self-centeredness?
Is it a lack of courage and strength? Is it a lack of belief in God's
direction, protection and involvement in the world? Is it evil? Is
it worldly focus? Is it caring more about what people think than what
God thinks? Or, can it be a combination of any of these wanderings?

In my case, it most often is some combination of these. As I grow spiritually, as I spend more time reading God's Bible, as I pray more,
and as I have more fellowship with my brothers and sisters, my patience and understanding grows. My fears lessen and I feel stronger.
I have come to realize that people with spiritual maturity are not
weak, but in fact they are powerfully strong. This is the grit and
resolve of the early Christians who faced the Romans' lions. We as
a world society need to create an appreciation for this strength.
The more this strength exists, the better chance we'll have in the
future to avoid conflict and war.

\subsection[With People]{With People, Ecclesiastes 7:21--22}
\index[ecc]{07:21--22@7:21--22} \index{patience!with people}

Throughout the Bible, there are warnings to control the tongue,
to listen to every word that is spoken. As Christ has
taught us, the tongue speaks what the heart has to offer. Unfortunately, most of our hearts in this world are laden with stress, sadness, and evil. The spoken word often is slanted, emotionally warped, and filled with vengeance. 

In business, employees often talk with disdain and dislike for their
management, regardless of how righteous they might be. Similarly,
management often talks with the same disdain for employees. Neither
can live without the other. A company would not exist with only management or with only employees. 

I think that it is important for all of us to remember that this is
the world. It is not heaven. While we strive to improve the world---as we should continuously---it will remain the world. We must learn to be more patient and understanding. 

Christians and others of spiritual maturity must also recognize
that even our mature brothers and sisters have their ups and downs
emotionally. Our patience and courage can be cyclical. The events
of the world, the battles with evil, and the maturing trials that
we all go through will weigh heavy on our hearts from time
to time.

A friend, who is a full time missionary for Christ, tells
me that he reaches emotional valleys where he just doesn't care. These valleys occur nearly once a month. The valleys drive him to solitude in the wilderness, where through prayer, God revitalizes him. However, he still has the valleys. We all do, and therefore, we should recognize that the valleys in others, will result in words spoken that should be ignored.

Learn to ignore words laced with and sourced from evil. Learn to absorb them, redirect them, and heal the speaker. Through God, we will have this strength.

\section[Wisdom in Personnel Management]{Personnel Management, Wisdom, Proverbs 19:19}
\index{management!personnel} \index{management!wisdom} \index{wisdom} \index[pro]{19:19}

Please go back and read verse 19 again. Now read verses 18 and 19 together. Does the perspective change?

Hopefully you feel the same level of responsibility in educating, training, and proper rearing, whether of children, or of your employees. This verse applies to child raising and to professional development.

The key to letting them pay the penalty is recognizing and determining
what the penalty will be. We can neither let them overly hurt themselves, nor their comrades, nor the business of the company. 

Time and again, I find myself recognizing a fault in a staff member,
bringing it to their attention and finding resistance in repenting.
I then remember the wisdom of my first employer who preached that
all people have a difficult time in accepting constructive criticism
to change the way they do their work. We need to let them pay the
price if it is not too steep, and then they will learn.

I've come to believe that good management does permit a developing staff to reasonably pay the price to learn. It is quite the natural thing to do. It is the substance of our world, and one of the main purposes for this life. We all must mature through these trials.


\section[Physical Fitness, Responsibility for]{Physical Fitness, Responsibility For, 1 Corinthians 6:19--20}
\index{physical fitness}
\index[1co]{06:19--20@6:19--20}

Our bodies are not our own. They are owned by Another. We are simply
renters and we are responsible for care and maintenance. Do not abuse the rental property!

The bodies that have been given to us vary from one to another. Some of us have weak, frail and limited-functioning bodies. Others have wonderfully performing bodies. Some have bodies that are quite beautiful aesthetically. Others, perhaps less so.

Regardless, let us all recognize that our obligation is to take care of our body. God has given us this rental property for our use in this world as we mature our souls. Let it carry the soul through the trials that are prepared for us. In some cases, recognize that the body is quite a part of the trial itself---especially in the case of painful illnesses. 

As the body carries us through this world, it should be in the best
condition possible. It should not hinder us in the missions that God
has planned for us. Take care of the body, so that we can all get
to the work at hand.

\section{Planning}


\subsection[Input]{Input, Proverbs 15:22}
\index{planning!input} \index[pro]{15:22}

Making informed decisions often requires making informed plans. 

My goals upon graduation from college included having my own business, being my own boss, and not having to listen to others. Of course, I was na\"{\i}ve, self-centered, and greedy.

I maintained these goals through my first work assignment which lasted approximately five years. The company's founder had built an organization that focused on him, served him, and frankly, was bottlenecked in performance by him. Being egocentric, I thought that the organization was perfect and I simply wanted to be the King of the Organization. This work experience reinforced my original dreams and goals. 

During my second assignment, which lasted nearly ten years, I worked for a larger firm that relied upon teamwork. Teamwork was promoted to assure that the whole was greater than the sum of the individuals.
This approach made all individuals dispensable and increased productivity in a synergistic fashion. Stress was reduced. Vacations were uninterrupted. The flow of work and productivity never seemed to cease. Just as importantly, the planning was superior. It was superior because it included a broad level of participation by many folks from many backgrounds.

With this broad spectrum of perspectives---experienced, or youthful and energetic, optimistic or wary---the outcome of the planning always seemed far superior.

Planning, when it includes many folks' input, takes on the look of
modeling. When many folks add input, the planning is mentally
predicting the process and the outcome, just as if you were modeling
a prototype. This is sound business, sound family planning, and sound
personal practice. The participants also learn and grow from the experience. 

There is one final perspective worth mentioning. In an organization
of many people, it is assumed that new ideas and changes to existing
procedures must be implemented by many people, those making up the team, the task force, the department, or whatever form of organization is used.

New ideas or changes in procedures require the support and discipline of the people who will be assigned to implement the changes. The implementers of the orders or decision to change must make it happen. If key representation by the group intended to implement is part of
the initial decision making process, the chances for success are much greater.

The fact that the implementing group had input means that they also have a stake in ownership of the ideas. When little bumps, obstacles, or difficulties are encountered on the implementation path, an informed implementer will navigate around or through the obstacles. 

\subsection[Long Term]{Long Term, Proverbs 19:21}
\index{planning!long term} \index[pro]{19:21}

Please also refer to \index[jam]{4:13--14}James, chapter 4, verses 13 and 14 for reinforcement of this perspective.

It is not that these verses are against planning,
because planning is most often critical to human success. In several
instances, our Savior advised us to plan our lives and actions.

However, these verses are alerting us to the fact that our planning must include God and His plans for our lives. Comprehensive planning on our part should include evaluating our options, our desires, and our plans, relative to God's plans, God's goals, and God's purpose. 

Why, you ask? Because God's plans, God's goals, and God's purpose will prevail. 

I encourage you to regularly confirm your plans with God's will. Confirmation can be made by reviewing your plans as compared to His will as recorded in the Bible. Another method of confirmation includes talking directly with God---by direct request in prayer. I recommend that for effectiveness, since we all have problems in hearing and seeing, that you spend time both reading the Bible and in prayer. Include in your prayers a request for God's supreme involvement and guidance in your planning.

May your plans serve Him!


\subsection[Methodology]{Methodology, Proverbs 20:18}
\index{planning!methodology} \index[pro]{20:18} 

Making informed decisions often requires research to gain the proper
information. Gaining the proper information in war is the difference
between victory and defeat, between life and death. Gaining the proper
information in business is the difference between success and failure,
between profit and loss, and between solvency and bankruptcy. 

Several years ago our engineering business got the bright
idea of opening another office in a different geographical location
for the specific purpose of building geographical diversity. We recognized that most of the time, the economic ups and downs were regional. We reasoned that if we picked another complementary region for another office, we would neutralize the impact of a local economy drop in business through sharing workload across regions.

The logic seemed good and the sense of adventure of operating in a different location was stimulating. One of my partners suggested that we should discuss such a proposed operation with future, potential clientèle. In effect, we were conducting a survey in order to collect advice on our campaign. The guidance and advice that was given through the survey was accurate, the office was opened and it immediately blossomed as a business.

Business planning and its methodology is very similar to war making.
It has many parallels in marketing services and products to new geographical areas, in dramatically changing the way products are manufactured, and in the discipline and organization of the people required in both business and war. Gaining advice and guidance by collecting information
is a never ending activity in business. It is not a passive activity
that should be left to its own devices. Information access and feedback
are pro-active systems that need to be engineered into all business
organizations. 


\subsection[Regulatory Compliance]{Regulatory Compliance, Nehemiah 2:6--9}
\index{planning!regulatory compliance}
\index[neh]{02:06--09@2:6--9}

Time and again I've heard non-believers talk of
the antiquity of the Bible. They are constantly trying to convince
people that our society has advanced beyond the teachings of the Bible,
that the teachings of the Bible are no longer appropriate for today's
life style.

I can't understand their perspective. For me, the Bible explodes with meaning when I read it. My daily reading of the Bible provides me with something for the day and task at hand.

Perhaps one of the two of us is ignorant. Is it possible that I don't
really have a grasp on modern living? Or is it that the non-believers
have no knowledge of the Bible? I find the Bible to be loaded with
present day values and principles.

The Book of Nehemiah is a wonderful lesson on project management, if we listen and recognize it. % ``Though we hear!''

Just this past week I was called in panic by one of my clients who
was pursuing a major construction project, unfortunately with another
engineer. The client had hired a local engineer (in the Boston area)
for the project. While the local engineer had outlined a schedule
and plan that included the typical construction permits, the engineer
had overlooked all of the politically based regulatory approvals.

\index{Bible!characters!Nehemiah}When apprised of the situation, I couldn't help but say to myself, ``Nehemiah wouldn't have made this mistake. Nehemiah was thorough in management and planning.''

In fact, Nehemiah had his regulatory approvals in place before
he took action---travel, material acquisition, and construction.
In fact, his approvals were in place before he made his intentions
public.

As a result, his activity could not be stopped through legal means. As a general rule, do not make anything public, unless it is properly organized and prepared for presentation, unless it is the proper timing in the sequence of approval events, and unless the way has been paved by proper behind-the-scenes authorizations.

In this Boston case history, not all approvals were obtained in the proper sequence and political opponents were blocking progress.

\index{Bible!characters!Nehemiah}Yes, I know that Nehemiah was God inspired. Nevertheless, we can still learn from Nehemiah about his project management skills. As you read more of Nehemiah, please note his dedication and servitude to God. The relationship is a model for all of us.

If you care to read more of his project management brilliance, continue with Nehemiah. He will continue to impress you with his wisdom. 

\subsection[Strategy vs.\ Tactics]{Strategy vs.\ Tactics, Proverbs 16:9}
\index{planning!strategy vs.\ tactics} \index[pro]{16:09@16:9}


At first glance, it would appear that this Proverb indicates that a man sets his own strategic course, and then God worries about the tactical details. However, further contemplation of the words and the spirit of the Proverb relay an additional message. 

Some years ago, I attended a management course in negotiating. I was
pleased with the course and teacher, because the teacher was promoting
that in good negotiations, the final negotiated settlement should
be perceived by all to be acceptable, fair and good---a win-win situation. The teacher, who had decades of involvement in complex negotiations,
also promoted that successful negotiations involved preparation. The
preparation should include an analysis of the alternative positions
of the parties, strengths and weaknesses, and then the development
of a strategy---not tactics.

The differences between strategies and tactics must always be understood and recognized. They are smaller components of the overall strategy. As smaller components of the overall strategy, they are actions which support the overall strategy. As we develop and conduct tactics, we must be ever mindful of the strategy and how we are fulfilling the goals of the strategy. The strategy
is the heart of our program. 

We are given freedom of choice in our lives to determine our course. Once we choose our course, set our plans, and establish our strategy, then we must expect God to direct our steps, assist us with the tactics, and present to us the ``adventure'' (as I call
it) of this life and the path that we're taking.

The more I grow spiritually, the more I become aware of my partnership
and affiliation with God. I rely on God's help and direction in everything at this stage of spiritual development. He outlines the strategy, the tactics, and He captains my ship through the rough waters of this world.

Why do I think this way, you might ask?

Well, for one reason, I have asked Him to lead me, guide me, and protect me. I consistently ask Him for courage, wisdom and strength. This was a conscious decision.

Now, after a few years of ``permitting'' God to lead my life, I have a perplexing question. I believe that God had a greater part in leading me before I requested His guidance. Is it possible that He was there all along and I just didn't recognize Him? 

\section[Prayers of Leadership]{Prayers of Leadership, 2 Chronicles 1:6--12}
\index{leadership!prayers of} \index[2ch]{01:06--12@1:6--12}

As managers of people and teams, prayers to God for guidance, direction, and assistance are important. Please recognize that the great prayers and requests by leaders in history were not for victory, not for wealth and success, and not for personal gain.

The great leaders prayed for principle-based outcomes. The prayers were for courage, wisdom, health, and happiness. They were not prayers to avoid the trials, but for strength to perform well in the trials. This is the sophisticated, intelligent, and universal perspective.

May I suggest that time be given to planning and writing prayers.
Spend time planning for your next prayers with God. As you plan the
prayers, please think of principled outcomes for the events facing
you, and be sure that you in fact, want those outcomes, and are sincerely
endorsing your words to God. ``If it is in your heart, then you will
surely want it.'' 

\section[Priorities, Survival\slash{}Opportunities]{Priorities, Survival\slash{}Opportunities, Deuteronomy 11:13--25}

These words quickly remind me of our simple priorities in surviving
in this world. Why is it that we wander from God? Why is it that we
become absorbed by the things of this world? Why is it that we forget
the purpose of our existence?

Man, of which I am one, is a strange creature, with a substantial
attention deficit disorder, especially when it comes to remembering
the commands of God. 

In studying these verses, God does not promise the latest in technology,
the latest in travel experiences, nor success in the music business.
These are the thoughts of man and the world. What God promises, if
we uphold His law, is a natural world that can bring forth food and
nourishment, if we work the land. He'll provide us with everything
but our hard work. God promises us the basics of survival. 

I am most reminded of these verses on a beautiful day in the outdoors.
I do recall once when I was playing golf on such a day, I was complaining
bitterly about the quality of my game to one of my colleagues, who
happened to be a Christian brother. Dave turned to me and sadly reminded
me that I should fall to my knees and thank God for the beauty of
the day, and certainly stop complaining about the golf game. I was
taken aback by his chastisement, but Dave was right. I was caught
up in the things of this world. We need to continue to remind ourselves
of the basics.

\section{Professional Development}
\index{professional development}

\subsection[Attitude]{Attitude, Luke 8:4--18}
\index{professional development!attitude}
\index[luk]{08:04--18@8:4--18}


Professional development and \index{professional development!spiritual}spiritual development have a lot in common.
Both require persistence, vigilance and a never ending quest for more.
I believe that important to both (from an individual perspective) is that we must be driven by our spirit and heart. 

Professional or spiritual development should never be driven by fame,
fortune, or the other man-made things of this world. In fact,
the things of the world will divert our attention from our further
development and understanding. They destroy the foundation for what
should be driving us and making us ultimately the consummate professionals.

My son and his friends used to hang out at our place and often we discussed what each of the youngsters' careers might be. I was always asking them to identify what it is that makes them happy and interests them, both in and out of school. I continued to share with them the advice
that my father gave me, which is to pursue a career path that makes
me excited and happy. It should be a career that is fun on a rainy
Monday morning, inside and out. It should be a career in which you never stop learning and actually enjoy the learning for its sake alone. 

I believe that my father could have said that we should each find a career for our heart. With that attitude, we will be successful, because we are focusing on our career for all of the proper reasons.

It is the same for our spiritual development. It is driven by
the heart and we are reminded that the heart and the spirit make up the driving force in both, as Christ's parable states.

\subsection[Prerequisite]{Prerequisite, Luke 8:15}
\index[luk]{08:15@8:15}
\index{professional development!prerequisite}

As discussed above, the heart and righteousness of a person, is the basis for their spiritual development. So is it with professional development.

Each year we recruit nearly a dozen youngsters from colleges in the area, either as fresh graduates, interns for the summer, or for co-op commitments. Our Senior VP of Engineering is a wonderful teacher and spends a great deal of time working on the virtues and values of the
young engineers. As he tells me often, ``I can teach them anything of a technical nature, but I can't give them the heart and resolve to become the best engineer possible''. So, in the screening of candidates, such elements as ``heart'', ``nobility'', and ``character'' are the prerequisites. These qualities are just as important as the quality of their degree, the accrediting of their college, and the quality of their education. 

I don't believe that engineering is the only field in which heart,
nobility, and character are important. Is it not true of any position
in life, of any path in life, and of any responsibilities in life?
It seems to me that parenthood that is successful might have these prerequisites.

In addition to my Bible reading, I read fiction, biographies, and business writings---nearly non-stop. At times, my wife accuses me of being a ``readaholic''. In my business and biography readings, I've never read that a successful individual reached his or her level of success because he or she wanted to make more money, or experience more power and control, or collect more toys and things.
The source of most success is the \index{fulfillment}fulfillment of heart yearnings.
Heart and soul fulfillment is not only the key to professional development, but also to happiness in general. What is important to me is that
my heart yearnings are the yearnings that are acceptable to God.

\section[Promise of God]{Promise of God, Psalm 89:14--37}
\index[psa]{089:14--37@89:14--37}
\index{promises!of God}

I am presently riding in the window seat of an airliner, flying homeward
over the Rockies, watching the sun disappear, and delighting in the
early evening moon and stars. I am reminded of the spectacular universe
that God has created for us. I am reminded of my friend Chuck's repeated
prayer during our Thursday morning Bible study. Chuck often offers
thanks directly to God for His consistency, His resolve to maintain His promises and commandments, and His never ceasing presence. When Chuck offers this prayer, my mind often drifts to ponder the immensity of God's thinking, the obviously different perspective of God's reasoning, and how different we might be from God. 

Just the perspective of time is so different between God and us. God
is used to existing with a perspective of eternity, while we live
with such short term goals and thinking. God's workings often are
woven through many years, perhaps centuries, while we plan in days
and weeks.

As we all know, life was not quite a bowl of cherries for the Israelites. But as we know, God kept His promise. Jesus Christ was
and is the fulfillment of that promise. 

\section{Reality}
\index{reality}

\subsection[Eternal Perspective]{Eternal Perspective, John 6:61--63}
\index[joh]{06:61--63@6:61--63}
\index{reality!eternal perspective}

Many a day hunting in the mountains, my father would make a comment
like, ``now this is reality!''. Reality to some is a physical or
dimensional thing. Reality is something that can be touched, broken,
and certainly sensed by sight, sound, touch or smell. As one of my
agnostic neighbors offers, ``if it doesn't make footsteps, I don't believe in it''. He never answers me when I ask him of such things
as radiation, or how much does an electron weigh?

From a universal and eternal perspective, these words in John are
telling me about reality from God's perspective. The things
of the spirit make up this reality, not the things of the flesh. I
believe that all of us will be able to better understand this fact
when we give up our bodies and pass to the ``next dimension''. Until
that time, I believe that it is important to understand that the spiritual
world is the real world. It is the driving force in nearly all things
in the physical world, the world of physical dimension and forces
and flesh. The spiritual forces, values, and passions are what motivate
us and make up our souls.

I can understand if this is a difficult concept to grasp; however,
understanding and appreciating the spiritual world will benefit you
immensely. I pray that you hear these words of mine, re-read and analyze
the words of Christ in John, and study the concept until it is understood.
For myself and others, understanding ``reality'' will open the gates
to further spiritual maturity.

\subsection[Practice\slash{}Belief]{Practice\slash{}Belief, Colossians 2:16-19}
\index{reality!practice and belief}
\index[col]{2:16--19}


\index{Bible!characters!Paul}
St.~Paul's words to the Colossians define reality for us. In
this case, it is the reality of our belief. It is the reality of the spiritual and the foolishness of the physical world. It is
the reality of the church, the body of Christ. The reality is Jesus Christ.

We are reminded that being a real Christian is not just doing good
acts, not just being a good person, and not just being a giver to
the poor. While those criteria are often used in describing the characteristics
of a Christian, the Christian is Christ-centered---that is, oriented toward Christ. That is reality. That is the fact. 

In some of the churches of the world, the leaders seem to be wandering
from basic doctrine. Some believe that the doctrine is outdated, that
the Bible is no longer applicable, and that there needs to be a new
sensitivity to people and their deep-seated needs. It seems
to me that these folks are in a state of spiritual confusion and
their hearts are wandering. They are intent on creating new spiritual
practices and procedures, some even set to new music and a new
mindset. I've been told by some that they are searching for the fabric
and force of the universe. Wow!

The answer is simple. Focus on Jesus Christ and acknowledge Him as
the Son of God, the Prince of Peace, and the Ruler of the Universe.
It is as simple as that!

\subsection[Spiritual]{Spiritual, Colossians 2:20--22}
\index{reality!spiritual}
\index[col]{2:20--22}

\index{Bible!characters!Paul}
As we read above, St.~Paul certainly had a strong grasp of reality.
He was expert in describing and defining the differences between the
physical world and the real world; i.e., the spiritual world. 

The spiritual world is filled with many forces as strong as, if not
stronger than those of the physical world. I share this perspective
as not only a mere mortal, but as a mortal that is limited in perspective
and vision. I am new to the spiritual world, but I have respect for
its strength and absolute belief in it. 

There has been much written about the healing power of attitude,
hope, and positive thinking. Our favorite movies are often
based upon the lives of people with loving hearts. Our heroes lived
their lives, and their actions flowed from their character, their
beliefs, and their passion---and yes, from their souls and hearts.
These are the characteristics of the real world, the spiritual world.
It is what motivates us and directs our lives, and makes the world
good or bad. It is the stuff that poets write of, that operas are
performed for, and that libraries are filled with. 

\subsection[Worldly Love, Galatians]{Worldly Love, Galatians 5:5--7}
\index[gal]{5:05--07@5:5--7}
\index{reality!worldly love!galatians}

As we explore the world and define reality, a further definition of reality might be this: reality is that which Christ values. It might be the best test of what is reality that there is. We should simply ask, ``is it something that
Christ values?''. If it is, it is real. If it isn't something Christ values, it is not real.
As believers, our faith is reality, and our faith is important to Jesus Christ.

For example, \index{Bible!characters!Paul} as St.~Paul points out, it is our faith expressing itself through love that counts. This is reality. This is what is valued by Christ Jesus. We are reminded that all else is unimportant. What is
real is our faith, as it is valued by Christ. Our faith must be placed
on the highest level of priority in our lives. Our faith cannot be
ignored as something that can be deferred to a later date, postponed
until we have time, or left until later in our life when we think
that we might need God. 

My day to day life is loaded with stress and demands. It is very easy for me to get wrapped around the axle of my job and its activities. From time to time, I forget what is really important. The the mental and physical demands of our worldly lives, as heightened by improved communications, further promote the diversion from what is important.

When I'm lucky enough to be in the office, and not flying somewhere in a rush, I am usually faced with the following: 
\begin{itemize}
\item 4--6 scheduled appointments,
\item 4--8 unscheduled appointments with my staff needing advice,
\item 6--8 voice mail call backs,
\item 35--55 emails, and,
\item 25 pieces of mail, of which 1/3 is usually fax copies for immediate action.

\end{itemize}
This is an average day, and this is what I call the reactive part of the day. In other words, I must react to these communications
and meetings, in order to meet the demands of others. However, this reactive
part of my day is the least important part of the day. My mission is to be proactive as a manager and leader. 

It is my mission to predict issues before they happen, create visions for marketing, sales and production, and create goals and dreams for all of the teams and staff members. This part of my mission as the ``head coach'' of our team requires priority and it's how I need to spend much of my time. It is a priority.

With this little glimpse of my day, I believe that you'll understand why I need a break or reality check about once per hour. I do this by prayer, just to remind me what is real, and what is important. My faith is what is important, not only to me, but to Jesus Christ. By praying hourly, I can remind myself that my faith and my spiritual knowledge and development must be part of my job. Why? Because my Savior values it. 

\begin{center}
\emph{As I have just written these words, I finally understand them, and I thank God for the understanding.}
\end{center}


\subsection[Worldly Love, Ephesians]{Worldly Love, Ephesians 2:1--10}
\index[eph]{2:01--10@2:1--10}
\index{reality!worldly love!ephesians}

These verses give me further breakdown of the definition of reality.
Reality has a ``static'' condition and a ``dynamic'' condition.
The ``static'' condition might include defining reality through
the senses of sight, sound, touch, smell, and the dimensions that
man has created for each to define the senses. The ``dynamic'' condition
of reality, might be defined by the power of God in action in the
world, the actions in the world that result from faith in God, and
other such spiritual forces at work. In other words, reality is doing
the good works that God has prepared for us, because of our belief
and love, and because of God's grace and love.

Reality is not only a static dimension, but action in this world.
It is never ending, requiring perseverance, and energy. However, it
is driven (as we engineers define potential energy) by the spiritual
forces. 

May I present you with a challenge, that will hopefully further define
the spiritual world as the real world. Look back over your day, or
yesterday, and identify a special problem or challenge, whether it
was work or family. Identify the forces at work in this problem or
challenge. Most of the forces and the foundation for the circumstances
are based upon spiritual and relational developments. It seldom is
simply the physical things of the world. Our world is a world of unseen
forces and things. It is a spiritual world.

\section{Rejection}
\index{rejection}

\subsection[Hometown]{Hometown, Luke 4:24}
\index{rejection!hometown}
\index[luk]{04:24@4:24}

You've probably heard of the definition of an expert. ``It is generally a classy looking person, with an attaché case, who has traveled at least 50 miles.'' If not from such a distance, the value is perceived as less.

Yes, it is the truth for all of us. I believe that this rejection applies to all. 

In the engineering design business, our clientele are expected to
pay valuable fees for our engineering expertise. The perception of
value is critical. I recall in more than one instance losing assignments
to engineering firms of less quality charging higher fees from
distant cities. The selection of such ``distant experts'' was justifiable
in the minds of the agents. I was once told ``why, they're from as far off as (some distant place), they must be special''.

While we may find human perception comical, we have to respect it
in business and also in teaching. It is really quite human for all
of us to carry these perceptions. Our brothers and sisters may be
telling us things that never seem to sink in, until we hear it from
a ``distant source''.

Recently, at work, I became quite upset when a bright, young engineer resigned. In his resignation to me, he profusely lied about his intentions and why he was leaving. As I learned more of his lies during the next few weeks, my rage grew. This young man's immediate manager, whom I consider to be a great Christian brother, started to tell me about my rage and my need to turn it around. ``Who was he to tell me?'', I thought.

A few days later, I read Romans 12\index[rom]{12:00@12} again. I got the point immediately, and offered forgiveness to the youngster. I immediately wrote the ``young liar'' my forgiveness and wished him well. As I reported my change of spirit to my Christian brother (who had been telling me the same as Romans 12 for weeks), I found myself immediately embarrassed. I had not accepted his advice immediately, because it wasn't perceived as expert enough for me.

I believe that in a world in which perception is sometimes more important than reality, we must be sensitive to these real truths. In
our endeavors, whatever they are, we must be vigilant of perceptions, postures, and points of perspective. When traveling, we'll always have
a stronger listener. We should not let the local rejection frustrate our efforts, but be cognizant of the tendencies, as Christ was. 

\subsection[Respect]{Respect, John 13:20}

\index{rejection!respect}
\index[joh]{13:20}

What is the cause of rejection? Is rejection motivated by the lack of respect for the individual that is rejected? Or, does it go beyond the individual to represent a lack of respect for all of the beliefs
and relationships of the individual?

I find these words in the book of John to be quite comforting, while at the same time serving as a reminder that we are not alone. From the personal perspective, the words remind me that my own rejection is not a lonely
situation, if I am at one with God. My rejection then becomes
a joint rejection of me and my Master---my God, His Son, and the
Holy Ghost. I am simply the front for the rejection. 

Quite often, rejection that brings loneliness and desperation is
personal rejection. This rejection is self-centered in focus and is often devastating to folks. The anxiety and self-pity furthers
the self-oriented focus of those rejected. Becoming God-focused is to promote selflessness. With selflessness and a focus on
God, there is no such thing as personal rejection. Our rejection by
man and the world is to be expected, but we are never alone. Just
as the world has a difficult time respecting God and his righteousness, we will honorably bear some of His rejection. 

We are truly part of the team when we focus on the other teammates and the mission---not on ourselves and our needs. The same holds true in business and in business success. With focus on the team and the mission of the team, not on personal gain and self-esteem, we all find greater comfort and success potential amongst a synergistic group. While I am very confident of my abilities (some would call me cocky), I am very comforted knowing that I am part of a team at work and in the universe. Knowing that my teammates are on my side and pushing for my mission to succeed, working behind the scenes to make \emph{our} success happen, is very comforting.

If my team and I are rejected, it will be a rejection that I will bear with honor and a quiet confident smile, as it furthers my spiritual maturity.

\subsection[Spouse]{Spouse, Ephesians 5:22--33}
\index{rejection!spouse}
\index[eph]{5:22--33}

As I write these thoughts, I am sitting in a lonely hotel room just outside of Limerick, Ireland. I've been here for more than a few days and I really do miss my family. Before I opened my laptop to read and write, I had called my wife, Natalie, just to hear her voice and tell her how much I missed her. When I left several days ago I had simply hollered out, ``see ya later'' and I had walked out without as much as a hand squeeze or a kiss.

I frankly hadn't valued our relationship that much and in some respects I was rejecting her for my work. I was ignoring the importance of our relationship, with my attention drawn to my travels, my work, and other worldly things. 

After a half hour of telephone conversation with Natalie, I was embarrassed
before God by my attitude earlier in the week, disappointed in my
behavior and my lack of love for my family, and apologetic to God
for my sins. I then pulled out my Bible, and read the verses of Ephesians.
Wow! We can really wander from the path so easily. These verses are
powerful to me and to my marriage. 

I must spend more time reading them and remembering Ephesians 5.
These verses will help me as I am one of those fellows that needs
to work at my marriage and my family relationships---they don't just
happen. These verses remind me of how I should be treating my wife. When I read these verses I am reminded again that I've not
taken the time to appreciate her love for me.

\section[Rest and Relief from Excess Business]{Rest and Relief from Excess Business, Proverbs 24:30--34}
\index{rest}
\index[pro]{24:30--34}

Like most things in life, excesses are to be avoided. Moderation in
all cases is more acceptable. Excess work requires dilution with a
touch of rest and mental diversion. The dilution should not be ``too
much'', as the Proverb warns. Similarly, excess work will result
in damage also.

Often I've mentioned to friends that I could handle retirement now,
years before most would retire. Usually, I mention something like
this out of the need for some well deserved rest or diversion. However,
I'm not sure if I'm suggesting a complete retirement, even though I'm really
looking for a change in what I'm doing. Personally, I don't have the
patience to sit around too long. I've found that when I vacation in the islands, just a few days are sufficient to resuscitate my curiosity and energy. By the end of one week, I'm raring to go (as they say), to return to my work.

I've heard reports that our present society is experiencing wealth and spare time at its highest level in history. I believe that we need to plan and pace not only our work, but our rest. Speaking from my own experience, proper blends of each will build better health and mental fitness.

\section[Scheduling Management]{Scheduling Management, Nehemiah 2:6}
\index{scheduling}
\index{management}
\index[neh]{02:06@2:6}

``On time and on budget!'' How many times have we heard this phrase in America as one of the goals of project management. ``On time delivery'' is also a term used to describe the tightness of production, distribution and inventory control management. Delivering a service or a product requires \emph{committing to a schedule}. Many folks talk about it, but few actually commit and then deliver. From my experience, very few people can really commit to a schedule and then deliver their product or service in accordance with their schedule. However, I've observed those folks that do deliver on their committed schedule to be of a high level of integrity and resolve.

I can't tell you how many times I've been involved in project meeting
discussions, when at the end of those meetings I'd hear the project
manager finally say those fatal words, ``let's set some dates when
we'll have this all done''. \index{Bible!characters!Nehemiah}Nehemiah, in my opinion, is one of history's great business and project managers. He was asked by the king to commit to a schedule and he did. He also delivered. I would expect that in his dealings with his lieutenants that he also asked them to set dates and make commitments.

Top managers demand that dates be set and that commitments to levels
of performance be established. If it can't be measured, it can't
be managed. Set dates to begin, set dates to monitor interim progress
against a benchmark, and set dates to finish. Dates to finish should
include dates just prior to finishing, in order to review and assure that the
finishing date is being kept. Project work should not be a matter
of holding your breath and waiting for surprise results. Project
work should be deliberate, well thought out, and flexible in plan
and schedule to absorb unplanned hurdles. How is it that God's Word
knows these things?

\section[Security of Information]{Security of Information, Proverbs 20:19}
\index{security!information}
\index[pro]{20:19}

Many pieces of information in business should not be made public.
It is not because the information is that secret, but because I've
seen information, particularly business information of different companies,
manipulated in an evil fashion. I'd like to suggest that all
information and procedures should be approached as if they were public,
as if they were posted on the company bulletin board, almost as if the angels and the saints and the remainder of the universe knows of the information. Certainly, God is looking over our shoulder, so that the information should be borne of righteousness.

However, that is not to say that the information should always be
made public. It should not. Such data as these should be kept confidential:

\begin{itemize}
\item compensation or pay levels (because greed could create confusion, not logic),

\item  vacation and benefit levels (again, greed could step in), and,

\item changes to organization prior to adequate presentation (if a package is not wrapped properly, the impact of its presentation is often lost).
\end{itemize}

Since there is a need for confidentiality and security of information,
there are fundamental security procedures that should be followed.
This verse from Proverbs warns us of at least one principle. In order
to protect information, only share it with those who need to know. In addition, identify those folks who have a tendency to further ``share'' the information. Because of their past habits, either by
your suspicion of such, or by a reason that is identifiable, you should
not share confidential information with them. Do not blame them if
there are leaks. Blame yourself. You should not have shared it with them.
It is like posting the information on the bulletin board.

I am surrounded with many good people in my personal and business life.
I am always surprised by their ignorance of the importance of confidentiality,
and I am further amazed at their lack of understanding of the timing
of delivering information. Most folks around me think out loud. They simply continue to share information, as if they are running
out of thoughts. They tell it all. Furthermore, they have
neither appreciation nor thought for the importance of timing in the delivery
of information. Rather than hold information until the timing is perfect
for reception by others, or well received by others, they simply get
it off their chest as if it is burning a hole in their pocket.

Most folks do not value or acknowledge the need to orchestrate communications.
Life is like a theater production or poem. It has timing, rhythm,
and plot. Communication needs timing and organization for proper reception.
There is a time and place for everything. 

Please recognize those around you and realize that confidentiality,
proper timing of delivery of information, and ultimate secrecy is
in your hands, not others. It is out when it is in their hands.

May I suggest that you listen to the wisdom of the ages. Keeping secrets
and keeping information confidential until the appropriate time is
our responsibility. Confidentiality starts with us.

\section[Self-Help, Self-Analysis]{Self-Help, Self-Analysis, Romans 2:1--11}
\index[rom]{02:01--11@2:1--11}
\index{self-help}
\index{self-analysis}

Perhaps there is no such thing as self-help. Perhaps we should redefine self-help to be defined this way: ``with God's help, we might improve ourselves''.
And, at least in my case, there is much to be done.

As we all know, to make the world a better place, we can most effectively
start by improving ourselves. Yes, there is always plenty of evil,
plenty of poor examples of mankind, and yes, even plenty of folks
that appear to be worse than we are. However, as we are warned, we
should start with ourselves, not with others. With God's help, we
can begin maturing our souls, and with His kindness, love and righteousness,
we will start on the road to improvement.

On the trail to \index{self-improvement}self-improvement, these verses share with us some
steps that will provide us with some self-analysis techniques.
As we all know, it is very easy for us to see the misdeeds and wrong doings of others, but we often overlook our own faults, sins and shortcomings.
Take the opportunity to recognize that what bothers us in others may well be part of our act, also.

So, simply list those irritations
and sins in others that bother you and then ask yourself if it doesn't really exist in your heart or your actions, too. If you're
honest with yourself, you'll find the same shortcomings in your heart and in your actions driven by that heart. Identifying the problem and the source of the problem will go a long way toward resolution.

If you find yourself without fault, may I suggest that you take things
a step further and ask one of your greatest critics to identify your
faults. Most often you'll receive a perspective beyond your ability
to sense. Do not let your emotions cloud the vision. The perception
of others most often is reality. While we don't want to admit it,
most of us live a lie or two in many parts of our lives. These lies
are very hard to face. When we do face them, it is often very painful
and disappointing. 

Business management, when combined with a righteous heart, will experience
this pain. We all must be diligent at developing our management skills,
our image, and most importantly, our heart. The road to improvement starts with identifying the impediments in our lives, what stimulates
us, and ultimately our shortcomings.

\section[Sharing Generosity]{Sharing Generosity, Luke 6:38}
\index{sharing}
\index{generosity}
\index[luk]{06:38@6:38}

We are not born generous. We must be taught to share, from the time
we begin to interact with other people as youngsters. Generosity and
sharing are in fact tests of our character, our righteousness, and
our soul. Possessiveness is typical in most people. We all hope for
excess wealth, so that we might be smug in our generosity,
if not successful in the eyes of others. 

Have you thought lately about sharing what little you have? Have
you thought about sharing in your business with your staff, with your
community, and with your church? Have you asked yourself about what
you might be doing with that precious little time of yours, to contribute
time to the needy, so that they might have hope and the world might become a better place for all? Have you prayed to God, asking Him
to direct you in your generosity, to lead you in His chosen direction
for your generosity?

It does get scary. If you're like me, I'm
protective of the family and business piggy bank, and even more protective
of my time, which never seems to be plentiful. It is very difficult
for me to take the step of expanded generosity. It is a test
and a very hard test. Nevertheless, we must take the test and I pray
that God gives us the courage to pass the it.

There are many verses like this in the Bible. They are verses that
warn us of the temporary value of our possessions and that we should
not be possessive of them. We should be recognizing how the possessions
might be put to better use by giving them to the very needy, or to
those who can make better use of them. This verse is also a testimony
of the wonderfulness of God and His ability to create perfection in
the world. If you were to create a test of the heart, how would
you do it? God has told us in this verse that we will be judged on
our own standards and our own actions. His wisdom will always continue
to amaze me.

Let me share with you a way to make this test easier. May I suggest
that you spend some time with God, admitting and reviewing your selfishness,
your possessiveness? Then ask God how He wants you to make best use
of these gifts that He has given you. Don't fight the battle alone.
The battle is so much easier with Him.

\section[Social Conformance]{Social Conformance, Proverbs 16:7}
\index{social conformance}
\index[pro]{16:07@16:7}

Being a new Christian, I can clearly remember some of my anxieties
about truly living my faith. One of my primary fears was that I would
not be socially accepted. In fact, it was worse than that. I was expecting
that I would be rejected by society, my friends and family, because
I would be different. This fear was reinforced by my own rejection
for those who would attempt to show me the way to God. I resisted
their efforts disdainfully and hoped not to see them again. I am
truly embarrassed to share this with you, but I was such a shallow
fellow, with very little courage, with very little understanding,
and quite insecure in my existence. I was unsure of the purpose of
my existence.

I was so weak and I was so wrong. There was really nothing to fear.
While I can tell you that I have not become an outcast of society since becoming a Christian, I believe that there are better examples
in the world. Take a look at some of your friends who have accepted
Christ and attempt to evaluate their relationship with the world.
Yes, they probably have begun to reject the things of the world that do not conform to their new values, but I find that they are
more loving, more understanding, more compassionate, and even more
in harmony with other people. It is very difficult to be a Christian
and to be a social outcast. 

We are not alone in our endeavors. We are different people when we
finally submit our lives to God. We become very bearable people, loving
people, and generally respectful people. We are people of hope, joy,
excitement, and discretion. How can folks of our social order not
accept us, unless they have an evil agenda? If so, then perhaps they
need us even more than we supposed. They may be the lost souls that
we've been sent to contact. In \index[1pe]{3:15}1 Peter 3:15, we are reminded to share
the word with gentleness and respect.

Gentleness and respect are social values that are accepted in all of the societies of which I know. 

\section[Spiritual Perspective]{Spiritual Perspective, 1 Corinthians 2:10--16}
\index{perspective!spiritual}
\index[1co]{02:10--16@2:10--16}

As an engineer, I have been trained and educated to look at problems
within the world using our technology---often physics, mechanics, and thermodynamics. These technologies are sciences that provide me
with a platform for observing and analyzing the problem or challenge.
They provide me with units, they provide me with dimensions, and they
provide me with repeatable, predictable relationships. If I do not work within these parameters, I will fail to reach a conclusion
or a solution. 

In a very similar dimension, the spiritual perspective must be followed
in order to understand the teachings of God. Our beliefs are based
upon a spiritual perspective, our hope for the future is based upon
a spiritual perspective, and our growth and maturity of soul is based
upon a spiritual perspective.

This is true for others in the world, too.
I believe that we have to recognize this, and if we are to serve God
in our lives by sharing the Word with others, then we must focus
on reaching the spiritual heart of an individual. If our associates
do not have a spiritual perspective, chances are pretty good that
they'll not receive our message. As these verses state, they'll probably
consider the words as foolishness. 

I've been asked by ``brothers'' when to share the Word. My best answer is ``when God wants you to do so''. I do, however, believe that we each have to focus on our own hearts and feel the
Spirit of God within us. With this focus, I can feel spiritual intensities in my relationships with friends and business associates
that seem to indicate that they are truly listening. Of course, listening
and accepting are two different things. However, as a lowly servant,
I can't do it myself, and I am just one of many in the ranks. I will
not give up. I will continue to probe for the spiritual openings,
the spiritual soft spots, and the spiritual opportunities.

\section[A Formula for Stability in Business]{A Formula for Stability in Business, Proverbs 12:7}
\index[pro]{12:07@12:7}
\index{stability in business!a formula for}

A few years back, my engineering design business appeared to be ``heading
south''. Bankruptcy seemed imminent. My expenses were growing; my
workload was falling; and no matter how I tried to correct our course,
it seemed to worsen. I remember receiving a call from a Christian
brother, who asked me to lunch. I exposed my heart to him. 

Since my friend had been through hard times in his own business, I
was hoping that he'd give me some financial advice, a tip to improve
the sales of my firm's services, or something of the wisdom of the world. Instead,
my brother in Christ asked me about what I was doing in this trial
to mature, to use this opportunity to witness, and to ultimately glorify
God. My first reaction was disbelief, if not disappointment, because
of what appeared to be a lack of caring. 

I sat and listened for well over an hour, as my friend shared with
me his hard times of the past and how he now realized how important
it was to use these opportunities to serve. As I continued to listen,
I realized that he was correct. I was shaken to the bone by my selfish
approach to the problem. I had actually forgotten the purpose of my
mission in this world. I had also forgotten why my business existed.
Just a few years earlier, I had realized that God had helped me build
this business as a platform for many of His activities, of which I
was simply a small part. Some of His activities included:

\begin{enumerate}
\item introducing his Word to many of my employees,

\item providing a means of income for the betterment of the employees' families,

\item providing a stage for promoting Christian business practices, and,

\item providing a development ground for many of His soldiers. 
\end{enumerate}


I spent several days pondering my posture and building a plan. I
gathered my staff leaders together, and told them that I was convinced
that we would, as a team, pull through this and that we were being
tested to see how good we really were. I was convinced that we would grow and learn from the experiences. And we did on all accounts.

I called my friend several weeks later, after overcoming the business
problems with teamwork, and thanked him for his words. He shared with
me that he did not recall what he had said. He did tell me that he
had felt like a conduit and those words weren't really his, but that
of Another.

\section[Spiritual Strength]{Spiritual Strength, Luke 6:46--49}
\index{strength!spiritual}
\index[luk]{06:46--49@6:46--49}

As a degreed Architectural Engineer with many years of experience,
I can share with you that a building will not stand long with a weak
foundation. The foundation is the building's point of contact with
the ground. It not only supports the weight of the building, its occupants
and contents, but it also provides lateral resistance to movement.
Wind and water and earthquake cannot hurt a building structure that has an adequate foundation. It will withstand much and persevere.

We are told that the strength of our spirit is built upon the foundation
of practice. To simply hear the Word, and say ``yes, I believe'' is not enough. We must become immersed in the spirit of God, and practice,
worship, and witness. How we put God's spirit to work in our life
and the degree to which we do it is proportional to the strength
of our spiritual foundation.

\section{Stress Management}
\index{stress management}

\subsection[Stress Levels]{Stress Levels, 2 Peter 1:5--9}
\index[2pe]{1:05--09@1:5--9}
\index{stress management!stress levels}

I've repeatedly advised my team managers to be aware of their stress
levels, and of the stress level of others around them, especially
near the end of each week, and especially when they are in need of
a well-deserved vacation. Our work and the natural pressures of family life can consume us. I often use the term ``being wrapped around the
axle'', which means to be consumed or wrapped up in the turmoil of
things. When we are wrapped around the axle, the experience becomes so dizzying that we certainly can not see the solutions, or the plan. We can be so immersed in the stress that we cannot even
see the level and intensity of our stress. 

These verses from \index{Bible!characters!Peter}St.\ Peter remind me of the steps of spiritual growth,
and the events that direct us toward managing stress. I need to redirect
my focus away from thetu business of the world, and focus on my own
self development in the things and mysteries of the universe. I need
to focus on my own spiritual growth. I need to move to the next level
of my development or maturity. And most importantly, I need to do
all of this through my life as it exists. My life does have a spiritual
perspective to it. I need to find it, and blossom from it.

My fear is that I don't take the time to remind myself of such things.
When I don't remind myself, my existence wanders further from the
truth. As I drift further from the truth, my stress level grows. My
stress is best managed by not drifting too far from reality, spiritual
reality that is.

\subsection[Remembering Our Purpose]{Remembering Our Purpose, 1 Peter 5:1--4}
\index[1pe]{5:01--04@5:1--4}
\index{stress management!remembering our purpose}

As managers of people, St.\ Peter's words identify the important elements
of our position. He shows us where the focus of our work should be.
Let's face it, when managers do focus on St.\ Peter's values, those
things which cause stress seem to be insignificant. 

These verses are personally very meaningful to me. I had bought my business and helped it grow just before I had grown to know and trust Christ with my life.
I was consistently referring to my business as mine.
I also received all challenges to my business as a
challenge to me. I was overly possessive and overly protective. I
actually thought that I had earned this success myself. I had thought that I was owed these rewards and this position in life.
I received the compliments of our success and nearly broke my arm patting myself on the back. 

These verses have been instrumental in changing my life. They have
had a profound impact on me. I remember reading them at breakfast
several years ago. For the obvious reasons, I found myself staring
out the window and then crying. My emotions were of happiness that
God had finally exposed me to one of the mysteries of my universe,
and that God had been so patient with me, and not left me while I
was such a sinner. My son asked if I was OK, and I told him yes, that
I was finally in great shape.

I believe that managers should be very thankful that God has such
faith in them to put them in that position. They should see their
position (or assignment) as an opportunity to serve God. And as the
business might become successful, accept it as simply a blessing,
and possibly a smile from God. It is a sign to keep up the good work.

\subsection[Source of Rest]{Source of Rest, Matthew 11:28--30}
\index[mat]{11:28--30}
\index{stress management!source of rest}

As I write these words, I have been surrounded by pressure, too much
work for one man, with some very critical personnel decisions pending.
I was sitting here worrying about my business problems, preparing
to read my Bible, when the telephone rang. The call was one of my
engineers telephoning to share with me that the wife of one of his
teammates was successfully recovering from her lung transplantation
operation. The operation was her only medical chance for survival.
At this writing, she's still critical, but stable. I was asked by my
engineer to pray for her. I did.

As I picked up my Bible, and turned to these verses of Matthew, I
started laughing at myself for becoming stressed out. Why is
it that we (especially I) forget about what is important in life?
Why is it that I have to be figuratively slapped to remember
where my focus and leadership must be? As a leader, I am to be calm
and collected. I am to be the model of serenity, while all around
me may be stressed.

I am weary and I am burdened. Where do I turn? Yes, the answer is
simple, and the answer is always the same---Jesus Christ. His commitment\index{commitment!of Jesus}
to us is to comfort and calm us in our stress and turmoil. He can
be trusted. He will deliver. He is gentle and humble in heart, and
He is in charge. He can do more than we can. I do pray that I do not
wander far from God, and that I learn to keep my eyes on my Leader,
my Master and my Savior. 

\subsection[First Things First]{First Things First, Matthew 6:28--33}
\index[mat]{06:28--33@6:28--33}
\index{stress management!faith}

What do these words say to you? To me, they clearly state that stress
is inversely proportional to faith. The more faith, the less stress
exists. The less faith, the more stress eats at us. To me, it is that
simple.

Verse 33 is the cornerstone of Christ's message. This verse establishes
our priorities for our life, and for stress management. First and
foremost, our efforts must be directed toward our spiritual life,
its growth and maturity. How do we do it? Verse 33 has the answer.

So often, my stress has been brought on by worldly things. I worry about finances, business workload, my children's success in school
or on the playing fields, or simply accomplishing all of the many
worldly tasks that usually overwhelm me. The stress drives me to emotional
peaks that cloud my thinking, force me to react overly quickly to
many things, and I often start feeling sorry for myself---almost
not really caring about my mission. These are my symptoms. These are
the alarms that remind me to refocus my life. ``But seek first \dots''

\section[Teaching Methodology]{Teaching Methodology, Proverbs 21:11}
\index{teaching methodology}
\index[pro]{21:11}

Teaching is more than simply sharing wisdom. Teaching includes the methods of delivery of the information, and the alternative techniques for explaining and displaying the wisdom. Most often the teaching of complex subjects involves a gradual procession of logical demonstrations. 

Teaching also involves tailoring your methods of delivery to the listeners, students, or audience. There should not be one standard method of teaching for any communicator. Communicators and teachers must be adaptable.

In my business, we are often making presentations and being interviewed for engineering assignments. Often, the client's interviewers are not familiar with our firm, its organization, or even engineering design procedures. Our interviews and presentations are therefore educational opportunities. Once we determine what it is that we need to share or impart, then we need to determine how we can best communicate it and at what level of detail the message will be best appreciated and received. The supporting visuals, graphic support, dress, appearance, the tone of delivery---forceful or soft-spoken---and the emotion with which we present are many of the options and alternatives.

The delivery methodology is critical to the teaching. Sometimes the delivery technique may be the key to success. Sometimes it may be more important to teaching than the wisdom itself. While the wise man may quickly recognize wisdom, others may not. In fact, in our world most people do not.

Yes, sometimes the mocker has to be punished to permit the teaching of wisdom to the simple. Sometimes our actions are important to set the stage for the delivery of wisdom. Our actions are most often noticed as part of our delivery, together with the words. The methodology of our teaching becomes an integrated collection of actions, props and the wisdom itself.

\section{Time}
\index{time}
\subsection[A Perspective Of]{A Perspective Of, Psalm 103:15--16}
\index{time!perspective of}
\index[psa]{103:015--016@103:15--16}

These words are timely in my life. My son and I are spending a rainy weekend in the mountains at our family hunting camp. The mission for the weekend is to prepare the camp for our annual family deer hunt and to attempt to recover our meadow from the forest. In nature, animal and vegetable growth take over what might have been man's domain, and return it to the wild. When nature is complete, signs of man's existence are gone.

As we entered the camp yesterday, we found the cabin inhabited by flying squirrels, who left their chestnuts throughout the furniture and beds. The floor was covered with the remains of their feasts.
All paper products were destroyed or used for nesting. Without our
clean-up each year, the camp would simply become a part of the forest
within a short period of time, without a trace of our involvement.
We have to insert our energy into keeping the camp usable and livable.

In engineering terms, we refer to this organizational energy for man
to structure the world as entropy.\footnote{One of the graphics on the front cover of the book could be the formula for entropy, which is (I think) $\textrm{d}S = \frac{\delta Q}T$, a formula I got from Wikipedia representing a steady state entropy in a closed system?} It is the energy involved in the
creation of the sculpture of Mt.\ Rushmore, the Empire State Building,
the Aswan High Dam, and other man-made things. However, entropy must
be applied repeatedly over time, to avoid nature's neutralization---nature's tendency toward chaos.
We must attend to maintenance. 

However, even with attendance to maintenance and keeping after things, our lives are short from the perspective of the universe.
Our existence is for a brief moment. The physical things of our lives
and even the impact that we make in the world can be viewed as insignificant---\emph{except
for the spiritual dimension}. Except for a man's spiritual development
and his spiritual contribution, a man's life is really insignificant.

Think about the millions of people who have toiled, built, worked, and even have been martyred over the past two thousand years.
Most of them will remain nameless and have been forgotten. Then, in the
same fashion, think of Jesus Christ, His disciples, and some of the
early leaders of His church. Their spiritual contributions continue.
Their teachings, their examples, and their wisdom remain with us. 

I am convinced that my only chance for making a difference is in my
own spiritual development. I will not become the President of the
United States; I will not become the greatest engineer ever;
but I can develop my soul further, to prepare it for an eternal existence.
I do believe that spiritual development, whether for one's self or for
others, is a lasting effort. The impact has eternal perspectives.
Spiritual development of one's soul is not subject to the frailties
of the world.

\subsection[Future Perspective]{Future Perspective, Matthew 6:25--34}
\index{time!future perspective}
\index[mat]{06:25--34@6:25--34}

Again, this multi-perspectival set of verses brings us some universal wisdom. This perspective of wisdom addresses our propensity to always worry about those things over which we have no control---in this case the future.

I can recall holding my infant son and daughter years ago and pondering the wonders of what they might be eventually doing with their lives.
I can remember sharing my wonderment with my wife and I believe that it actually bothered her. She'd quickly change the subject to something at hand (like changing the diaper), admonishing me for day
dreaming. She'd often suggest that I was dreaming my life away. She
finally admitted to being frightened by the thought of the future,
you know with old age, death and all.

You know, from the point of the physical world, the future is not
very attractive to man. It is something to worry over, if not deny.
With age, we lose our energy, physical prowess, fresh appearance,
and in some cases our mental sharpness. It is not a happy thing
to which we should look forward.

However, return to the text, particularly verses 33 and 34. These verses have us focus on our spiritual
existence. From the spiritual perspective, which I believe to be the
true and real world, we really do not have worries in the future.
God is taking care of the future planning. From the spiritual
perspective, there are plenty of present day challenges that
require our attention and energy. And if you're one of those folks
like my wife that truly does worry about the future, then listen to
our Savior's words and let God handle the future for you.

\subsection[Future Planning]{Future Planning, James 4:13--17}
\index{time!future planning}
\index[jam]{4:13--17}

Time and again, as I prepare and plan for the future, I do so with the misconception that I am actually in control of my destiny. I actually think that if I make plans and follow my plans, that the future will unfold in the fashion that I planned. Wow! How wrong can a man be? These verses in James offer us a perspective that works very well in the world of family planning, business planning, and other future planning.

I believe that we should try to plan our way. I believe that planning for the future prepares us to be better people today. However, I do believe that we can ``over focus'' on the future, forgetting that much of our future is in God's hands. I believe that our plans for the future, should be spiritually based, God serving, and flexible to allow for changes brought to bear by God.

Presently, I am dissatisfied with our engineering firm's performance.
My staff is in a rut, my leaders are questioning their own proven
ability, and I'm disappointed as well. So, a few of us are re-engineering
our organization, or planning how we should be organized to take us
into the future. We've identified the symptoms, identified the principles
of performance under which our organization should perform, and listed
the key people in the organization around which to build our firm
for the future. We've built a prototypical conceptual model of the
organization. We've then created a review challenge consisting of
outside professionals and advisors, then key leaders of the future,
and then the staff at large. The process of planning for the future
includes much room for change, flexibility in implementation, and
a development process that involves many of our staff, outsiders,
and God. Many of us who are believers, pray to God, asking for His
guidance, His will, and His blessing on what we are doing. It no longer
becomes our plans for our business future, but it becomes what God
would have us do in the future.

To some this is frightening. I find it very comforting. I am totally
satisfied with future plans that are God led and God blessed.

\subsection[God's Perspective]{God's Perspective, 2 Peter 3:8}
\index[2pe]{3:08@3:8}
\index{time!God's perspective}

As St.\ Peter advises, understanding time is an important part
of our spiritual development. We need to explore beyond the limits
of man's four-dimensioned world and understand time from the universal
perspective.

Man has defined time with the development of clocks, time scales and
calendars. It is a relentless march, an unstoppable inertia, and a
fixed function in many mathematical formulae. It appears to be something
over which we have no control; and we cannot alter. We are at times slaves to its power. Then, St.\ Peter comes along and tells us that
God has control over it. 

God has given us some glimpses into His ability to dominate time.
The glimpses come in the form of our mind's dominance over time. As
an example, in many instances, we blow time by keeping our mind
involved in reading, television or some other diversion. In other
instances we slow time down by focusing on it, and perhaps enjoying
the smell of the flowers. Our math and physics tell us that time's progress could be reversed if we could travel beyond the speed of light. Man has not accomplished this yet, but the concept has promoted many movies and books.  

I have enjoyed taking video movies of my children's athletic events.
We've used the slow motion replay to better develop their techniques
for swimming, diving, ice hockey and golf. I've often wondered if
God and His angels use such a similar control of time speed, in their
intercession in our lives. As an example, as God and His angels protect us from harm and accidents, there are many close calls. In many
of my personal close calls, I really don't know how the harm
was avoided. When the collision appeared to me to be unavoidable, somehow, it was avoided. In other words, I wonder if God and His angels work at different, adjustable speeds of time than we are capable of perceiving? 

When I was eighteen years old, I was swimming at a water filled quarry.
There was no beach and just one trail to the water. My friends and I enjoyed jumping and diving from the stone cliffs around the spring-fed swimming hole. Our only resting area near the water, was a large rock
just below the water's surface that we called Table Rock. It was the
size of a large table, where you could find rest from swimming. About
65 feet above Table Rock was a limestone perch, that was great for
diving. With just a little push-off, you could easily clear Table
Rock. 

However, one evening, as I was taking my fourth or fifth dive from
the perch, I slipped on the accumulated water on the limestone. As
I was heading for the water, I noticed that I was headed right for
the middle of Table Rock. I was excited, if not traumatized.
I did not panic, but kept an eye on my balance and approach to the
rock. As I entered the water, my momentum carried me into the depths
of the beautiful blue-green water. Was this heaven? As I swam to the surface and paddled over to the path from the water, my friend Chip
asked if I was alright. I said yes, almost dazed and in some sort
of mild trance. He pointed to my chest, and I looked down to see the
blood from the brush burns I had received just skimming Table Rock. 

I laughed and told Chip that the brush burns were nothing, compared to what I thought was in store for me. For some reason, I was still
alive. I shouldn't have been, because I was headed right for the middle of
Table Rock. Somehow, I missed it. I am convinced of God's intercession
at the last infinitesimal second of my dive toward Table Rock. He
has control and dominance over time.

Fifteen minutes can be a very different period of time from one person
to another. The last fifteen minutes of a man sentenced to death,
perceptively travel differently than the last fifteen minutes of a
wait before a blind date arrives. Similarly, perception and control
may be one. If our existence is eternal, then time takes on a different
perception. If our existence is limited, then time is not flexible.

Time is critical in cooking, and many other things in life. There is time for a fast searing cook---and there are times for a slow deliberate roast. When one's existence is eternal, such as God's existence, then the correct timing can be counted upon, from His perspective,
not ours.

\subsection[Life]{Life, Ecclesiastes 3}
\index{time!life}
\index[ecc]{03:00@3}

And yet here we have another perspective on time. In Ecclesiastes,
we find that time is not simply a dimension used in our science, not
simply a mathematical unit, but it is a description of the passages of one's life. On my third or fourth reading of this chapter, I found myself recalling each passage in my life. At the age of 50,
I had experienced all of the passages listed, and I felt that some
of my passages were noteworthy.

When I look at these passages, I find myself taking the perspective
of a Hollywood screenwriter, recognizing the various passages that
might be used in a movie, but also the extremes of the passages, the
irony of the passages, and the contradictions of the passages. The
presentation of opposites makes me think of all of the trials of life, along with the good times---times to love and hate.

This perspective of time gives us a clear vision of God's perspective
of our life. I call it the universal perspective of our lives and
time. In some cases, it may be a recipe of what our experiences on
this earth are to be. This perspective of time presents us with the
formulae of our lives. It describes for us the universal perspective.
Frankly, I much prefer this outlook on time than that of a scientific
unit. When will man learn?

\subsection[Lifetime]{Lifetime, James 1:2--4}
\index{time!lifetime}
\index[jam]{1:02--04@1:2--4}

Please read James 1:2-4, then return to Ecclesiastes chapter 3\index[ecc]{03:00@3}. When reading these verses in this sequence, they bring so much logic and
meaning to my life and the lives of others. What appear to be disasters and tragedies in my life and that of others are simply tests. They
are trials that require perseverance. The perseverance promotes wisdom
and maturity. 

Why is it such a mystery to the world that this is God's plan? In
our society today, folks seem to believe that this world and its joys are the end in itself. The denial and outright rejection of God by folks during trials and tribulations further promote this short-sighted
perspective on the purpose and meaning of life.

As the joys of this life are glimpses of what God has in store for
us, the trials are there for our spiritual growth. The verses tell
us that ``perseverance must finish its work so that you may be mature
and complete, not lacking anything''. ``Not lacking anything''
tells us that we grow from these experiences. Something is added to
our make-up. We are more complete. Without the trials, there is little
meaning or purpose to our existence in this world. So, turn your life around and give thanks for the trials. Look them in the eye with a confident smile of understanding and a wink of commitment.

\subsection[Units Of]{Units Of, Genesis 1:3--19}
\index{time!units of}
\index[gen]{01:03--19@1:3--19}

Throughout this past century (which consists of 100 years, in each of which there are 365$\frac{1}{4}$ days, of which a day has 24 hours) scholars, lawyers, scientists, and priests have debated questions of time, such as how the earth was formed, how man was created, and other mysteries of God. These verses in Genesis have been at the heart of it all. From verse 3, the term ``first day'' was used. By use of this term, many folks interpret God's words to mean 24 hours. As a result, theories of evolution have been discounted by fundamental believers, and others have been on the decision making fence about what to believe.

I am always impressed by the argument presented by Clarence Darrow,
famed defense lawyer in ``The Scopes (Monkey) Trial'' earlier in
the last century, when legal action was brought against a school teacher
for suggesting that Darwin's theory of evolution indicates that man
may have evolved over time---certainly more than a day. The trial
was popularly referred to as the ``Monkey Trial'', in that Mr.\ Scopes
suggested that under Darwin's theory of evolution, man may have evolved
over time and man's earlier forms may have even looked like monkeys---but
this is not part of the theory. Mr.\ Darrow presented some basic questions
about these verses. He asked that if God defined and governed a day with darkness and light on day four, what was a day on days 1--3, before God defined a day? Could it have (been by our standards) more
than 24 hours, perhaps 25 million years?

I would further suggest that days 1--3 may have been days that God observes, in which case we cannot perceive them as usable units in our dimension. Remember \index[2pe]{3:08@3:8}2 Peter 3:8, ``with the Lord a day is like a thousand years, and a thousand years are like a day''. There are
many aspects of God's universe that man and his science do not understand.
Let me suggest that even though our technology moves forward in further
defining God's world, the unknowns are many more than we can imagine.
I am a man of science in a field of science. I spend many of my spare
hours further exploring science or reading of scientific development.
I can share with you, the more I learn of science, the more I know
of our ignorance. There is an infinite level of unknowns to explore.
Only God has all of the technology and all of the answers. His definition
of days 1--3 are to be simply accepted as ``God days'', beyond our comprehension
of time.

Let me also suggest that man's propensity to debate the presentations
of God in the Bible are man's futile and silly attempt at being god-like. The debate will consistently lead to man's technology
as being in contention or conflict with God's Word. As a man educated
in the ways of science, I have never found conflict between science
and God's Word. Most of science is simply the definition of God's
world and the phenomena of God's universe. I believe that it is important
for man to consistently and indefatigably explore God's universe,
and to better understand our existence and God. But we must always recognize and give credit to the Founder and Creator for the universe and joyfully praise Him for His works.

\section[The Limits of Trials]{The Limits of Trials, 1 Corinthians 10:13}
\index{trials!limits of}
\index[1co]{10:13}

Temptation accompanies all trials. We are always tempted to take the easy way out, give up, denounce our relationship with God, or even blame God for our problems. There is always temptation to submit, to give in, to relinquish our commitment to God and generally roll over to the pressures of the trials. 

As a youngster, I recall doing exercises on our wrestling team, and as we were trying to push our bodies to a new and higher level of conditioning, there was always the temptation to cheat. There were many ways to fake the appearance of a sit-up, a push-up, or other exercises. When the pain of physical stress was too great, many of us would temporarily seek escape by cheating. Who were we cheating? We were cheating ourselves and our team. 

Spiritually, our lives are filled with many trials. Some of the trials are brought about by physical circumstances, from business circumstances, and often by personal relationships. We are tempted to take the easy way out. 

This verse is a verse that should be read every Friday morning and
etched in the mind of the businessman. You can bear it. There is nothing
too great in any of the trials thrown at you. And God will provide a way out. It will be His way out, so that you can persevere. It will mature you.

\section[Trust-Building]{Trust-Building, Non-committal Posture, Romans 5:6--11}
\index[rom]{05:06--11@5:6--11}
\index{trust!building}


Relationships built upon mutual trust and common, righteous values
are priceless. These relationships are the things upon which wonderful
marriages, friendships, and business relationships are based. They
are also the foundation upon which successful teams are built.

Trust can be demonstrated though acts that promote trustworthiness.
In this evil world of business deception and deviousness, the acts
cannot be portrayed or interpreted as a means to gain something, or
to conceal a hidden agenda. They must be pure. They must demonstrate
clarity of purpose and not expect reciprocal treatment. The
acts must be built upon a foundation of principled love, honor, and
righteousness, with no payment expected.

Trust-building acts are best offered when the act is not expected,
when there is nothing noticeably to be offered in return for the act.
Just as God provided His Son, when we were worthy of being His enemies,
we should also consider developing trust building acts. Perhaps
we don't have to focus just on our enemies, but we should be consistently
vigilant for chances to act positively, when it is not expected
of us. Try it, if you want to build a trust-based relationship. 

\section[Truthfulness, Testing of]{Truthfulness, Testing of, Deuteronomy 18:21--22}
\index[deu]{18:21--22}
\index{truthfulness!testing}

Times haven't changed much. Our world today is filled with lies and people who lie. They present themselves as being capable of things that they are not, of having experiences that they did not have, and of even believing things that they do not. Unfortunately, the truthful folks of the world are often discounted and mistrusted, just as true prophets cannot be immediately distinguished from false prophets.

The speaker at my daughter's college commencement was \index{people!Bradley, Ben}Ben Bradley, retired Editor of the Washington Post. Mr.\ Bradley had been the editor in charge of releasing his reporters' story on Watergate. His message to the graduating students was the importance of honesty.
``Never, ever lie!'' was his message. He gave many examples of lying by our top government officials during the past few decades, and then described the unfortunate consequences not only for those doing the lying, but for the rest of us. 

\index{people!Roosevelt, Theodore}Theodore Roosevelt often preached and encouraged all to be honest. He believed that the foundation of our democracy is the voice of the people in selecting leaders; and the integrity of the elected officials. 

Why do so many of our leaders even have to be concerned about honesty?
Why do we have to consistently remind our youngsters and our people
about honesty? Lies and the telling of lies are commonplace in our
world.

However, in reverse, honest and truthful persons are special in this
evil world. They stand out as elevated examples and stars. They are
applauded, awarded, and in business, promoted. Who are they? These
verses in Deuteronomy advise us to record and remember the words.
They promote our investigation of a person's honesty. The verses implicitly tell us to keep track of who is honest and be joyful for honest men.

\section[Wealth, Perspective of]{Wealth, Perspective of, Philippians 4:11--13}
\index{wealth}
\index[phi]{4:11--13}

What is wealth? What does it mean to be wealthy? Let me offer some
of the definitions that I've been exposed to in my life.

\begin{enumerate}
\item To be wealthy is to be listed in ``Forbes'' magazine as being one
of the wealthiest. (RJB)\footnote{We should  at least spell out these sources, if not footnote the exact source (issue, date, page number).}
\item To be wealthy is to be the talk of others. (WRS)
\item To be wealthy is to be the top investor in one's community bank. (TAP)
\item To be wealthy is to have the freedom to do what one wishes to do, not what one has to do to survive. (JSB)
\item To be wealthy is to have the ability to go shopping and buy something without having budgeted and saved for at least six weeks. (JVR)
\item To be wealthy is to not want for anything. (EMB)
\end{enumerate}

Being wealthy is a relative term. My father was raised on a small,
mountain dairy farm during the Great Depression. He always explained
that his family was well-off and wealthy in hard times. He would always
add that his family had little money, but they had all that they needed.
His mom would bake dozens of pies each week (I believe on Thursday's)
and everyone in the valley would comment about the aroma from Mellie's
pies. He would describe the generosity of his mom in feeding the
many drifters and lost souls that would come by. Most of the drifters
were from the cities, where there were no jobs, no money, and no food.
Dad would tell us that on the farm, they had everything that they
needed and more. ``Boy, we were wealthy'', he would conclude. 

Wealth is one of those terms that needs a point of reference and perspective.
Are we living our lives under the measurable standards of God, others,
or our own standard? Let me suggest that peace and understanding
can be found in God's standards and in our own lives when we adopt
and embrace God's standards. Then wealth takes on a different meaning.
It includes great satisfaction and wholesomeness of life.

\index{Bible!characters!Paul}
St.~Paul gives us yet another definition of wealth that I can easily
grasp. It is contentedness. It is being content in every situation
in life. And it is being content, because of the strength that God
gives us. It is being able to do anything and everything.\footnote{This isn't necessarily true. You may want to expand on or correct this.} Be content in all situations through the strength of Christ Jesus!

\section[Will Power]{Will Power, Definition, 1 Corinthians 6:12--20}
\index{will power}\index[1co]{06:12--20@6:12--20}

I have experienced many vices. By insurance company standards, I've
been a smoker, addicted to nicotine and tobacco products. For nearly
thirty years, I have smoked, stopped for a year or five at a time,
and re-started. When I had been able to quit for a period of time,
the quitting was very difficult for not only me, but for those around
me. 

It has been nearly two years since my last smoke.\footnote{Do you want to correct this number?} As an addict, I am
reminded to state my habit in this manner. It reminds me that I can
be back in the habit and addiction within the stroke of a match. The
last time I quit, I further complicated the withdrawal process by
being overweight and out of shape. However, I had finally reached
a point of resolve in which I really wanted to quit the habit. I spent
nearly two months readying myself for the withdrawal. I separated
linked activities, like drinking coffee and smoking. I prepared myself
by buying nicotine patches that would assist with chemical withdrawal.
I announced to my children and my wife---the greatest critics of my
addiction---that I was quitting and gave them the date. I knew that
they would maintain pressure on me. 

Nearly one week before my scheduled quitting date, I began to have
second thoughts---what a wimp I was! A Christian brother suggested
these verses, suggesting that it was not my body to destroy. And in fact, it isn't. That evening, I re-read these verses, closed my eyes, bowed
my head, and I asked God for His forgiveness. I was wrong in developing
the addiction to start with, in not maintaining my body at a proper
weight and condition, and in wavering on the nicotine withdrawal.
I cried a great deal. I cried at my failure, at my human frailty,
at my weakness, and ultimately at God's grace and patience.

For the first week of my withdrawal, I read these verses intensely
each morning. I can tell you, this was the easiest withdrawal and
with prayer, will be the end of nicotine in my life. 

\section{Work Ethic}

\subsection[Enjoying Work]{Enjoying Work, Ecclesiastes 3:22}
\index[ecc]{03:22@3:22}
\index{work!enjoyment}

Yesterday, I met a friend who had lost his contracting business several
years back. He had called me for help, and I was able to identify
some openings of employment with some other contractors. He's been
with one of those contractors ever since. I asked him how he was finding
the work, the people and the challenge of working for someone, instead
of being the employer or boss.

He told me that he enjoyed his life at this point. He doesn't have
the headaches of management, finance, and law, but still has the sense
of accomplishment. He finds himself with more time for his family,
and he feels that with their lifestyle, that is precisely what they
need. 

He told me that it wasn't easy going from the position of employer
to that of an employee at first. He was depressed. He then decided
that he should be the employee and worker that he always wanted working
for him, when he had owned his own business. He took on that role
with a smile and an upbeat attitude. He asked all of those around
him to do the same, since he was a team leader for this large contractor.
He's excited about his work and feels that he and his team can compete
with anybody, anywhere. He does enjoy his work. He also has respect
for his supervisors, because of his experience in that role.

I often wonder when I read these verses from Ecclesiastes, what the
world might be like if all of us had my friend's enjoyment for his
work. My friend's joy was not natural. He worked at it and had to
develop it as a goal. He attempted to work through his depression
and make something of his job. My friend Harvey actually said to
me yesterday, ``wouldn't the world be special, if everyone enjoyed
their work---can you imagine the quality of life that we'd have?''.

\subsection[Commitment and Resolve]{Commitment and Resolve, 1 Corinthians 9:24--27}
\index[1co]{09:24--27@9:24--27}
\index{work!commitment}
\index{work!resolve}

As I read these verses and share my experiences, I am five hours from
home on a Sunday morning, preparing to share in worship with my son.
We are attending an ice hockey tournament in which my son's team
has won all of their games over the past two days; and they are now
scheduled for the championship round this afternoon.

Yesterday they competed against an excellent team. The game was close,
but our kids pulled through with hard work and a never say
die attitude. It's quite possible that the other team was more
talented. Yes, commitment and resolve are required in competitions
such as this. Commitment and resolve are required when we are apparently
cheated, when we are behind or down, and when it appears
that there might be little to no hope.

Our lives are competitive in general---not just in sports. Throughout
history, nearly all of man has had to compete, not only to win, but
sometimes also to live and be left alone. Competition in life and the universe requires preparation. It requires training that builds stamina, endurance, perseverance, capability, and a mental attitude of commitment and resolve.

The last product may the most important. The mental attitude that
comes with training is one that includes a never dying hope. It is
knowing that no matter how bad the situation, you, your body,
and your God will overcome. There is always hope. 

My secretary's son has just completed the first of many steps in Navy
special forces training. His goal is to be an elite \textsc{SEAL} anti-terrorist
soldier. The training has been very difficult. His completion odds
for this first step were one in twenty-five. He made it and shared
with his mother that the training will prepare him to handle the worst
of situations and to survive.

I believe that it is important for all of us to recognize what our
mission in life really is. What is the purpose of our existence? If
we're to be written up in the future, what is it that we and God want
the testimony to say. Once you find your mission, it is important
to rally all of your resources to meet the challenge of that mission.

\index{Bible!characters!Paul}In verse 27, Paul outlines the importance of controlling all of the
resources that you have and tailoring them toward serving you and your
mission. We are in a competition. Let's recognize it, train for it,
and use all of our resources in becoming the winner.

\subsection[Goal Setting]{Goal Setting, Proverbs 16:26}
\index[pro]{16:26}
\index{work!goal setting}

My father used to say that he worked to keep a roof over our heads
and food in our bellies. Concern for our house\slash{}home along with meeting
the payments on the grocery bills were what made him work so hard.
He had a full time job at a factory, plus he had a welding shop
in the garage that was ever filled with customers wanting to have
something mended. My dad worked continuously, never complaining. Hard,
continuous work had become a habit. In fact, when given the time to
rest, he wouldn't keep still very long. 

When about the age of six, I first became aware of my father's work
ethic and worried if I'd have the stuff to be like my father
in the future. As a youngster and one of my father's helpers, I felt
that I was lazy and without drive. While my father seemed to have
a tireless spirit, almost like that of a paced distance runner, my
stamina was strong, but short-lived. Sometimes I would wonder just
how I was going to get along in this world and sometimes I even had
nightmares about being homeless and destitute. In my youthful years,
I did not fully believe that God would, in fact, lead me through this
life the way that He has.

In retrospect, I realize I had set short term goals and targets for
myself toward which I could sprint, in order to obtain them. As each goal was met, I would move toward the next. This start and stop for a breath approach to life is a proper analogy for the trials of my life. It is very different from my father's life. 

I should also share with you that God has always outlined for me
goals and targets. There has always been an attractive, highly valued
goal to target. I believe that most folks need these to provide incentive
to strive, persevere, and to mature. We are humans with a mind and
we need reasons for our labors. We need to know why we do what we
do and not just do them by habit. There needs to be reason which
promotes passion and intensity in our endeavors.

When I train project managers in our firm, I suggest to them that
a design project is simply a series of goals and sometime crises.
A good project manager needs to identify the phased goals, predict
the crises, and develop a strategy to address the goals and to align
his team in the attack. There is reason and logic, combined with a
desire to instill passion and intensity.

It seems to me in this half century of my life that
our lives must have purpose. With purpose, there is hope. With hope,
we do not simply live our lives and look for the easy way out. With
hope, we live our lives attempting to achieve our established goals
with the passion and intensity of knowing that we are maturing our
souls, while grateful for the fact that we are serving God. 

\chapter{Financial Matters}
\index{finance|see{money}}
\index{money}

While \index[1ti]{6:10}\index{evil!love of money}``the love of money may be a root of all sorts of evil'',\footnote{Note that this text was originally cited incorrectly, ``money is the root of all evil''. 1 Timothy 6:10 says, \emph{For the love of money is a root of all sorts of evil, and some by longing for it have wandered away from the faith and pierced themselves with many griefs.}} it is also
the tool that we use to value commodities, services, construction,
products, and nearly all exchanges of value. Like many of the
tools and things of the world, man has managed to corrupt it and abuse
it. Because of the abuse, money can sometimes carry a negative connotation.

I am convinced that there is a place in today's commerce for the Christian
business person. I also believe that the Christian business person
must manage financial matters responsibly, and in accordance with
the guidance provided us by the Word.

Successful financial management requires identified and stated, purposes
and goals. It requires deliberation, understanding and planning. In
many instances, misguided purposes and evil-based goals are the cause
of man's downfall in handling money. However, I'd like to suggest
that Christian brothers and sisters in all levels of business and
life, can be responsible financial managers. Furthermore, let me suggest
the following perspectives, which become positive reasons to develop
responsible financial managers, whether for others, our businesses,
or our families:
\begin{enumerate}
\item Money is often a measure of how well we labored for someone else; 
\item Money allows us to feed, house, clothe, and educate our families; 
\item Money supports Christian mission efforts; 
\item Money is needed to keep our businesses afloat; 
\item Money can serve our purposes, which we pray are righteous.
\end{enumerate}
We cannot avoid financial matters. Let me suggest that we nurture
a positive attitude and approach financial management as something
that we can do responsibly in the service of God.

\section[Business Focus]{Business Focus, Luke 12:29--31}
\index[luk]{12:29--31}\index{business!focus}

It is important to establish \index{business!objectives}\index{business!goals} business goals and objectives. I believe
that they should be confirmed or renewed on a regular basis. I also
believe that the goals and objectives should be shared with all of
the team members, and all of the employees, and staff members---as
a means of aligning the efforts of all involved. I also believe that
financial goals and objectives should be outlined, perhaps even prioritized
in their level of importance relative to other business goals and objectives,
which might change from time to time.

These beliefs are based upon
my own experiences, both in my successes and failures. Put them
in the category of Lessons Learned.

My experience tells me there is a need to establish a priority for all of our goals and objectives. I suggest that the financial
goals and objectives, as compared to other strategic goals and objectives,
might be toward the lower end of the scale in priority.

Let me explain.

Our businesses, in most instances, are not simply a means to make
money. They are activities that hopefully produce a product or service
that makes the world a better place. They are a forum where people
of similar interests can work together as a team to out produce the
sum of the individual efforts. Our businesses serve as classrooms that provide people with maturity, knowledge, wisdom, and an ever-increasing sense of development. The people composing the business should be
filling a void in the marketplace, becoming an integral part of our society.

Our businesses are also a context in which we can share the Word. They are a daily podium with all of the trials that life has to offer. It is rich with opportunity for sharing the Word.

In order to keep our businesses running, we need to market and sell
goods or services; we need to perform the work efficiently; and we need to be compensated for the work. Compensation is generally financial exchange; i.e., money. Management of the money is financial
management. We use money as the unit of measure, in determining the
cost to sell and market our products and services, in determining
the cost to produce or deliver our services, and in determining how efficiently (profitably) we are performing. 

While we use money as one of the units of measure in managing our
companies, it is not the primary source of focus in most successful
businesses. Our primary management focuses are the guiding forces that make our companies successful. The guiding forces might be considered the parallel to our ``treasures'', or ``where our heart is''.

The guiding forces might include efficiency, personal\slash{}professional
growth, product development, market share or capture, or quality of life. Financial management is often a maintenance level focus, in simply assuring that we are stable.

It is not my intention to promote an attitude of financial ignorance, nor a perception of financial
avoidance. Financial management is mandatory for business success,
but it simply is not the top attraction. It is not the master,
unless we let it become such.

If financial success becomes our primary business focus---becomes our master---then don't be surprised when
our businesses lose their purpose, spirit, and adventure! According to the Word, we are doomed.

I recognize that in the world of public corporations, the never-ending watch over stock values, financial management, financial imagery, and financial reporting and control are critical for perceived success. Public corporations and the sale of stock through images created by financial reporting is another topic to be discussed elsewhere. The money changers are consistently creating a buzz of their
own, both positive and negative. They need something to think about, worry about, talk about, and generally fuss over. 

Nor is it my intention to downplay the significance of financial management, or to decry the value to our society of accountants and financial managers. These folks are an integral part of our commerce and often find themselves in leading positions of our companies. Yet, when they become the leaders of our companies, their focus has to go beyond financial management, into their company's productivity, marketability, and into future planning. Successful corporate leadership, even if you're an accountant or financial analyst, must further focus on manufacturing performance, new product and service development, public and client relations, image, culture, people chemistry and many other areas. Finance and money management are simply a portion of the scope of the job. All of the areas of focus are important; none can be ignored. 

I pray that all of us are sure of what we hope for, that our
business efforts, our company's management, and our workday are founded upon principles that are righteous. I pray that we are contributing 
positively to making our world a better place.

\section[Business Philosophy]{Business Philosophy, Exodus 22:25--27}
\index[exo]{22:25--27}
\index{business!philosophy}
\index{business!collateral|see{collateral}}
\label{collateral_1}

Yes! There is a time for not collecting interest and there is also
a time for not demanding collateral!

I can remember several years ago, when I shared with my banker that I was not going to be paid what I was owed by one of my clients. My banker then quickly asked if I had the client's house as collateral. I said no, and his face demonstrated an expression of grimace. He then suggested that I could kiss that \$200,000 goodbye. He then went on to tell me that I needed to monitor these situations more closely and to extract a level of pain from these people, if not leverage some collateral from them. 

Recognize that there is the business of investing to build on your money by collecting interest. In fact, corporate America is based upon this premise with our stock market and our system of investment. Folks invest their savings for the purpose of earning interest on their money. In exchange, the recipients are given the loans to build their businesses. In our system, corporations are advertising (within the guidelines permitted by the Securities and Exchange Commission) that they'll give you a return on your investment, if you'll assist them in their capitalization. This interest building and charging is very different than collecting interest from a brother or sister in need. Our gifts and existence come from God and He is providing us with opportunities to help His children, our brothers and sisters.

Similarly, we should not expect to demand or provide collateral for all of our investments and loans, particularly not for loans to individuals in need. If they had collateral for the loan, chances are good that they would not be asking you for the loan. 

I also believe that it is important to look at this situation from both sides of the fence. Put yourself in the shoes of the borrower. You are in need. You are reasonably sure that the loan you need will be enough to allow you to reach the level of performance that you expect and you're willing to take the risk. You are willing to lower your pride, borrow from another person, and you commit to repaying your debt. 

If we are to let our ``yes, be yes'' and our ``no, be no'', then we must, without qualification, be prepared to repay the debt. If it takes more time than expected, that is acceptable, but we will repay the debt. No claim of bankruptcy, or other excuses should stand in the way of our \index{commitment!to repay debt}\index{debt!repayment}commitment to repay. With this commitment, we should not place the treasures of our existence in jeopardy---like using our homes for collateral. 

I signed away our home as collateral when I borrowed money nearly fifteen years ago, in order to buy my present business. I do not believe that the added pressure of seeing my family on the street was incentive for my performance. If anything, it detracted from my performance.

The five-year loan was repaid in three and one-half years. Nearly five years later, I found that the bank had not released my lien. They joked about it when I complained to them and I suggested that they should display more honor than this. This was the same banker mentioned above that was intent on extracting pain from loan defaulters. This same banker lost his bank in a state take-over due to questionable practices.

My reason for mentioning this is that I've taken a stand against such unreasonable collateral. People need to live. They should not be risking their family's necessities for their businesses. I will never do it again, nor do I recommend that anyone do it. Yes, it has become standard practice in our financial community to have houses put up as collateral. No, it does not mean that it is right. I firmly believe that it is wrong.

Let's put these thoughts into proper perspective. God gives us our abilities, our situations of profit, and our challenges in the business world. Is it right to extract pain from those who owe us money? Have we lent them \emph{our} money or \emph{our} possessions? Or are we simply the landlords\footnote{``stewards'' is a better term, maybe?} of this money and these possessions? Is it possible that our calling and our reporting are to a higher authority than the standard of the day?

\section[An Example of Charity]{Charity, An Example, Luke 10:29--37}
\index{charity!an example}
\index[luk]{10:29--37}

What is your perspective on charity?

Several years back, one of my business partners suggested that we \emph{not} give the local children's hospital a donation, even though we had a profitable year. Instead, he argued that we should be donating the money to an institution that would give us more work in return for our charitable contribution. The children's hospital, while needing donations to survive, would not be building any new buildings, or renovating its present building in the near future; and building design was our business.

I was pleased that I didn't need to comment on this situation, as my remaining partners jumped on the greedy proposal, outlining that while we were attempting to be responsible business people in our community, our donations to the hospital were of value, if not critical to the hospital's operations and that we should continue our donations as long as we could afford it. 

We are all part of a world community, composed of different races, families, languages, and cultures. Regardless of our differences, we are one pitiful race under God that needs to hang together. When our brothers and sisters are in need, we all need to respond. God gives us the provisions to do such.

I grew up in rural central Pennsylvania in a time when much of the economy involved bartering and the trading of services. Cash was short. Our almost pioneer-like culture in Appalachia demanded that there be cooperation and charity amongst all. If a house or barn burned, we all responded by rebuilding it. If a roof needed to be replaced on someone's house who was in need, we attacked the roofing job as a team of neighbors. If someone's car broke down along the road, all stopped to assist. There was never any thought of race, retribution, compensation, or gold digging. Charity and giving, when it is common and a way of life, simply happens without record. There were no \textsc{IOU}s. It was a proliferation of anonymous giving. It became contagious.

Luke's description of the Good Samaritan is often presented and received as an exceptional example of charity. Charity should not be exceptional in our world. While some would consider it extraordinary, it can and should be quite ordinary. I have lived in situations and environments where it is common and expected. It seems to me that in locations where there are greater numbers of people, like in cities around the world, there is less charity.

Why does charity always seem to appear in just rural areas? It's almost as if the density of people and charity are inversely proportionate. People do not take care of each other in our cities and therefore we need to rely on our \index{charity!government}governments to provide help and charity to those in need. 

This is a sad documentary on the condition of most of today's world. Rather than promoting love, sympathy and charity at an individual level, we re-direct our conscience driven sense of responsibility to the government. I've spent a lot of time in countries in Europe with social governments and cultures. There, charity is generally poorly provided, the culture is void of volunteerism, a spirit of community, of fund raising for a cause, and of basic charity.

Of the people that I've met in these countries, all are dissatisfied with the system. They all seem to be hopeless about the ability to change. I see that trend developing in the United States. That trend says ``let the government handle it''. It's a trend that draws focus away from individual responsibility to addressing the needs of the people only through \index{government!programs}government programs.

I don't buy this approach and will never agree with it. All of us are on this earth to mature our souls. We bear responsibility for our individual actions. Our judgment by God is not on a community basis, but it is focused on the individual. We need to share this story throughout the world. Let's attempt to change the world and the people of the world. Let us---you and me---take the steps to make a difference. While we might be disgusted with the world, we need to start to change it into something that God would have it be.

\section[Collateral Principle]{Collateral Principle, Deuteronomy 24:6}
\index{collateral}
\index[deu]{24:06@24:6}

Common sense is needed in the business world. While this verse may initially appear as a ``no brainer'', there is deeper meaning and intent.

If you've not read my earlier editorial about collateral (see p.\ \pageref{collateral_1}), let me state that if collateral is available, it should be used as a down payment (or partial contribution) to finance the investment. If the investment is a wise investment, it should be just that, and not require collateral to neutralize the risk. If the basis for the loan is sound, then all resources should be made available for the commitment. If part of the profit making process is to be sacrificed, because it must be set aside for collateral, success for repayment of the loan is less sure. 

I should also point out that the need for collateral, is a statement of mistrust. In other words, ``I don't trust you, so I'll hold this item of value, until you can repay me''.

May I suggest that rather than spending hours and days on writing very detailed, if not overly complex contracts, that outline how collateral will be usurped upon repayment default, that banks and other lending institutions spend the same amount of effort investigating the worthiness of the investment and the trustworthiness of the person to whom the loan is to be extended.

I know that this is not how the world's banking community presently works. The banks use terms like ``services'' and ``products'' and some even brag in their advertising about how ``people oriented'' they are. Instead, they are ``procedure based'', and frankly, they might be the first overall industry in our society to be replaced by a computer. They lack trust, sympathy, emotion, and spirit---the ingredients of the soul. They are simply machines. Of course, is mandatory. It will remain an integral part of the banking business, as long as our banks are the mega-merged monsters that lack
real people.

However, as individuals and business persons, we have no excuse for being so heartless and insensitive. Let's use our heads and our hearts more. Let's spend our time getting to the heart of what makes folks tick. Let's get to know the people, their companies, their cultures, their goals and objectives, and then we can make decisions on our investments or lending of money. Are they the kind of people with whom we want to associate? Do they embrace our values, or are they following another ``leader''? What will it be like in a crisis with
them? Should we entrust them with the money with which God blessed us? Is God's money being used to forward His Kingdom?

\section[Financial Contentment]{Financial Contentment, Philippians 4:10--13}
\index[phi]{4:10--13}
\index{contentment!financial}

``The peace of God in a man on earth!'' is the phrase that I wrote
in the margin of my Bible as I first absorbed these words. 

Often I have found myself wanting more money. I ``dream'' of doing
things at my work that require more money, like purchasing better
equipment. Thoughts like, ``if I only had lottery winnings'' pass
my mind. I then have to remind myself that I am not even making the
best use of what I have, at least not in God's eyes. As I concentrate
and reflect during these moments, I find myself reviewing my own waste
and decadence (relative). I frankly think that I am getting wrapped
up in the world's love of money---not my mission.

Several years back, when I was ``risking our home'' to buy my business,
my wife was very supportive, and had to finally encourage me by stating,
``we've been poor before, and we can be poor again if need be!''
She went on to give me examples of how we'd survive, ``no matter
what''. I'm not sure that I would have gone further with the venture
without her words and thoughts of encouragement. She was right. We
would survive regardless, if it were God's plan.

Contentment, particularly financial, is a blessing, and I believe
a part of the peace of God. I pray for all, that they find such contentment
and satisfaction that come through focusing on God and His love.

\section[Courage to Persevere]{Courage to Persevere, Hebrews 12:1--13}
\index[heb]{12:01--13@12:1--13}

Years ago, while traveling on business with a friend, I asked my friend if he was going to watch a series on the Jewish holocaust in his hotel room that evening. He stopped and looked at me with a calm and confident smile and stated that he was not going to watch the documentary, because he had lived the holocaust. As a tank commander in the Hungarian army, he was fortunate enough to have survived extensive combat and to have been a prisoner of war in Germany. As a \textsc{POW}, he spent over a year cleaning up a concentration camp, including carrying corpses to mass burial sites. He told me, ''if you ever hear anyone doubt that it happened, send them to me for proof. I have the proof''.

His name iss Dez Bacsujlaky. He told me that he thinks about the experiences of the holocaust whenever his present day life seems to be overly burdensome. Dez told me that when ``his enemies seem to be coming down around him'', either socially, financially, or otherwise, and the crisis in which he is trapped seems like the end, he thinks back
to his holocaust experiences and the relative comparison always makes his present crisis seem small. In effect, Dez puts things into perspective. He often says that ``thinking of the war and the
holocaust was a benchmark experience''. He was removing himself from the present circumstances and comparing his crisis to another, in order to
measure its relative importance. He told me that this process of ``benchmarking'' erased all fear when he joined the Hungarian revolution of 1956 against the Russians. As a ``Molotov cocktail throwing rebel'', he never feared for his life, but kept his focus on his objective. When fear began to creep into his mind, he would ``benchmark''.

\index{Bible!characters!Paul}
St.~Paul's words also remind us to benchmark our lives. He is encouraging us to mentally step aside from the pressures of the immediate crisis and challenge, and similarly take stock of our lives. Our benchmark as Christians is found in verse 2 of Chapter 12 of Hebrews. Christ is our focus, our Savior, and our ``Benchmark''. We are not to let the trials, hardships, and financial challenges of this world ``take us down''. They will continue to be in our path, and we are matured
by these experiences. Thank God! 

I recommend that these verses will serve you well in times of financial crisis. They are words of solution and of courage in difficult times. I can vouch for them. They send me to my ``Benchmark''.

\section[Payment of Debt]{Payment of Debt, Romans 13:8}
\index[rom]{13:08@13:8}
\index{debt!repayment}

My wife and some of my friends maintain an old philosophy regarding
indebtedness. They don't want to carry debt. Debt bothers them, subconsciously
nagging at them. As my wife says, it gives her a stomach ache, if
not chronic indigestion. So, rather than impose the stress that indebtedness
imposes on them, my wife and my friends live a ``cash based'' financial
existence.

Time and again I have evaluated paying cash for a new car, or financing
various portions of the new car cost. Time and again, the evaluation
mathematically promotes ``borrowing'' for the new car, and investing
the cash that would be used for the car, in a conservative ``sinking
fund'', such as mutual funds, a CD, or even an interest bearing checking
account. During the life of the loan, I will spend less for the new
car by borrowing, at conventional rates. However, my wife is our personal
financial manager. She will only permit one small car loan at a time,
which demands that we pay cash for our additional vehicles. In a family
with several young drivers that are active and in need of transportation,
this creates a large cash demand. My wife explains that when we owe
money to others, they come first in their repayment, and she is always
bothered by the debt from an emotional perspective. Do you think that
this is silly?

Step back a second, and define what promotes loans, and why they exist. 

A loan is the sharing of resources from one who might have extra resources,
to one who needs and can make good, if not vital use, of the extra
resources. As mankind was developing its society and civilizations,
a loan was an act of mercy during an emergency. To give a loan to
someone in need, was an act of grace from one to another in a situation
of desperation. As servants of God, we Christians look forward to
the chance to offer loans to those who are in need. Just the offer
of a loan at times of desperate need, is soothing to those in need,
to know that they have a solution.

However, today's society promotes the use of loans, as a financially
wise practice. I hear advertisements for loans, promoting them for
the silliest of reasons, certainly not for situations of desperation.
I must admit that there are times when ``the business line of credit'',
a house mortgage, and car loans are blessings to us. However, I can
now understand my wife's concern for indebtedness. It is not wise
to ``ride the line of credit'', to keep debt on the credit cards,
and to constantly be obligated to others. As we believers know, our
lives will be ``steeplechases'', filled with trials, hurdling many
obstacles and many ups and downs. When we are stable and ``up'',
we should be ``building nest eggs'' for those trials ahead of us,
when we might need it. Reduce that debt to a manageable level, if
not eliminate it. Save the loan making for when we are desperate for
it. This principle is based, not upon the interest formulae and life
cycle cost comparisons, but from the perspective of our long-term
success as businesspersons and leaders.

A close Christian brother of mine, lost his architectural business
a few years back. He had four partners, with one of the partners assigned
to manage the finances of the firm. He ran the finances under the
``popular'' belief that by operating in an indebted posture, they
would be making money from the interest game. In other words, invest
extra income at a higher rate than that of the debt. As a result,
he rode his line of credit to the hilt (\$700,000). 

When their firm reached a bump in the road (a hole in the work load),
they had to consume their invested money. In order to keep going,
they had to take an additional half-million dollar loan. The partners
guaranteed the loan with their homes. They took the loan at the beginning
of a mild recession in their marketplace. The lack of work atop their
debt payments made repayment an impossibility. They couldn't even
keep up with the interest payments, let alone repay the principle.
In effect, their indebted posture brought them to bankruptcy when
they hit another low in their business. They lost their company, their
homes, and they lost all of their savings.

By the way, my brother in Christ has restarted his own business. He
reports that he will continue to run it on a cash basis, with no debt.
He is now in his fourth successful year. 

Follow the wisdom of this verse, and keep the debt level of your firm
low, if not eliminated. It will give you much flexibility for navigating
the highs and lows of your business life, with ease.

\section[Treatment of Debtors]{Treatment of Debtors, Leviticus 25:35--38}
\index[lev]{25:35--38}
\index{debtors!treatment of}

Today is Christmas Eve, 1997. My son has been off delivering food and gifts to those who barely survive, yes---in a Santa Claus outfit! My neighbors are involved in taking care of the indigent.
My wife supports those in need through various charities in our neighborhood.
My staff at work take care of the homeless on our corner. Have
we really lost the spirit of common-wealth that our forefathers
in America had and which they quickly demonstrated at the drop of
a hat?

God is very clear in these verses in his directions and orders to the
Israelites. Note that I said \emph{directions and orders}. They are not to
be defied, not to be questioned, and not to be debated. In fact, God
added the words ``I am the Lord your God \dots'' as a point of authority. \emph{We must obey His words}!

When we have debtors who owe us, we must be sensitive to their needs.
If they are just living and they need their resources for existence,
we must ``back-off'' and let them live. We must not charge them
interest, and be flexible in their repayment terms. We must not forget
that God is responsible for our existence and all that ``we have''.
It is not really our property, but God's. We all need to be reminded
of this fact, recognize this truth, and follow His Word!

\section[Fairness, Perception of]{Fairness, Perception of, Luke 7:41--47}
\index[luk]{07:41--47@7:41--47}

\begin{quote}
But he who has been forgiven little, loves little.\footnote{It may be that you have misunderstood this text. Jesus is pointing out that those who do not recognize the love that has been demonstrated to them in their forgiveness, likely won't be able to lovingly forgive others.}
\end{quote}

This is not ``tit for tat'',\footnote{An interesting phrase with several etymologies. The most convincing I've seen is that it's from the Dutch for ``this for that''.} or ``even Steven''. In fact, we are reminded of the perceptions of human beings. As most folks are self-centered, we must be reminded that fairness in their eyes is often not equity in their minds, but something that may be slanted just a bit toward their benefit.

This is a powerful principle to keep in mind when we are negotiating settlements, fees and compensation. In a similar vein, the demonstration of fairness to others often requires more than what is fair. It requires going the extra mile in aid, service and compensation. Only then is fairness or your demonstrated sense of fairness, recognized by others.

I grew up in a rural part of Pennsylvania. Much of the economy of
the area was founded on bartering. From the earliest times of settlement in Appalachia, folks traded services and commodities. Every
person was expected to a have a valued specialty. Specialties
included carpentry, chimney witching,\footnote{I don't know this phrase, but I guess it means chimney building, maintenance, and repair? Sort of like ``Ditch Witch''?} furniture making, tool making,
clothing making, etc. A person was expected to provide these services,
not for a fee, but in return for other services and commodities, when
they were needed. The ``when'' is the key to the bartering system. It is also the key to a ``giving'' community. 

When someone ``needed'' a service, his or her specialty was not
always needed in exchange. A mental ``IOU'' (not written, backed
with guarantees and collateral\index{collateral}) was established. The needed service
was valued at the time it was given, and the recipient was indebted
by honor and integrity. There would come a time in the future, when
the recipient of the valued service or commodity, could reciprocate.
What was promoted was an environment of giving. It was considered
the norm to have given much more than was received, and it was the
norm to save and build a bank of ``IOU's''. The result of this bartering
economy, was a community and society based upon generosity, an innate
welfare system, and an extra appreciation of the services and commodities
provided, well beyond that which is experienced in a fee for service
economy. Fairness was perceived from a human value system, not from
an accounting perspective.

When accounting is added to the formula or the economy, fairness is
only perceived in standard pricing arrangements. However, standard
pricing doesn't really exist in a free economy, as prices are subject
to change. Supply and demand establishes pricing, or value. Therefore,
fairness and ``just value'' are a point of perception. 

Establishing just value and fairness---via perception, is a key ingredient
in business success. In addition to conducting one's business with
these values, an effort must be put forward to actually ensure that
the perception of value and fairness is in place. It falls into the
realm of marketing and sales on the front end, delivery of service
and commodity during the project, and ``post satisfaction'' programs
after the services or commodities are delivered. Satisfied customers,
clients, neighbors, etc. need to have their perceptions and expectations
managed as part of a business delivery plan. If the value of the services
or commodities can be communicated, and the perception established,
the better the future business will be. 

In the business world and in our everyday life, perception of fairness
is often more important than the reality of fairness. It is the perception
of fairness that can't be taken for granted when either providing
a service or commodity, or in forgiving debt. Perception becomes reality
when dealing with people. 

\section{Financial Cycles, Deuteronomy 8:1--5}

I do not know of any persons who have been in business for over five
years who have not experienced the ``ups and downs'' of business.
I am included in this group. While I have not yet lost a business,
I have come close. Conversely, I have had extremely profitable years
to contrast the ``downs''. I have had years of growth in which my
only response was to simply thank God. The growth and profits were
so unexplainable. Oh sure, my partners and accountants had all sorts
of explanations and ``ratios'' to define the growth and success,
but they had no explanation for the source of the work from which
the profits flowed. They are so smart in worldly matters, yet so ignorant
of the world and the universe.

While I've been in business ownership for the last decade and a half,
the previous phases of my life included cycles of financial stability
and indebtedness as a student and then as an employee of other companies.
Now that I'm over fifty, my perspective is built upon a repertoire
of experiences that vary substantially. My perspective has developed
a universal view of life. Our life is but a speck in the scale of
time of the universe, but this speck of time is an ``experience'',
and a series of trials that are intended to better our soul, perhaps
make it more compatible with God's so that we might live with Him
for eternity. 

Think of it! The bad times, or the ``lows'' of our financial experiences,
might in fact be positive builders of our character and our ability---perhaps
even our soul!

As a youngster, I competed in wrestling at the high school and collegiate
levels, at institutions known for their wrestling. I had the opportunity
to experience the ``best of the best'' in the sport at the time.
I was just recently discussing the value of sports in raising children
with a close hunting friend. He shared with me a definition of an
athlete that did not focus simply on athletic ability. His definition
of an athlete included a person who participated in athletic events,
who ``worked hard'' to further develop his or her abilities and
experience the satisfaction of that development. The definition had
nothing to do with winning, or elitism. It focused on the ``personal
development of individuals''. My friend asked me to list the reasons
why wrestling, as a sport, seems to produce so many athletes. Of course
I shared all of the \index{ethics!work}work ethic required to participate in wrestling,
but then added that with these victories, wrestling always seems to
include some terrible defeats. These defeats are public failures (no
place to hide behind teammates). They are usually brutal, physical
assaults, and they are attacks on a wrestler's pride and courage.
However, they are extremely important in building strength of character,
and a balanced view toward athletics and life. Without the ``brutal''
defeats, a self-centered character prevails.

Similarly, in business, the defeats almost seem to scar you, and remind
you how easy it is to fail. In a like fashion, the successes often
come without warning and without plan. They are pleasant blessings
that remind us of the rewards of hard work, and also the blessings
of a loving and compassionate Father.

I have recognized the cycles of life, as part of our agenda, in maturing
our souls. I pray that you develop a similar perspective---a perspective
of consistent hope, faith and the knowledge that God is with us, regardless
of how high the highs are, and how low the lows are. 

\section{Financial Stress, Hebrews 13:5--6}

As my wife will quickly attest, I am not a good candidate to even
discuss the topic of ``financial responsibility''. I can honestly
say that I have only been ``stressed-out'' over financial matters
in a few instances in my entire life. My wife often thinks that my
attitude toward money and finance to be irresponsible. She often lovingly
treats me as one of the kids when it comes to money. After I hand
over my income checks to her, she'll give me my ``allowance''. On
vacations, she carries our cash, and deals it to me one dollar at
a time. 

I don't mind her attitude. Frankly, it is comforting and motherly.
At the same time, there is some merit to this attitude. My personal
perception of the importance of money is different than the norm.
I have always considered money to simply be a tool and a media that
our society uses to exchange services. In addition, I've never been
one to worry about having money, nor losing money. 

My attitude toward money, which promotes somewhat of a stress-free
attitude, comes from being financially poor as a youngster. Living
in a rural area, our hobbies, pass-times and activities were centered
on God's natural environment. Most of the activities cost little to
nothing in the form of money. ``Prized'' investments like a new
shotgun or hunting rifle, were always within an ``annual-savings''
reach. If money was needed to fill the gaps (vehicles and gas money),
there always seemed to be odd jobs that would generate ready cash.
I do not recall wanting for much of anything. Tubing on the river,
ice skating on the canal, building forts in the hay mows, riding the
horses, climbing the mountains and the cliffs, swimming in the quarry,
and many other similar activities cost us little to nothing in cash.
The folks of our family, and most of my friends, were rich in love
and experience---not in money. We didn't seem to need it. 

Few of my career decisions have been based upon finances. Almost all
of my career and professional development decisions have been based
upon improving my ability and value in my design skills and services.
I suppose in retrospect, I can say that I've enjoyed a blessed ``high
road'' of principle and value. While I've only had three places of
employment since leaving the army after college, the ``moves'' from
one place of employment to the other were driven by emotion and the
need for a different exposure to our industry. I moved from my first
job to my second job with a 30\% cut in pay, but with the chance to
experience the design of much larger buildings and projects. Within
two years, my salary was doubled. I moved from my second job to purchase
my own business. I took a bank loan, along with a partner, to buy
the business and paid it off in three and one-half years. During those
three and one-half years, my salary level was 35\% less than my second
job. Obviously, my career moves were not made for financial reasons.
What drove me to my own business was an emotional attitude, an attempt
to prove that my ideas might be of value. 

Some would say that I've been quite foolish. Others would say that
I am na\~{A}\textasciimacron ve. By the way, I've heard all of these
types of comments even from friends. Perhaps so, but who's the judge
in this matter? I've followed my judgment and instinct, based upon
my beliefs, and God has given me a ``broad'', easy to identify path,
that has bestowed upon me much satisfaction. 

In many ways, I suppose that my approach to life is quite simple.
Money is of little value to me. While I don't cling to money, it has
some importance as a media of exchange. I can share with you that
I do become upset if a person or company violates their financial
agreement with my company or me. The violation of the agreement or
financial terms is an act of dishonesty, from my perspective. Even
if they're slow in paying me, I find it annoying. Introspection would
indicate that the failure of principle, integrity, honor and morality,
are the basis for my intensity of emotion over the violation, I pray. 

I've had my accountants shake their head at my lack of appreciation
for their values. The world of business is not just greed, or taking
advantage of others, or being ``mean and inflexible'' (these are
qualities I heard just yesterday from a financial advisor as being
needed in our company, since they work in all of the business world).
I believe that we'll all realize someday in the future, that our world
of business was our training and test ground for the maturity of our
souls. I also realize that my perspectives on the business world are
contrary to the norm.

I have been robbed from time to time in this life of mine. I have
lost articles of value. I have felt violated. I have felt that my
most valuable entity, time, had been stolen. Money is time, and time
is money. I have been robbed of time in my life. Is that fair for
another human being to do? I don't believe so. While we are all a
bit different from one another, I don't believe we're too far apart
on perspectives. 

I can personally point to many times in my life when money has been
received that was not hard earned, and other times when money was
hard to earn, when I needed it badly. I believe that we go through
these trials for the maturity of our souls, not the building of our
``nest egg''. As a friend of mine repeatedly says, ``I've never
been to a funeral where there was an armored car carrying the deceased
persons wealth in the procession.'' 

I also can see in retrospect, that worrying about money and the accompanying
stress of financial challenges, are not healthy for us, nor necessary.
God will continue to care for us, direct us, and if appropriate, reward
us. 

Please step back, and look at the stress created by money from a universal
perspective. It is something that can be handled, controlled and suppressed,
if not extinguished.

\section{Inheritance}

\subsection{Value, Proverbs 13:22}

Have you ever received an inheritance? If you have, I'd guess that
some of the inheritance is quite valued by you, while some of the
inheritance has little value for you. While I've never received any
substantial endowment from my ancestors, I do have a few items that
remind me fondly of my early years with my ancestors, and the love
of my mother and father and great uncles and grandparents. 

On the surface, this Proverb appears to be promoting the development
of substantial estates and trusts, that are of such substance and
size that they have an impact on not only your children but also your
grandchildren. In addition, a ``good man'' will find a way or method to make it happen, safeguarding the estate.
And yes, the ``sinner'' will not be assured that his children will even receive their inheritance. 

However, my first perspective of this Proverb was different and continues
to evolve as I reread the verses. When I first read this Proverb,
I understood that God was telling me not to simply develop an inheritance
of money and material things, but that I should be responsible for
establishing values and heritage, that my children would pass on for
generations---values such as an appreciation for our natural world,
an appreciation for love within the family and activities which promote
such, wholesome pastimes such as hunting, fishing, golf, skiing, and
other sports, and ultimately and most importantly, a true and sincere
love of God. Learning and education, along with promoting the never-ending
quest for more, are a key ingredient of these values. These are the
things of which long lasting inheritances are made. In review of the
above listing, I don't see a lot of ``tangible assets'', as my accountant
would say.

\subsection{Value of, Proverbs 20:21}

When my children were ages one and five, I had a medical trauma that
could have easily taken my life, as my doctor advised both my wife
and me as I was being discharged from the hospital. The startling
news upset my wife more than it bothered me. She asked me, while driving
me home from the hospital, ``is there anything that you want me to
share with the children, in case you die in the near future''. Her
question took me aback, and I was embarrassed to respond that I'd
need to think about it for some time. Natalie's question led us to
a multi-year quest in identifying values, principles and heritage
for the children---their inheritance.

While I won't bore you with the details of their legacy, we have focused
on ``building experiences'', as a means of transferring the inheritance.
Nearly all of their inheritance will be intangibles. Most of the inheritance
is education and values. These experiences expose our children to
the ``thrill'' of our values, provide them with a practicum to experience
God's world, and give them a sound appreciation for the value of their
heritage. These experiences, hopefully will be repeated for years
to come, as a means of sharing the same values with their descendants---our
grandchildren and great-grand children. With God's blessing, we will
hopefully help our children to share the same experiences with our
grandchildren, promoting their ``carrying the flag'' further into
the generations.

These are not quick transfers. They are not ``bags of money'', but
character, value, principle, and God-loving promotions, that take
time to develop, similar to the time commitment of a good education. 

\section{Loans}

\subsection{Collection of, Deuteronomy 15:1--3}

When I mentioned to a friend, that I thought we should have a seven
year forgiveness of debt period in America, we both laughed with glee,
thinking of how our suggestions to non-Christians might be received.
Isn't it a shame that we can't? Our financial system is so driven
by debt, that I think that we can call it debt-based. Even our \index{government!debt}national
government is in debt. I just read a statistic last week that stated
that nearly 30\% of Americans were carrying over \$5,000 in debt on
credit cards. About the same time, I saw a sign in front of a used
car lot that promoted the sale of cars, by offering a ``free credit
card''?

It may be appropriate to read Chapter 14, and the remainder of Chapter
15 in this section of Deuteronomy. Please do it with an eye toward
what God was attempting to create for the Israelites, in their new
land of richness. There would be little to no need for indebtedness,
certainly no need for interest. In fact, God expected that they, the
Israelites, would be lending and sharing their riches with the remainder
of the world. Certainly, if any of the Israelites were in need, their
brothers or sisters would lend to them out of compassion, with no
mention of interest. In a land of riches, provided by God, interest
is not required. Similarly, if a loan is made from the riches provided
by God, then it should not be a problem to obey God and forgive all
of their debt every seven years.

Take a few moments to ponder what a cash economy, rather than a debt
economy, might produce: 
\begin{itemize}
\item It would require ``saving practices'' prior to purchases, that would
offer time for research and scrutiny on products---reducing the impact
of false advertising. 
\item It would eliminate the exorbitant interest charged on credit card
debt---a spiraling, compounding debt from which many never seem to
free themselves. 
\item There would be no need for a credit check when purchasing products,
since cash would be the exchange base. Credit checks are simply a
symptom of a trust-sick society. 
\item Loans would not be the norm, but would be requests in times of urgency\slash{}emergency,
and acts of mercy. Loans would be special acts of giving on the part
of one person to another. 
\item There would be fewer financially based scams. We would simply pay
the requested amount for services or products rendered. 
\end{itemize}
My, how we've wandered from God's direction and intent for us. Granted,
we have produced many alternatives to cash, that in some cases promote
improved security. However, in most variances from God's direction
to the Israelites, we have permitted unfair practices to be promoted
and the art of ``money changing'' to reach new heights of deceit. 

I can share with you, that as individuals, we can begin to make a
difference. We can begin to promote God's values in the fashion that
He intended them to be used. For those of you who have been blessed
by God, consider the blessings to be His gifts to you, maybe even
His test to you. Are you worthy of His blessings? 

\subsection{Interest, Deuteronomy 23:19--20}

God directed that the Israelites were not to charge interest to their
brothers. This was forbidden ``so that the Lord your God may bless
you in everything you put your hand to in the land you are entering
to possess.'' Some would say that perhaps this verse and law from
God is no longer applicable. Is it?

What does it mean to be a brother? As an Israelite, as one of the
original tribes, it was quite clear in genealogy as to who were brothers.
As descendants of Abraham, all members of the tribes were brothers
and sisters. 

When I sit in our church and worship God, from time to time I look
around me and feel very pleased and comforted that I'm surrounded
by brothers and sisters in Christ, and surrounded by the souls with
whom I might share eternity. We are children of God, with a common
Father, and all rich in the inheritance of Christ. We are brothers
and sisters. 

Brothers and sisters care for each other. Brothers and sisters come
to each other's rescue when in need. Brothers and sisters pray for
each other when there is a need. They feed one another. They take
the time to ``be there'' in time of distress and disaster. Brothers
and sisters do not take advantage of one another. When there is no
one else to turn to for help, they turn to one another. Do brothers
and sisters in these situations charge interest?

\section{Money}

\subsection{Debt, Romans 13:6--8}
\index{responsibility!debt repayment}
\index{responsibility!money}
\index{money!debt}
\index{commitment!to repay debt}
\index{debt!repayment}

Pay your debts! Pay your taxes! Honor your responsibilities and commitments!

These are not suggestions by God. They are His direct commands. They
are not optional practices that when times are good, you might simply
``think'' about them.
\index{Bible!characters!Paul}Through St.~Paul, God is directing us, no
different than the way an officer in the military issues a ``direct
order'' to an enlisted man.

I have several clients that are forever violating their contracts
with me. In some cases, my contracts call for payment within thirty
days, and some that call for payment of invoices within fifty-five
days. When the contracts are violated, I call the clients to bring
their obligation to their attention. Since they are repeat clients,
for whom we've been working for years, they tell me that they'll be
a ``bit late on the payments'', just for awhile as a means of ``cash
management''. I shake my head in disgust with the way people create
terminology for not living up to their commitment. ``Cash management''
is a term that is not only used frequently, but it is a disguise for
a practice that is dishonest. It is stealing. When commitments are
made, contracts put in place, and agreements understood, a service
or product is provided. Under the terms of the agreement, compensation
should be made promptly. ``Cash management'' does not give anyone
the right to ignore, delay or postpone commitments. 

For some years, my firm has avoided the practice of delaying or stretching
its payments to its service providers. Recently, our accounts receivable
has grown significantly, to the point where our normal ``three month
backlog'' in accounts receivable has grown to ``six months''. For
years, we've taken pride in paying our debts promptly. It became standard
practice. Our service providers used to give us discounts for the
quality and timeliness of our payments. Unfortunately, we've recently
been ``driven to the wall''. For the first time, we've had to contact
our service providers and ask them to provide us with forgiveness,
and we've had to create ``deferred payment'' programs. We're forced
to this condition by our client's lack of honesty. Things have got
to change.

In a recent meeting, our executive staff discussed being more ``selective''
in our clientele. We will be firing some of our existing clients,
even if we have to reduce our revenue and capacity for doing work.
If I wasn't fearful of legal retribution, I'd love to tell you the
names of the clients who are cheating us. You'd be very surprised
since they are the top corporations and institutions in our country.
However, it disturbs me when I realize that nearly all of our long-term
honorable clients are part of our problem. Their business practices
are driven by MBA's taught to ``manage cash''. It appears to be
another deceitful ``money changer'' practice to me. Do folks in
business schools spend that much time creating and then teaching such
disgusting and deceitful practices, and then memorialize them by giving
them an economic name and entity, as if they were honorable? Perhaps
we've identified a focal point of our mission, where words and acts
of honor, value and integrity might be unusual, if not foreign, but
well received? 

Let's all start with basics. Let's begin to listen and follow the
directions of God. It is my belief that if it is our intent to comply
with God's directions, He'll provide us with the means to do it.

\subsection{False Value, Proverbs 11:28}

What is the purpose and importance of money in our lives? I believe
that we all have friends or acquaintances whose primary purpose in
life is the making or management of money or material possessions---houses,
cars, collections, etc. They are probably members of the ``club''
that believes ``he who collects the most toys, wins!'' They do believe
that their salvation is in their material collections. They can live
comfortably within their man-made environment that they've built ``for
themselves''. They have wonderful healthcare services to handle the
``bumps in their physical existence''. Chances are pretty good that
they will give to some popular charities before they die. I should
also point out that these folks are reasonably popular, they have
many well-to-do friends, and they belong to just the right social
organizations. Their focus is on serving themselves, and gaining the
respect of others. They establish their value system on what the world
believes is valuable. Sometimes they're living their lives for what
others think (living through other's eyes), not what they want to
do themselves.

Their golf game is more important than raising their children. Their
house and car are a higher priority than their child's tuition. Tennis
is more important than volunteer work at the hospital. The focus is
on ``me and what others think of me''. 

I know an awful lot about these folks, these friends of yours and
mine. The reason I know so much about them, is that I was (am) one.
In fact, I find myself from day to day, fighting the urge to return
to my ``trusting in riches''. It is a difficult thing to do, this
focus on God and others. We need to work at it. There is almost a
tendency or current that drags us toward focusing on materialism and
self-indulgence. It does require discipline, and resolve. I pray that
we all, with God's help, succeed in placing value on the correct goals
and on the right things in life. 

\subsection{Investment of, Proverbs 3:9--10}

With financial success, we begin to think about what to do with the
extra money. We think of mutual funds, stocks, real estate, and precious
metals. The amount of interest gain is critical. Growth of the investment
is a factor. There are folks who make a living just giving us advice
on how to invest our money. 

These verses in Proverbs give the Christian businessperson an additional
perspective on his or her investments. There is a priority to our
investments that God ``orders'' us to follow. Our ``first fruits''
go to God. It is a sign of our respect, and it honors God. It illustrates
to God that we are aware of the source of our financial success. It
is not our labor, but God's blessing and grace that has given us this
success. 

I can personally testify to the accuracy of these statements. They
are a path to success. I work hard, and God has given me the tools
for success in my business. However, I can also share with you that
without God's direction and intercession, my business and I are going
nowhere. Time and again over the past decade, God has led me through
unknown territory to pastures of success. 

I believe that we all need to recognize this, and to recognize that
we are His servants. As His servants, we need to honor our Master.
In addition to dedicating our ``first fruits'' to God, we need to
recognize that God has blessed us with success, not to make our lives
easier, not to put us in the limelight, but to give us the flexibility
and opportunity to further serve God. ``To those whom much has been
given, much more is expected.''

\subsection{Receivables, Timing, Romans 13:6--8}

Folks have become so focused on gaining interest from their money,
that taking extra time to pay their bills is common. The lack of promptness
in bill paying is deliberate and supposedly justified by the interest
gained while one holds on to another's money. The theory of course,
being that the longer you hold on to the money, the more interest
you will gain from it. My MBA friends advise me that this is ``good
financial'' practice. It is a method of improving ``your cash flow''. 

Have you ever not received a paycheck, compensation, or reward for
work well-done? Most of us have experienced this. Did you ever have
your employer ``miss'' a paycheck, particularly when there were
groceries to buy, tuitions and mortgages to pay? It hurts people to
delay or postpone compensation to them for their work. The delay in
payment affects their reputation, and their ability to pay their bills.
It certainly affects their credit rating, which is somewhat a measure
of their integrity.

When two parties come to an agreement on a transaction, or job, or
contract, or service, the terms of payment sometimes are recorded
and honored. Where I grew up, if the terms of the payment were not
discussed, it was understood that payment would be made ``immediately''
upon delivery of the work or service or product. Before you walked
from a store with a new product, you paid for it. At worst, you agreed
that the bill would be paid at the end of the month as part of your
credit program. 

\index{commitment!to repay debt}
\index{debt!repayment}
In my business life, I have seen and heard most of the approaches
used by folks to avoid paying their bills and meeting their commitments.
``The check is in the mail'' is a common response denoting dereliction
of duty. I've had clients ``cut the check'' and mail it two months
later. I've had clients lose the checks. I've had clients who simply
lie about their payments. I've had clients who just never pay their
bills. I've had to assign a full time person to just ``collecting''
receivables. In my small business, I've had to borrow as much as \$750,000
at a time, just to cover my expenses because my clients do not pay
in a timely fashion (in this particular case, it was delinquency on
the part of a government agency). The reasons that I've been given
for the lack of payment or late payments are too numerous to count.

Don't tell me it is ``good cash management''! It is close to robbery,
if it is not robbery. It is robbery of time, robbery of integrity,
and robbery of a person's life.

After reading these verses, how can Christian businesspersons not
pay their bills and not meet their responsible commitments? ``Give
everyone what you owe him'' is a command. Just do it! It is our obligation.

\section{Philanthropy}

\subsection{Righteousness of, Proverbs 21:13}

From time to time, I hear my partners complain of our \index{government!welfare|see{charity}}\index{government!charity}government
welfare programs, as being give-away programs. I find myself sympathizing
with them. Why do we need welfare? We now need welfare programs primarily
because we, as a society, do not voluntarily take care of those in
need. Care of the indigent must be legislated. This in itself is a
symptom of the sickness of our society.

Giving and caring starts with each of us. As God has blessed us, more
is expected of us. We need to use our wealth to care for those in
need. Our giving may be just the spark that is needed for a struggling
soul to fight his way to an independent life. It may be the money
needed to educate the poor. It may be the assistance needed to feed
the hungry. It may be the medical care needed to heal the unfortunate.
If we don't, who will?

\subsection{Value of, Proverbs 19:17}

Living in the city, I am approached daily by beggars, homeless and
scam artists for ``spare change''. Years ago, when I first moved
to the city, I found the begging to be heartbreaking, and I always
submitted to sharing. Soon I found the same folks doing the same begging,
over and over again. My friends would tell me of some of the scams
that they had been exposed to, and I had recalled submitting to them
also. The scams were mostly healthy and healed people lying about
their need. One fellow showed me a ``fake colostomy'' and asked
for money for a cab to the nearest hospital. 

As a result of these scams, I then developed a hardened heart, and
I would not share. I was afraid of being taken. However, how many
truly needy did I deny?

I don't believe that it is for us to judge who is worthy of our gifts.
God will direct the giving and the receipt of our gifts. He will make
use of it, as we are approached. It is His wealth to distribute. It
is also His world in which we live. Time and again I am amazed at
God's workings in my life and in the lives of all of us. The network
of ``players'' in this life of ours, the fabric of intricate intervention,
and the control of what appears to be an unfathomable complexity of
players, are points of amazement to me and others. Surely, God's intelligence
and organizational skills are much greater than mine, to the point
where I can't even comprehend such a level. I believe that it is our
responsibility to recognize God's greatness in these matters, and
leave ``the results making'' in His hands. His purpose will be served. 

Don't hesitate to give to the needy, to those who beg of you. Let
God worry about how it is being used, and let God direct your actions.
It is in fact, God to whom you are giving.

\section{Planning Financial, Luke 14:28--30}

Yes, these verses were presented by Christ as parables to outline
the ``preparations and planning'' needed when anyone of us makes
the commitment to follow Him. Planning and preparations are also required
by us in our lives of service to God. 

Planning for activities, generally requires the following fundamental
steps:
\begin{enumerate}
\item Assess the available resources that can be used in the endeavor. 
\item Determine the level of resources that will be needed for the endeavor. 
\item Establish a plan to reconcile the differences in resources available
versus those needed. 
\item Identify and test a plan of implementation, on a step-by-step basis. 
\item Confirm your commitment, resolve your plan and move forward. 
\end{enumerate}
These basic steps can be followed for nearly any level of financial
planning. In the case of a building project, with which I am most
familiar, the basic steps are the same in principle as those stated
above, but carry the application of building a structure as follows:
\begin{enumerate}
\item Complete a net worth assessment and statement, which outlines funds
that might be available for the new building project. Also, from a
design perspective, determine the available space and compare it with
projections of what might be needed, to establish the scope of the
building project. 
\item Based upon the projected scope of the project, complete a cost estimate
for the project, with some level of contingency built into the estimate,
to cover unexpected expenses. 
\item Chances are usually pretty good that there will not be sufficient
cash for the endeavor, and some level of financing will be required,
or some assets will be liquidated to provide the cash.  
\item A schedule of activities will be planned, and probably charted on
some sort of time schedule chart (like a bar chart, or a CPM\footnote{Define this.})
that will identify key milestones for the critical path of the project
delivery approach. 
\item Contracts and agreements will be prepared, discussed, reviewed, and
prior to final commitment, a review will be conducted to confirm the
appropriateness and accuracy of steps 1--4, above. Upon review and
acceptance, the project will move forward, never to look back again. 
\end{enumerate}
Yes, these same steps are required in committing our lives to God.
Step through the planning sequence, and assure that the total commitment
is there, and that the resolve to live your life for Christ is in
place. Once the commitment is in place, move forward, and you'll find
no need to ``look back''. 

\section{Poverty, Explanation of, Deuteronomy 15:11}

Poverty is one of those conditions that ``is of the world''. Poverty
is foretold and predicted not only in Deuteronomy, but also in Matthew
chapter 26. Poverty is a condition, and yet it is also a trial. While
poverty will continue to exist in the world, it does not mean that
we are to be complacent about it. These verses in Deuteronomy tell
us not to be complacent about poverty. Poverty, as a trial, has two
perspectives---those in poverty, and those who can do something about
the poverty of others. 

Poverty is a condition into which some of us have been born, or to
which others have been led. Poverty creates a condition of limited
flexibility and strength to recover and fight back. It is a condition
of desperation. It is a condition that results in begging for assistance.
Often it is a condition that lacks hope. It lays open the soul, and
displays a character of complete dependence. ``I can't do this by
myself!'' Some of us have reached this level of desperation, even
without financial problems. Many of us know that God is listening.
Because of God's grace, His vigilance and His promise-keeping, there
can be and should be hope.

As God's servants, we must take our cue from Him. He helps. He reaches
out to the desperate. He fills the hands of the impoverished. With
God, all of us will survive the trial of poverty.

Take the other perspective of poverty for a brief moment---the test
of our response to poverty and those who are destitute, when we are
blessed with resources and God-given wealth. As God provides us with
a means to house our families, and educate our children, we cannot
and must not ignore the impoverished. Exposing us to poverty, and
providing us with an opportunity to respond, is a trial before God.
Ignoring those in need is failure of the trial. We cannot continue
to ignore the impoverished. If it is difficult, we must ask for God's
help, just as the impoverished meet their trials head-on, using God's
assistance. 

I believe that a big part of our mission in doing something to help
the impoverished, is our attitude in our acts. What is the position
of our heart when we help the poor? Are we sincere and compassionate?
. I believe that these acts of giving are placed before us for the
purpose of maturing our hearts and souls. The act of giving is meaningless
if it doesn't come from the heart. 

\section{Profit}

\subsection{A formula for, Proverbs 14:23}

``Nothing ventured, nothing gained!'' Profit requires action, not
just words. Benefit comes from hard work, and yes, sometimes-painful
steps are involved.

My wrestling coach would scream out, ``Two more sprints, two more
ropes to climb, two more sit-ups and two more push-ups!'' Having
reached exhaustion, I was often in tears, but knew that the additional
training would reap benefits in the future contests. The pain was
nearly unbearable, but somehow we all would find a way to ``do more
exercises''. Sports and athletics have many analogies for living.
After all, isn't that the primary purpose and value of sports? Isn't
the primary purpose of sports and athletics to better prepare our
children for life---to expose them to the value of hard work, training,
discipline and teamwork, to let them feel what it is like to win the
highest laurels, and experience the depression of loss and failure
(with limited consequences), to expose them to the pain of sacrifice
and the discipline of training. That is the case in a lot of life.
The pain and sacrifice are what make great athletes and successful
business persons.

We all have friends that talk of doing many and great things, but
they never deliver. Let us not be part of this lot. Let us be people
of action. Let us set our goals and work toward them. Let us strive
to succeed in business, as servants of God. Let us not just talk about
it. Let's do it.

\subsection{Management, Proverbs 16:8}

The temptations of the devil are everywhere in get rich quick schemes. Patience and commitment to righteousness are required to
work toward goals in an orderly fashion. 

When I first read this verse, my notes in the margin were ``are your
products and rates righteous?''. My father used to preach to me often
of providing something of worth to the world when I grow up. ``Don't
just make a living off of folks. If you're simply selling them something,
sell it at just a slight margin. If you mark-up goods too high, it'll
eventually be disclosed, and you'll lose customers forever.'' It
would seem that he was preaching moderation. While I may not have
understood him as a youngster, I can share with you from my present
perspective, that he was correct. He was accurate from a business
perspective, and correct from a Christian perspective. 

The message is moderation---moderation in gain, moderation in fees
and charges, moderation in profit, and moderation in investment. Invest
moderately in things of substance. Think long term growth and foundation
building activities.

Each year at tax time, in our business, we are forced by the tax laws
to make a decision on how much profit to retain (and how much tax
to pay on the profit) and how we should use the profit. Since we are
in a ``low margin'' business, the amount of profits is slim by comparison
with most industries. The importance of how we use our reinvested
profit is critical. Time and again, there is the temptation for the
``get rich, quick'' investments. 

I believe that in these types of situations strategic level planning
is critical. Profit investment, and management toward the benefits
of such an investment should be in accord with your strategic plans.
If you don't have strategic plans for these situations, then you might
be driven toward the ``shallow'' investments. Think of what God
would have you do with His blessings! Think of what Jesus would do
with your profits! Think of the growth, substance, and value of your
company from the perspective of the people involved, the products
or services provided, and the future of the company! Think new products
and services! Think of a better informed and a better-educated staff!
Think of new tools and improved technology for the mission. These
are the things that make sense to me for profit investing.

\section{Salary, Payment of. Deuteronomy 24:14--15}
\index{responsibility!payment of salaries}
\index{salary!payment of}
\index{salary}

There is no excuse for mistreatment of employees or workforce, at
any level. Most of us in our working career, started at the bottom.
Some proceed up the ladder to different levels of supervision. Supervision
means management---at any and many levels. Regardless of the level
of supervision, there are responsibilities demanded of management.
One of these responsibilities is management's treatment and attitude
toward the workforce. The workforce is to be respected. It is to be
treated fairly and timely with compensation for services and labor
rendered. It is not optional. It is a command from God.

Over the years, as I've sat in salary\slash{}benefit committee meetings for
various companies including my own, I've been surprised at the insensitivity
of most people over the importance of not only timely pay, but the
amount of increases on a regular basis. Increases are for the purpose
of keeping up with inflation and to reward growth in ability and productivity.
The insensitivity is not always caused by self-centeredness and greed. 

Most often, I have observed the insensitivity to compensation for
the workforce to be found in ``middle management'' , not upper management,
in corporate America. Middle management does not seem to consider
itself as management, nor as labor, but in some intermediate level
that adheres to the values of neither. Middle management does not
normally have the drive and vision of top management (or they are
self-centered in their drive to be top management), and are normally
incapable of expressing a vision to the staff. Similarly, middle management
does not have the respect of labor that top management has. 

Much of the successful top management generally knows that the value
of the company is in the drive and attitude of its labor force (often
I've found some to have a focus on the stock value and the shareholders---a
path to disaster in the long run). When top management turns over
compensation responsibilities to middle management, it is turning
over a responsibility to a group that does not have an innate understanding
of the strength of the company, and adequate respect for the workforce.
Middle management sometimes doesn't understand the importance of the
mission, and at the same time, they cannot feel the pressures that
the workforce is feeling. They need to feel the pressures.

One last thought on these verses. Can you imagine a fate worse than
having someone in your charge ``crying to God against you''? To
me, this kind of situation is a very sobering, if not a humbling thought.
It may be the kind of ``negative'' prod that we need to keep our
companies focused on righteous missions. Rather than having our Christian
brothers and sisters ``crying against us'', we should be posturing
our management and our goals, to having our brothers and sisters thanking
God for us and His guidance of us. Let's all work toward this goal.

\section{Savings, The Process, Proverbs 13:11}

At the age of fifty, I now realize the wisdom of these verses. Savings
do take time. ``Little by little'', they accumulate to levels where
they can be used for major investments such as homes, college tuition,
and other ``once in a life time'' commitments. I've never played
the lottery, received any significant inheritance, nor did I ever
rob a bank. All of my money in my life has been received ``little
by little'', with hard work. 

I must admit that I've always been blessed with enough income, even
when it was little to nothing. As my wife and I tried to get our life
started (just after my military duty), we were both blessed to find
jobs. We realized pretty quickly, that we could pay the bills from
my salary, so we decided to ``bank'' all of her income, in the hope
that within a few years, we might have the funds for a down payment
on a house. Just as we had hoped, we were able to do such in about
five years, and we began our house search. I share this with you to
point out the concept of time, and how time is integral to the concept
of savings. During those five years, we didn't suffer, but we learned
to be satisfied within our established budget. In fact, we look back
at those years with fondness. 

``Little by little'' is the only means that most of us will ever
have, to generate a savings account. Since the time factor is included,
mentally we have to simply start the process and keep it going. It
is really quite simple, but it requires a bit of analysis. Start with
itemizing your expenses, and identifying which expenses are not necessary.
Mathematically, put those excess expenses aside as savings. Savings
must become a ``habit''. Similar to a long, hard and enduring task,
it doesn't make a lot of sense to keep monitoring the situation. Just
make the payments to the savings account on a frequent and regular
basis, and when the calendar alarm ``goes off'', enjoy your accomplishments.
Or as my father used to say, when we were in the middle of a long
hard project, ``don't look up, keep your head down, and keep working
until it is done---just get it done!''. 

Was it Ben Franklin that was quoted as saying a ``watched pot never
boils''? So it is with savings accounts.

\section{Taxes, Payment of, Romans 13:6--7}

I remember the look in the eyes of my parents when I told them of
the taxes that I had to pay after I moved from the country to the
city. They were flabbergasted at how much more the taxes in the city
were, than in the country. As a matter of record, the taxes in the
city were 2000\% more. We spent many hours discussing how the money
was being used, and conjecturing on the amount of corruption obviously
involved, since the roads were in poor condition, the water tasted
funny, the sidewalks were in terrible shape, snow removal was a joke,
and the crime level was well-beyond control. Of course, much of the
basis for our discussion came from a perspective of ignorance. We
had not been exposed to the ways of the city, and the execution of
the social programs that are required when dealing with such dense
populations. 

Over the years, I have become more familiar with the various ways
that my tax dollar was being spent, and in most cases, there was little
corruption. In fact, there always seems to be a need for more dollars.

When I was sharing the extent of the services with my country based
family; I was amused when my father added, ``With all of those social
programs, no wonder city folks don't take care of each other.'' 

Regardless of how we feel about the taxes, the use of the taxes, and
the potential for corruption, we still must pay our taxes. It is even
more righteous to pay your taxes without complaint or delay in this
democracy of ours. Remember that we have the vote and the right to
select and direct our government through its constitution. All too
often, our reluctance is based upon ignorance, or greed. 

In any event, or under all circumstances, our obligation is defined.
Pay your taxes, and meet your obligation as a citizen. 

\section{Wealth}

\subsection{A Perspective, Luke 18:18--25}

Having read these verses, may I ask you the following questions:
\begin{enumerate}
\item Is Jesus advising that a wealthy man cannot enter the Kingdom? 
\item Is Jesus advising that a man must be broke and destitute to enter
the Kingdom? 
\item Is Jesus establishing a financial ``threshold'' over which we should
not exceed, for fear of not entering the Kingdom? 
\end{enumerate}
I don't think so! Please jump to Matthew 6:19--21, with special emphasis
on verse 21. I believe that Jesus is advising us on our focus, our
obsessions, or as the psychiatrists say, our ``cathecting'' of material
things, wealth, and money. Where is our heart? 

I've had three close friends lose their businesses, homes, and assets
in the past ten years, due to poor business decisions on their part.
I am still fascinated by the metamorphosis in their character and
values brought on by the experience. The same can be said for their
family. The phases of change in attitude, accompanying their trail
to bankruptcy, and eventually spiritual wholesomeness included the
following:
\begin{enumerate}
\item As their businesses started to fail, they were tempted to not pay
their creditors. They were able to begin thinking for themselves,
to ignore, if not overcome the ``deceit-laden'' advice that others
provided. Some of this advice included lying, illicit practices of
deceit, and other practices less than righteous. Instead, rather than
following the easy path of deceit, they stood firm and took a stand
of responsibility. 
\item As they started to lose their homes of many years (in one case the
center of their life), they had to face the ``importance'' of material
things in their life---or as they've come to call them the ``things
of life''. With proper adjustment in their hearts, their focus in
life changed from materialism, to the wholesome and righteous if not
spiritual aspects of life.  
\item At the financial bottom of the experience, having lost everything
of material substance, they experienced a ``sense of relief'', of
``pressure off their back''. In all three cases, my friends asked
for my forgiveness for their arrogant attitudes, and their flaunting
of their material successes in the past. I felt as if God had removed
the cloaks from their bodies and souls, and for the first time, I
was meeting my friends. 
\end{enumerate}
On the rebound, as each has formulated new businesses(all are self-employed),
their businesses are driven by God, His rules, and there is an apparent
mission. Some of the characteristics of all three businesses include:
\begin{enumerate}
\item They are cash-based, non-debt managed businesses.  
\item The offering of the first fruits of income, are celebrated fervently. 
\item The staff of each business is characterized as fiercely spiritual,
and hard working. 
\item The businesses have a stated mission, which is not laden with financial
goals, but noble goals. 
\end{enumerate}
Yes, Jesus was speaking to us, businesspersons, in these verses in
Luke. Our primary focus, purpose and obsession can not be money and
financial success for its own sake. If it is, we are in for a hard
time of it. I pray that we continue to base our purpose, our mission,
and our ``budgets'' on God's goals, driven by His hands, not our
own. I pray that all of my brothers and sisters be blessed financially
by God to the level that they need, and to the level that they can
invest in noble, righteous and God-centered ventures.

\subsection{Disposition of, Psalm 49:16--20}

``You can't take it with you!'' When I first read these verses,
that was the simple message that jumped off the pages for me. However,
like most of the verses of the Bible, each reading, with different
emotions, with different perspectives created by recent experiences,
creates new meanings and directions.

On my last reading, the following messages hit me: 
\begin{enumerate}
\item On this earth, we people put many social and peer pressures upon one
another. One of the most common is the desire for more and more wealth.
I even find myself ``bragging'' or ``talking-up'' someone's wealth,
and the extent of that wealth. Just yesterday, I used the term that
``he was wealthier than the Federal Reserve'', in describing one
of our clients. The phrase was used in a bragging fashion, as if we
all should be humble in front of such wealth. What a mistake! I'm
such a sinner! 
\item I am reminded that as we ponder these points of wisdom, our perspective
is critical. I am forever asking my children to take a universal perspective
on things. ``Pull yourself away from the world into space, and look
at life through a ``snooping'' telescope. Most proceedings seem
insignificant''. Similarly, there is a universal perspective that
includes time. As an example, astronomers talk of the universe and
its evolution from a distant perspective, and in a fashion where 50,000
years, in comparative scale, is a brief moment. As we make our decisions,
particularly financial decisions, we must think of the impact of these
decisions on both a short and long term basis. Long term, may be well
after our death. What was the impact of our life? Certainly, our financial
wealth will have little meaning to the proceedings of the world, when
you project 300 years into the future. So what should you be doing
with you wealth?

\item Wealth for the sake of wealth is wasteful, decadent and sinful. Wealth, whether inherited, earned or won, should be converted into things of action, missions with purpose, and acts of righteousness,
in God's eyes. Invest in education, in experiences that build wisdom
and character, in programs that give people hope, nutrition and livelihood,
in environments and habitats that support God's mission, and in other
such proceedings that are within the parameters that God would support. 
\item If the Lord blesses you with wealth, remind yourself that He is expecting
``much more from you''. Put your wealth into action in His behalf!
Don't just think and talk about it! Don't just sit on it for a rainy
day! Make things happen in the world! Make life better for the low
in spirit, for the destitute! Use your wealth to build ``hope''
in the eyes of the downtrodden! Teach those who are uneducated! Feed
those who are hungry! 
\end{enumerate}
If you are a Christian brother or sister, you will know what I mean
when I suggest that you invest your wealth to expose the world to
the workings and mystery of God. Let them marvel at His workings,
His majestic ability and His kingdom, as we do.

\subsection{Goals, Luke 12:32--34}

What do you value in your life? With regards to wealth, do you value
the accumulation of money, at least from the perspective of reaching
a level of income that is comparable to your friends and adversaries?
Or do you value what wealth can do for you? 

As most of us have to work for a living, from time to time we try
to determine what it is that we're doing and why we're doing it. We
are setting goals for our long-term development. Sometimes those long-term
goals include levels of financial income or compensation. I believe
that it is fair to say that in our society, nearly every formal station
or position in most industries includes a salary range that the marketplace
can bear. The compensation levels are established to attract the qualified
person to fill those positions. If they are not what we call ``competitive
compensations'', we will probably not compete well in the marketplace.
Usually, when we think of jobs and professions, income is categorized
with the position. It's quite normal. 

However, as our Savior warns us; the income should not be the primary
focus nor the priority for a job. Our ``treasure'' should not be
the compensation for the work, but the enjoyment, satisfaction and
contributions of the work. These should be our ``treasures''. With
these treasures in mind, our heart will drive us to great levels of
success, at which point the compensation will flow. It will flow to
the level that is required to meet your needs and satisfy you. 

This is a message that I share with all of the many engineering students
that I counsel. When they ask ``what major to take in college'',
I respond with these points of wisdom. Find your niche in God's world.
What is that excites you, even on a rainy Monday morning? Find that
niche! Let your emphasis in your careers be on doing well, advancing
our society and the way it does things, teaching others how they can
be better, and make what it is that you become a ``profession''.
And never, never, never, make career decisions based upon the average
income of one career path versus another. Such decisions have led
to the biggest career mistakes I've seen in my life. 

Set your career sights on a path that warms the desires of your heart,
from the perspective of enjoying the work. If you like the work, you'll
do it well, and you'll probably excel in that field. With your heart
in this posture, God will compensate you well enough.

\subsection{Purpose of, Luke 16:13}

Financial wealth is not an end in itself. It is simply a tool, a means
to another end(s). 

Following are some basic beliefs that I've adopted from a friend,
a successful Christian business person, that have served me well in
business:
\begin{itemize}
\item What we charge for our services and products should be a ``fair''
price. 
\item Profit or margin in our pricing should be modest. 
\item Earnings should be invested in developmental and growth activities
that make the world a better place to live. 
\item Always keep in mind the Source (God) of our wealth, and look to the
Source for how the wealth is to be used and its impact. 
\end{itemize}
Wealth is a blessing, not an achievement. Wealth bears with it, the
responsibility of reinvestment in God's mission. When we are blessed
with wealth, we should remember that God trusts us and expects us
to use His wealth wisely, as His servants. As wealth permits us the
flexibility to do new things, we must recognize that we need God's
help and direction in its investment. If we simply trust in our own
desires, we may and probably will not use God's wealth to His advantage,
but to ours. It might become the purpose of our existence, which is
a ``dead end street''. 

I pray that all of us not forget the purpose of wealth, and that we
continue to serve God with His blessings, not ourselves, and I pray
that none of us serves money and wealth.

\subsection{Source of, Deuteronomy 8:17--18}

This past year\footnote{What year was this? Or perhaps you would like to say: `I remember one particularly profitable year for my company \dots{} etc.} was a very profitable year for my company. At our recent
leadership meeting, I was met with the applause and the pride of my
partners in celebrating their accomplishments. Being human, I became
caught up in the celebration, and for a while, I forgot that our accomplishments
were the blessings of God. I must admit that it is quite easy for
us to forget the source of our wealth and profits.

Just yesterday, I was making a presentation to a client for new work.
The client is a Christian. In our discussions, we began to discuss
the source of our mutual success. There was much mundane talk of angles,
services, promotions and other worldly things. We paused and looked
at each other and he said to me, ``Come on, we both know the source
of our success. If there's any doubt in our minds, just look back
at how we obtained our last ten commissions. They came out of nowhere!'' 

Steven's ``nowhere'' was indicative of our human floundering. We
attempt to manipulate the marketplace, try marketing schemes, and
focus our marketing and sales in the profitable arenas. ``Nowhere''
means that the work came from marketplaces and clients that we didn't
attempt to influence through our marketing. ``Nowhere'' is Steven's
way of saying, it came from God, in spite of our efforts.

\subsection{Spending\slash{}Distribution, Luke 16:1--9}

I have had to read this parable over and over to gain the various
truths that it expresses. I suppose that I am as ``dense'' as most,
and yes, I am probably frustrating my Master, just as the disciples
once did. My first impressions were to despise the lack of honesty
with the Master's manager. He was at first giving away the Master's
rightful debt. After my third or fourth reading, I suddenly saw something
else that ``struck home''.

Verse 9 is very direct, if not an order to us. It specifically tells
us to make good use of our wealth, which our Master has given us.
It also tells us to use our wealth in solving the ills and pains of
others. However, who and what are the ``eternal dwellings''? I do
not believe that Jesus was really advising us to ``buy friends'',
but to focus on investments and use of our wealth in behalf of what
might be our eternal destiny. Specifically, the spending and distribution
of our wealth should be focused on serving God. 

Serving God with our wealth might include: 
\begin{itemize}
\item Educating those who cannot afford it (scholarships); 
\item Housing those in need; 
\item Feeding those who can't feed themselves; 
\item Healing those who are lacking health; 
\item Providing work for the unemployed; 
\item Supporting God's work through missions and the church; 
\item Providing comfort and diversion for those in need. 
\end{itemize}
There are many ways for us to distribute our Master's wealth. Our
hearts and souls must guide these investments. Our hearts and souls
are the ``receivers'' of God's words, our communication method is
prayer, and our point of reference is the Word of God.

\subsection{Spiritual, Luke 18:29--30}

Spiritual wealth will offer eternal compensation. 

Time and again I listen to the pains of Christian brothers and sisters,
when they ``cry out'' over the lack of belief and faith of a close
family member. There is always genuine despair and concern for their
loved relatives. ``Why won't they just listen and give it a chance!'' 

My experience is that there will always be these types of anxieties
with our loved ones. As individual souls, each of us must personally
step up and accept God. The \index{commitment!to God}commitment to God must ultimately be an
individual act. As brothers and sisters, we can enjoy many joint activities.
Finding Christ can be one of these activities. Accepting Christ, is
an individual act. It is an act that is the difference between ``life
and death''. We love our friends and relatives, but they must make
a decision. We cannot do it for them. God gives them the right to
decide for themselves. 

All that we can do is be persistent and consistent in our sharing
of the Word and our beliefs. While we should not give up hope for
their souls, we must realize that after we ``are saved'', we are
different. We have left our non-believing brothers, sisters and parents,
if that be the case. Our perspective on the universe, its purpose,
is different than those of the world. We should not give up, and we
should pray for our loved ones. We may be blessed to have them catch-up
to us, spiritually. 

One of my daily prayers is for my loved ones, in which I ask God to
``enable them, and guide me to guide them, that ultimately they might
become worthy servants in Your eyes''. 

Several of my closest friends have entered either the ministry or
the mission fields. I've attempted to spend as much time visiting
them as is possible. Unfortunately, their commitments, along with
mine on the homefront, don't permit us to see each other. I believe
that in some cases I may not see them again, in this life. However,
my prayers are with them, I know that we are eternal brothers and
sisters, and eventually I pray, we will spend eternity together. 

Just as little birds leave the nest to live their lives, we also must
consider similar breaks in bonds to serve our Leader in our own missions.
Our love for our family does not diminish, but we must recognize our
obligations. Our wealth is in spiritual trinkets and spiritual currency.
This is a wealth in which we can languish and fill our purses. 

\subsection{Unjustly Earned, Jeremiah 17:11}

Time and again you've been told of folks who unjustly earn wealth,
premature advancement in their careers, and other unearned laurels.
These people are often the envy of the rest of us, but should they
be? In fact, if it is unjustly accumulated, can I really use the term
earned?

Most people are not prepared nor equipped spiritually to handle unjustly
accumulated wealth. It becomes their downfall. They not only squander
the wealth, but they most often let it destroy their lives, their
satisfaction with their life, and they quickly lose sight of the purpose
of their existence. A good analogy of this condition is the cause
and effect relationship of premature advancement in business.

Early in my own career, I was warned by my sages in the profession,
not to look for money in my career, but to look for career building
education, that would lead to my own personal value in the marketplace.
With this increased value, delivered by me at a confident and comfortable
ability level, the marketplace would compensate me justly. Similarly,
as a supervisor, I should advance people in their roles and compensate
them, at a paced and stable level. Over advancement, and unusually
high ``pay raises'', lead to a focus on money. With focus on money,
there is the desire for even more money---a never-ending thirst for
more.

I've watched many folks who ``grew too fast'' in business and the
corporate world. They reached positions beyond their just value and
beyond their capability (``The Peter Principle''). In nearly every
case, they destroyed themselves with stress and greed. The stress
developed because they were put in a position of responsibility beyond
their ability. They had little confidence in their ability, and certainly
no experience based understanding of the events occurring around them.
They worried continuously of the unknown. The greed was developed
because of the ``excess wealth''. Their focus was shifted from their
personal development of their capabilities to a new focus. The new
focus was their quest to secure more laurels and money. This shift
in focus has destroyed many people, and will continue to do so, as
these verses teach us.

These verses tell us to pity those who unjustly gain wealth. They
also warn us to beware of these perils in our own lives.

\section{Windfall, Easy Money, Proverbs 13:11}

My stockbroker friends tell me that on Wall Street, ``Pigs die, and
bears get rich!'' There's a lot of merit to this statement, as I
can personally attest.

While I'm not certain about the exact intent of Proverb 13:11, ``dishonest
money'' comes to my mind as ``unearned'', ``unexpected'', ``unwarranted''
and ``unjust'' money. It seems to me that ``windfall'' money falls
into this category. However, my further thinking leads me to believe
that there may be some cases where ``easy money'' may be a blessing
in disguise, or perhaps an answer to a prayer. I think that it is
important to focus on our perspective of why God blessed us with the
income. 

Income of any sort, should be valued, not ``dwindled''. Income is
God's means of blessing us, compensating us, and feeding us. I believe
that we are to value God's gesture, and give the gesture respect,
thanks, and serious deliberations on what it is that we're to do with
the income. While most of our income in this life, comes in small
and hopefully, regular amounts for our work, there may be times when
we receive unusual amounts. When this does happen to us, there is
a natural tendency to ``squander it'', ``splurge'', and eventually
waste it. Beware! ``To those whom much has been given, much more
is expected!'' 

Financial responsibility does not include wasting God's gifts to us,
but to make good use of the blessings. Windfalls should be approached
with deliberation on how we might best serve our Master with this
blessing. It is indeed, a test for His servants. How well will we
perform as His servants?

\chapter{Marketing and Sales}

What are marketing and sales?

Some would define marketing as the act or procedure of bringing goods
or services to the marketplace in preparation to offer the sale of
such. Marketing are activities that permit the sale to be made, or
that promote sales.

Sales has been defined as the actual transaction of exchange or the
agreement to exchange goods and services in exchange for money or
services (bartering).

In a free economy, marketing is generally required prior to sales.
Marketing would include advertising, communicating directly on the
merits of goods or services, and educational programs on the value
of the goods or services, all of which may be needed for the sales
to occur. Marketing also includes such subroutines as name recognition
programs, focus research, and response research. Marketing can include
advertising, with many different types of messages. Marketing sets
the stage for sales, and in many instances, creates the necessary
image for the product or service to be sold.

In the USA, marketing is needed in nearly every business and marketplace,
and by nearly every ``seller'' in order to reach nearly every ``buyer''.
The extent of the marketing may be driven by: 
\begin{itemize}
\item the level of competition; 
\item the level of ignorance in the marketplace of one's services and goods; 
\item and, the extent of communication with ``buyers'' that might exist
through previous relationships. 
\end{itemize}
In some cases, such as retail product markets, marketing is a big
portion of a business budget. Billions of dollars are spent annually
for this marketing. This investment becomes the financial base for
other businesses, such as the television industry. The amount of marketing
effort is most often directly proportional to the competition in the
marketplace.

New products and high technology products and services are often not
understood by the consumer. Marketing of these products and services
often requires education of the consumer or buyer, to justify their
need, to establish their value, and to create a ``sense of desire''.
The educational approach to marketing can be ``straight forward'',
indirect, or even a prompting of one's own conclusions. The value
of the services and goods must be transmitted in the educational message
or process.

Some would say that they do no marketing in their businesses, but
simply answer the phone from past clientele. Let me suggest that they
did market with their client in the past, and that they are now reaping
the benefits of the previous marketing. Perhaps it was the extra effort
in the way they delivered their service, the extra maintenance and
responsive service in the repair of their products, the friendly way
that they posted messages to their client, or some other form of contact
and communication. In a nutshell, marketing is ``results-driven''
contact and communication with buyers and consumers.

Let me be bolder by suggesting that all of us are involved in marketing
and sales, in nearly every aspect of our lives. Most people in this
world have been given the freedom of choice in most matters. In order
to ``influence'' people to make the right choices, we market and
sell them in some of the following ways:
\begin{enumerate}
\item We market and sell to our children in their choices of behavior, sports
selections, college and career selections, and even in how they dress. 
\item We market and sell to our friends when we invite them to join us on
a recreational outing that they've not experienced before, and might
be a little reluctant to attempt---skiing, sailing, theatre, etc. 
\item We market and sell to our friends when we politically campaign for
candidates and platforms. 
\item We market and sell to our friends and fellow citizens when we participate
in community action groups. 
\item We market and sell to others when we attempt to raise funds for charities
and other non-profit organizations. 
\item We market and sell to our spouses when we want to do something as
a couple that we've not planned or done before. 
\item We market to others when we share our joy of God (the final sale is
by God).
\end{enumerate}
Marketing and sales are a big part of all of our lives. As a professional
engineer, our code of ethics does not permit us to formally advertise
our firm and our services. However, this limitation does not restrict
us from marketing our services to focused clients. Our techniques
are just different, but they are a real, demanding investment of time
and money, and they must be effective for us to survive and prosper
in business.

I am sure that there are limitations in all of our marketing activities.
I believe that it is critical to know what our limitations are, respect
those limitations, and develop honorable and righteous approaches
to marketing. I also believe that marketing activities are a never
ending quest, constantly changing as with most marketplaces, and requiring
new methods for communicating with the buyer.

The Bible, with its marketing and sales principles which ``jump off
the pages at you'', is a business winner. It can help you win from
the perspective of dynamic, long-term, and deep-rooted success, in
marketing and sales. The success of our Savior and His disciples,
as recorded in the New Testament, is proof. His church, which has
been growing for 2000 years, is testimony to the value of the marketing
and sales principles that you are about to review. While I have attempted
to share the meaningfulness of these verses from my perspective, I
only hope that you develop your own view, and take the ``truths''
to greater heights of success and glory.

\section{Advertising}

\subsection[Delivery]{Delivery, Proverbs 27:2}


He's ``ringing his own bell again'' was what my mother used to say
when she caught me bragging about my athletic accomplishments. Not
only did her comments humble me, but it also ``discredited'' what
I had been telling my friends and relatives. When most folks sense
immodesty, and perceive conceit, they generally ``turn off'' their
listening of the speaker and discredit the message as self-serving.

I was told today by one of our marketing folks, that in interviews
with executive management of corporations and institutions, one should
always be prepared to strongly, and confidently, express one's personal
ability to accomplish the task, and not to always be totally modest.
I can't argue the point, that when we are claiming to be up to the
task, we should be prepared to state our willingness, and yes, it
should be communicated in a strong and confident fashion. I do not
consider this delivery to be immodest. Frankly, it states a positive
level of willingness and commitment. But let me suggest that this
``forward'' approach should only be considered as a response to
a question, and it should not be considered as part of an advertising
campaign, or as part of an information campaign.

When advertising, when personally presenting your company's goods
or services, or when providing testimony in favor of a product or
service, the presenter and his integrity will be judged. This is quite
common in today's society, and it is one of those sins that is so
easy to do---judging others. Let's face it, with the extent of people
manipulation, the lies that are thrown in our direction every day,
and the basic evil surrounding us, we need to listen intensely and
to be critical of what it is that we hear. It is for these reasons,
that I believe most folks become critical of advertisers. It is for
these reasons, that Proverbs and I recommend that you consider having
someone else---an independent reviewer, an unimpeachable source,
and an honest reference to your good products or service,``ring your
bell'' for you.

Note that when someone goes to court, character witnesses are generally
not relatives. The next time you watch TV commercials, note the number
of times that independent references are used---the man on the street,
the successful user of a product, an award for excellence or safety,
etc. It is almost as if the manufacturer of the product or service
is saying, ``don't believe me, just listen to our satisfied customers''. 

In our professional services firm, we often have to make presentations
of our credentials and experiences in order to be selected for new
work. During these presentations, which we call interviews, we've
standardized introducing one another, so that we can take a few moments
to promote each other's values to our client. When sharing information
on our past projects or accomplishments, we've noticed improved communication
with listeners, when we present the projects with ``non-involved''
persons, who talk about the project and its success without a question
of modesty. 

Some things never change, and the wisdom of this Proverb is one of
them. Modesty is still an admired trait. 

\subsection[]{Integrity, Luke 14:1--11}


My old boss used to say ``you have to get their attention, before
they get the message''. While I can accept that theory, particularly
with the youngsters of the world, I believe that many folks have a
tendency to take the spirit of ``getting their attention'' too far.
In some cases, I've been exposed to advertising schemes that make
``outlandish'' claims of performance, as part of the ``attention
getting'' introductions. In some cases, I'd call the claims simply
lies.

A few years back, a popular car manufacturer was using comical adds
in which the typical car salesman made outlandish lies about the cars
and their performance. It became such good comedy, that it didn't
offend folks, but it struck me as a sign of the times. I personally
enjoyed the commercials. I believe that the parody was enjoyable,
while teaching the need for integrity. And the message was, don't
believe the claims of others, look for folks of honesty and integrity,
and research auto performance.

Let me suggest that there are some universal truths and principles
involved here, and I believe that the summary of the truth is in verse
11. ``For everyone who exalts himself will be humbled, and he who
humbles himself will be exalted.'' Note that this truth, as stated
by Christ, involves at least two people. The two parties are the ``humbler''
and the ``exalter''. 

The second party, is the one receiving the message, or the one in
charge of making a decision about accepting the information, the action
or the gesture. The second party or the recipient is the party in
charge, since it seems to have the power to humiliate. In advertising,
it is the targeted consumer. It is hoped by the advertiser that the
potential consumer will accept the integrity of the claims, and proceed
to buy the goods or services. If not, the business will probably not
survive.

For those of us who believe, let me suggest that the second party
might also be God. God is the judge of our intentions, our integrity,
and the Party who will humble us or exalt us. God is working throughout
all of business today, as I believe that business is a testing ground,
a place of trials, to test and ``fire cure'' Christians. Let me
suggest that if you have any doubt as to the integrity of your advertising,
whether for your company, or for your own personalized services, take
the advertising to God in prayer. God will clearly show you, whether
the advertising has integrity. He will direct you through your conscience.

Posture, Luke 18:9--14:

Bragging about oneself, or one's products or services in advertising
may produce ``flashes'' of success. In discussing these verses with
other business friends, our belief is that the short-term success
is just that. In fact, self-boasting and over-bragging of one's products
can do long term harm to sales success. 

Over the last three decades, I have taught engineering courses at
the local universities. At the start of each semester, I would give
the students a brief biography of my background, and the perspectives
from which I was teaching. I taught them that I felt that it was important
to understand the perspective, experiences and posture from which
anyone was sharing information. The benefits include:
\begin{enumerate}
\item Understanding the limits of one's experiences, to the extent that
there may be more to the situation than is being shared by the speaker; 
\item The balance of theory and real-life experience in the perspectives;
i.e., how much did the speaker really experience first hand; 
\item The general educational background of the speaker, and the speaker's
prejudices, if any; 
\item The speaker's exposure to life, the breadth of the speaker's experiences,
and society's acknowledged awards to date. 
\end{enumerate}
I would tell my students that in the ``quest for the truth'', which
is the lot of an engineer, they should be critical of the sources
of information, including teachers. I know that I'm not alone in promoting
this approach to be mindful of the source of information. Most educators
are in agreement on this point. So in summary, we can believe that
most ``educated folks'' are taught to be critical of the source
of information.

In addition, I believe that most folks are cynical or distrustful
by nature, due to the rampant dishonesty and deception in the world.
We are dealing with reasonably cynical minds today. 

Therefore, does first person ``testimony'' of products or services
make sense in a doubting world, that by education and exposure to
evil, scrutinizes the source of our message? I'm afraid that ``bragging''about
our own products, services and abilities, is not effective advertising.
Advertising in the ``first person'' will be seen as immodesty---not
to be trusted, possibly self-serving. 

May I suggest that you consider ``going with the flow''. In other
words, let your Christian beliefs shine. As a Christian, you've probably
already learned how to be a servant to others. Being a servant of
Christ doesn't mean that you withdraw from society, not permitting
your voice to be heard. Quite the contrary! 

Being a servant of Christ means that your words should be from your
heart, always truthful and God inspired. Keep your focus on Christ,
as His servant, and prepare your advertising approach from that perspective.
As His servant, you've got to believe that your message will be heard
and received with greater success than under the strength of your
own abilities.

\subsection{Principle, Luke 11:33}

I once had a client tell me that our engineering firm was the ``best
kept'' secret in our region. While on one hand, it was a compliment;
it was also a testimony as to the ineffectiveness of our marketing
programs---if they really existed. I believe that if you have a worthwhile
product, idea, service or whatever, you should share the good news.
``Sharing'' it through advertising, preaching, writing, or simply
``putting it on its stand, so that those who come in may see\dots  .''. 

However, in today's world, is everything that is worthwhile, advertised
and held in high esteem? No, of course not! In a similar fashion,
is everything that is advertised worthwhile? No, of course not!

Upon pondering these statements, my engineering firm's marketing has
been driven by analysis of what is virtuous, valuable, and righteous
in what we do for folks. Once we properly define the perspectives
of our clientele, and how they most value our services (market analysis),
we ``formalize'' these aspects of our service (giving them entity
and substance), and promote them through our discrete advertising
campaigns. This is putting our lit lamp on its stand, rather than
keeping it hidden. Because of our limitations in advertising, we also
share all of these points of value with our staff, encouraging them
to share these values in the third person approach to clientele, both
existing and potential. 

I believe that we all know that in these verses, our Savior was referring
to the sharing of the Word. In other words, once we are able to understand
the value of the ``light'', we should not keep it a secret, but
share the Word, putting it on its stand for all to see and hear. I
also believe that there is a strong parallel to righteous businesses
that are founded upon Christian principle. These are businesses that
are strongly contributing to making God's world a better place. The
products and services of these companies are ``to be admired and
desired'', and should be put on a pedestal so that all can consider
them. 

Let me ask you to evaluate the righteousness of your products and
services, and as you find them worthwhile in God's eyes, put them
on a stand, sharing with the world their value, their worth, and their
righteousness.  

\subsection{Spokesman Strategy, John 4:4--42}

Our Master, Jesus Christ, was the most effective marketer to ever
walk the earth. These verses in John, tell us of some of His techniques
and methods, and they are certainly worthy of analysis. They are worthy
from the perspective of following His path, and these methods do work
in business. 

Each and every person that we meet in life is a potential ``spokesman''
for our services, products and messages. Notice that Jesus started
this event with the Samaritans, by ``teaching'' one woman at the
well, and then letting her spread the word to gain the attention of
many other Samaritans. Then, the other Samaritans came to Jesus and
received the Word directly. The approach seems to be a simple one
of multiplication, but there is more to the method than simply having
one person spread the word to others.

Jesus spent some of His precious time, from His short life, to ``impress''
the Samaritan woman at the well. He taught her, He chastised her,
and He shared with her His power and insight. Note that He did not
send her off to spread the word. He so impressed Her that she naturally
wanted to share His existence. He didn't say, ``Now, go tell everyone
of My wondrous existence and My ability.'' Instead, He let her come
to her own conclusions. In effect, His words and His actions ``prompted''
her conclusions. With her own conclusions and impressions, she was
inspired to share His existence and power. So it is with all successful
marketing programs. Let the human mind work. Let people come to their
own conclusions. Give them the logic, the values, and the reasoning.
And also, let God work for you, and then let the humans share their
excitement of a ``new found'' power, method, product or service.

Jesus introduced Himself to the Samaritans through His ``spokesman'',
the woman by the well. While she spread word of His existence and
power, they did not take her word, but had to come see for themselves.
Note that they then stayed with Him for two days. 

I target 30 minutes of interview and contact time in my business,
and if my personal interview reaches thirty minutes of listening,
then I consider the contact a success. I also use Christ's methods.
I am presently leading a marketing campaign in a new geographic region,
to sell our engineering design services. I started with a ``name
recognition'' campaign, by sending postcards with our name on it
to over 400 potential customers. One month later, I followed up on
this initial contact by sending letters and brochures to the 400 companies---to
the CEO's and CFO's, hoping that they would become spokesmen for us.
I will then hire an independent spokesman to make the initial contacts,
simply requesting that the potential clients ``hear me out''. From
experience, the effectiveness of the campaign will depend upon how
well I educated and inspired the spokesmen, just as Christ inspired
the woman.

Please revisit verse 42. Like most people, the Samaritans were impressed
with the words of the woman from the well, but they had to hear it
for themselves. This act and the time allotted for it, are the goals
of my marketing. If I can just spend one half-hour with them, they
don't have to simply rely on my spokesman's words. Words from another
are sometimes fleeting. Wisdom from first hand sources lasts longer
and is more meaningful. Successful marketing should include a testimony
from the source. I wish I could have been there, to share those two
days. ``..,and we know that this man really is the Savior of the
world.'' While we can't be there, Christ is with us. His spirit fills
us. We are there. We've heard His words, and seen His actions. Let
us be His spokesmen.
\begin{enumerate}
\item Let me summarize this referenced marketing methodology:
\item Identify your spokesman, and train, inspire and commission your spokesman. 
\item Let your spokesmen do their job, of sharing the information in your
behalf. 
\item Follow-through with the recipients of the spokesman's message, presenting
the facts yourself, to confirm the message and the truth of the message. 
\item Make the sale possible! 
\end{enumerate}

\section{Appearance, Image, Proverbs 15:30}

\label{appearance-image-proverbs-1530}

Take the ``high road'', and market your goods and services with
joy and pleasantness, and be of good spirit. Focus on the merits of
what you're marketing, its value and its benefit. Do not use negative
marketing of your competitor's goods and services. Life in this world
has enough ugly moments and situations. There's plenty of bad news
to go around, already. You do not have to add to it. In fact, ``good
news'' and joy are generally a well-accepted anomaly. 

A close friend and client of mine is constantly buying services in
the design business. He shared with me that one of his trick questions
is to ask marketers what they think of their competition. They lose
the possible sale and the relationship, if they offer a negative response.
While you may be objective, your first words should always be those
of respect.

People listen to those who have hope and a good spirit about themselves.
Be positive, light of heart, high in integrity, and honest. To gain
respect for what you have to say and present, you must win the heart
and ear of the listener first.

\section{Client Relations Principle, Proverbs 23:1--3}

Most often when you meet with a client, the client has had prior knowledge
of the products or services that you represent. Hopefully, the client
already has a positive perception of your products or services at
this point in the relationship. In most cases, it is your responsibility
to simply reinforce the positive perception. Do not let your actions,
words, or methods detract from this positive perception.

On the marketing and sales ``trail'', there are many opportunities
to share the marketing message, and to reinforce the positive aspects
of your services or products. The opportunities can come at mealtime
(as these verses relate), on the golf course, and through many other
``contact opportunities''. The successful marketer must always recognize
his or her responsibility, and maintain self-control. Exemplary and
outstanding people succeed in marketing by focusing on the good in
their services and products.

There is a tendency in all of us to ``drag others down'' to our
level. ``Misery loves company.'' On the marketing trail, you'll
meet many clients that will attempt to share their low values. Perhaps,
these low values are some of your weaknesses. Resist and abstain from
these tendencies and temptations. Maintain your cognizance and control.
Let your services and products sell, with your exemplary conduct and
character. While there is a need to understand others' plights and
perspectives, maintain your values in your personal conduct. Good
people are in the minority in this world, and they are generally admired
and respected. 

\section{Faith, Example of, Luke 7:1--10}

Even Jesus acknowledged the ``great faith'' of the centurion. The
centurion was absolutely sure that Jesus could heal his servant. There
was no doubt in his mind. He even intercepted Christ on the way to
the bedside of his servant, to let him know that his belief was strong
enough to believe Christ would heal from a distance. While these verses
point out the strength of what all of our faiths should be, they also
share a methodology of sales and marketing. 

Please take the time to note the centurion's methods, since they were
successfully deployed to promote Christ's healing: 
\begin{enumerate}
\item First, the centurion sent ``sales people'' to plead with Christ.
His sales people were folks with whom Christ would identify---elders
of the Jews. These folks could gain Christ's attention. 
\item The ``sales people'' pleaded with Christ, and provided personal
testimony to the worth and character of the centurion. They vouched
for the centurion and the fine work that he had done for the nation
and the church. 
\item The centurion then pulled a ``bold sales'' move. He sent a second
set of salespeople to Christ, to advise Him that he didn't even have
to make the complete visit, but to simply use ``His remote powers''.
In other words, the centurion did all of his sales through others
and by remote procedures. The centurion didn't even attend to Christ,
but instead explained why it wasn't necessary. 
\end{enumerate}
The centurion trusted his methods of communication. They might compare
to the use of catalogs, sales persons, Internet advertising, and other
indirect contact methods. The key to his success however, was his
trust and faith in Christ.

The centurion's use of sales people to whom Christ would relate, compares
to using people with access, or people that are part of the ``good
old boy'' network. Many existing sales agencies profess this method
as the foundation of their success.

In addition, the centurion prepared the sales people with testimonies
that let Christ know of the character of his life. Virtue and integrity,
the characteristics of the centurion, are still valuable entities
in today's world as sales references.

I should also point out that the Centurion had a sizable staff, if
not an army to manage. When many are being managed, the manager relies
on trusted servants---or he delegates successfully.

In addition, the centurion made it convenient for Christ to heal the
servant by ``remote control''. All impediments were removed, and
the centurion achieved the results that he wanted. Jesus was amazed
at the centurion's faith and consideration. Is it possible that the
centurion knew that Christ's time was short and precious? Is it possible,
that the centurion recognized Christ's purpose, and took every step
possible to make it convenient for Him?

I've used the centurion's methods in marketing and selling engineering
design services. In many instances, when approaching ``very busy''
clientele, the centurion's methods work. Imagine yourself in a busy
time, being bothered by sales people. You appreciate someone who doesn't
want to consume your time, and simply wants to share a quick effective
sales message. I personally appreciate this approach, and I believe
there are others who also appreciate it, not only in giving, but also
in receiving the sales approach.

\section{Image Projecting, Proverbs 27:19}

This past week, I joined a Christian brother in a joint sales presentation
that was an organized competition. Our presentation was the last of
four to a board of directors of a major institution for the design
of a major building project. The project would be the largest, if
won, for my brother and his design firm.

We did pull the ``stops out'' for preparations. We carefully planned
our hour-long presentation, and equipped ourselves with the latest
technology for presenting images along with our message. We rehearsed
the entire presentation at least six times, making refinements each
time. The last rehearsal was about two hours before ``show time''.
We had a quiet dinner to relax and gather our thoughts before the
presentation in the evening hours.

As we sat reflecting over dinner, my friend asked me what ``image''
he should project. I believe that the Spirit of God was active in
my response, when I told my friend to ask God to be with him, and
let his heart reflect his genuinely wonderful character. If my friend
could let the board of directors know of what was in his heart, he
would have no problem convincing the board of directors that they
would enjoy his work over the next few years on this project. I suggested
that he spend some quiet time with God, reflecting on his soul, what
he valued, and his love for God. He did, and his presentation was
nothing less than spectacular. When my friend spoke, it was from his
heart, and there was no mistaking his sincerity and commitment to
the project. 

There are times when all of us do not like who we are, and what image
we are projecting. The tools for re-programming ourselves lie within
our hearts. There is a limit as to what we can selectively project
as an image, until our true selves, and what lies within our hearts,
shines through to the world. Recognize that you cannot hide what lies
within your heart for very long. Make sure that your heart is worthy
of exhibition to the world, at all times. 

\section{Impressions of Wealth, Proverbs 12:9}

Coming from a financially poor family background, my parents were
always spouting this Proverb to us. I would find it amusing when my
older brother would suggest that this Proverb was meant for ``rich
folks'', and we ``poor folks'' should ``put on the dog''. His
comment, along with my parents spouting this Proverb, confused me
to no end. In effect, the rich are trying to appear poor, and the
poor are trying to appear rich. Can't the rich and the poor meet ``somewhere
in the middle''?

However, over the years, I have come to see a deeper meaning in this
Proverb. I believe that there are few situations, if any, for ever
giving anyone the impression that I am, or anyone of us, is wealthy.
First of all, God knows better. Secondly, we might be creating expectations
upon which we may not be able to deliver if someone is in need. Thirdly,
even if we might be wealthy, we are gaining nothing in giving the
impression of wealth, and we may be discouraging others.

We should be living our lives for God, our family, those in need,
and ourselves. I cannot see the value in living our lives to make
impressions of wealth to others. What a waste of time it is? 

Recognize that in this Proverb, there is inference that taking action
with your resources is more important than making impressions of your
success. Have a ``servant'' to assist you in your endeavors. Make
your wealth work for you, not to impress others. Let's bring this
thought into the 21st century. Was your last car selected for your
needs, or for what other's thought of it? Do you live in the community
and neighborhood for the wholesomeness of the environment, or do you
live in an area where your income places you? Do you dress for church
to show respect to God and others, or to make an impression on others? 

Human pride is a weakness among us all. It is difficult at times to
be humble. We all need to work at it, and somehow remind ourselves
of the reason for our existence, and that is not to simply impress
others with our success and ability.

\section{Information Secrecy, Ecclesiastes 10:20}

In marketing and sales, there are many instances when discretion and
confidentiality of information is critical. Some examples include:
\begin{enumerate}
\item If you receive early word of an upcoming project, purchasing activity,
or other opportunity prior to the time that your competitors will
know of it, keeping the opportunity confidential will give you an
enviable position to market your services prior to the competition,
almost sole source. You are in a position to win the project without
competition, simply through negotiation. Once the competition becomes
public knowledge, you are now one of several competitors, all ``touting''
the same line. 
\item If you have unique services or product features, it will simply be
a matter of time until your competition recognizes and analyzes your
uniqueness. By maintaining a ``lid'' on the secret, you may a gain
a few more months or days of having a competitive edge. 
\item On large, targeted sales efforts, maintaining any ``privileged information''
in confidence may be the difference between your success or your failure. 
\item In most marketing and sales campaigns, unique approaches to advertising,
contact confirmation, and follow-through are developed and implemented.
If successful, your competition will then take months to respond and
counter your campaign. If kept in secrecy as long as possible, your
investment in the campaign may reap more profits for a longer duration.
\end{enumerate}
However, in my experience, nothing remains confidential or secret
forever. Just as good ideas and activities become public knowledge
eventually, so do evil practices. All improprieties, unfair practices,
immoral or unethical activities, and sinful acts are known immediately
by God, and eventually become known, or exposed to the people of this
world. So that while secrecy may be important temporarily for practical
reasons, recognize that all will become public knowledge, and your
information, words, and actions will be your responsibility to bear.

\section{Marketing}

\subsection{Courage, Luke 11:5--10}

Yes, Jesus was obviously referring to our personal courage when petitioning
God in prayer. However, just as Christ uses the analogy of a father
and a son, His teachings are adaptable to marketing.

In my experience, both boldness and forwardness are needed for marketing
and sales. Most (if not all) of us are born with a natural shyness.
This shyness may well be part of our protective system, assisting
us in avoiding danger and threats. However, it must be overcome to
be effective in marketing and sales. As Christ teaches us, ``boldness''
is the reason the neighbor and friend shared the provisions for the
visiting party. 

Please recognize a very important detail of this message. It was a
friend who was bold. Developing a friendship or a good relationship,
is paramount to effective marketing. Without the friendship, the ``boldness''
may have been received obtrusively, if not leading to a rejection
of the request. My brother is one of the most successful marketing
and sales people that I've run across. His methods are very simple.
If you've got a good product or service, it will sell, if you build
a relationship with the buyer or consumer. First and foremost on the
marketing trail is the need to build positive relationships. These
relationships will clear the way for the reception or consideration
of your ideas, services, products and requests. The relationship will
permit even a ``bold'' request in the middle of the night for provisions---not
for yourself, but for another person.

My brother promotes spending time with potential clients in activities
in which business is not discussed. He recommends activities that
require concentration, team work, and a level of focus that provides
one with a diversion from the typical business day. He prefers hunting,
fishing, golf and then sight-seeing in a ``spectacular environment'',
in descending order, as activities which work well for the building
of relationships. He actually holds formal meetings with his staff
to discuss and classify the maturity of the relationships with his
potential and existing clients, to determine if the relationships
are mature enough to absorb any discussion of work or sales.

What is boldness? Let me define it in the context of marketing, as
something out of the ordinary. Something that ``catches eyes'',
something that is ``pleasantly different'', something that makes
you as the marketing contact or executive, stand out in a world that
is ``flooded'' with marketing. Marketing requires activities that
make the buyer stop and think, make the buyer give your product or
service some consideration, and make the buyer give you a chance.
Yes, there may be some bewilderment on the part of the buyer, and
it should be out of admiration for your courage. The buyer should
begin to feel obligated to you for at least the chance to give you
consideration, if not make the purchase. Some would say that ``repeat
calling'', or repetitive advertising provides this boldness. To me,
repeat activity is the military approach of massing your resources
to make an impact. I'd like to think that there are more clever ways
of establishing your ``boldness'', and committing the buyer to an
obligation. The efficient use of our resources is always a consideration
for our small business, as I suspect in most small businesses.

Yes, these discussions are treading on the edge of good and bad behavior.
In business in this world, you are surrounded by both good and bad
behavior. We are, and we are amongst sinners. We should be the ``rose
amongst the thorns''. We should not be afraid to compete with those
who are evil. We should be bold in our commitment, focused on
our leadership, and confident that we'll succeed. We should not be
afraid of the edge between good and bad, and I believe that
with a proper heart, and spiritual guidance, our boldness will
be very positive, and well received.

\subsection{Personnel, Proverbs 25:13}

``Like the coolness of snow at harvest time,'' paints a wonderful
picture. What relief it is!

As a high school youngster, I recall harvesting hay during the hottest
and most humid days of the summer. I recall one time, in mid afternoon,
when there wasn't sufficient water to drink, I was covered in sweat
and dust, and I was returning to the barn with a load of hay. I was
not looking forward to loading the hay into the barn, because at this
time I was thinking of how a turkey must feel, baking in the oven.
As I neared the barn, I could see my boss's wife standing near the
barn doors with something in her arms. As I arrived, I recognized
a watermelon that she had pulled from the spring. She pulled out her
knife, and cut a quarter of the watermelon just for me. As I buried
my face in the melon, the coolness and refreshment that it provided
were more than words could say. I liken it to the quote from Proverbs,
above---``Like the coolness of snow at harvest time'', when all
are hot and exhausted and in need of refreshment.

In this age of deceit, misinformation, lies, and a marketing flood
in the media, there is need for relief and refreshment. The key word
is ``trustworthy''. The messenger, or marketing person, must be
trustworthy not only to his employer and master, but also to his consumer.
Trustworthiness is a characteristic that is a key ingredient to successful
marketing. Trustworthiness in itself will carry a marketing person
far. It even compensates for other deficiencies in appearance, and
technical delivery. While a conscientious marketer will resolve other
deficiencies through hard work and training, the marketer must be
righteous. If not trustworthy, a marketer will be quickly dismissed
for the deceit that he bears.

``Is he\slash{}she believable? Is he\slash{}she telling the truth?'' These are
questions that most professional buyers are asking themselves at every
turn of an interview or screening of products and services. If you
are contemplating marketing and sales, or if you are hiring a marketer,
remember the number one ingredient---trustworthiness. 

\subsection{Strategy, 1 Corinthians 9:19--23}

People usually give sincere consideration to the advice of those they
consider friends, or of those whom they respect. The relationship
is critical. Relationships are most easily and most often built, amongst
folks with common identity. Common identity might simply be understanding
their problems, talking their language, having had similar experiences,
or coming from similar backgrounds. ``He's one of us!'' is a cry
that all marketers like to hear. It means that there is brotherhood,
and brotherhood breeds consideration. Brotherhood will permit the
relationship to be built, and then the advice and wisdom to be imparted,
and hopefully considered.

Most of us humans, need to identify with a group. Very few of us are
``Lone Rangers''. It is our nature to be of a tribe, members of
a sect or ethnic group, and we even advertise it with our hats, jackets
and sweaters that all bear a common identity. In some cases, sports
and sports teams fill the void of a group. Common identity is common
in sports and with sports fans.

\index{Bible!characters!Paul}
St.~Paul was a master at building common identity, at building the
relationship, and then imparting the message. Common identity does
not happen by accident. It requires listening, study and research.
We are all very different people, each from different backgrounds
and each with different chromosomes. Who is the audience, the consumer,
and those to whom the message is directed? What do they enjoy, what
do they fear, and what are their priorities in living? What are their
values, how do they speak, and what is humor to them? How do they
dress, what do they eat, and what do they believe? What can they afford?
We call the investigation that is conducted to answer these questions,
marketing research. This marketing research will provide us with the
details and criteria for our successful marketing campaign. 

Sometimes, there is insufficient time and resources to conduct adequate
research. Sometimes we are caught in an ``on the spot'' interview,
opportunity, or contest. In these situations, we must be alert to
all characteristics of our audience. Follow the eyes, the body language,
and determine their heart. Define the subjects, and adapt to the situation
with integrity, trustworthiness, and genuine love for your consumers.
\index{Bible!characters!Paul}Like St.~Paul, you will begin to quickly adopt the characteristics
of common identity, not for deception, but in sympathy, love and compatibility.
You will understand your consumers, you will be aware of their problems,
and you will fulfill their needs. 

\section{Relationships, Forcing of, Song of Songs 3:5}

How many times were you told as a youngster, not to force or push
love? ``Let love take its course!'' ``Relax and let it develop
as it wants to!'' ``If it is meant to be, time will tell!'' Do
not force love. Do not force relationships.

One of the most difficult aspects of marketing and sales to learn,
is the patience required in building relationships. Relationships
most often take time to build. They don't happen as ``love at first
sight''. At the same time, the relationship requires just the right
blend of ``boldness''. ``Boldness'' is not ``forcing a square
peg in a round hole''. Boldness is taking the initiative to communicate
your desires, your offerings, and your intentions. Boldness is not
arrogance, it is not forcing submission, but it is the spark that
simply defines the relationship.

Appropriate boldness: 
\begin{enumerate}
\item Stimulates you to contact another;\label{stimulates-you-to-contact-another} 
\item Introduces you to another;\label{introduces-you-to-another}
\item Promotes your offerings, services, benefits, and illustrates ``what
makes you attractive or not'';
\item Quite naturally, without human manipulation, it takes you and another
to greater levels in your relationship. 
\end{enumerate}
``Appropriate boldness'' is a term that I've used to illustrate
what I think each of us has in our repertoire in promoting relationships.
In marketing and sales, as well as life in general, each of us cannot
just ``sit back and wait'' for the relationships to develop. We
must assist in developing the relationships, with ``appropriate boldness''.

Again, let us be warned that ``appropriate boldness'' is what initiates
the growth of a relationship. It does not force the relationship,
nor drive it. It is simply the ``firing of the starting pistol''
for each phase. Yes, sometimes there are misfires. Sometimes we will
find ourselves in situations where relationships are quite unlikely
to occur. I've had several of these situations in my life, where I've
met folks ``who just don't like me''. I've met some folks ``who
downright hate me'', and unfortunately, they've never had the chance
to know me. Perhaps it is not just me that they hate, but everyone.

On the other hand, I've had clients that were ``close'' within a
few days of interaction. Other clients have taken two to thee years
of ``quarterly contact'', until the relationship was at the point
where they'd consider our services. I do recommend that you attempt
to identify the ``pace of the relationship development'', more from
a passive perspective. It is like watching the rain. ``How long will
it last?''

After all, in marketing and sales, most of us are after a ``quality
relationship'' that endures time, difficulties, and competition from
others. 

\section{Relationships, Foundation for, 1 Corinthians 13:1-7  }

I have many friends who participate in businesses that they define
as ``void of relation\-ships---just money based''. They buy and sell
nondescript commodities for a margin, they sell products to faceless
masses whom they believe trust their clever advertising, they manufacture
and sell a patented product for which there is no competition, through
several layers of sales distributors to a client that simply cares
about price, and they provide services to clientele that they don't
know, and don't care to know. 

I feel sorry for these friends of mine. They seem to be missing a
lot of life. Part of the ``secret'' of my joy in my labors, is the
``love'' in the relationships that I've experienced and developed
over the years. This love is for folks from all kinds of backgrounds,
from all walks of life. As our paths have crossed, the relationships
have evolved.

Perhaps I have been the instigator, using a degree of ``appropriate
boldness'', in the development of these relationships. In the business
of professional design services, ``relationships'' are absolutely
mandatory. Folks in our business, if hoping for business success,
must recognize this simple fact. Frankly, it is a blessing. It is
a wonderful life, for which I am grateful to God.

Each and every person with whom you come in contact in your business
day, is a relationship that requires attention. That attention, in
whatever form, must be founded upon ``love''. You must have love
in your heart, and then your intentions and actions in these relationships
will ``blossom''.

If you're like me, I sometimes find myself intolerant of others, impatient
with mediocre performance by others, and sometimes I develop a sense
of elitism and arrogance over others. Rather than ``loving'' others,
I find myself attempting to avoid and ignore others. However, I am
reminded frequently enough, that I am God's servant, and my mission
is to serve Him. I also believe that my mission is to serve Him by
serving others, embracing them in times of trouble, and guiding them
when provided with wisdom. It means ``loving'' all of those who
cross my path. Yes, it takes work and deliberateness, just like marketing.

\chapter{Success and Failure in the Business World}

I have experienced both success and failure in the business world.
I must admit, however, that my experiences have been moderate---no
dramatic failures, and no extra-exciting successes. I know that they
are moderate from my exposure and assistance to friends and brothers
who have experienced the extremes of success and failure.

The terms, success and failure, are somewhat over-used in today's
world. They are terms that are used in judging performance. Performance
for most of the worldly folks, is financial performance, which in
itself tells you about most of the world's condition and perspective.
Some folks concentrate on their success rate so much that they forget
about the purpose of life.

Success\slash{}failure, wins\slash{}losses, and profits\slash{}losses are the reports and
focus of our society and our businesses for the most part. The effort,
the trials, the growth through hard work, and the maturing under fire
are often over-looked.

Having grown-up in the world of sports, I believe that there is value
in striving to win. However, winning is not the ``only thing''.
The ``striving to win'' is most important in my mind and experiences.
The ``striving to win'' is what makes us better people. That is
our mission, to make the most of our selves and our lives in God's
eyes. Because of my sports experiences, I view most of the situations
of life as a sports competition. There are competitions of finesse,
brute strength, endurance, and wisdom. Our lives are filled with these
competitions and situations that require our development of not only
our athletic skills, but also our discipline, resolve, and humbleness.
Life's situations keep coming at us, day after day. Our mission is
to respond to them in a positive fashion, with an attempt to win.
However, I doubt that we will win them all. We will lose some of them.
However, losing does not make us failures. We have fought the fight,
we should have put our best effort forward, and we competed. In the
process, we've grown and matured.

As Christians, we need to focus on the final success---our acceptance
into eternal life by God. With this focus, our trials and competitions
take on new meaning. The successes become less exciting, the failures
result in less grief, and the effort becomes the point of focus. We
become stronger from the trial. We begin to immediately look for the
next competition, all in the spirit of being better in God's eyes.

\section{Allegiance, Selecting People, Acts 9:1--22}

In this world, with all of its trials and competitions, success and
failure often depend upon the allegiance of one's teammates. Each
of us has a time when we have to select teammates for a mission, whether
in business or other aspects of life. When you approach this decision,
don't overlook the unlikely. Don't ignore your adversaries, your competition's
best people, and don't ignore the least likely of candidates.

Obviously God chose Saul to spread the word of Jesus Christ. Saul
was one of the least likely candidates, but at the same time, he was
one of the most qualified. He was a declared enemy of Christianity,
a persecutor of Christians, and he was willing to go out of his way
to attack Christians. There might not have been more of an adversary
of Jesus Christ on this earth than Saul, at this time in history.

At the same time, Saul's redeeming characteristics included his strong
religious education and influence in the Jewish community, his Roman
citizenship, his marvelous ability to write, and his oratorical skills.
Saul was a man of determination. Determination is often exhibited
by those with big hearts, big passions, and a sense of mission. They
know what is meant by the order to ``just get it done''. 

Saul was not only an adversary, but he was a murderous enemy of Christ
and His followers.
\index{Bible!characters!Paul}However, upon his conversion to Paul (as we say
on the ``road to Damascus''), the servant and disciple of Jesus
Christ, he became invaluable. His allegiance was unquestionable, his
commitment to success was unimpeachable, and his effectiveness in
spreading the word became a model for our church.

I have noticed the same of converted adversaries in the business world. 

They can provide testimony to the righteousness of the cause, from
a dual perspective. Their commitment to the cause seems to be actually
greater than most. They are a steadying force within the team. The
converts often have a greater sense of mission. They also have a sense
of gratitude and revelation.
\index{Bible!characters!Paul}Perhaps it is the model of Saul\slash{}Paul from which the term comes, ``I have seen the light!''

\section{Revenge, Retribution, Romans 12:17--21}

When we are wronged by someone, it is quite natural for us to seek
revenge. Overcoming the desire for revenge requires strength of character.
It requires discipline, self-control and resolve. In a world filled
with evil, it is very unlikely that any of us will live our lives
without being wronged by someone. Do not fret and whine, even if it
is a natural tendency. I ask you to ``turn your perspective upside
down''. Consider these events as opportunities. They are opportunities
to illustrate to God your maturity, your discipline, and your commitment. 

Time and again, I have been wronged in business. I have been denied
fair compensation even in contracted relationships. I have been unfairly
eliminated from competitions for work by lies and misinformation.
I have been personally and corporately slandered. While I do not condone
these actions and situations, I know that they will continue to happen.
My immediate reaction to these situations is to obtain swift and \index{justice!harsh|see {revenge}}harsh justice, which means personal revenge. I usually require a few days
of ``simmering'', to calm down, before I take action. The few days
of simmering usually involve Bible study and prayer. With meditation,
these events become opportunities to illustrate to the world and to
God, my discipline, resolve and commitment to what is right---in
reality, a chance to show God my development. 

Most often, my solutions include the following principles:
\begin{enumerate}
\item Let God handle the revenge; 
\item I am responsible for showing the wrongdoer his or her error in an
attempt to assist in saving him or her; 
\item With God's help, I am strong enough to withstand the wrongdoing; 
\item My actions will be scrutinized and imitated by others, since I have
declared myself as a servant of God; 
\item God's command in these verses gives me the solution to the wrongdoing.
\end{enumerate}
I attempt to evaluate my actions to assure that revenge is not included
as a driving force, or as part of the solution. It is so hard. I am
not always successful. It is so hard, but I must learn. I must attempt
to mature my soul. It is so hard. Fearing only God and the devil in
this world, I have a tendency to handle things the way I feel I must.
It is so hard. I am a sinner. I will continue to learn self-control,
discipline, love, and brotherly kindness.

\section{Success}

\subsection{Attitude for, Romans 15:1--4}

As a youngster, I recall noticing that not all of us kids had the
same abilities in sports, academics, and other life ventures. The
differences were remarkable, if not fascinating to me. Being personally
blessed with many abilities, I used to marvel and enjoy the performance
of the ``equipment'' that God gave me. My successes and superiority
led to personal pride and arrogance. My failures or defeats were accompanied
by envy for others.

Contemplating the differences, I began to realize that there might
be a purpose to the successes and failures. I began to search for
purpose and meaning in winning and losing, beyond just the joy of
others' praise.

What is the purpose of success? I believe that there is a purpose,
but I'm not sure that I still know all of it. What I have learned
is that with success comes responsibility. Success also brings the
scrutiny of other people. Others want to know why it is that we are
so successful. It is important to have reasons that are explainable.
I believe that our successes are delivered by God. God decides on
who wins and loses. He is testing us with our successes and failures.
He wants to see what it is that we do with our successes. He seems
to be interested in finding how our souls and spirits react to winning
and losing.

As we approach a new mission, business quarter, or project, may I
suggest that we consider how we will handle success. What will we
do with the laurels of success? Will we put them to good use? Will
we give God proper credit for the success? Will we take the opportunity
while in the limelight to do good, to say something meaningful, and
to recognize the source of our success? Will we take the time to spread
``hope'' to those who are interested? When winning or losing, as
others inquire of our feelings, and respect what we have to say it
almost as if we are in the pulpit, in front of many people. They are
anxious to see our response, and will heed our words---good or bad.
In my experience, our attitude, character and the righteousness of
our motives seem to determine our continued success or downfall.

\subsection{Complacency, Luke 12:16--21}

Businesses operate in campaigns. They have tax years, fiscal years,
and quarterly reporting of financial records as a minimum. Each period
of reporting is a campaign. If the business is successful during the
reporting period, there seems to be a natural tendency toward complacency.
The euphoria of success seems to breed a sense of satisfaction that
hinders initiative and creativity toward the future.

In most successful businesses, success in a campaign breeds a drive
to perform even better than before, breeds creative thinking on how
the profits can be reinvested to do even better things, and seems
to develop a sense of mission for the future. Success, in these cases,
seems to spark thinking for the next campaign. If it doesn't spark
drive, it should. ``To those whom much has been given, much more
is expected.''

While Jesus was warning us about avoiding a money-focused life in
these verses, I am also reminded that with success, we need to be
developing future plans for not only further gains, but also for wisely
investing this success. We should not simply be storing our wealth
and living ``off the fat'' of it, but making the fruits of our success
work in a positive fashion. A positive fashion, for those of us who
believe, is investing in improving the world, assisting in the spreading
of the Word, and in promoting the things of God.

Let me give you an example. Several years ago, an exhausted Christian
brother returned from missionary duty in Mongolia. Being well educated
in the construction industry, he put his mind to making a living to
support not only his family, but to develop a retreat to ``refresh''
missionaries, all while he was tithing. He has worked hard, invested
wisely, and he is already making his plans for the retreat. He will
not ``store up'' his wealth, but instead he will make his wealth
work for God's mission. My friend has a clear understanding of his
purpose, which was developed in prayer with God. 

I've also observed another unique characteristic in my friend, that
I've tried to adopt for myself. Success comes from hard work. The
hard work needs to be paced with brief respites or breaks for diversion
and refreshment. However, upon reaching our goals, we should not simply
retire or rest on our laurels. Life is very short, as I can attest
in my fifty fast years. Cognizant of the shortness of life, we are
to continue to strive for further accomplishments. We must not rest.
As life saps our strength and energy with age, we must find new, slower
paced ways to serve. We must not simply give up, but we must continue
the effort, fight the fight, and assist in winning the battle. 

\subsection{Credit for, 2 Corinthians 10:15--18}

Let God be the judge of success---not man. If God defines success
and who is successful, our perspective on life dramatically changes.

Each of us should set goals for ourselves. Our goals should include
our mission to God, our hopes for our family, and yes, our dreams
for our businesses. Some would suggest that the goals that we set,
should be obtainable, while others would suggest that our goals should
be just beyond our reach. Most of us work very hard at our efforts
in approaching and sometimes obtaining these goals. Whether or not
we reach the goals, our success might be in the effort we put forth,
and certainly in how pleasing the effort is in God's eyes.

Let's face it, our personal ``successes'' are such a little part
of the overall, universal effort of God. We are small cogs in God's
machine. While we may be small parts, we are integral to God's plan,
and we are important in His plan. Our existence is a miracle in itself.
It is God's miracle. Let us not waste the chance, the opportunity,
and the important mission that God intends for us.

Often it is difficult for us to determine the mission that God intends
for us. It often requires a search. The search must include prayer
and Bible study. Most of my Christian brothers and sisters say that
God often uses our acquaintances, friends and family to disclose His
mission for us. He seems to show us through the experiences of our
lives, and often talks to us through our hearts. Sometimes the search
for our mission takes time. However, the search is worth it. There
can be nothing more satisfying in this world than being sure of our
mission and accepting His mission in our lives.

These missions are not always what we might want, nor are they easy
tasks. Use the disciples and apostles of Christ as an example. Their
lives were hard by comparison to our lives, and in some cases they
were aware of the pain of their future and their fate. However, the
rewards and the satisfaction of their missions for God were accepted
with joy and contentment. They sacrificed their lives to their missions.
As with the early disciples of Christ, the missions are worthy of
giving up our worldly life for an even better life---eternal.

Let us all strive for success in God's eyes so that He may consider
us His worthy servants.

\subsection{Perpetuity, Proverbs 27:23--24}

I have often become confused with success in this world. Whether in
business or in sports, the success has dizzied my direction and understanding
of my mission and my ``place'' in this world. The excitement and
exuberant pride clouds my thinking. I am nearly intoxicated, if not
dangerous from a logical perspective. The good news is that this condition
is not lasting. It is a temporary euphoria, which often leads to self
praise, complacency, and sometimes weakness. While I continue to suffer
from such conditions with brief moments of success, I am committed
to making ``me'' better. I need to heal this soul of mine, and mature
it more to God's liking. Lately, I've been using brief moments of
pause after success to immediately plan for the next campaign, to
assess my strength and weaknesses, and to confirm and reinforce my
capabilities.

My business is design engineering. It is a business that lives and
dies with the ability of my staff. My staff's hard work, their creativity,
their ``vision'', and their teamwork (bonded by a chemistry that
I've been trying to define for decades) is what appears to make for
successful performance. I've defined my business success to friends,
as simply an experience in a ``window in time''. I've experienced
it, lived through it, and was blessed to even see it. While I am supposed
to be the ``director'' of these teams, the originator of the design,
and the ``coach of the team'', I feel more like a simple observer.
I've been in the right spot at the right time---as directed by God.

While I know that my experiences have been God directed, God inspires
me to continue assessing the capabilities of my people (flock), and
to be vigilant of improving the staff. We cannot rest on our laurels,
but we must strive consistently to improve and better our methods.
We must not just keep up. We must lead. Disappointingly, sometimes
my folks are insulted by my words, thinking that I'm dissatisfied
with their efforts. I love their efforts, but I will not condone complacency. 

With God's blessings, guidance and direction, we will be long term
servants. We must continue to not only improve ourselves and our methods,
we must also focus on the generations to come.

Several years ago, I was trained in \index{people!Deming, W.\ Edwards}Mr.\ Deming's \textsc{TQM} (Total Quality
Management) programs. Mr.\ Deming suggested that the economic success
of Japan and industries in other countries was based upon a vision
toward long term success. I never had the chance to ask Mr.\ Deming
if he was a student of the Word. If not, the Word confirms his beliefs.
Success is not a simple flash in time. Success is a multi-generational concept, of repeatedly striving for ever-extended goals.

\subsection{The Perils of, Nehemiah 6:1--19}

I am very pleased with the economic system in America. It is founded
upon competition for success, and it provides a level of fairness
for the competitors that cannot be found anywhere in the world. It
has permitted a ``poor hillbilly'' like me to follow my dreams of
becoming a design engineer, performing my work in 26 countries around
the world.

However, in a world that is filled with evil, I have found that it
is human nature to be jealous and plot against those who are perceived
as successful. There are always adversaries, and yes they might be
competitors, who are envious. Their envy drives them to attempted
destruction of good efforts, spreading of lies and bad will against
others, and in some cases their collusion to join a winning team (jumping
on the band wagon). It seems to me that every time someone accomplishes
something good, there is someone else attempting
to deride the success. 

I've learned to enjoy my successes in private, keeping it between
God and myself. I have a friend who owns and operates a very large
business, employing thousands, and ever-growing. You'd have a hard
time recognizing his success from his dress, demeanor, his hobbies,
or even his car and house. He keeps a very low profile, and professes
that ``a surfacing whale gets harpooned''. Success should not be
followed by bragging in public. Bragging results in more harm than
good.

In the world of business, we use our past and present ``successes''
to advertise our services and products. Such advertisement aids in
developing new work and new clients. However, it also brings out the
ugliness of mankind. Our adversaries and those loaded with envy, will
use such advertising against us. Be wary. We should dampen our enthusiasm
for our success, and focus our advertising of the success, to only
those for whom it is intended. We should not be bragging, or the evil
of the world will bring us down.

\section{Winning}

\subsection{A Mental Concept, Revelation 21:1--4}

I recently returned from a trip to the other side of the state, where
I experienced something that all parents hope for. I watched my son
help lead his high school ice hockey team to their first state championship.
I gathered most of our family for the trip, and they added to the
cheering section for our little school. The celebration of the victory,
the elation of winning, and the satisfaction of watching our young
men mature as a team, were wonderful experiences. I thought to myself,
``well, we've nurtured them for many years, and it is satisfying
to know that we parents have somewhat ``won'' in the mission of
raising our young men.'' Our boys had overcome tremendous odds, they
had a sense of hope in their fight, and they had emerged as victors.
Their last three victories were close, one-goal games. It was almost
as if the script had been written, and they had played it out. Their
happiness and amazement was expressed in their hugs and tears.

The elation is still with me. However, when I read these verses from
Revelation, ``winning'' takes on a new meaning. Our trials in this
life, our wins and losses in the world, are miniscule as compared
to the ``winning'' described in these verses. These words describe
the ultimate victory. It is a victory over death, over evil, over
pain, and the disappointments in our life.

With these verses foretelling victory, and the experiences of winning
in the world, I am led to believe that our wins have purpose. It is
almost as if we're elected to the position. We experience an unusual
sense of well-being. Our wins are to prepare us for the ultimate victory.
Our victories might be a glimpse of what will be. In the new order,
the victory will be well worth the fight. 

\subsection{A Song for, Psalm 18}

Let's face it, for those of us who believe, winning and losing is
in the hands of God. Our mission is to continue to fight the fight.
Only with God's help, will we win.

In the elation of victories in our life, we must be wary of the sin
of pride. We must always recognize the source of our victories---God.
I also believe that when we do experience a victory, or a defeat,
we need to acknowledge the goodness and the wisdom that God has brought
us. We need to concentrate on how God has directed us, and His purpose
in the victory or the defeat. We are maturing through these trials
and contests. As His servants, we must improve and grow. Let the excitement
of others, the ``back-slapping'' and the self-serving perception,
rest with others. As God's servants, let us make use of our trials
in serving Him. Our focus is different and more mature than the non-believers
of the world. We must be God focused.

\subsection{An Attitude for, Matthew 6:25--34}

I was told by my high school football coach that we must learn to
win on the practice field and in the fitness room. Our attitude must
be developed there before we can win on game day.

I've been using this advice for some years in training young wrestlers
that ask for help. I attempt to identify the attitude that they'll
need to win in a match, and then establish a grueling physical development
program for them. I ask them to use the ``pain'' of building their
strength and endurance as the catalyst for their attitude. As they
are ``fighting through the pain'' of physical training, they should
be developing the thoughts that they will be repeating when facing
a competitor. One of my favorite exercises is a dead-hand hang from
a bar for two to three minutes. The pain is unbearable at first. As
the pain sets in, I ask the young wrestlers to mentally work through
it, thinking of their investment in their physical development, and
that their competitors will try to discredit their investment by defeating
them. I refer to the attitude that is developed from this training
as ``controlled rage''. The attitude for winning is developed in
training.

In business, I was told by my superiors that winning is to be achieved
by developing our skills, our techniques, and our use of the tools.
If we build a foundation of winning, then we will in fact be successful
in our endeavors. 

The key to Christ's words in these verses is verse 33. ``Seek first
\dots  '' This is the foundation for our winning the ultimate victory.
Our persistence in seeking ``His Kingdom'' will set the stage for
our win. Our attitude in these matters is critical in life. 

I had a very stressful negotiation for new contract terms with a client
today. The client, while new to me, has a reputation for being unreasonable,
if not volatile. I spent yesterday finalizing my preparation, and
last night in a hotel, reviewing my approach. This morning, I rose
early, and spent a few hours in the Word. I re-read these verses.
I realized that my mission was not to obtain the additional compensation
that we needed for the project, but to ``witness'' to the potentially
un-reasonable client. I prayed for God's guidance, and during the
prayer, I cried. My emotional outpour was created by my sudden realization
that God was ``right there with me'', calming me. I could feel God,
and He seemed to be smiling at my sudden ``realization''. 

As I entered the conference room to meet my nemesis, I smiled to myself,
and stated, ``but first seek the kingdom of God''. My associates
were flabbergasted. We spent our time with our client, talking about
the good and bad of our efforts, how we can make things better, and
how our client might make ``our teamwork'' better. At the end of
the meeting, my client asked me to send him a proposal for the revised
contract terms. I will.

What a day this has been! I feel so accomplished. I have never been
so confident in the power of prayer. I am so amazed at the strength
and ability of God. I am now sitting on a homebound airplane, looking
at the sun slip over the horizon, from a perspective above the clouds.
I have been thanking God for His steadfastness, and help, and I am
amazed at how different God's perspective is on situations in the
world. Oh, we humans are just so ignorant of the real world, and the
real issues.

I have been born-again for just over fifteen years. I can share with
you that even with the thousands of hours of studying God and His
word, there is much more that I have to learn. The depth of the study,
the extent of the ``seeking'', and the length of the search has
been satisfying and immense. I pray for all of us that our journey
in finding ``His kingdom and His righteousness'' will be unencumbered
and fulfilling.

\subsection{Goals, Philippians 2:1--16}

God exalts those who humble themselves, just as He did with His Son.
Our goal should be to have God exalt us, in all of our endeavors. 

In all that we do, we should be approaching our trials ``without
complaining or arguing''. Our goal should be to ``become blameless
and pure'', even though we live in the middle of a ``crooked and
depraved generation''. 

Last evening, as my son and I toured New England colleges, I saw a
bumper sticker that stated profoundly, ``My boss is a Jewish carpenter''.
I thought to myself, what a wonderful perspective. Regardless of our
position in our companies and institutions, we all have a boss. Our
relationship with our boss must be right. We are taught to fear and
respect our boss. We cannot be good servants unless we respect and
fear our boss. You might be king of the world, but you still have
another boss. As our boss, Christ sets an example for us, and He sets
our goals for us. 

I pray that as we live our lives, work our jobs, and raise our families,
when we face decisions and trials, we think of what our ``Boss''
would have us do. What He would have us do should be our goals. Winning
can therefore be defined as meeting the goals of what Christ would
have us do. Our complete existence and focus should be on our Boss. 

\subsection{Reason for, Matthew 7:7--12}

Several years ago, after winning a golf tournament with my brother
as my teammate, the two of us sat in the clubhouse of the country
club, and spent several minutes just staring and smiling at each other.
Our actions were a combination of gloating, thankful for the success
of our golf game, and an expression of love for each other. We've
been close throughout our lives even though geography separates us.
My brother spoke first, asking if I remember when we'd lie in bed
as kids (we shared the same room growing up) and wonder what we'd
be doing when we grew up. Whether ``we'd go to college'', ``get
married'', ``have a house'', were the types of puzzling questions
for us.

Well, as we sat in the country club locker room, surrounded by reasonable
decadence, with our wonderful families waiting in the opulent dining
room for us, before we returned to our fine homes, our interesting
careers, our good friends and our wealth beyond our imagination, the
questions had been answered.

My brother and I have gone through this soul searching several times.
Bobby, my brother, always asks me the same questions. Did you ever
imagine that we'd be so ``well off''? Why is it that we've been
so blessed? We both agree on the satisfaction with our family, professional,
and social success. But the gnawing question is always, ``Why?''.
We both agree that we desired, prayed for and worked for our winning
and our success. Our Father has answered our prayers, as the Bible
promises He will. 

I've also developed some further observations on this matter, including: 
\begin{enumerate}
\item As most Christians, while we are yet imperfect, we do try to live
by the laws set before us; 
\item As most Christians, we recognize that our success brings responsibility
to further God's causes; 
\item As most Christians, we know that our success is not ours but God's,
and therefore our success is a reason to further our life's mission
to serve God. 
\end{enumerate}
We don't win. God permits us to win. When we win, we need to discover
His reason for the victories. What is the meaning of the winning?
What is in store for us in service to God? 

Winning should be accompanied by a prayer of thanks to God. The prayer
of thanks might be best directed toward God's confidence in us. That
He has chosen us to win, means that He honors us as His servants.
There is no greater honor. It is a position of trust. 

Winning should also be accompanied by a prayer requesting wisdom.
The prayerful request might be asking God to show us the next steps
in our mission to serve Him, and how our winning serves that mission. 

In the words of a good friend who was a collegiate All American linebacker
and spent more than a decade playing professional football, ``winning
brings responsibility''. Yes, he is a devout Christian.

\subsection{Recipe for, Hebrews 12:1--13}

\index{Bible!characters!Paul}
When I read these verses, my vision of St.~Paul is one of a football
coach at practice, or one of a drill sergeant in basic training for
the army, or one of a leader in a survival training group, or one
of an inspiring business leader-a squad boss doing his\slash{}her job. The
imagination sees a strong and emotional figure, with arms waving,
and a strong, pleading voice. Visions of Knute Rockne or Joe Paterno
jump to mind.

Some of the ingredients in the recipe for winning as provided by St.~Paul, include, in paraphrased, terms: 
\begin{enumerate}
\item Practicing hard---prepare for the future competition 
\item Performing Repetitions---the methodology of practicing, which will
build mental and muscle memory 
\item Learning to endure pain---part of strength building 
\item Studying---building wisdom in the maturing process 
\item Testing---developing our character in the experience of living. 
\end{enumerate}
All of these activities, referenced in the verses, seem to create
a detailed program for developing our ability to focus on the goal,
and becoming more mature in our existence. God is developing us. That
is the purpose of our existence on earth. The trials of this life
are the activities of the maturing process. With that in mind, perhaps
there is much reason to give joy for the trials. 

I have personally been blessed with many ``winning experiences''
in athletics, business and family relations. I can share with you
that in order to win, once a target is established, it requires focus
and hard work (maturing of mind and body). The focus must eliminate
the diversion of other activities and interests (to a great extent)
and build resolve. \index{commitment}\index{resolve|seealso {commitment}}Resolve is the absolute commitment of one's resources
to ``win''. Resolve brings with it some elements of commitment,
including: 
\begin{enumerate}
\item ``Nothing will stand in my way, nor divert my progress!'' 
\item ``No pain is too great to overcome in my quest for the goal!'' 
\item ``It is as important as life itself!'' 
\end{enumerate}
I'm sure that you can think of many more sayings that illustrate resolve,
or absolute commitment. In my experience, this level of commitment
is required to ``win''. It is required no matter what the competition,
no matter the field of endeavor. However, since I've found my Savior,
I've found extraordinary strength and ability. I'm not alone. For
each mission, competition, or activity in our lives, if we take the
time to establish the purpose within our mission for God, and assuming
it is blessed by God, then we have an unbeatable team.  
\begin{quote}
World or God? Matthew 8:18--22:
\end{quote}
What drives us? Is it what others think of us? Is it what the world
expects of us? Are we doing what we're doing for worldly things and
worldly reasons? 

Christ is very direct in His words to the disciples. He is to be followed.
He also advises that not all people will follow Him. He lets us know
that others will follow the things of the world, and not of Him. We
are to be wary of straying from Him. We are to be ever vigilant of
the purpose for which we strive in our jobs, our lives and in all
of our endeavors. We are to be His servants, and only His servants.

All of our ``wins'' should be scrutinized. Are they for Him? Are
the wins serving Him? If not, we need to change our path, our goals
and our purpose. We are not to serve the world and the desires of
the world. We are to take a different path, and if necessary, leave
the goals and desires of the world behind. Then and only then, will
we truly win. 

\chapter{Competition, the Christian Perspective}

The desire to compete is one of those traits that seems to come with
being human. From the time we are children, we race with others, we
wrestle with others, and we compete with others in many different
ways. I might also add, that as children, we compete with joy. Our
spirit in the competition usually creates smiles and laughter, whether
winning or losing. It is a spirit of joy, excitement, and absolute
delight. It is what I expect our eternal existence to be. It is play
with intensity and love, but never meant to cause hurt.

Our lives seem to be one long competition. Upon completion of one
event, there is another competition before us. That is our lot. What
is also our lot is our spiritual development as a result of that competition.
Competition is what makes us better. Competitions are trials and tests.
``Give joy for the trials!'' Competition goes beyond sports and
athletics. Competition is the spirit of the effort for business success,
the effort for academic success, and even the courting of one's spouse.
This world sometimes seems to exist for the purpose of bettering our
souls through competition. It is not so important as to whether we
win or not, it seems to be how well we ``try'' in the competition.
The value and ``form'' of our effort in the competition seem to
be what is important, as reflected in the wisdom of the Bible. As
they say, ``it is not if you win or lose, but how you play the game''.
That is not to say that you do not try. It is not to say that you
do not focus and place tremendous demands upon yourself when you're
competing. Quite the contrary, it means that you attempt to compete
to the best of your ability. You apply all of your resources, all
of your strength, and all that you have. You should and must attempt
to win, to succeed and to be the victor. It is in the ``giving of
your all'' that you mature.

I also believe that the win or the loss, is in the hands of God. We
must plan, train, and do all that we can in our power to prepare for
the competition. We must then compete to the best of our ability.
If God desires our success, it will happen. If not, we will tie or
lose. Tying and losing are misnomers. We learn from the experience,
and we mature in the experience, the stress, and the contest of the
event. We win, regardless, because of the maturity that we gain from
the experience. Therefore, please do not fret about competition. Enjoy
the excitement of it. Enjoy the maturing from it. Know that God has
planned competition for us, for our growth, and for the maturity of
our souls, that we might be worthy to live with Him.

\section{Adversaries}

\subsection{Treatment of, Proverbs 25:21--22}\label{adversaries-treatment-of-proverbs-2521-22}

Did you ever notice how many of the best athletes in the world seem
to have respect, if not love for their competitors and certainly for
their game. The best football players viciously knock their opponents
to the ground, then reach down to help them up, and give them a pat
on the back. In ice hockey, after sixty minutes of drilling opponents
into the ice and boards, they shake hands. In soccer, while during
the game displaying an attitude like that of Attilla the Hun, the
opponents hug one another, showing respect and admiration for one
another. In sports, the best of the players display a love for their
opponents, while the mediocre and poor players display hate. It is
almost as if the good players step back from the game and view it
from a universal perspective. They seem to understand that the games
and life are not personal, that the trials will go on to mature us.
They understand the game and life, and have a deeper respect for the
game and the players. They also know that the thrill of their competition
is nothing without the tremendous efforts and abilities of their adversaries.

Conversely, the weaker players, displaying mediocre skills and abilities,
are often the individuals who display petty hatred, un-sportsmanlike
behavior, and a general disrespect for their competitors. They appear
to be totally self-centered, not caring about others, or the game
in which they are competing. My hockey playing son calls them ``cheap
shot artists''.

In these verses, God is telling us to have love and respect for our
trials, our games, and our businesses and the competitors within each
endeavor. Be respectful of the adversaries, and speak and treat them
with respect. Yes, it is difficult. It is also a test of faith. Trust
in God and follow his directions. It is almost as if verse 22 is the
challenge. It is almost as if God is saying ``you must trust me first''.
Yes, it is a test of faith. 

If you follow God's direction in this matter, I think that you'll
also find that you'll gain the respect of others. It is ``taking
the high road'' and ``doing the right thing''. With these thoughts
and feelings in your heart, you are well on the way to maturity. 

\subsection{Treatment of, Luke 6:26-31}

Our adversaries in competition should be respected. They are usually
as sensitive as we are. They are competing on the same, hopefully
level playing field. I must admit that there is a natural tendency
to attack, if not hate our adversaries. That is the evil in us.

In these verses, Christ is setting a challenge before us. It is a
test of immense proportions. Christ is telling us to advance spiritually
to the point of over coming our natural tendencies. It is a test of
our faith in Him, and in His ability to guide and protect us. We must
believe that He will overcome in our behalf. If we do not believe,
and we follow our evil instincts instead, we have failed the test.
We must believe and have faith in our convictions.

I liken it to my game of golf. In golf, you have to believe that you
will strike the ball well, or putt the ball straight. Without confidence
in your swing or stroke, you will most likely fail. If you do not
believe in yourself, you are doomed. We've all experienced a lack
of confidence and conviction. In golf, we call it the ``yips''.
We are not sure of ourselves, the thought of failure creeps into our
minds, and we usually fail.

Christ is setting our course for us in these verses. It is not our
option, and in fact, it is a command from God. We must treat our adversaries
as we would want to be treated.

As I prepare for this day of work ahead of me, I am concerned with
having to work with a person who has wronged me in the past. Some
years back, this associate lied to one of my clients for his own personal
gain. My client dismissed my firm from the assignment based upon a
lie. Over time, the client came to realize the truth and now I am
back working with them. I am anticipating a day of trial---my trial.
I was wronged, and I haven't forgotten it. However, Christ is telling
me to forget it, and He will overcome. This is my test of faith, today.
Will I be up to following my Savior's direction? Will I treat my adversary
as I would want myself treated. I pray that I will. I pray that I
will have sufficient faith to pass the test.

\emph{Post hoc}: Yes, it was another timely verse from the Word. My
meeting not only went well, but I built a trusting relationship and
made a friend. I trusted in God, and received my peace. The project
is moving wonderfully. 

\section{Enemies, Treatment of, Romans 12:17--21}

When I am at a loss for a topic to read from the Bible in the morning,
I turn to Romans 12. As a chapter, it is ``loaded''. It is also
a great preparatory message for the morning, prior to work, sports,
or competition of any kind.

These verses also tell us of our personal responsibility, to ``live
at peace with everyone''. We are to work at living in peace with
everyone. To me, this means that we as individuals are to do everything
possible to be at peace.
\index{Bible!characters!Paul}God's words through St.~Paul are qualified
to include ``If it is possible'' and ``as far as it depends on
you''. Yes, sometimes it is not possible and sometimes it doesn't
depend on us as individuals. But, in America, each of us has a voice
and the opportunity to take constructive action, perhaps more so than
in any other society in the world. I believe that these ``qualifying''
clauses are not to be construed as limiting qualifiers, but as enabling
qualifiers. We need to do as much as we can to promote peace within
our means.

Looking at the history of the world over the past few thousand years,
and as Christ has advised, there will continue to be war in the world.
If it is unavoidable, I believe that we need to fight in wars using
all our might and ability, with God's presence. I believe that we
should apply as much force, cunning, and strength as is available
to us, to force as quick a conclusion as is possible. With God's presence,
we will defeat the enemy.

Upon the enemy's defeat, we must quickly change from a war maker to
a peace-maker, helping the surviving enemy to see the error of their
ways and to provide the enemy's survivors with nutrition and the means
to grow in God's ways. I am proud of the history of the U.S.A. during
the 20\textsuperscript{th} Century, with the Marshall Plan to help
Europe recover from World War II, and Gen. MacArthur's plans for the
revitalization of Japan after the same war. These are typical examples
of righteousness. In addition, the U.S.A. not only treated its enemies
with care and compassion, it converted its enemies into allies and
friends. This can be done on a personal basis also.

It requires control, discipline and wisdom. In fact, these are the
things that I believe we need to petition in our prayer to God in
such matters.

\section{Forgiveness of our brothers, Luke 17:3--4}

I am constantly amazed at God's grace. I do not know anyone in this
life that has as much patience with me as God does. I constantly find
myself having to ask God for His forgiveness. My sins are frequent
and many. But yet, God never leaves me, and always forgives me.

These verses remind me of God's grace, and just as importantly, they
remind me of my responsibility to try to emulate God in forgiving
others. Just as God is patient with us, we are to be patient with
our brothers and sisters.

Yesterday, I sent an E-mail to a brother who needed to hear my forgiveness.
He had accepted an offer to run one of our branch offices, and at
the last minute, he rescinded his acceptance of my offer of employment
to continue working with a ``questionable'' group of folks. I know
that his decision had to be God inspired, because it was based upon
prayer. After praying and confiding in other Christian brothers, it
became clear that my forgiveness was important. I do hope that my
forgiveness will eliminate any chains that bind my brother in completing
his mission for Christ.

I might remind you that these verses also outline that, if necessary,
we are to rebuke, then forgive our brothers. As a family in Christ,
we need to help each other, direct each other, and inspire each other.

\section{Honesty, Fairness, Deuteronomy 25:13--16}

Thousands of years ago, God directed and ordered the Israelites to
be fair in their business and competition tactics. They were to be
consistent in their use of standards and measurements. I also believe
that this translates into their pricing tactics. Today, it is a test
of faith for each of us.

Time and again in my business practices, I am challenged on my fees
and charges. Even though our firm established standard fee structures
some time ago, which include a modest profit for our further development,
we are consistently accused of ``gouging practices'' in our pricing.
In our alliances, and long term relationships, we open our books to
those clients, and they are always surprised that we were requesting
fair fees prior to the disclosure. Is it possible that many folks
in this world will attempt gouging, if they can get away with it.
I believe that we all know the answer to that academic question. The
result of such greed creates a world of mistrust and suspicion of
our intentions. We are all expected to be of the type that would take
unfair advantage of others. It is somewhat discouraging, but take
heart. 

Fair, honest and righteous business practices will win over time.
It takes time to prove our trustworthiness. Long term thinking is
required.

Just this morning in a sales interview, after I illustrated through
a slide show, our firm's ``world class'' experience, I was flabbergasted
when I was asked what makes our firm unique enough to be hired by
this client. My first thoughts were to respond by saying ``Didn't
you watch the slide show, because there's no one with this level of
experience?'' I realized that this ``stilted world'' of ours creates
mistrust. Obviously, my client didn't really accept or believe a lot
of what I shared in the slide show. I then began to ``ramble somewhat''
about some of our virtues, and then remembered my only real assurance
of our ability to perform, was our ten year record of having over
90\% (by revenues) repeat client work. Service firms are not hired
repeatedly, if they do not have a high level of performance and a
dedication to service. With this testimony, my client began to shake
his head in positive agreement, and I simply concluded the successful
sales presentation. 

With repeat client business over 90\% each year, it is obvious that
our clients have learned to trust us for the delivery of service and
fairness of fees. However, each new client brings a challenge to our
fees, and gives us an opportunity to shine in an evil world. I personally
monitor our client relations by watching their acknowledged trust
in our character. It is usually the second and third project that
brings the trust. 

I'm sorry to say that on the first fee negotiation with a new client,
the client's standard practice is to reduce our fees downward. If
they don't gain a concession, as a standard, they consider the negotiations
a failure. The client's standard practices are based upon their previous
exposure to less than righteous business individuals in the workplace.
Frankly, this pattern is quite common. However, I've always felt that
exposing our books, our margins and our past performance illustrates
our willingness to have an open relationship, based upon trust. This
approach will build toward a long term, trust based association. After
all, we are all under ``direction'' from a greater source of authority,
and we really have no choice in the matter. If we are to truly be
worthy servants, we must adhere to His directions.

\section{Planning Competition, Luke 14:28--33}

These verses are obviously directed at the level of deliberation,
planning and commitment required to be a disciple of Christ. They
also provide me with a checklist for approaching, planning and competing
in life.

Competition has been fierce in my life. It has been fierce in athletics,
in business, and in the spiritual wars. Consider the following checklist
for planning competition (inspired by these verses):
\begin{enumerate}
\item Accurately assess your own strength and resources; 
\item Assess, with reliable intelligence, your competitor's strengths, weaknesses,
and resources; 
\item Analyze the differences and develop strategies and tactics to benefit
your position; 
\item Train and build resolve; 
\item Follow your plan, and stick to it with courage. 
\end{enumerate}
This checklist promotes the principle of responsible management in
the preparation for and the conducting of competitions. Frankly, those
competitions that I've approached lackadaisically, I've lost. I not
only lost them, but I was embarrassed in them. 

Please don't overlook point 4, above, in the checklist. It may be
one of the most important activities in preparing for competition.
Our attitude and ``focus'' in competition are critical to success.
Regardless of whether we're competing in sports or business, our resolve
will make a significant impact on our performance, and on how we are
perceived by our adversaries. 

From time to time, I work with young wrestlers trying to help them
in their development. I always share with them the wisdom of my collegiate
coach. My coach felt that we should develop our competitive resolve
while experiencing the pain of training or strength building. We were
to develop an attitude of ``controlled rage'', using the pain of
the training as a stimulant or catalyst that could be stored and called
upon when we needed it during competition. 

In order to compete well, our mental composure should include discipline,
focus, and absolute commitment to winning the goal. Some call it resolve.
From personal experience, resolve building should include eliminating
the distractions of other activities, simplifying one's life to reduce
complications, carrying a positive, ``can-do'' attitude, and mentally
``modeling'' (some refer to this as visualizing) for the absolute
victory. In the heat of competition, whether sports, business, combat
or spiritual wars, our minds must not be diverted or cluttered with
outside concerns. We must be absolute in our resolve to win.

\section{Power Struggle, Resolution of, 1 Corinthians 3:5--23}

\index{Bible!characters!Paul}
St.~Paul was a master at diplomacy, and at recognizing situations
of conflict and disruption in the young church. These verses display
that, and they also offer us methodology for handling power struggles,
that might otherwise hinder our missions, our businesses, and our
ability to compete.

As with most activities involving people in this world, greed and
pride interfere. Group programs, involving the structured involvement
of several people, always seem to have power struggles for control,
fame and glory. I seem to have these struggles from week to week in
my business. I can't say that anyone in our company is immune to egocentric
focus, believing that they individually are the basis for our success---including
``yours truly''. I am constantly having to resolve these struggles
amongst my associates, and yes, sometimes the conflicts began with
greed or pride.
\index{Bible!characters!Paul}However, St.~Paul gives us the principles for resolving
these struggles. 

\index{Bible!characters!Paul}
I've tried and tested successfully St.~Paul's approach as follows:
\begin{enumerate}
\item First, identify and confirm the struggle, and the contestants in the
struggle. 
\item Remind everyone involved of the mission and goals---refocusing on
the mission away from individual gain or perspectives (in some cases,
intentionally degrading the importance of personal credit). 
\item Convince all involved that individual gain, personalities, and fame
are less important than reaching the goals of the mission. 
\item Take the initiative to personally endorse the goals above one's personal
gain or pride, even complimenting the contestants with whom you are
struggling. 
\end{enumerate}
Several years ago, my father shared with me that my grandfather had
had a terrible argument with his brother. The argument had developed
over some perceived inequities in spending money to support their
hobbies---hunting and fishing. The feud between the brothers lasted
years, right up until my great uncle's death. My father had attempted
to encourage my grandfather to resolve the dispute with his brother,
as his brother lay on his deathbed. He was successful. Dad reported
that after my great uncle had died, my grandfather professed his sorrow
at being so stubborn and wasting all of those good years in a power
struggle over something that was really trivial and worldly. They
had missed several years of each other's company, hunting and fishing
together, because of their personal feud over less than \$20. Don't
let the pettiness of the world get in the way of your success or enjoyment.
In fact, work at removing the materialistic and pride-centered pettiness
of the world.

Most of the disputes and power struggles in which I've been involved
appear similar after the fact. In many cases, I find myself laughing
at myself for the silliness of letting myself get ``wrapped around
the axle'', as they say. I suppose that such conditions are quite
common. We are human. 

\index{Bible!characters!Paul}
However, there is a ``higher ground'' and a ``greater calling''.
Recognize that many of our activities are part of the mission of furthering
the Church of Jesus Christ, and of maturing our souls so that we might
be worthy to exist with God for eternity. Reminding ourselves of the
purpose of our existence, and following St.~Paul's methodology to
resolve struggles, seems to make these disputes and struggles go away.

\section[Trust, Fairness]{Trust, Fairness, Luke 16:10--12}\label{trust-fairness-luke-1610-12}
\index[luk]{16:10--12}
\index{trust}\index{fairness}

Trust is an absolute term. There are no partial levels of trustworthiness.
Either you are trustworthy, or you are not. While Christ was using
people and their relationships as examples, I believe that He was
also referring to our relationship with God. How can God trust us,
if we are even the least bit questionable in our trustworthiness?
How can He ``let us loose'' in Paradise for an eternity, if we are
not trustworthy?

Similarly, in sports, business, and other forms of competition, fairness
is an absolute term. Either we are fair, or we are not. Our treatment
of our competition, our treatment of our cohorts and teammates, and
our treatment of our employees will determine our fairness and trustworthiness.
Our actions will be the proof of our soul. As Christ states, God's
fairness to us will be based upon our fairness to others. 

As I write these words, I am quaking in my boots. I can recall several
instances when I have not been fair with others. My greed and my desire
to win over-rode my sense of fairness. I suspect that I am not alone.

Time and again, we have been short-changed, cheated, and misled by
unfair tactics and unfair people. I am reminded that even in these
situations, we must still remain trustworthy, fair and righteous.
It is our obligation not only to others, but most importantly, to
God. We must continue to ``take the high road''. It is a test. It
may be one of the toughest tests that our souls face in this world.
However, let us all face this test with the knowledge that we must
pass the test lest we not join God for eternity. Passing this test
is a requisite. 

Let's look at another perspective of these verses of ``trust''.
Can we be trusted with what God gives us? As we know, all of what
we have, from material things to relationships, is the gift and blessing
from God. Do we recognize the source of our blessings? If we do, God
gets the credit, and then the ownership. All comes from God. If so,
then God owns all, and for those material things that He gives us,
whether perceived as earned or not, we are to set aside the ``first
fruits'' for His purpose. Do we always remember this? Are we trustworthy
with God's gifts and blessings? 

I pray that we all pass the test, repeatedly.

\chapter{Law and Contracts}

Law, contracts and enforcement are needed in this evil world in which
we now live. Our world is not paradise, and frankly, it is far from
it. In my opinion, we people are a big part of the problem.

We do not conduct our lives always as God would have us do. Even as
devout believers, we all go through cycles of ``ups and downs''
in our spirit of commitment and our actions. Some days we are very
God inspired and our works are driven and representative of what we
believe God would have us do. On other days, and hours, our actions
can be questionable, if not downright evil. This is due to our
imperfections, and our propensity for drifting from the ``way''.
It is for this reason that laws are written and enforced. It is our
evil waywardness that has driven traders, businesspersons, and others
to develop and rely on contracts to define transactions---even in
marriage.

Christ tells us to ``let our yes be yes, and our no be no!'' However,
in today's world, very few folks will simply accept your word, alone.
That observation in itself offers you a ``telling glimpse'' of our
world and its condition. Most folks require some means of legal retribution,
such as that defined within most contracts. Another tool of insurance or retribution in agreements today is requiring \index{collateral}collateral as part of the transaction. Why have such retribution clauses? Because,
people cannot be trusted. Where I live, you can't pump your own gas
without relinquishing your credit card or cash before the pump is
activated. Why?

\index{evil!prevalence of}Evil and evil people are prevalent in this world of ours. They do
not hesitate to hurt each other for their own gain, and they do not
begin to attempt to bear responsibility for their actions.

I'm sorry to say, but I believe that our legal system, our need for
contracts, and even our need for the professions involved in the definition
of the law and the enforcement of the law, are required. They are
all required due to our evil. I'm sorry to say that for the time being,
we need law and contracts. We simply cannot live and make transactions
in a world submerged in, if not suffocated by, evil, on the strength
of our word and faith alone.

Please do not construe my statement as one of surrender. It is a statement
of observation, which identifies the situation. It is a situation
of evil which can be overcome. We can make a difference, and with
God's help, we know that improvements can be made. With God's help,
perhaps revival of the spiritual values in the world can become a
reality. By understanding our shortcomings, and working on bettering
ourselves, I believe that we will be fulfilling part of God's intent,
to mature our souls. I pray that we all reach a level of maturity
and righteousness, which is acceptable in God's eyes, warranting our
eternal life with Him.

\section[Contract Performance and Conformance]{Contract Performance and Conformance, Luke 17:7--10}


Pride and self-centeredness sneak into all of our lives, as we complete
missions, succeed in business, and win in sports. As Christ reminds
us in these verses, we forget that performing and maturing in this
world are our missions. We should be reminded of our posture in the
world, in the universe, and with God. ``We are unworthy servants;
we have only done our duty''. 

In the past few years, I have noticed an increase in the level of
gratitude on the part of my clients, at the completion of our project
efforts. Attempting to pinpoint the reason for the increase, I have
interviewed my clients. I find that they are pleased with our performance,
because my engineers have done what they said they would do and what
they were contracted to do. How preposterous, in this day and age!
So, isn't that what they should have done? ``We are unworthy servants;
we have only done our duty''.

Further investigations and discussions with folks have revealed to
me that very seldom in this day and age can we expect to receive the
services for which we originally agreed, or even contracted. There
is more emphasis upon the hype of sale, than upon the success and
completeness of the delivery. Something is definitely wrong with our
world, since we can't begin to believe each other. We must be able
to communicate. Lies cloud our communication. Contracts and legal
proceedings will not make it any better. Legal obfuscations, or plays
on words don't help. They further complicate the world and hinder
the survival of good. We have to begin to mean what we say, meet our
commitments---simply, and let our ``yes be yes''. ``We are unworthy
servants; we have only done our duty.''

\index{commitment!to contracts}It seems to me that all first steps in correcting our problems must
start with the individuals involved in the problems---it is a people
thing, not a process thing. One individual provides the services,
goods or products, and another individual receives them. One provides
and one receives. The receiver has made a commitment for compensation,
based upon complete and satisfying delivery. Upon completion of the
delivery of the goods, services or products, a simple ``thank you''
would not be out of order, but it is not necessary. The provider,
upon delivery, should receive his\slash{}her agreed upon compensation, and
that is all. ``We are unworthy servants; we have only done our duty.''

Why is it that we seem to make our life much more complicated than
it need be? We confuse the simplicity of our relationships, not only
with people, but with God. I pray that we all keep our ``eyes on
the ball'', focus on our God, and simplify our lives. ``We are unworthy
servants; we have only done our duty.''

\section{Disputes}


\subsection[Resolving New]{Resolving New, Matthew 5:25--26}
\index[mat]{05:25--26@5:25--26}
\index{disputes!resolving new ones}

``Quickly'' and ``directly'' are the key thoughts of this message.
Christ is telling us, and commanding us, to keep an open relationship
and constant dialogue with all of our associates, particularly our
adversaries.

It seems to me that when communications break down between two parties,
whether friends or foes, confusion and evil reign. Paranoia develops
threatening images which are not real. Ignorance invents motives that
are unfounded. The silence increases our fear. 

In contrast, when we keep an open relationship and a constant dialogue,
we build confidence, understanding, and trust. Open dialogues and
consistent communication require our energy. It must be a proactive
effort. It is not a passive process. 

Similarly, disputes with our adversaries can be resolved through quick,
direct and open relationships. We should not have to rely upon others,
like lawyers, the court system, or arbitrators. We must be courageous
Christians, who face our adversaries with respect, strength and wisdom.
We must look to Christ for our guidance, strength and direction. We
are not alone. We must face our commitments with Christ, to resolve
these disputes of the world. It is our lot.

Some would advise us that we should rely upon our legal system for
resolution of disputes, and that that is the purpose of the legal
system. From my readings of the Old Testament, it seems that the legal
systems of the world today, are based upon God's directions to the
Israelites. Our legal systems today provide processes through which
we can attempt to resolve the disputes within the world, brought on
by evil. They are last resorts, and unfortunately, they are
presently needed. However, regardless of their importance, it is not
necessary to rely upon them as the primary method of resolution. The
court system produces ``winners and losers'', and very seldom (to
never) does it produce friends and renewed friendships. It most often
produces bitterness, if not a sense of wrongdoing and vengeance. Perhaps
these verses are tips for a better solution? 

Let's face it, for those of us who believe, and for those of us who
are servants or soldiers of Christ, these words are not tips, but
they are commands.

\subsection[Resolving Old]{Resolving Old, Deuteronomy 25:1--3}
\index[deu]{25:01--03@25:1--3}
\index{disputes!resolving old ones}

As men and women live their lives, it is almost certain that disputes
will arise as they interact with one another. When disputes do arise,
it is important to attempt to resolve the issues and ultimately, the
disputes. While a simmering or calming period may be advisable, so
that logic might reign, procrastination beyond a calming period might
lead to long term anguish. Unresolved disputes often fester into deep-rooted hatred. Sometimes the hatred can be passed from generation to generation within families and communities.

In the referenced verses, God is directing the Israelites to take
those unresolved disputes to court. As organized by God, the courts
and the judges are to provide fair and just judgment. The judges are
to arbitrate the dispute, issue judgment, and to assign punishment
as it is warranted. Throughout the Old Testament, it is implied that
God will guide the judges, providing them with wisdom and direction.
God works through the judges. However, please note that the dispute
must be brought to light, so that it can be resolved in a fair and
just manner, so that life and people can move on, and put the dispute
in the past---forgiven and forgotten. 

I can't help but think of the horrible crimes against humanity in
the world today, that might have been avoided with the quick and fair
resolution of disputes. The hot spots are Serbia, Ireland, Palestine
and Iraq. Some of these crimes have been portrayed by the media experts as the work of God in defense of His religions. The ignorance of the
cynical media persons of this world astounds me.

Of course, we know
better. Without resolution of a dispute, the pain continues. More
violence is usually added to a dispute, along with heartache, pain
and personal disaster. The hatred becomes self-perpetuating and ever-growing.
Historical perspectives point to a continuation of unresolved disputes,
some of which have existed for decades, centuries, or millennia. I
do wish that these disputes had been resolved in the past, if not
with direct efforts, with efforts guided by the court.

God's formula for justice is not just quick resolution, with his guidance
in judgment, but also fair punishment. Note that God is advising that
the punishment imposed by the courts and judges should be proportional
to the crime, neither too severe, nor too lenient. In fact, the limits of the punishment
are to be targeted to permit the punished individuals to return to
society without being degraded.

The purpose of the court seems to be to resolve the dispute and then put the dispute in the past, so that those in dispute might return to life and society to continue their lives. God's compassion is clear. Our compassion must be directed similarly. We are to issue the punishment, but with a focus toward
healing and returning to normal life. After resolution, we are not
to dwell on the hurt, the infraction, or the emotions of vengeance.
We are to return to our lives---to forgive and forget. I pray that
we all gain some of God's compassion and grace.

\subsection[Healing Mistakes]{Healing Mistakes, Proverbs 26:20}
\index{resolving mistakes}
\index{disputes!healing mistakes}
\index[pro]{26:20}

Some years back, I worked for a fellow who was a dictatorial manager.
As a youngster, learning the design engineering business, I made mistakes
as all youngsters do. However, my manager would rub my nose in my mistakes for months and years. He constantly reminded me of
all of my shortcomings. Eventually, I reached an emotional threshold
where I had to leave that firm, mostly because of my manager's constant
and undue harassment.

When I gave my notice of leaving, my manager sat with me, attempting
to convince me to stay. He couldn't understand why I was unhappy with
his consistent criticism of my actions. Frankly, I'd probably still
be with that firm today if he had simply pointed out my shortcomings
as they occurred or shortly thereafter, discussed how I might improve
things on the next project, and then let the hurt die out. Wanting
to be a mature design engineer, each and every mistake really hurt
me. I was personally embarrassed, somewhat disgusted with myself,
and actually angry with myself. What I didn't need was someone else
prolonging and deepening the pain. However, my old boss lived by the
motto, ``living near the top is like living nearer the fire, it gets
hotter with every move''.

In business, hiring of responsible people of character is foremost in building a people oriented team. People of character take
their mistakes and shortcomings seriously. Inflicting more hurt from
the outside as a manager is not often needed. I suggest that as
a manager of others, it is important that we take the time to discuss
the mistakes or shortcomings, plot a course to avoid such in the future,
and then move on. Continuously reminding folks of their shortcomings
gains nothing and only causes more pain, and eventually emotional
anxiety. Maturity is not the result. Damage to the relationship, if
not the individual, is the product.

Swift resolve is critical to improving the world. Whether we are resolving
disputes, or correcting mistakes, the swiftness and timeliness are
critical. Our human minds are wonderful machines. The mind quickly
forgets the pain, but not necessarily the lesson of the dispute or
mistake, as long as the gossiping and the reminding does not
continue. 

\subsection[Justice In]{Justice In, Deuteronomy 17:2--7}
\index{disputes!justice in}
\index{justice!in disputes}

As I continue to spiritually mature, I've found God's world and universe
to be one of absoluteness. It is not a world of gray---it is black
and white. It is more a world of yes and no---not maybe. In these
verses, God advises us on the process to clear away the questions,
to give definition to the ambiguities, and to reach a conclusion with
absolute certainty. God is absolute and consistent in His promises
to us, and in His covenant. I believe that He expects us to conduct
ourselves in a similar fashion.

\index{evil!defined}
Note that He defines what evil really is. He then provides us
with a methodology of determining what is true. His methods, of relying
on more than one witness, and then to include the witnesses in the
administration of the punishment, become a process of assurance in
search of the truth. The process includes checks and balances, relying
on human commitment to determine the truth. His process also includes
complete investigation. His direction of a thorough investigation is what commissions our police forces worldwide. ``You must purge the evil
from among you'' is an order to not acquiesce. We cannot sit back
and permit evil to exist and persist. We are required to take action.

As I write these thoughts, our country is experiencing a political
and moral crisis involving our leader. He has admitted to evil. However,
many are rising to his support, suggesting that we overlook the evil
due to the good condition of our economy, peace in the land, and our
leader's other accomplishments. These verses tell us that that we cannot overlook the situation. Yes, we are to forgive and move on, but we
must take a stand on evil. We must follow God's direction in this
matter, then, with justice we can heal our wounds and proceed with our lives.


\section[Law and Grace]{Law and Grace, Romans 4:1--16}
\label{chap7:law_and_grace}
\index{grace and law}
\index{law!and grace}
\index[rom]{04:01--16@4:1--16}

\begin{center}
\emph{For related thoughts on law and grace, see \ref{chap2:law_and_grace} and \ref{chap2:law_v_grace} on pages \pageref{chap2:law_and_grace} and following.}
\end{center}

Law and grace are often points of confusion to many as they read the
Bible. I am one of those who has searched, discussed and asked for
help in these matters.
\index{Bible!characters!Paul}St.~Paul's verses from this chapter are very helpful in understanding the differences, and yet similarities of grace and the law. 

Please re-read verse 4, and take the time to think about the attitude
and perception of the worker. While I believe that wages are the obligation
of the employer, take the time to think about this from the perspective
of the worker. Most of my employees, who are leaders in the workforce
of my business, are not focused on their compensation. The successful,
enlightened, and very happy folks in my work force do not look upon
their work as a mere ``for hire'' situation. They perform their work with a sense of mission, a sense of contribution to the greater whole, and a genuine love for their work. They take on the air of ownership
of our firm, even if they do not own stock. Their performance is not
motivated by work rules and accompanying compensation, but is driven
by a sense of mission, contribution and self-satisfaction---equivalent
to faith.

In other words, it is not as if their work contributions
are driven by contract, whip, or salary, but their contributions
and work are self-motivated by the individuals' love for their work.
This concept is analogous to our Christian works being inspired
by our faith. Our faith is powered and generated, not by God's laws (even though we must comply with them), but by God's grace. 

The real differences in concept between law and grace lie at the
source of the inspiration of our Christian works. Are our works driven
by peer pressure, God's law, or by direction from the church? If any
of these are the drivers of our works, we can say that we live by law and we need to develop a better (if not real and loving) relationship with God. These are outside sources for our driv, and we are simply reacting to them. If these outside sources are our only drivers,
we might as well be robots. We, and our souls are not thinking, or believing for ourselves.

Our lives, our works, and our activities should be inspired by our
faith and our belief in God---generated by His grace. Our existence
and our lives should be driven by our belief and our sense of love
and commitment. If our lives are guided by our love of God and our
works inspired by His teachings, chances are real good that
we will be living within the laws. 

Recognize that laws are created for those who do not have faith and
for those who require laws to define the difference between right
and wrong. Our faith as Christians is not simply based upon laws; our faith is built upon the foundation of God's grace. Our religion is based upon His love and compassion, modeled from His grace. Our religion is not one of law enforcement and law compliance only. 

This message can be found throughout the Gospels and I've come to
believe that this message may be one of Christ's primary objectives
in His mission on earth. As an example, He has consistently boiled
down the ten commandments into two commandments. His two commandments
require our love for others and for God (please refer to \index[mat]{22:37--40}Matthew 22:37--40).

With true and real love for others and for God consistently
in our heart, we then will be complying with God's goal for our faith.
And yes, an audit will then prove that we are complying with the ten
commandments, and thus, complying with the law of God. From my personal experience, the works generated by God's love and inspired by His
teachings, will continue to bear fruit.

\section{Law}
\index{law}

\subsection[Compliance With]{Compliance With, Romans 13:1--5}
\index{law!compliance with}
\index[rom]{13:01--05@13:1--5}

I'm sure you've heard the criminals in cops and robbers movies
say, ``Laws were made to be broken''. Conversely, these verses give
us the reasons why the laws are not meant to be broken.

Time and again, especially in tax laws, we are tempted to bend
the law and interpret the tax laws a little bit toward our liking.
Please be reminded of these verses when you are working on your taxes
next. Laws are not meant to be broken. Compliance with the law is
an order from God, and this order is a test of our faith. It is a
matter of conscience. Like sharing your first fruits with God,
we must consider this test as part of the thread between God and
us. It may be that critical.

\index{Bible!characters!Paul}
As St.~Paul wrote these inspired words, most of the governments and
the law makers of his era were not as democratic as that found in
the U.S.\ today. While Rome had a fine early democracy, it was a democracy
that was not friendly to Christians.
\index{Bible!characters!Paul}Regardless, Paul was inspired
to encourage all of the followers to have respect for the laws. The
laws, the lawmakers, and the law enforcers were and are the workers
of God. Their authority is God-commissioned. Hopefully, it is even
more so today in America, where we elect our lawmakers on a regular basis and we have access to their ear. With our right to vote, we
into place our lawmakers and, in some cases, our law enforcers. We have even more reason to respect and comply with the laws, than in the
Roman times of St.~Paul.

Compliance with the law does not mean acquiescence in the face of
abuse of power by our lawmakers, leaders or law enforcers. It does
not mean turning our faces away from corruption. It seems to me that compromising a little, leads to sacrificing quite a bit. Yes, we've got to take
a stand out of respect for God's commissioning of the laws. Just as
we as individuals are to respect the law and it's makers and enforcers,
we are to assure that the makers and enforcers also respect the law
and enforcement of it. 

\subsection[Versus Grace]{Versus Grace, Romans 6:1--23}
\index{law!versus grace}
\index[rom]{06:01--23@6:1--23}

Christians and believers throughout time have faced the difficulty
of understanding the difference between right and wrong, sin and righteousness.
Before we became believers, we ignorantly lived in sin and (speaking
personally) I was quite ignorant of the difference between sin and
righteousness. My basis for judgment in my life was what felt good,
what my heroes in the movies portrayed, and what was within the law
of the land.

After coming to know Christ, and growing in His spirit
and knowledge imparted in the Bible, my basis for judgment in my life
has changed dramatically. My ignorance of the difference between sin
and righteousness has slowly evaporated. However, my growing understanding
of the difference between sin and righteousness has also raised a
new level of consciousness that bears with it responsibility. I am
now aware of the differences and I am expected by God to attempt
to control my decisions accordingly.

Do I continue to sin? Of course, knowing that I am still human, it
doesn't take me long to build quite a list of sinful actions just
in the past few hours. Must I do something about the sinning? ``Yes'' is the answer to that question! By acknowledging our sins before God, and with a raised
sense of conscience of and responsibility for our sins, I believe that
we each have a better chance of doing something about correcting our
sinning---and this something is classically referred to as repentance.

As we mature, as we grow more spiritually compatible with God, and
as we become more responsible believers, the responsibility
and inspiration for our actions change from outside influences, like
the law and others, to a guiding righteousness from within.
I believe that as we continue to grow spiritually, we are also preparing
ourselves for our eternal life with God. As we progress beyond simple compliance with the law and because of the growing righteous from within, we gain freedom.

Our freedom is a test in itself. Will we become more like the companions that might be acceptable to God, for eternity in His Kingdom? Can we do it on our own? No! We must constantly remind ourselves that it is only through God's grace that we can mature and be saved. With
God's grace, we will become strong enough to overcome the lure and
enticement of evil and sin. With God's grace, we will shed the desire
for evil, and with God's help, we will become strong, disciplined
and obedient to His wishes. We will be living His wishes because we
want to, not because it is simply the law.

\section{Parenthood}
\index{responsibility!parental}
\index{parenthood}

I have included Parenthood and the issues of parenthood under ``Law
and Contracts'' for several reasons. 
\begin{enumerate}
\item First, most business people are also parents, and there is a need
to constantly remind one another of one of our primary responsibilities---parenthood.
\item Secondly, the procreation of children should be accompanied by an
understanding of the serious consequences and responsibilities involved,
and the ``contracted'' commitment to the proper raising of children.
\item Thirdly, business management and parenthood have many similarities.
The process of creating and promoting teamwork in business is nearly
the same as in parenthood and the family life. The focus on growth
and maturity for all in business is the same as in the household when
raising children. When raising children, we are not always teaching,
but often we find ourselves leading by example and certainly learning
from the experience. 
\end{enumerate}
As we focus on improving our world and the people within the world,
I personally believe that one of our greatest shortcomings, and potentially
most fruitful areas for improvement, is in the raising of children.
Poor child-rearing and the lack of proactive efforts in family building
is not germane only to the financially poor and destitute. Poor child-rearing
and lack of family focus is common to all economic classes. It is
a plague on our society, country and world.

My children are now nearly grown and on their own. I pray repeatedly
for their protection and guidance. I also ask God to enable them that
they might become worthy servants in His eyes.

Approximately ten to fifteen years ago, when they were about to enter
their teenage years, I suddenly came to the realization, that my children
may be my greatest contribution to this world. What a shock to my
system. For my entire life, I had been focusing on my development
and my contributions to the world. Yes, I had led a self-centered
life. As a result of this revelation, my focus was now redirected
from myself and my accomplishments to my children, and their potential.
Concurrent with this revelation, I also realized that my children
were much better people than I. I suddenly realized that in the ``engineering''
of our kids, my wife and I had exposed them to a broad spectrum of
experiences that should make them far better people than we have the
potential to be. Their education, exposure to the cultures of the
world, exposure to God fearing people who are not afraid to discuss
their faith openly, exposure to wonderful sports competitions, exposure
to history through extensive travel, and our complete commitment to
their raising has many of the ingredients for building ``worthy servants''
of God.

History may prove that as a mission, child rearing and our focus on
its importance, may either be our greatest victory, or our demise.
And yes, our Bible is loaded with much more on child-rearing and the
development of the family (enough to fill a book by itself), than
I've listed below. However, I've included these few ``loaded'' verses
directed toward the ``business person\slash{}parent''.

\subsection[Discipline]{Discipline, Proverbs 29:15--18}
\index[pro]{29:15--18}
\index{parenthood!discipline}
\index{discipline}

While I find wisdom in each of these verses, I find the overall principled
message in the combination of the verses---which is really quite
a poem to discipline, self control, and character. These verses promote
the building of discipline and character in our children as they are
being raised, so that the ``job of raising them'' is complete, and
then the children can live their lives under their own ``trained''
guidance.

Human beings are not robots. They have emotions, and yet logic. The
two, logic and emotions, cannot be separated. Within their emotional
composition, people must apply logic to determine for themselves the
difference between right and wrong or the difference between legal
and illegal. This judgment must be based upon an established foundation
or benchmark---usually training and education. Simply reasoning with
children doesn't always penetrate their emotional make-up. More severe
measures are often required to gain their attention, forcing them
to concentrate on the issues, and ultimately have them absorb and
adopt the wisdom. Children cannot be left to themselves. If so, we
simply have ignorance, and little or no wisdom for determining what
is right and wrong.

It is the same with team members in business. When recruits are brought
into business from the university environment, they are ``rough blocks
of stone'' waiting for the sculptor's hand. The recruits have a foundation
of education, but they are not usually familiar with the process of
business, the ethics of the market place, and the procedures necessary
for their continued growth. Formalized training programs are yet required
to develop them into productive team members. In addition, they are
usually searching for role models of behavior in the world of business.
Focus and effort on the further development of young business persons'
character is a must. Structured training that explains the character
of business persons is required. The training should include not only
desirable character traits, but also the character traits of what
might be expected from ``less than principled'' adversaries. For
those of us who ``believe'', we know that the best source for structured
education is the Bible.

\subsection[Education]{Education, Hebrews 6:1--8}
\index{parenthood!education}
\index{education}
\index[heb]{06:01--08@6:1--8}

These verses talk to us, not only as parents, focused on the education
of our children, but also with regard to our own education. In reading
and re-reading these verses, our maturity and knowledge of God is
very much like our gaining of wisdom in any subject. The steps to
knowledge include: 
\begin{enumerate}
\item Building and accepting the foundation upon which our understandings
and our beliefs are based; 
\item Through maturity in the belief, and through gaining greater wisdom
in the subject, we can put our wisdom and understanding into productive
and active use---as engineers say, ``moving from the potential to
the kinetic''; 
\item Maintaining a solid posture and being steadfast in our commitment
to the foundation of our principles and belief. 
\end{enumerate}
In other words, we need to learn the basics, to gain further maturity,
to use the information portrayed in acts, and to be steady and consistent
in our belief and actions. It sounds simple enough, doesn't it?

However, it is not, at least from the perspective of our commitment
to God. I marvel at some of my Christian brothers and sisters in their
steadfastness and their apparent commitment to God, day in and out.
\index{Bible!characters!Paul}St.~Paul warns us of the dangers and penalties involved in wandering from the way, particularly when we are well aware of the consequences,
and when we have committed our souls and our existence to Jesus Christ.
This is not something to take lightly. What's at risk is our eternal
life, and the potential ``public disgrace'' of our Savior.

It appears to me that we return again to the concept of absoluteness
in this world. In other words, there is no ``gray level'' of commitment---it
is black or white. We are either totally committed to Christ, or we
are not. Our commitment is forever, or never---time cannot and should
not change our position. We either are, or we are not. We are either
saved, or we are not. We either belong to Christ, or we do not. We
will have eternal life, or we will not. We will enjoy the ``drink
of rain'', or we will burn.

\subsection[Fatherhood]{Fatherhood, Ephesians 6:1--4}
\index{parenthood!fatherhood}
\index{fatherhood}
\index[eph]{06:01--04@6:1--4}

\index[eph]{06:04@6:4}Verse 4 does not leave much room for interpretation. It is a direct
order, and frankly, it is an order loaded with love and commitment.

As children grow, they are learning machines. They can absorb not
only the wisdom of books, teachers and preachers, but they can adopt
the values of their role models. As parents, we are the role models.
We have no choice in the matter. Our children look to us for the way.

On one trip home from an early morning hockey game, in which my son
had been ravaged by some larger boys, he seemed to be quite upset.
I asked him what was bothering him. He looked at me and asked if I'd
take him to the doctors office, to see how large he'd grow. I shared
with him that in our family, we come in all sizes, and I was sure
that the doctor would agree. I then asked him if he was feeling small
on the ice. He nodded affirmatively. I suggested that he was the fastest
man on the ice that day, and he responded with, ``I have to be!''
Finally agreeing with me that he'd grow larger, but to what extent
only God knew, he asked me how tall I was. When I told him, he then
looked at me, and said with comfort, ``That will be just fine.''
We are their role models---good or bad.

I believe that it is our responsibility as a parent, to provide them
with the best role model possible---a role model that they can ``honor''
in complying with one of the Ten Commandments. As their role model,
we are in a position to not only passively transfer wisdom, but we
are in a position to take an active role in their education.

From my brief decade or more of studying God's Word, the Lord's ways
and words are wonderfully complete, and yet simple. They provide wisdom
to depths beyond my comprehension, while offering a sense of understanding
that is nearly black or white---with no conflict. When we proactively
teach our children of the Lord's ways and Word, we are offering clarity
and definition to not only our children's lives, but to the entire
universe in their minds.

It is not always easy for children to understand the meaning of their
life, especially as it relates to the purpose of their existence in
the universe. Without understanding their purpose, the mission of
their life, and their relationship with the universe, they become
frustrated, anxious, confused, and yes, exasperated. Teaching them
the ways and the Word of the Lord, will provide them with not only
wisdom and understanding, but it will provide them with a peace that
they cannot find nor obtain anywhere else. Their wisdom and understanding
will provide them with a sense of purpose and ``belonging''.

Each day, the news brings us acts of violence by outcast children.
Our media tries to find answers as to where the community went wrong,
how we might change our laws, and how we might control children and
adults. These issues are simply scratching the surface of the matter.
The real solutions run deeper. Like most things in life, and in the
process of entering the Kingdom of Heaven, it is the hard and narrow
path. It takes work with spiritual depth. The real solutions can be
found in the teaching and love of the children. Their understanding
of their purpose and belonging in the world is a big part of the solution.

I am a living example of this truth. I spent most of my life curiously
searching for this wisdom---unfortunately, in the wrong places. When
I share with my brothers and sisters in Christ the frustrations of
my past wanderings, I usually receive compassionate head shakes of
disbelief. It is almost as if they are thinking, ``and It was there
all along, but he didn't open his eyes!''

\section[Tough Love]{Tough Love, Hebrews 12:7--11}
\index{parenthood!tough love|see {love}}
\index{tough love|see{love}}
\index{love!tough}
\index[heb]{12:07--11@12:7--11}

I just finished a long lunch with one of my clients, who shared with
me his recent hardship. In the process of coming through open-heart surgery, a trauma which he approached with reluctance, a problem developed.
As he was just coming out of his anesthetic fog, the head surgeon
of this top medical facility, advised him that they needed to go
back in and retrieve a left behind medical instrument. The
inventory of medical instruments and the resulting X-rays, confirmed
that the missing instrument was next to my friend's heart. The head
surgeon was taking the responsibility for the mishap, and was recommending
that they go back in, to retrieve the instrument as soon as possible.

My friend and client, in a drugged, frightened and frustrated state,
was being asked to approve the surgery. His emotions were cluttered
with anger at the mishap, self-pity for the situation, and wonder
over God's putting him in this situation. Atop all of this, he was
physically very weak.

As my client's fears grew over this big decision, he heard God say
to him, ``I was with you for the first surgery, and I will stay with
you for the second surgery, and all will be well!''

My friend was sure of the voice that he had heard. He stated that
the tone of the voice, the strength of the words, and the comfort
of the fatherly statement were definitely God's, and he had never
experienced such in his entire life. He told me that he felt the spirit
and presence of God, and stated that he has never felt such peace
in his entire life. He told me that the spirit of God also spread
from him, to his family (who were gathered around his recovery bed).
Prior to God's words, they were quite distraught over the situation.
I still get goosebumps thinking of his story.

While there is much more to the story, if not a book in itself, the
outcome of the experience was that it was quite a trial and a hardship.
I asked my friend how he now views this experience, and he concludes
that the experience was ``a blessing, and he thanks God for the experience''.

I can share with you that not only is my friend stronger in his belief
and commitment to God, but hearing of the experience, further strengthened
my intensity of belief, further heightened my joy of knowing God,
and further developed my commitment to be a servant of God.

We believe that God put my friend through this hardship to make him
more righteous and to give him a taste or glimpse of the peace that
only God can bring. My friend declares that his exposure to God was
accompanied with a sense that he didn't want to ever leave. It truly
``is a peace that surpasses all understanding and comprehension''.
I can't help but think that perhaps this ``tough love'' is in fact,
the purpose of our existence in this world and this life. It is to
mature us, so that we might be better companions for God, for eternal
life with Him.

And yes, just as we are receiving tough love from our Father, we should be sharing it with our children.
They must also endure hardships and they must mature as we are.
Life cannot be too easy for them. If it is, they will not mature.

Tough love is not always easy when you are a parent.
From the time our children rest in our arms as infants, we are postured to help, guide and protect them.
I often forget that they must learn to be independent, and that measured trials and maturing suffering will build them.
Sometimes I wonder if the successes and victories in their lives do not have less of a maturing impact, than the defeats in their lives.

I have had to punish my children from time to time.
I refer to it as mental discipline building.
My punishment has been psychological more than physical, and ``engineered'' to have them focus on the wrong-doing.
My engineered approach includes my severe disappointment with their attitude and action, combined with a significant display of love and compassion.
It forces my children to concentrate on the principle and value that I am reinforcing.
Recognizing my absolute love for my children, instituting punishment is not easy and contrary to my normal desires.
However, I know that in the long run, they need it, and will grow from it.
In fifty years, I'll let you know how well it worked!

\backmatter

\chapter{Acknowledgements}

I thank God for the timely wisdom and experience of my friend Lynn, and I thank God for His exposing the wisdom of His word to us.
While we are slow to comprehend His word, and slow to accept His word, He is ever patient and graceful with us. Thank you.


\chapter{Indexes}

\section{Index Of Topics}

\vspace{-3em}\renewcommand\indexname{}
\printindex

\clearpage

\section{Index Of Bible Texts}

\printindex[gen]
\printindex[exo]
\printindex[lev]
\printindex[num]
\printindex[deu]
\printindex[jos]
\printindex[jdg]
\printindex[rut]
\printindex[1sa]
\printindex[2sa]
\printindex[1ki]
\printindex[2ki]
\printindex[1ch]
\printindex[2ch]
\printindex[ezr]
\printindex[neh]
\printindex[est]
\printindex[job]
\printindex[psa]
\printindex[pro]
\printindex[ecc]
\printindex[son]
\printindex[isa]
\printindex[jer]
\printindex[lam]
\printindex[eze]
\printindex[dan]
\printindex[hos]
\printindex[joe]
\printindex[amo]
\printindex[oba]
\printindex[jon]
\printindex[mic]
\printindex[nah]
\printindex[hab]
\printindex[zep]
\printindex[hag]
\printindex[zec]
\printindex[mal]
\printindex[mat]
\printindex[mar]
\printindex[luk]
\printindex[joh]
\printindex[act]
\printindex[rom]
\printindex[1co]
\printindex[2co]
\printindex[gal]
\printindex[eph]
\printindex[phi]
\printindex[col]
\printindex[1th]
\printindex[2th]
\printindex[1ti]
\printindex[2ti]
\printindex[tit]
\printindex[phm]
\printindex[heb]
\printindex[jam]
\printindex[1pe]
\printindex[2pe]
\printindex[1jo]
\printindex[2jo]
\printindex[3jo]
\printindex[jud]
\printindex[rev]

\cleardoublepage

\end{document}
